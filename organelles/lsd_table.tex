% Created 2020-08-26 Wed 15:26
% Intended LaTeX compiler: pdflatex
\documentclass[landscape]{article}
\usepackage[utf8]{inputenc}
\usepackage[T1]{fontenc}
\usepackage{graphicx}
\usepackage{grffile}
\usepackage{longtable}
\usepackage{wrapfig}
\usepackage{rotating}
\usepackage[normalem]{ulem}
\usepackage{amsmath}
\usepackage{textcomp}
\usepackage{amssymb}
\usepackage{capt-of}
\usepackage{hyperref}
\usepackage{longtable}
\usepackage[margin=0.45in]{geometry}
\date{}
\title{Lysosomal Storage Disorders}
\hypersetup{
 pdfauthor={Matthew Henderson},
 pdftitle={Lysosomal Storage Disorders},
 pdfkeywords={},
 pdfsubject={},
 pdfcreator={Emacs 26.3 (Org mode 9.3.7)}, 
 pdflang={English}}
\begin{document}

\begin{longtable}{llllll}
\caption{\label{tab:org3db5f8f}Lysosomal Storage Disorder Biochemistry}
\\
Disease & Protein & Accumulation & Biomarker & Diagnostic & Gene\\
\hline
\endfirsthead
\multicolumn{6}{l}{Continued from previous page} \\
\hline

Disease & Protein & Accumulation & Biomarker & Diagnostic & Gene \\

\hline
\endhead
\hline\multicolumn{6}{r}{Continued on next page} \\
\endfoot
\endlastfoot
\hline
\textbf{Mucopolysaccharidoses} &  &  &  &  & \\
MPS I (H, H/S, S) & \(\alpha\)-Iduronidase & DS, HS & GAGs(U) & E(L,F)Pd, M & \emph{IDUA}\\
MPS II (Hunter) & Iduronate sulfatase & DS, HS & GAGs(U) & E(L,F,P), M & \emph{IDS} (XL)\\
MPS III A (Sanfilippo A) & Heparan sulfatase & HS & GAGs(U) & E(F), M & \emph{SGSH}\\
MPS III B (Sanfilippo B) & Acetyl-\(\alpha\)-glucosaminidase & HS & GAGs(U) & E(L,F,S), M & \emph{NAGLU}\\
MPS III C (Sanfilippo C) & \small{Acetyl-CoA:glucosamine N-acetyltransferase} & HS & GAGs(U) & E(L,F), M & \emph{HGSNAT}\\
MPS III D (Sanfilippo D) & N-acetyl-glucosamine-6-sulfatase & HS & GAGs(U) & E(L,F), M & \emph{GNS}\\
MPS IVA (Morquio A) & N-acetyl-galactosamine-6-sulfatase & KS, CS & GAGs(U) & E(L,F), M & \emph{GALNS}\\
MPS IV B (Morquio B) & \(\beta\)-Galactosidase\footnotemark & KS & GAGs(U) & E(L,F,)Pd, M & \emph{GLB1}\\
MPS VI (Maroteaux-Lamy) & Arylsulfatase B & DS & GAGs(U) & E(L,F)\footnotemark, M & \emph{ARSB}\\
MPS VII (Sly) & \(\beta\)-Glucuronidase & DS, HS, CS & GAGs(U) & E(L,F)Pd, M & \emph{GUSB}\\
MPS IX (Natowicz) & Hyaluronidase & Hyluronic acid & - & E(L,F), M & \emph{HYAL1}\\
\hline
\textbf{Sphingolipidoses} &  &  &  &  & \\
Fabry & \(\alpha\)-Galactosidase A & Gb3 & Gb3(U) & E(L,F,S)Pd, M & \emph{GLA} (XL)\\
Farber & Acid ceramidase & Ceramide & - & E(L,F), M & \emph{ASAH1}\\
GM1 & \(\beta\)-Galactosidase\textsuperscript{\ref{orgff49c04}} & GM1 ganglioside, KS, & Oligos(U) & E(L,F)Pd, M & \emph{GLB1}\\
 &  & oligos, glycolipids & GAGs(U) &  & \\
GM2 Tay-Sachs & \(\beta\)-Hexosaminidase A (\(\alpha\) subunit) & GM2 ganglioside & - & E(L,F,S)Pd, M & \emph{HEXA}\\
GM2 Sandhoff & \(\beta\)-Hexosaminidase A \& B (\(\beta\) subunit) & GM2 ganglioside, oligos & Oligos(U) & E(L,F)Pd, M & \emph{HEXAB}\\
Gaucher & \(\beta\)-Glucocerebrosidase & Glucocerebroside & Chito\footnotemark(S) & E(L,F), M, BM & \emph{GBA}\\
Krabbe & Galactocerebrosidase & Galactocerebroside & Psychosine(B) & E(L,F), M & \emph{GALC}\\
MLD & Arylsulfatase A & Sulfatides & Sulfatides(U) & E(L,F)Pd\textsuperscript{\ref{orgfb290b0}}, M & \emph{ARSA}\\
Niemann-Pick A \& B & Sphingomyelinase & Sphingomyelin & Chito\textsuperscript{\ref{org6d95a1b}}(S) & E(F), M, BM & \emph{SMPD1}\\
\hline
\textbf{Activators} &  &  &  &  & \\
GM2 AB variant & GM2 activator protein & GM2 ganglioside, oligos & Oligos(U) & M & \emph{GM2A}\\
MLD/Fabry & Saposin B & Sulfatides, Gb3 & Sulfatides(U), Gb3(U) & M & \emph{PSAP}\\
Krabbe & Saposin A & Galactocerebroside & Psychosine(B) & M & \emph{PSAP}\\
Gaucher & Saposin C & Glucocerebroside & Chito\textsuperscript{\ref{org6d95a1b}}(S) & M & \emph{PSAP}\\
\hline
\textbf{Glycogenoses} &  &  &  &  & \\
GSD II (Pompe) & Acid \(\alpha\)-glucosidase & Glycogen & CK(S) & E(L\footnotemark,F), M & \emph{GAA}\\
\hline
\textbf{Oligosaccharidoses} &  &  &  &  & \\
Aspartylglicosaminuria & Glycosylasparaginase & Aspartylglucosamine & Oligos(U) & E, M & \emph{AGA}\\
Fucosidosis & \(\alpha\)-Fucosidase & Glycoproteins, & Oligos(U) & E(L,F)Pd, M & \emph{FUCA1}\\
 &  & glycolipids, &  &  & \\
 &  & fucoside-rich oligos &  &  & \\
\(\alpha\)-Mannosidosis & \(\alpha\)-Mannosidase & Mannose-rich oligos & Oligos(U) & E(L,F), M & \emph{MAN2B1}\\
\(\beta\)-Mannosidosis & \(\beta\)-Mannosidase & Man(\(\beta\)1-4)GlnNAc & Oligos(U) & E(L,F), M & \emph{MANBA}\\
Schindler & N-acetylgalactosaminidase & Sialylated & Oligos(U) & E(L,F), M & \emph{NAGA}\\
 &  & asialoglycopeptides, &  &  & \\
 &  & glycolipids &  &  & \\
ML-1 Sialidosis & \(\alpha\)-Neuraminidase & Oligos, glycopeptides & Bound SA(U), & E(F), M & \emph{NEU1}\\
 &  &  & oligos(U) &  & \\
\hline
\textbf{Lipidoses} &  &  &  &  & \\
Wolman/CESD & Acid lipase & Cholesterol esters & - & E(L,F), M & \emph{LIPA}\\
\hline
\textbf{TMP defect} &  &  &  &  & \\
\textbf{Transporters} &  &  &  &  & \\
Sialuria (Salla) & Sialin & Sialic acid & Free SA(U) & M & \emph{SLC17A5}\\
Cystinosis & Cystinosin & Cystine & Cystine(L) & M & \emph{CTNS}\\
Niemann-Pick C & NPC1 or NPC2 & Cholesterol, & Chito\textsuperscript{\ref{org6d95a1b}}(S) & Filipin, M, BM & \emph{NPC1},\\
 &  & sphingolipids &  &  & \emph{NPC2}\\
\textbf{Structural Proteins} &  &  &  &  & \\
Danon & LAMP-2 & Cytoplasmic debris, & - & M & \emph{LAMP2} (XL)\\
 &  & glycogen &  &  & \\
Mucolipidosis IV & Mucolipin & Lipids & - & M & \emph{MCOLN1}\\
\hline
\textbf{Protection Defect} &  &  &  &  & \\
Galactosialidosis & Protective protein cathepsin A & Sialyloligosaccharides & Bound SA(U), & E(F,L)\footnotemark, M & \emph{CTSA}\\
 &  &  & oligos(U) &  & \\
\hline
\textbf{PTP defect} &  &  &  &  & \\
MSD & Formylglycine-generating enzyme & Sulfatides, & Sulfatides(U), & E\footnotemark, M & \emph{SUMF1}\\
 &  & glycolipids, GAGs & GAGs(U) &  & \\
\hline
\textbf{Trafficking defect} &  &  &  &  & \\
ML-II/III & GlcNAc-1-P transferase & Oligos, GAGs, lipids & Oligos(U), GAGs(U) & E\footnotemark, M & \emph{GNPTAB},\\
(I-cell/Pseudo Hurler) &  &  &  &  & \emph{GNPT}\\
\hline
\textbf{Degradation defect} &  &  &  &  & \\
Pycnodysostosis & Cathepsin K & Bone proteins & X-ray & M & \emph{CTSK}\\
\hline
\textbf{NCLs} &  &  &  &  & \\
NCL 1 & Palmitoyl protein thioesterase & Saposins A and D & EM of skin & E, M & \emph{PPT1}\\
NCL 2 & Tripeptidyl peptidase 1 & CV(c) & EM of skin & E, M & \emph{TPP1}\\
NCL 3 & CLN3, lysosomal TMP & CV(c) & EM of skin & M & \emph{CLN3}\\
NCL 5 & CLN5, soluble lysosomal protein & CV(c) & EM of skin & M & \emph{CLN5}\\
NCL 6 & CLN6, ER TMP & CV(c) & EM of skin & M & \emph{CLN6}\\
NCL 7 & CLC7, lysosomal chloride & CV(c) & EM of skin & M & \emph{MFSD8}\\
NCL 8 & CLN8, ER TMP & CV(c) & EM of skin & M & \emph{CLN8}\\
NCL 10 & Cathepsin D & Saposins A and D & EM of skin & M & \emph{CTSD}\\
\end{longtable}\footnotetext[1]{\label{orgff49c04}GM1 and MPS-IVB appear to be a continuum of phenotypes}\footnotetext[2]{\label{orgfb290b0}p-nitrocatecol sulfate is used for \(\alpha\)(MLD) and
\(\beta\)(Maroteaux-Lamy) arylsulfatase assays. \(\beta\) is inactivated with
0.25M sodium pyrophosphate. \(\alpha\) is inactivated with barium}\footnotetext[3]{\label{org6d95a1b}5-7\% of the population have an AR defect in the
chitotriosidase gene, which leads to false-negative values}\footnotetext[4]{\label{orgba045ab}GAA measurement in leukocytes is unreliable due to
neutral \(\alpha\)-glucosidase activity in acidic conditions. Acarbose can
be used to inhibit neutral \(\alpha\)-glucosidase activity in leukocytes}\footnotetext[5]{\label{org681cffa}Leads to \(\beta\)-galactosidase and neuraminidase deficiency}\footnotetext[6]{\label{orgd484b25}Decreased activity of multiple sulfatases}\footnotetext[7]{\label{org21dc58e}Activity of MPR targeted lysosomal hydrolases is elevated in
plasma and decreased in cultured fibroblasts}
\begin{description}
\item[{Abbreviations}] PTP, post-translational processing, TMP,
transmembrane protein; MSD, multiple sulfatase deficiency; SA,
sialic acid; CV(c), Subunit c of ATP synthase; Chito,
chitotriosidase; EM, electron microscopy; U, urine; B, blood; BM,
bone marrow biopsy; L, leukocytes; F, fibroblasts; S, serum; P,
plasma; Pd, pseudodeficiency
\end{description}
\end{document}