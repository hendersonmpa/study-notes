% Created 2020-01-30 Thu 10:20
% Intended LaTeX compiler: pdflatex
\documentclass{scrartcl}
\usepackage[utf8]{inputenc}
\usepackage[T1]{fontenc}
\usepackage{graphicx}
\usepackage{grffile}
\usepackage{longtable}
\usepackage{wrapfig}
\usepackage{rotating}
\usepackage[normalem]{ulem}
\usepackage{amsmath}
\usepackage{textcomp}
\usepackage{amssymb}
\usepackage{capt-of}
\usepackage{hyperref}
\hypersetup{colorlinks,linkcolor=black,urlcolor=blue}
\usepackage{textpos}
\usepackage{textgreek}
\usepackage[version=4]{mhchem}
\usepackage{chemfig}
\usepackage{siunitx}
\usepackage{gensymb}
\usepackage[usenames,dvipsnames]{xcolor}
\usepackage[T1]{fontenc}
\usepackage{lmodern}
\usepackage{verbatim}
\usepackage{tikz}
\usepackage{wasysym}
\usetikzlibrary{shapes.geometric,arrows,decorations.pathmorphing,backgrounds,positioning,fit,petri}
\usepackage{fancyhdr}
\pagestyle{fancy}
\author{Matthew Henderson, PhD, FCACB}
\date{\today}
\title{Mitochondrial Disease}
\hypersetup{
 pdfauthor={Matthew Henderson, PhD, FCACB},
 pdftitle={Mitochondrial Disease},
 pdfkeywords={},
 pdfsubject={},
 pdfcreator={Emacs 26.1 (Org mode 9.1.9)}, 
 pdflang={English}}
\begin{document}

\maketitle
\tableofcontents


\section{Mitochondria}
\label{sec:orgd30ff33}
\subsection{Background}
\label{sec:org1aedb58}
\begin{enumerate}
\item Maternal Inheritance
\label{sec:orgb10051f}
\begin{itemize}
\item Mitochondria and therefore the mtDNA, usually come from the egg
\begin{itemize}
\item the egg cell contains relatively few mitochondria
\item these mitochondria divide to populate the cells
\end{itemize}
\item Sperm mitochondria enter the egg, but do not contribute genetic
information to the embryo.
\begin{itemize}
\item paternal mitochondria are marked with ubiquitin for destruction
inside the embryo.
\end{itemize}
\item mitochondria are randomly distributed to the daughter cells during
the division of the cytoplasm.
\end{itemize}

\item Heteroplasmy
\label{sec:org26ca81e}

\begin{itemize}
\item Heteroplasmy is the presence of more than one type of organellar
genome within a cell or individual.

\item It is an important factor in considering the severity of
mitochondrial diseases.
\begin{itemize}
\item can also be beneficial
\end{itemize}

\item Microheteroplasmy is present in most individuals.
\begin{itemize}
\item hundreds of independent mutations, with each mutation found in
about 1-2\% of all mitochondrial genomes.
\end{itemize}
\end{itemize}

\begin{figure}[htbp]
\centering
\includegraphics[width=0.6\textwidth]{./mitochondria/figures/heteroplasmy.png}
\caption[heter]{\label{fig:org2cb8b59}
Heteroplasmy}
\end{figure}

\item Muller's Ratchet and the Mitochondrial Bottleneck
\label{sec:org61e661c}
\begin{itemize}
\item Entities undergoing uniparental inheritance and with little to no
recombination can be subject to "Muller's ratchet"
\begin{itemize}
\item the inexorable accumulation of deleterious mutations until
functionality is lost.
\end{itemize}

\item mitochondria avoid this buildup through a developmental process
known as the mtDNA bottleneck. 
\begin{itemize}
\item selection acts to remove those cells with more deleterious mtDNA
\end{itemize}
\end{itemize}

\begin{figure}[htbp]
\centering
\includegraphics[width=0.6\textwidth]{./mitochondria/figures/bottle_neck.jpg}
\caption[bottle]{\label{fig:orgb2f1ad8}
Mitochondrial bottle neck}
\end{figure}

\item Fusion
\label{sec:org6283701}
\begin{itemize}
\item response to cellular stress
\item enables genetic complementation
\begin{itemize}
\item fusion of the mitochondria allows two mitochondrial genomes with
different defects within the same organelle to encode what the
other lacks.
\item corrects for mtDNA damage
\end{itemize}
\end{itemize}

\item Replication and Fission
\label{sec:org0241915}
\begin{itemize}
\item Mitochondria divide by binary fission, similar to bacterial cell division
\item mammalian mitochondria replicate their DNA and divide mainly in response
to the energy needs of the cell, not in phase with the cell cycle.
\begin{itemize}
\item When the energy needs of a cell are high, mitochondria grow and
divide.
\item When the energy use is low, mitochondria are destroyed
or become inactive.
\end{itemize}
\end{itemize}
\item Human Mitochondrial DNA
\label{sec:orgcc57fa8}
\begin{itemize}
\item a circular DNA molecule \textasciitilde{} 16 kb
\item encodes 37 genes
\begin{itemize}
\item 13 for subunits of respiratory complexes I, III, IV and V
\item 22 for mitochondrial tRNA
\begin{itemize}
\item 20 standard amino acids, plus extra genes for leu and ser
\end{itemize}
\item 2 for rRNA.
\end{itemize}
\item One mitochondrion can contain two to ten copies of its DNA.
\end{itemize}

\begin{figure}[htbp]
\centering
\includegraphics[width=.8\textwidth]{./mitochondria/figures/mitochondrial_genome.png}
\caption[mtdna]{\label{fig:org7effa4a}
Human mitochondrial genome}
\end{figure}

\item Alternative genetic code
\label{sec:org2945c6b}
\begin{itemize}
\item The mitochondria of many eukaryotes, including most plants, use the
standard code.
\end{itemize}

\begin{table}[htbp]
\caption[mito code]{\label{tab:orgf06d5aa}
Exceptions to the standard genetic code in mamalian mitochondria}
\centering
\begin{tabular}{lll}
Codon & Standard & Mitochondria\\
\hline
AGA, AGG & Arginine & Stop codon\\
AUA & Isoleucine & Methionine\\
UGA & Stop codon & Tryptophan\\
\end{tabular}
\end{table}

\begin{itemize}
\item AUA, AUC, and AUU codons are all allowable start codons.
\item Some of these differences are pseudo-changes in the genetic code due
to the phenomenon of RNA editing, common in mitochondria.
\end{itemize}

\item Mitochondrial Disease
\label{sec:org6da47ac}
\begin{itemize}
\item About 15\% of mitochondrial disease is caused by mutations in the
mitochondrial DNA that affect mitochondrial function.
\item Other mitochondrial diseases are caused by
\begin{itemize}
\item mutations in nuclear DNA
\item acquired mitochondrial conditions (drugs, toxins)
\end{itemize}
\end{itemize}
\end{enumerate}
\section{Pyruvate}
\label{sec:org4d80c72}
\subsection{Introduction}
\label{sec:org88bee2f}
\begin{enumerate}
\item Disorders of Pyruvate Metabolism
\label{sec:org883f297}

\begin{itemize}
\item Pyruvate Carboxylase Deficiency
\item Phospoenolpyruvate Carboxykinase Deficiency
\item Pyruvate Dehydrogenase Complex Deficiency
\begin{itemize}
\item Dihydrolipoamide Dehydrogenase Deficiency (PDHC E3)
\end{itemize}
\item Pyruvate Transporter Defect
\end{itemize}


\begin{figure}[htbp]
\centering
\includegraphics[width=0.7\textwidth]{./pyruvate/figures/pyruvate_disorders.png}
\caption[TCA]{\label{fig:org16a65db}
Disorders of Pyruvate Metabolism}
\end{figure}

\begin{enumerate}
\item Pyruvate Reactions
\label{sec:org0ee4449}

\begin{description}
\item[{pyruvate dehydrogenase complex}] decarboxylation \(\to\) acetyl-CoA
\end{description}

\ce{pyruvate + CoA + NAD+ <=>[PDHC] acetyl-CoA + CO2 + NADH + H+}

\begin{description}
\item[{pyruvate carboxylase}] carboxylation \(\to\) oxaloacetate
\end{description}

\ce{pyruvate + ATP + CO2 <=>[PC] oxaloacetate + ADP +Pi}

\begin{description}
\item[{alanine transaminase}] transamination \(\to\) alanine
\end{description}

\ce{pyruvate + glutamate <=>[ALT] alanine + \alpha-ketoglutarate}

\begin{description}
\item[{lactate dehydrogenase}] reduction \(\to\) lactate
\end{description}

\ce{pyruvate + NADH <=>[LDH] lactate + NAD+}
\end{enumerate}
\end{enumerate}

\subsection{Pyruvate Carboxylase Deficiency}
\label{sec:orge7efad8}
\begin{itemize}
\item PC is a biotinylated mitochondrial matrix enzyme.
\begin{itemize}
\item carboxylation of pyruvate to oxaloacetate
\end{itemize}

\ce{pyruvate + ATP + CO2 ->[PC] oxaloacetate + ADP +Pi}

\item important role in:
\begin{itemize}
\item gluconeogenesis
\begin{itemize}
\item urea cycle indirectly
\end{itemize}
\item anaplerosis
\begin{itemize}
\item \(\downarrow\) 2-ketoglutarate \(\to\) \(\downarrow\) glutamate
\item \(\downarrow\) TCA intermediates lowers reducing equivalents
\item redox equilibrium between 3-OH-butyrate and acetoacetate \(\to\) acetoacetate
\item \(\downarrow\) 3-OH-butyrate/acetoacetate ratio
\end{itemize}
\item lipogenesis
\begin{itemize}
\item oxaloacetate + acetyl-CoA \(\to\) citrate
\end{itemize}
\end{itemize}
\item 
\end{itemize}
\begin{enumerate}
\item Biotin and Bicarbonate
\label{sec:orgecdc722}
\begin{itemize}
\item PC requires biotin and bicarbonate
\item Metabolic derangements associated with PC are observed in:
\begin{itemize}
\item Biotin deficiency and biotinidase deficiency
\item The three biotin-dependent carboxylases:
\begin{itemize}
\item Propionyl-CoA carboxylase (PCC)
\item 3-methylcrotonyl-CoA carboxylase (3MCC)
\item Pyruvate carboxylase (PC)
\end{itemize}

\item CA-VA deficiency
\begin{itemize}
\item results in dysfunction of all four enzymes to which CA-VA
provides bicarbonate as substrate in mitochondria
\end{itemize}
\end{itemize}
\end{itemize}

\item Clinical Presentation
\label{sec:orgd8b459f}
\begin{itemize}
\item French phenotype (type B), most severe
\begin{itemize}
\item acute illness 3-48h after birth
\item hypothermia, hypotonia, lethargy, vomiting
\item severe neurological dysfunction
\item death prior to 5 months
\end{itemize}
\item North American phenotype (type A)
\begin{itemize}
\item severe illness between 2 and 5 months of age
\item progressive hypotonia
\item acute vomiting, dehydration, tachypnoea, metabolic acidosis
\item severe intellectual disability
\item progressive with death in infancy
\end{itemize}
\item Benign phenotype (type c)
\begin{itemize}
\item rare
\item acute episodes of lactic acidosis and ketoacidosis
\item near normal cognitive and motor development
\end{itemize}
\end{itemize}
\item Genetics
\label{sec:org7ded3bb}
\begin{itemize}
\item Autosomal recessive with incidence of 1 in 250000
\item PC is a homo-tetramer
\item PC protein and mRNA absent in 50\% of French phenotype
\item American and Benign phenotypes have cross-reacting material
\item Mosaicism has been observed with prolonged survival
\end{itemize}

\item Diagnostic Tests
\label{sec:orgd7de88c}
\begin{itemize}
\item PC deficiency should be considered in any child presenting with
lactic acidosis and neurological abnormalities
\begin{itemize}
\item with hypoglycemia, hyperammonemia, or ketosis
\end{itemize}

\item \(\uparrow\) L/P with \(\downarrow\) BHB/acetoacetate in severely affected patients
\begin{itemize}
\item pathognomonic in neonates
\end{itemize}

\item post-prandial ketosis, hypercitrullinemia, hyperammonemia, low glutamine

\item CSF lactate, alanine and L/P are elevated, glutamine decreased

\item PC activity in cultured skin fibroblasts
\begin{itemize}
\item can not distinguish severity
\end{itemize}
\end{itemize}

\item Treatment
\label{sec:orgf354d10}

\begin{itemize}
\item Currently, no treatment.
\end{itemize}
\end{enumerate}

\subsection{Phospoenolpyruvate Carboxykinase Deficiency}
\label{sec:org41c1b72}
\begin{itemize}
\item PEPCK has cytosolic and mitochondria isoforms
\item Cytosolic PEPCK deficiency is secondary to hyperinsulinism
\begin{itemize}
\item insulin represses expression of the cytosolic form
\end{itemize}
\item Mitochondrial PEPCK deficiency has not been clearly demonstrated
\end{itemize}

\subsection{Pyruvate Dehydrogenase Complex Deficiency}
\label{sec:org800d1b7}
\begin{enumerate}
\item Pyruvate Dehydrogenase Complex
\label{sec:org9872bb7}
\begin{itemize}
\item PDHC decarboxylates pyruvate \(\to\) acetyl-CoA
\begin{itemize}
\item thiamine dependent
\end{itemize}
\item PDHC, KDHC and BCKD have similar structure and mechanism
\item Composed of:
\begin{itemize}
\item E1 \(\alpha\)-ketoacid dehydrogenase
\item E2 dihydrolipoamide acyltransferase
\item E3 dihydrolipoamide dehydrogenases
\end{itemize}
\item E1 is specific to each complex
\begin{itemize}
\item Composed of E1\(\alpha\) and E1\(\beta\)
\end{itemize}
\item E1 is the rate limiting step in PDHC
\begin{itemize}
\item regulated by phosphorylation
\end{itemize}
\end{itemize}

\begin{table}[htbp]
\caption{\label{tab:org282a28a}
Pyruvate Dehydrogenases Complex}
\centering
\begin{tabular}{llll}
Unit & Name & Gene & Cofactor\\
\hline
E1\(\alpha\) & pyruvate dehydrogenase & PDHA1 & thiamine pyrophosphate (TPP)\\
E1\(\beta\) &  & PDHB & \\
E2 & dihydrolipoyl  S-acetyltransferase & DLAT & lipoate, coenzyme A\\
E3 & dihydrolipoyl dehydrogenase & DLD & FAD, NAD+\\
\end{tabular}
\end{table}



\begin{figure}[htbp]
\centering
\includegraphics[width=0.6\textwidth]{./pyruvate/figures/pdhe1_phos.png}
\caption[pdhe1]{\label{fig:orgd7c909b}
Activation/deactivation of PDHE1}
\end{figure}


\begin{figure}[htbp]
\centering
\includegraphics[width=0.7\textwidth]{./pyruvate/figures/pdhc.png}
\caption[pdhc]{\label{fig:orgcdccd3f}
Pyruvate Dehydrogenase Complex}
\end{figure}

\item Pyruvate Dehydrogenase Complex Deficiency
\label{sec:orga97e3b4}
\begin{itemize}
\item PDHC deficiency provokes conversion of pyruvate to lactate and alanine rather than acetly-CoA
\item Metabolism of glucose \(\to\) lactate, produces 1/10 ATP compared to
complete oxidation via TCA and ETC
\item Impairs production of NADH but not oxidation
\item NADH/\ce{NAD+} is normal, \(\therefore\) normal L/P
\begin{itemize}
\item ETC deficiencies \(\to\) \(\uparrow\) L/P
\end{itemize}
\end{itemize}

\item Clinical Presentation: PDHE1\(\alpha\)
\label{sec:org4d897f7}
\begin{itemize}
\item Majority of cases involve the X encoded \(\alpha\)-subunit of the dehydrogenase (E1)
\begin{itemize}
\item PDHE1\(\alpha\) deficiency
\item developmental delay, hypotonia, seizures and ataxia
\end{itemize}

\item Common presentations in hemizygous males:
\begin{enumerate}
\item neonatal lactic acidosis
\begin{itemize}
\item most severe
\end{itemize}
\item Leigh's encephalopathy
\begin{itemize}
\item most common
\item present in first 5 years
\end{itemize}
\item intermittent ataxia
\begin{itemize}
\item rare
\item ataxia after carbohydrate rich meals \(\to\) Leigh's syndrome
\end{itemize}
\end{enumerate}

\item Females with PDHE1\(\alpha\), uniform presentation, variable severity
\begin{itemize}
\item dismorphic features
\item moderate to severe intellectual disability
\item seizures common
\item severe neonatal lactic acidosis can be present
\end{itemize}
\end{itemize}

\item Clinical Presentation: PDHE1\(\beta\)
\label{sec:org14de35d}
\begin{itemize}
\item very rare
\item similar to PDHE1\(\alpha\)
\end{itemize}

\item Genetics
\label{sec:orgbbc3d77}
\begin{itemize}
\item All components of PDHC are encoded by nuclear genes
\item Autosomal except E1\(\alpha\) on Xp22.11
\begin{itemize}
\item \(\therefore\) most PDHC deficiency is X-linked
\end{itemize}
\item No null E1\(\alpha\) identified except in a mosaic state
\begin{itemize}
\item suggests E1\(\alpha\) is essential
\end{itemize}
\end{itemize}

\item Diagnostic Tests
\label{sec:org52545f4}
\begin{itemize}
\item Lactate and pyruvate in blood and CSF
\item CSF lactate is generally \(\uparrow\) compared to blood
\item Urine organic acids
\begin{itemize}
\item lactic and pyruvic acid
\end{itemize}
\item plasma amino acids
\begin{itemize}
\item alanine
\end{itemize}
\item L/P ratio is usually normal
\item Skin fibroblasts for PDHC
\begin{itemize}
\item also lymphocytes, separated from EDTA <2days
\end{itemize}
\item PDHE1\(\alpha\) genotype in females is useful
\end{itemize}

\item Treatment
\label{sec:orgb757401}
\begin{itemize}
\item Early adoption of ketogenic diet may have a benefit
\item Thiamine responsive forms
\item DCA is a pyruvate analog, inhibits E1 kinase, keeps E1 dephosphorylated (active) (Figure \ref{fig:orgd7c909b})
\end{itemize}

\item Pyruvate Transport Defect
\label{sec:orgdfc87ba}
\begin{itemize}
\item MPC1 mutations have been described in 5 patients
\item mediates the proton symport of pyruvate across the IMM.
\item \(\therefore\) metabolic derangement similar to PDHC deficiency
\item No treatment
\end{itemize}
\end{enumerate}

\subsection{Dihydrolipoamide Dehydrogenase Deficiency}
\label{sec:org67c013a}
\begin{enumerate}
\item Dihydrolipoamide Dehydrogenase
\label{sec:org8f2b2dc}
\begin{itemize}
\item DLD (E3) is a flavoprotein common to all three mitochondrial
\(\alpha\)-ketoacid dehydrogenase complexes
\begin{itemize}
\item PDHC, KDHC, and BCKD
\end{itemize}
\item Combined PDHC, TCA , BCAA defect
\begin{itemize}
\item \(\uparrow\) lactate , pyruvate,
\item alanine, glutamate, glutamine, BCAA
\item urinary lactic, pyruvic, 2-ketoglutaric, BC 2-hydroxy \& 2-ketoacids
\end{itemize}
\end{itemize}

\item Genetics and Diagnotic Testing
\label{sec:orgeeb9a1f}
\begin{itemize}
\item DLD mutations AR
\item 13 unrelated patients revealed 14 unique mutations
\item Blood lactate, pyruvate
\item plasma amino acids
\item urinary organic acids
\item Pattern of abnormalities not seen in all patients at all times.
\end{itemize}
\end{enumerate}
\section{Tricarboxylic Acid Cycle}
\label{sec:org0e9ceeb}
\subsection{Introduction}
\label{sec:orga36dc0a}
\begin{itemize}
\item Pathways for oxidation of fatty acids, glucose, amino acids and ketones produce acetyl-CoA
\end{itemize}
%%\setchemfig{lewis style=red}
\centering
\chemname{\chemfig{\lewis{0.,H}-\lewis{0.2.4.6.,{\color{red}C}}(-[6]\lewis{2.,H})(-[2]\lewis{6.,H})-\lewis{4.,{\color{red}C}}(=[2]O)-[,,,,decorate, decoration=snake]SCoA}}{acetyl-CoA}
\begin{itemize}
\item Part of aerobic respiration
\begin{itemize}
\item ETC regenerates \ce{NAD+} from NADH
\end{itemize}
\item Cofactors:
\begin{itemize}
\item niacin (\ce{NAD+})
\item riboflavin (FAD and FMN)
\item panthothenic acid (CoA)
\item thiamine
\item \ce{Mg^2+}, \ce{Ca^2+}, \ce{Fe+} and phosphate
\end{itemize}

\item TCA net reaction
\end{itemize}

{\tiny\ce{Acetyl-CoA + 3NAD+ + FAD + GDP + Pi + 2H2O -> 2CO2 + CoA + 3NADH + FADH2 + GTP + 2H+}}

\begin{itemize}
\item release of energy via oxidation of acetly-CoA
\item one molecule of glucose breaks down into two molecules of pyruvate
\item Pyruvate is converted into acetyl-CoA, which is the main input TCA
\end{itemize}

\begin{enumerate}
\item Disorders of the TCA cycle
\label{sec:org3cf561d}

\begin{itemize}
\item \(\alpha\)-Ketoglutarate Dehydrogenase Complex Deficiency
\item Succinate Dehydrogenase Deficiency
\item Fumarase Deficiency
\end{itemize}

\begin{figure}[htbp]
\centering
\includegraphics[width=0.8\textwidth]{./tca/figures/TCA_disorders.png}
\caption{\label{fig:orgcb78731}
TCA disorders}
\end{figure}


\begin{figure}[htbp]
\centering
\includegraphics[width=0.5\textwidth]{./tca/figures/gr2.png}
\caption{\label{fig:orgc5a9bd1}
Model for a functional splitting of the Krebs cycle reactions into complementary mini-cycles.}
\end{figure}

\begin{itemize}
\item To compensate for defects the TCA can split into complementary
mini-cycles (Figure \ref{fig:orgc5a9bd1})
\begin{itemize}
\item uses aspartate-amino acid transferase
\item (A) would allow to convert pyruvate up to \(\alpha\)-KG, even when the second mini-cycle (B) does not function.
\item accounts for the urinary excretion of \(\alpha\)-KG in patients
presenting with defect of \(\alpha\)-KG, SDH or fumarase activity.
\item could produce reduced equivalents to sustain the normal oxygen
uptake measured in circulating lymphocytes or cultured skin
fibroblast from these patients.
\end{itemize}
\end{itemize}
\end{enumerate}

\subsection{\(\alpha\)-Ketoglutarate Dehydrogenase Complex Deficiency}
\label{sec:orgb25f433}
\begin{itemize}
\item KDHC is a \(\alpha\)-ketoacid dehydrogenase analogous to PDHC and BCKD.
\end{itemize}

\ce{\alpha-ketoglutarate + NAD+ + CoA ->[KDHC] Succinyl CoA + CO2 + NADH}

\begin{table}[htbp]
\caption{\label{tab:orgee6d025}
\(\alpha\)-Ketoglutarate Dehydrogenase Complex}
\centering
\begin{tabular}{llll}
Unit & Name & Gene & Cofactor\\
\hline
E1 & \(\alpha\)-ketoglutarate dehydrogenase & OGDH & thiamine pyrophosphate(TPP)\\
E2 & dihydrolipoyl succinyltransferase & DLST & lipoic acid, Coenzyme A\\
E3 & dihydrolipoyl dehydrogenase & DLD & FAD, NAD\\
\end{tabular}
\end{table}

\begin{itemize}
\item E1 subunit is the thiamine dependant substrate specific dehydrogenase
\begin{itemize}
\item Not regulated by phosphorylation.
\end{itemize}
\item E2 subunit dihydrolipyoyl succinyl-transferase is also specific to KDHC
\end{itemize}

\begin{enumerate}
\item Clinical Presentation
\label{sec:org08f473e}
\begin{itemize}
\item Similar to PDHC (section \ref{sec:org800d1b7})
\item Developmental delay, hypotonia, opisthotonos and ataxia
\begin{itemize}
\item seizures less common
\end{itemize}
\item Present as neonate and early childhood
\end{itemize}

\item Genetics
\label{sec:orgbe3e8ec}
\begin{itemize}
\item AR inheritance, encoded by nuclear DNA
\item E1 gene mapped to 7p13
\item E2 gene mapped to 14q24.3
\item Molecular basis of KDHC deficiencies is not resolved.
\item \(\alpha\)-ketoglutarate dehydrogenase deficiency is sometimes a feature of DLD deficiency
\end{itemize}

\item Diagnostic Tests
\label{sec:org73f2cb1}
\begin{itemize}
\item Urine organic acids
\begin{itemize}
\item \(\uparrow\) \(\alpha\)-KGA, \textpm{} other TCA intermediates
\item \(\alpha\)-KGA is a common finding, not specific for KDHC deficiency
\end{itemize}
\item Blood lactate
\begin{itemize}
\item Normal or increased L/P
\end{itemize}
\item KDHC activity
\begin{itemize}
\item \ce{^14CO2} release from \ce{[1-^14C]} \(\alpha\)-ketoglutarate (or \ce{[1-^14C]} leucine)
\item cultured skin fibroblasts
\item muscle
\end{itemize}
\end{itemize}

\item Treatment
\label{sec:org225c893}
\begin{itemize}
\item None to date
\end{itemize}
\end{enumerate}

\subsection{Succinate Dehydrogenase Deficiency}
\label{sec:org1369c94}
\begin{itemize}
\item Four subunits compose Complex II of the mitochondrial respiratory chain
\end{itemize}

\begin{table}[htbp]
\caption{\label{tab:org58665d9}
Succinate Dehydrogenase | Complex II}
\centering
\begin{tabular}{ll}
Subunit name & Protein description\\
\hline
SdhA & Succinate dehydrogenase flavoprotein subunit\\
SdhB & Succinate dehydrogenase iron-sulfur subunit\\
SdhC & Succinate dehydrogenase cytochrome b560 subunit\\
SdhD & Succinate dehydrogenase cytochrome b small subunit\\
\end{tabular}
\end{table}

\begin{itemize}
\item The SdhA subunit contains an FAD binding site where succinate
is deprotonated and converted to fumarate.
\end{itemize}

\ce{succinate + ubiquinone ->[CII] fumarate + ubiquinol}

\begin{itemize}
\item Electrons removed from succinate transfer to SdhA
\item transfer across SdhB through iron sulphur clusters to the SdhC/SdhD subunits
\begin{itemize}
\item SdhC/SdhD are anchored in the mitochondrial membrane.
\end{itemize}
\end{itemize}

\begin{figure}[htbp]
\centering
\includegraphics[width=0.5\textwidth]{./tca/figures/SuccDeh.png}
\caption{\label{fig:org59888b1}
Succinate Dehydrogenase | Complex II}
\end{figure}

\begin{enumerate}
\item Clinical Presentation
\label{sec:orgc08eee1}
\begin{itemize}
\item Very rare disorder with highly variable phenotype
\item Complex II is part of the TCA cycle and ETC
\begin{itemize}
\item phenotype resembles defects in respiratory chain
\end{itemize}
\item Clinical picture can include:
\begin{itemize}
\item Kearns-Sayre syndrome
\item isolated hypertrophic cardiomyopathy
\item combined cardiac and skeletal myopathy
\item generalized muscle weakness, \(\uparrow\) fatiguability
\item early onset Leigh encephalopathy
\end{itemize}
\item Also:
\begin{itemize}
\item cerebral ataxia
\item optic atropy
\item tumour formation in adults
\end{itemize}
\end{itemize}

\item Genetics
\label{sec:orged55bf5}

\begin{itemize}
\item All components of Complex II are encoded by nuclear DNA.
\end{itemize}

\begin{table}[htbp]
\caption{\label{tab:org8848efb}
Succinate Dehydrogenase Genetics}
\centering
\begin{tabular}{ll}
Gene & Location\\
\hline
SDHA & 5p15.33\\
SDHB & 1p36.13\\
SDHC & 1q23.3\\
SDHD & 11q23.1\\
\end{tabular}
\end{table}


\begin{itemize}
\item AR with highly variable phenotype
\item Case of affected sisters with one identified SDHA mutation suggested
dominant transmission
\item Mutations in SDHB, SDHC and SDHD cause susceptibility to familial
phaeochromocytoma and familial paraganglioma.
\end{itemize}

\item Diagnostic Tests
\label{sec:orgc1806dd}
\begin{itemize}
\item Unlike other TCA cycle disorders Complex II deficiency does not always
result in characteristic organic aciduria
\begin{itemize}
\item succinic aciduria.
\end{itemize}
\item Organic acids can show variable amounts of lactate, pyruvate, succinate, fumarate and malate
\item Measurement of complex II activity in muscle is the most reliable
means of diagnosis
\begin{itemize}
\item there is no clear correlation between residual complex II activity
and severity or clinical outcome.
\end{itemize}
\end{itemize}

\begin{figure}[htbp]
\centering
\includegraphics[width=0.5\textwidth]{./tca/figures/gr4.jpg}
\caption{\label{fig:org247be4b}
Coupled spectrophotometric assay}
\end{figure}

\item Treatment
\label{sec:org14699c2}
\begin{itemize}
\item In some cases, treatment with riboflavin may have clinical benefit
\end{itemize}
\end{enumerate}

\subsection{Fumarase Deficiency}
\label{sec:org2c44fec}

\begin{itemize}
\item Fumarase (AKA:fumarate hydratase) catalyses reversible
hydration/dehydration of fumarate to malate
\end{itemize}
\ce{fumarate + H2O ->[FH] malate}
\begin{itemize}
\item Two forms: mitochondrial and cytosolic.
\begin{itemize}
\item The mitochondrial isoenzyme is involved in the TCA Cycle
\item The cytosolic isoenzyme is involved in the metabolism of amino acids and fumarate.
\end{itemize}
\item Subcellular localization is established by the presence/absence of an N-terminal mitochondrial signal
sequence
\item Deficiency causes impaired energy production
\end{itemize}

\begin{enumerate}
\item Clinical Presentation
\label{sec:org91e9cc6}
\begin{itemize}
\item Characterized by polyhydramnios and fetal brain abnormalities.
\item In the newborn period, findings include:
\begin{itemize}
\item severe neurologic abnormalities,
\item poor feeding,
\item failure to thrive
\item hypotonia.
\end{itemize}
\item Fumarase deficiency is suspected in infants with multiple severe
neurologic abnormalities in the absence of an acute metabolic
crisis.
\item Inactivity of both cytosolic and mitochondrial forms of
fumarase are potential causes.
\end{itemize}
\item Genetics
\label{sec:org1feb22e}
\begin{itemize}
\item AR inheritance, encoded by nuclear DNA
\item Single gene and mRNA encode mito and cyto isoforms
\end{itemize}
\item Diagnostic Tests
\label{sec:orgdfc8b8e}
\begin{itemize}
\item Isolated, increased concentration of fumaric acid on urine organic
acid analysis is highly suggestive of fumarase deficiency.
\begin{itemize}
\item Succinate, \(\alpha\)-KGA can also be elevated
\end{itemize}
\item Molecular genetic testing for fumarase deficiency
\end{itemize}
\end{enumerate}
\subsection{Isocitrate Dehydrogenase}
\label{sec:org406da60}
\begin{itemize}
\item IDH exists in three isoforms:
\begin{itemize}
\item IDH3 catalyzes the third step of the citric acid cycle while converting \ce{NAD+} to NADH in the mitochondria.
\end{itemize}
\end{itemize}

\ce{isocitrate + NAD+ ->[IHD3] \alpha-ketoglutarate + CO2 + NADH + H+}

\begin{itemize}
\item IDH1 and IDH2 catalyze the same reaction outside TCA cycle and use \ce{NADP+} as a cofactor.
\begin{itemize}
\item They localize to the cytosol as well as the mitochondrion and peroxisome.
\end{itemize}
\end{itemize}

\ce{isocitrate + NADP+ ->[IHD1/2] \alpha-ketoglutarate + CO2 + NADPH + H+}

\begin{enumerate}
\item Clinical relevance
\label{sec:orgefea1f6}
\begin{itemize}
\item IDH3 deficiency is associated with retinitis pigmentosa
\item IDH1/2 mutations linked to malignant gliomas and acute myeloid leukemia
\item Mutations in IDH2 identified in half of patients
\end{itemize}
\end{enumerate}
\section{Electron Transport Chain}
\label{sec:org0f90628}
\subsection{Introduction}
\label{sec:orgf1a57cf}
\begin{enumerate}
\item ETC and OxPhos system
\label{sec:org194c55e}
\begin{itemize}
\item Responsible for ATP production
\item ETC complexes I-IV
\item OxPhos system complexes I-V
\end{itemize}
\begin{enumerate}
\item Chemiosmotic Coupling Hypothesis
\label{sec:orgcae59c4}
\begin{itemize}
\item the ETC and OxPhos are coupled by a proton gradient across the IMM.
\item The efflux of protons from the mitochondrial matrix creates an electrochemical gradient.
\begin{itemize}
\item used by the F\(_{\text{0}}\)F\(_{\text{1}}\) ATP synthase complex to make ATP via oxidative phosphorylation.
\end{itemize}
\end{itemize}
\end{enumerate}
\item Electron Transport Chain
\label{sec:org4fe928e}
\begin{itemize}
\item Energy from transfer of electrons down the ETC is used to pump
protons from the mitochondrial matrix into the intermembrane space.
\begin{itemize}
\item creates an electrochemical proton gradient (\(\Delta\)pH) across the IMM.
\begin{itemize}
\item largely responsible for the mitochondrial membrane potential (\(\Delta \Psi\)M).
\end{itemize}
\item ATP synthase uses flow of \ce{H+} through the enzyme back into the
matrix to generate ATP from ADP and Pi.
\end{itemize}
\item There are three energy-transducing enzymes in the electron transport
chain:
\begin{itemize}
\item NADH:ubiquinone oxidoreductase (complex I)
\item Coenzyme Q – cytochrome c reductase (complex III)
\item cytochrome c oxidase (complex IV).
\item also ETF-QO and mitochondrial GPD
\end{itemize}
\end{itemize}

\begin{figure}[htbp]
\centering
\includegraphics[width=\textwidth]{./etc/figures/etc.pdf}
\caption{\label{fig:org763bd3d}
Electron Transport Chain}
\end{figure}

\begin{itemize}
\item Complex I (NADH coenzyme Q reductase) accepts electrons from the
Krebs cycle electron carrier NADH
\item passes them to CoQ (ubiquinone)
\item CoQ also receives electrons from complex II (succinate dehydrogenase)
\item CoQ passes electrons to complex III (cytochrome bc1 complex), which
passes them to cytochrome c.
\item Cyt c passes electrons to Complex IV (cytochrome c oxidase), which
uses the electrons and hydrogen ions to reduce molecular oxygen to
water.
\end{itemize}

\begin{figure}[htbp]
\centering
\includegraphics[width=\textwidth]{./etc/figures/hsa00190.png}
\caption[ETC]{\label{fig:orge4fbeb1}
Oxidative Phosphorylation}
\end{figure}

\begin{figure}[htbp]
\centering
\includegraphics[width=0.6\textwidth]{./etc/figures/potential.png}
\caption[redox]{\label{fig:org476cfd1}
Electron flow to O\(_{\text{2}}\) and release free energy}
\end{figure}
\end{enumerate}

\subsection{Complex I}
\label{sec:org44311ab}
\begin{enumerate}
\item Complex I | NADH-ubiquinone oxidoreductase
\label{sec:orgc504301}
\begin{itemize}
\item catalyzes the transfer of electrons from NADH to coenzyme Q10
(CoQ) and translocates protons across the inner mitochondrial
membrane
\end{itemize}

{\small\ce{NADH + H+ + CoQ + 4H^{+}_{in} ->[CI] NAD+ + CoQH2 + 4H^{+}_{out}}}

\begin{itemize}
\item Mechanism: 
\begin{enumerate}
\item Seven iron sulfur centers carry electrons from the site of NADH
dehydration to ubiquinone.

\item ubiquinone (CoQ) is reduced to ubiquinol (\ce{CoQH2}).

\item The energy from the redox reaction results in conformational
change allowing hydrogen ions to pass through four transmembrane
helix channels.
\end{enumerate}
\end{itemize}

\begin{figure}[htbp]
\centering
\includegraphics[width=0.9\textwidth]{./etc/figures/c1.pdf}
\caption[c1]{\label{fig:orgaf7bbc4}
Complex I | NADH-ubiquinone oxidoreductase}
\end{figure}

\item Complex I Inhibitors
\label{sec:orgd7e76d7}
\begin{itemize}
\item The best-known inhibitor of complex I is rotenone
\begin{itemize}
\item commonly used as an organic pesticide
\end{itemize}
\item Rotenone binds to the ubiquinone binding site of complex I
\begin{itemize}
\item piericidin A a potent inhibitor and structural homologue to ubiquinone.
\end{itemize}
\item Hydrophobic inhibitors like rotenone or piericidin likely disrupt electron transfer between FeS cluster N2 and ubiquinone.
\item Bullatacin is the most potent known inhibitor of NADH dehydrogenase (ubiquinone)
\item Complex I is also blocked by adenosine diphosphate ribose – a reversible competitive inhibitor of NADH oxidation
\end{itemize}
\end{enumerate}

\subsection{Complex II}
\label{sec:org169ed23}
\begin{enumerate}
\item Complex II | Succinate Dehydrogenase
\label{sec:orgdbb01ab}
\begin{itemize}
\item see section \ref{sec:org1369c94}
\end{itemize}

\item Complex II Inhibitors
\label{sec:org7361203}
\begin{itemize}
\item There are two distinct classes of inhibitors of complex II:
\begin{itemize}
\item those that bind in the succinate pocket and those that bind in the ubiquinone pocket.
\end{itemize}
\item Ubiquinone type inhibitors include carboxin and thenoyltrifluoroacetone.
\item Succinate-analogue inhibitors include the synthetic compound malonate as well as the TCA cycle intermediates, malate and oxaloacetate.
\begin{itemize}
\item oxaloacetate is one of the most potent inhibitors of Complex II.
\end{itemize}
\end{itemize}
\end{enumerate}

\subsection{Glycerol-3-phosphate shuttle}
\label{sec:org4bd3a26}
\begin{itemize}
\item Oxidation of cytoplasmic NADH by the cytosolic form of the enzyme
creates glycerol-3-phosphate from dihydroxyacetone phosphate.
\item Glycerol-3-phosphate diffuses into IMM and is oxidised by mitochondrial glycerol-3-phosphate dehydrogenase
\begin{itemize}
\item uses quinone as an oxidant and FAD as a co-factor.
\end{itemize}
\item maintains the cytoplasmic NAD+/NADH ratio.
\end{itemize}

\begin{figure}[htbp]
\centering
\includegraphics[width=0.6\textwidth]{./etc/figures/GPDH_shuttle.png}
\caption[g3ps]{\label{fig:orgf93b361}
Glycerol-3-phosphate shuttle}
\end{figure}

\subsection{Electron Transferring Flavoprotein/ Dehydrogenase}
\label{sec:org53172a3}

\begin{itemize}
\item ETFs are heterodimeric proteins composed of an alpha and beta
subunit (ETFA and ETFB), and contain an FAD cofactor and AMP

\item ETQ-QO links the oxidation of fatty acids and some amino acids to
oxidative phosphorylation in the mitochondria.
\item catalyzes the transfer of electrons from electron transferring
flavoprotein (ETF) to ubiquinone, reducing it to ubiquinol.
\end{itemize}

{\small\ce{Acyl-CoA + FAD ->[ACAD] FADH2 + ETF ->[ETF-QO] CoQ ->[CIII] CytC}}

\begin{itemize}
\item ETF-QO deficiency results in \textbf{Glutaric Acidemia Type II} (AKA MADD)
\begin{itemize}
\item Discussed more in FAODs
\end{itemize}
\end{itemize}
\subsection{Complex III}
\label{sec:org8256a62}
\begin{enumerate}
\item Complex III | Coenzyme Q – cytochrome c reductase
\label{sec:org50b5323}
\begin{itemize}
\item Complex III is a multi-subunit transmembrane protein encoded by both
mitochondrial (cytochrome B) and the nuclear genomes (all other
subunits)

\item The bc1 complex contains 11 subunits:
\begin{itemize}
\item 3 respiratory subunits (cytochrome B, cytochrome C1, Rieske protein)
\item 2 core proteins
\item 6 low-molecular weight proteins
\end{itemize}
\end{itemize}

{\small\ce{QH2 + 2Fe^{3+}-cyt c + 2H+_{in} ->[CIII]  Q + 2Fe^{2+}-cyt c + 4H+_{out}}}

\begin{enumerate}
\item Mechanism
\label{sec:orgc0e3e6f}
\begin{itemize}
\item Round 1:
\begin{itemize}
\item Cytochrome b binds a ubiquinol and a ubiquinone.
\item The 2Fe/2S center and BL heme each pull an electron off the bound ubiquinol, releasing two hydrogens into the intermembrane space.
\item One electron is transferred to cytochrome c1 from the 2Fe/2S centre, whilst another is transferred from the BL heme to the BH Heme.
\item Cytochrome c1 transfers its electron to cytochrome c (not to be confused with cytochrome c1), and the BH Heme transfers its electron to a nearby ubiquinone, resulting in the formation of a ubisemiquinone.
\item Cytochrome c diffuses. The first ubiquinol (now oxidised to ubiquinone) is released, whilst the semiquinone remains bound.
\end{itemize}

\item Round 2:
\begin{itemize}
\item A second ubiquinol is bound by cytochrome b.
\item The 2Fe/2S center and BL heme each pull an electron off the bound ubiquinol, releasing two hydrogens into the intermembrane space.
\item One electron is transferred to cytochrome c1 from the 2Fe/2S centre, whilst another is transferred from the BL heme to the BH Heme.
\item Cytocrome c1 then transfers its electron to cytochrome c, whilst the nearby semiquinone produced from round 1 picks up a second electron from the BH heme, along with two protons from the matrix.
\item The second ubiquinol (now oxidised to ubiquinone), along with the newly formed ubiquinol are released.
\end{itemize}
\end{itemize}

\begin{figure}[htbp]
\centering
\includegraphics[width=0.7\textwidth]{./etc/figures/CIII.png}
\caption[cIII]{\label{fig:org8ec1ece}
Complex III | Coenzyme Q – cytochrome c reductase: two step reaction}
\end{figure}
\end{enumerate}

\item Complex III Inhibitors
\label{sec:org3389ae6}
\begin{itemize}
\item There are three distinct groups of Complex III inhibitors:
\begin{itemize}
\item Antimycin A binds to the Q\(_{\text{i}}\) site and inhibits the transfer of electrons in Complex III from heme b\(_{\text{H}}\) to oxidized Q (Q\(_{\text{i}}\) site inhibitor).
\item Myxothiazol and stigmatellin bind to distinct but overlapping pockets within the Q\(_{\text{o}}\) site.
\begin{itemize}
\item Myxothiazol binds nearer to cytochrome bL (hence termed a "proximal" inhibitor).
\item Stigmatellin binds farther from heme bL and nearer the Rieske Iron sulfur protein.
\item Both inhibit the transfer of electrons from reduced QH\(_{\text{2}}\) to the Rieske Iron sulfur protein.
\end{itemize}
\end{itemize}
\end{itemize}
\end{enumerate}
\subsection{Complex IV}
\label{sec:org3024036}
\begin{enumerate}
\item Complex IV | Cytochrome c oxidase
\label{sec:org03a8222}

\begin{itemize}
\item last enzyme in the respiratory electron transport chain.
\item large IMM integral membrane protein composed of several metal prosthetic sites and 14 protein subunits.
\item eleven subunits are nuclear in origin, and three are synthesized in the mitochondria. 
\begin{itemize}
\item contains two hemes,
\item cytochrome a and cytochrome a3,
\item two copper centers, CuA and CuB
\end{itemize}
\item the cytochrome a3 and CuB form a binuclear center that is the site of oxygen reduction.
\item receives an electron from four cytochrome c molecules and transfers them to one O\(_{\text{2}}\) molecule
\end{itemize}

{\small\ce{4Fe^{2+}-cyt c + 8H^{+}_{in} + O2 ->[CIV] 4Fe^{3+}-cyt c + 2H2O + 4H^{+}_{out}}}

\begin{itemize}
\item In the process binds four protons from the inner aqueous phase to
make two water molecules, and translocates another four protons
across the membrane, increasing the transmembrane difference of
proton electrochemical potential which the ATP synthase then uses to
synthesize ATP.
\end{itemize}

\begin{figure}[htbp]
\centering
\includegraphics[width=0.5\textwidth]{./etc/figures/CIV.png}
\caption[cIV]{\label{fig:orgaf04812}
Complex IV | Cytochrome c oxidase}
\end{figure}

\item Complex IV | Inhibitors
\label{sec:orgc8172ef}
\begin{itemize}
\item Cyanide, azide, and carbon monoxide all bind to cytochrome c
oxidase
\item Nitric oxide and hydrogen sulfide, can also inhibit COX by
binding to regulatory sites on the enzyme
\end{itemize}
\end{enumerate}
\subsection{Complex V}
\label{sec:org83d6b29}
\begin{enumerate}
\item Complex V | ATP synthase
\label{sec:org2c2fe50}

\begin{itemize}
\item ATP synthase is a molecular machine that creates the energy storage
molecule adenosine triphosphate (ATP).

\item The overall reaction catalyzed by ATP synthase is:
\end{itemize}

{\small\ce{ADP + P_i + H+_{out} <=>[CV] ATP + H2O + H+_{in}}}

\begin{itemize}
\item Formation of ATP from ADP and P\(_{\text{i}}\) is energetically unfavourable
\begin{itemize}
\item would normally proceed in the reverse direction.
\end{itemize}

\item To drive this reaction forward, ATP synthase couples ATP synthesis
to the electrochemical gradient (\(\Delta \Psi\)M) created by complexes
I,III and IV

\item ATP synthase consists of two main subunits, F\(_{\text{0}}\) and F\(_{\text{1}}\), which has a
rotational motor mechanism allowing for ATP production.
\end{itemize}

\begin{figure}[htbp]
\centering
\includegraphics[width=0.5\textwidth]{./etc/figures/atp_synthase.jpg}
\caption{\label{fig:org2beeb0e}
Complex V | ATP synthase}
\end{figure}

\item Complex V Inhibitors
\label{sec:orgcd62271}
\begin{itemize}
\item Oligomycin A inhibits ATP synthase by blocking its proton channel
(F\(_{\text{0}}\) subunit), which is necessary for oxidative phosphorylation of
ADP to ATP (energy production).
\item The inhibition of ATP synthesis by oligomycin A will significantly
reduce electron flow through the electron transport chain; however,
electron flow is not stopped completely due to a process known as
proton leak or mitochondrial uncoupling.
\begin{itemize}
\item This process is due to facilitated diffusion of protons into the
mitochondrial matrix through an uncoupling protein such as
thermogenin, or UCP1.
\end{itemize}

\item Administering oligomycin to an individual can result in very high
levels of lactate accumulating in the blood and urine.
\end{itemize}
\end{enumerate}
\section{OxPhos Disorders}
\label{sec:orgc17fcc7}
\subsection{Introduction}
\label{sec:org0bb04ef}
\begin{itemize}
\item Oxidative phosphorylation is the process in which ATP is formed as a
result of the transfer of electrons from NADH or FADH\(_{\text{2}}\) to O\(_{\text{2}}\) by a
series of electron carriers (Figure \ref{fig:orge4fbeb1}).
\item This process, which takes place in mitochondria, is the major source
of ATP in aerobic organisms
\end{itemize}
\subsection{Metabolic Derangement}
\label{sec:orga2ae407}
\begin{enumerate}
\item OxPhos Deficiency and Anaerobic Glycolysis
\label{sec:org0d30231}
\begin{itemize}
\item Insufficient ATP severely affects highly energy dependant tissues
\begin{itemize}
\item A complete loss of OxPhos is not observed in human disease.
\end{itemize}
\item In the absence of OxPhos cells survive using ATP from anaerobic glycolysis
\begin{itemize}
\item 20x less efficient, generates lactate
\item pyruvate \(\to\) alanine if glutamate is available:
\end{itemize}
\ce{alanine + \alpha-ketoglutarate <=>[ALT] pyruvate + glutamate}
\item Lactate, pyruvate and alanine are the typical products of anaerobic glycolysis
\end{itemize}

\item Factors Affecting OxPhos System
\label{sec:org0e70e62}
\begin{itemize}
\item \textasciitilde{} 90 subunits
\begin{itemize}
\item 13 subunits of Complexes I, III, IV and V encoded by mtDNA
\end{itemize}
\item mitochondrial replication, transcription and translation
\begin{itemize}
\item require \textgreater{} 200 proteins, rRNAs and tRNAs
\end{itemize}
\item Cofactors: coenzyme Q\(_{\text{10}}\), iron-sulfur clusters, heme, copper
\begin{itemize}
\item require synthesis and/or transport to OxPhos system
\end{itemize}
\item Cardiolipin required for cristae formation
\item Mitochondrial function
\begin{itemize}
\item protein import, turnover
\item fission, fusion
\end{itemize}
\item Toxic metabolites
\item \textgreater{} 1500 proteins in the human mitochondrial proteome
\begin{itemize}
\item other additional factors - lipids, cofactors
\item up to 10\% of human proteome may have role in maintaining mitochondrial function
\end{itemize}

\item genetic defects with impaired OxPhos phenotype 
\begin{itemize}
\item OxPhos Subunit
\item Assembly Factor
\item mtDNA replication
\item mtDNA transcription
\item cofactor
\item mitochondrial homeostasis
\begin{itemize}
\item fission and fusion
\end{itemize}
\end{itemize}
\end{itemize}
\end{enumerate}
\subsection{OxPhos Clinical Manifestations}
\label{sec:orge9fd97b}
\begin{itemize}
\item Clinical diagnosis is extremely challenging
\begin{itemize}
\item can affect any organ system
\item antenatal (IUGR, birth defects) \(\to\) elderly (myopathy)
\end{itemize}

\item Clinical Suspicion based on:
\begin{enumerate}
\item Constellation of symptoms/signs consistent with a mitochondrial syndrome
\item Complex multi-system presentation involving two/more organ systems,
best described by an underlying disorder of energy generation.
\item Lactic acidosis, characteristic neuro-imaging, 3-methylglutaconic
aciduria, ragged red fiber mypopathy.
\item Pathogenic mutation in a known mitochondrial disease gene.
\end{enumerate}
\end{itemize}


\begin{itemize}
\item Mitochondrial disease commonly presents with:
\begin{enumerate}
\item Myopathy
\item Encephalopathy
\item Leber’s hereditary optic neuropathy
\item Pearson Syndrome
\item Diabetes
\end{enumerate}
\end{itemize}

\begin{table}[htbp]
\caption{\label{tab:org46caf34}
OxPhos Clinical Manifestations}
\centering
\begin{tabular}{ll}
System & Manifestation\\
\hline
CNS & \textbf{myoclonus}\\
 & \textbf{seizures}\\
 & \textbf{ataxia}\\
Skeletal muscle & \textbf{myopathy, hypotonia}\\
 & \textbf{CPEO}\\
 & myoglobinuria\\
Marrow & sideroblastic anemia/pancytopenia\\
Kidney & fanconi\\
Endocrine & \textbf{diabetes}\\
 & hypoparathyroidism,\\
 & growth/multiple hormone deficiency\\
Heart & cardiomyopathy\\
 & conduction defects\\
GI & pancreatic failure\\
 & villous atrophy\\
Ear & \textbf{sensorineural deafness}\\
 & aminoglycoside deafness\\
Systemic & \textbf{lactic acidosis}\\
\end{tabular}
\end{table}

\begin{enumerate}
\item Myopathies
\label{sec:org678399b}
\begin{itemize}
\item Chronic progressive external ophthalmoplegia (CPEO)
\begin{itemize}
\item w/wo retinitis pigmentosa
\item most common clinical manifestation
\item muscle biopsy is diagnostic
\end{itemize}
\item Kearns-Sayre syndrome is a subtype of CPEO
\begin{itemize}
\item onset \textless{} 20
\item pigmentary retinopathy
\item cardiac conduction defect
\item ataxia, \(\uparrow\) CSF protein
\end{itemize}
\item Isolated limb myopathy
\end{itemize}

\item Encephalopathies
\label{sec:orged5c204}
\begin{itemize}
\item encephalopathic features:
\begin{itemize}
\item dementia/ID, ataxia, seizures, myoclonus, deafness, dystonia
\end{itemize}
\item MELAS: myopathy, encephalopathy, lactic acidosis, stroke-like episodes
\begin{itemize}
\item most common mito encephalopathy
\end{itemize}
\item MERRF: myoclonic epilepsy w ragged red fibres
\begin{itemize}
\item ptosis (drooping eyelids), ataxia, deafness
\end{itemize}
\item Leigh Syndrome
\begin{itemize}
\item most frequent presentation of MD in childhood
\item subacute necrotising encephalomyelopathy
\item several biochemical defects including: PDH, OxPhos
\item MRI - lesions affecting basal ganglia and/or brain stem
\item \(\uparrow\) lactate blood and CSF
\item hypo/er-ventilation, spasticity, dystonia, ataxia, tremor, optic atrophy
\item cardiomyopathy, renal tubulopathy, GI disfunction
\item \textgreater{} 75 genes(mt and nuclear)
\item Saguenay-Lac-St-Jean type incidence 1/2000, gene prevelance 1/23
\end{itemize}
\end{itemize}
\item Leber’s Hereditary Optic Neuropathy
\label{sec:orgedc5e34}
\begin{itemize}
\item most common cause of blindness in otherwise healthy young men.
\item maternally inherited and manifests in late adolescence or early
adulthood as bilateral sequential visual failure.
\item 90\% of patients are affected by age 40
\end{itemize}

\item Pearson's Syndrome
\label{sec:orgfcbda71}
\begin{itemize}
\item transfusion dependent sideroblastic anemia/pancytopenia
\item exocrine pancreas failure
\item progressive liver disease
\item renal tubular disease
\end{itemize}
\end{enumerate}

\subsection{Clinical Presentation Presentation}
\label{sec:org8205f53}
\begin{enumerate}
\item Neonatal and Infantile Presentation
\label{sec:orgd0a483e}
\begin{itemize}
\item Congenital Lactic Acidosis
\item Leigh Syndrome
\item MEGDEL: 3-methylglutaconic aciduria, deafness, encephalopathy and Leigh-like disease
\item Pearson's marrow-pancreas syndrome
\item MDDS: mitochondrial DNA depletion syndrome
\item Alper-Huttenlocher syndrome
\item Reversible infantile respiratory chain deficiency
\item Infantile onset Q\(_{\text{10}}\) biosynthetic defects
\end{itemize}

\item Childhood and Adolescent Presentation
\label{sec:orga62f911}
\begin{itemize}
\item Kearn-Sayre syndrome
\item MELAS: myopathy, encephalopathy, lactic acidosis, stroke-like episodes
\item MERRF: myoclonic epilepsy w ragged red fibres
\item NARP: neuropathy, ataxia, retinitis pigmentosa
\item LHON: Leber's Hereditary Optic Neuropathy
\item MEMSA: myoclonic epilepsy, myopathy, sensory ataxia
\item MNGIE: mitochondrial neurogastrointestinal encephalopathy
\end{itemize}

\item Adult Presentation
\label{sec:org2139f21}
\begin{itemize}
\item MIDD: maternally inherited diabetes and deafness
\item PEO: Progressive External Opthalmoplegia
\item SANDO: Sensory Ataxic Neuropathy, dysarthria and opthalmoparesis
\end{itemize}
\end{enumerate}

\subsection{Investigations}
\label{sec:org3267ae6}
\begin{enumerate}
\item Biochemistry
\label{sec:orgbeb68ff}
\begin{itemize}
\item blood lactate, CSF lactate
\item L/P \(\uparrow\) at rest, \(\Uparrow\) after excercise
\item renal tubular dysfunction: urine anion gap, pH, serum K
\item Plasma amino acids:
\begin{itemize}
\item alanine \(\propto\) pyruvate
\item ala/lys normally \textless{} 3:1
\item \(\uparrow\) gly in lipoic acid biosynthesis defects
\item \(\downarrow\) cit and arg in Leigh, NARP, MELAS and Pearson
\end{itemize}
\item Urine organic acids
\begin{itemize}
\item lactate, pyruvate, TCA intermediates
\item 3-methylglutaconic acid in Barth, Sengers, MEGDEL, ATP synthase deficiency
\item ethylmalonic
\item MMA in succinyl-CoA-ligase deficiency
\end{itemize}
\item Acylcarnitines
\begin{itemize}
\item flavin cofactor metabolism
\end{itemize}
\item Purine and pyrimidines (plasma or urine)
\begin{itemize}
\item MNGIE \(\uparrow\) thymidine and deoxyuridine
\end{itemize}
\item FGF-21, GDF15 and creatinine \(\propto\) mito disfunction in myopathy
\end{itemize}

\item Imaging
\label{sec:orge86cd0e}
\begin{itemize}
\item Cranial CT shows cerebral and cerebellar atrophy in many encephalopathic patients
\begin{itemize}
\item basal ganglia calcification may be seen in MELAS.
\end{itemize}
\item MRI in MELAS-associated stroke reveals increased T2 weighted signals in the grey and white matter
\item Symmetrical changes in the basal ganglia and brainstem observed in Leigh syndrome.
\end{itemize}

\item Histology
\label{sec:org57945ee}
\begin{itemize}
\item Muscle biopsy is diagnostic
\begin{itemize}
\item mitochondrial myopathy due to mtDNA mutations and LHON may have normal biopsies.
\end{itemize}
\item Ragged red fibres on Gomori trichrome staining, due to mitochondrial proliferation
\item fibres stain strongly for succinate dehydrogenase
\item fibres often negative for COX (complex IV) in CPEO, KSS, or MERRF but positive in MELAS.
\item Leigh syndrome patients may have no ragged red fibres and  COX-negative fibres only
\end{itemize}

\begin{figure}[htbp]
\centering
\includegraphics[width=0.5\textwidth]{./oxphos_disorders/figures/Ragged_red_fibers_in_MELAS.jpg}
\caption[rrf]{\label{fig:orgae32101}
Ragged red fibers - Gomori stain}
\end{figure}

\item Molecular
\label{sec:org11e51c1}
\begin{itemize}
\item no strict relation between phenotype and genotype.
\item mtDNA tRNA mutations are most common of the single base change abnormalities.
\begin{itemize}
\item A3243G in the tRNA\(^{\text{Leu(UUR)}}\) gene is most frequently found in MELAS
\item G8344A in tRNA\(^{\text{Lys}}\) in MERRF.
\item Many other tRNA mutations have been associated with other clinical phenotypes.
\end{itemize}
\item The primary mutations associated with LHON (G11778A, G3460A,T14484C) are in complex I genes ND4, ND1, and ND6.
\begin{itemize}
\item G11778A is most common, found in over 50\% of LHON families in the UK.
\end{itemize}
\end{itemize}
\end{enumerate}
\section{Fatty Acid Oxidation}
\label{sec:org084471e}
\subsection{Introduction}
\label{sec:org5a814e2}
\begin{itemize}
\item fatty acids are usually straight aliphatic chains with a methyl
group at one end (\(\omega\)-carbon) and a carboxyl group and the other
end
\end{itemize}

\definesubmol{x}{-[1,.6]-[7,.6]}
\definesubmol{a}{-[1,.6]\beta{}-[7,.6]\alpha{}}
\definesubmol{y}{!x!x!x!x!x!x!x!x}
\definesubmol{b}{!x!x!x!x!x!x!x!a}
%\chemfig{H{_3}C!y-[1]C(=[1]O)-[7]O{^-}}
\chemname{\chemfig{\omega{}!b-[1]C(=[1]O)-[7]O{^-}}}{stearic acid 18:0}

\begin{itemize}
\item non-systematic historical names most commonly used
\begin{description}
\item[{Palmitic acid}] discovered in palm oil
\item[{Stearic acid}] from the Greek word "stear", which means tallow
\item[{Oleic acid}] oleic means related to, or derived from, olive oil
\end{description}
\item position of a double bond is designated by the number of the carbon in the double bond that is closest to the carboxyl group
\end{itemize}


\definesubmol{x}{-[1,.6]-[7,.6]}
\definesubmol{y}{-[7,.6]-[1,.6]}
\definesubmol{d}{=[0,.6](-[7,0.25,,,draw=none]\scriptstyle\color{red}9)-[1,.6]}
\definesubmol{e}{!x!x!x!x!d!y!y!y}
\chemname{\chemfig{\omega{}(-[3,0.25,,,draw=none]\scriptstyle\color{red}18)!e(-[2,0.25,,,draw=none]\scriptstyle\color{red}2)-[7,.6]COOH}}{\small Oleic acid 18:1,\Delta{}$^9$}
\begin{itemize}
\item oleic acid - 18:1,\(\Delta^{\text{9}}\)
\begin{itemize}
\item 18:1 - 18 carbons and 1 double bond
\item \(\Delta^{\text{9}}\) - double bond between 9th and 10th carbon
\begin{itemize}
\item also 18:1(9)
\item distance from \(\omega\) methyl group, \(\omega\)-9
\end{itemize}
\end{itemize}

\item fatty acid chain length:
\begin{description}
\item[{very long-chain}] > C20
\item[{long-chain}] C12-C20
\item[{medium-chain}] C6-C12
\item[{short-chain}] C4
\end{description}
\end{itemize}


\begin{enumerate}
\item Mitochondrial fatty acid oxidation
\label{sec:org6c9f83a}

\begin{itemize}
\item involves three processes (Figure \ref{fig:org14547dc})
\begin{enumerate}
\item entry of fatty acids into mitochondria
\begin{itemize}
\item long-chain fatty acids are activated to coenzyme A (CoA) esters
in the cytoplasm
\item need to be transferred to carnitine to cross the inner
mitochondrial membrane
\item they are transferred back to CoA within the mitochondria
\item CPT I is the main site for the regulation of fatty acid
oxidation by cytoplasmic malonyl-CoA
\item medium and short-chain fatty acids enter mitochondria
independent of carnitine and are activated to CoA esters in the
matrix
\end{itemize}
\item \(\beta\)-oxidation via a spiral pathway
\begin{itemize}
\item each turn of the spiral shortens the acyl-CoA by two carbons and involves four steps:
\begin{itemize}
\item two dehydrogenation reactions linked respectively to FAD and NAD
\end{itemize}
\item \(\beta\)-oxidation is catalysed by enzymes of different chain
length specificities
\item enzymes for long-chain substrates are membrane-bound;
\begin{itemize}
\item three of the reactions are catalysed by the mitochondrial
trifunctional protein (MTP)
\item medium and short-chain enzymes are located in the matrix
\end{itemize}
\end{itemize}
\item electron transfer
\begin{itemize}
\item electrons are passed to the respiratory chain either directly
(from NADH to complex I) or via two transfer proteins (from
FADH2 to ubiquinone)
\end{itemize}
\end{enumerate}
\end{itemize}

\begin{figure}[htbp]
\centering
\includegraphics[width=0.9\textwidth]{./fao/figures/b_oxidation.png}
\caption{\label{fig:org14547dc}
\(\beta\)-oxidation}
\end{figure}


\begin{figure}[htbp]
\centering
\includegraphics[width=0.9\textwidth]{./fao/figures/Slide12.png}
\caption{\label{fig:org7424c36}
FA oxidation and ketone body metabolism}
\end{figure}


\begin{table}[htbp]
\caption{\label{tab:org1f6aeeb}
Acyl-CoA Synthetases: Chain Length Specificity}
\centering
\begin{tabular}{lrl}
Enzyme & Length & location\\
\hline
V.L. chain & 14-26 & pex\\
L. chain & 12-20 & ER, mito, pex\\
M. chain & 6-12 & mito - kidney, liver\\
acetyl & 2-4 & cyto, mito\\
\end{tabular}
\end{table}



\begin{table}[htbp]
\caption{\label{tab:org3fb023e}
Acyl-CoA Dehydrogenases: Chain Length Specificity}
\centering
\begin{tabular}{lrl}
Enzyme & Length & location\\
\hline
VLCAD & 14-20 & IMM\\
LCAD & 12-18 & MM\\
MCAD & 6-12 & MM\\
SCAD & 2-6 & MM\\
\end{tabular}
\end{table}


\begin{table}[htbp]
\caption{\label{tab:org3f223e2}
Other: Chain Length Specificity}
\centering
\begin{tabular}{lrl}
Enzyme & Length & comment\\
\hline
Enoyl-CoA hydratase,SC & >4 & \(\downarrow\) activity w \(\uparrow\) length\\
Hydroxyacyl-CoA dehydrogenase, SC & 4-16 & \(\downarrow\) activity w \(\uparrow\) length\\
Acetoacetyl-CoA thiolase & 4 & Acetoacetyl-CoA specific\\
Trifunctional protein & 12-16 & \(\uparrow\) activity w \(\uparrow\) length\\
\end{tabular}
\end{table}
\end{enumerate}

\subsection{Carnitine Cycle Defects}
\label{sec:orgcc0a47c}
\begin{itemize}
\item carnitine-mediated transport of fatty acids is rate-limiting in the
oxidation of fats
\item a defect anywhere in the pathway would be expected to lead to
inadequate formation of ketone bodies in response to fasting along
with inadequate gluconeogenesis and hypoglycemia.
\end{itemize}
\begin{enumerate}
\item Carnitine Transporter Deficiency
\label{sec:orgd29be63}
\begin{itemize}
\item AKA:primary carnitine deficiency, carnitine uptake deficiency
\end{itemize}
\begin{enumerate}
\item Clinical Presentation
\label{sec:org1ec516c}
\begin{itemize}
\item cardiomyopathy, cardiac failure, muscle weakness, liver disease
\item precipitated by infection, fasting, pregnancy or antibiotics containing pivalate
\item pivalate is excreted bound to carnitine, \(\downarrow\) carnitine concentration
\begin{itemize}
\item isobaric with C5-carnitine
\end{itemize}
\item some present in infancy with hypoglycaemia, liver dysfunction and hyperammonaemia
\item other children develop heart failure due to cardiomyopathy,
thickened ventricular walls and reduced contractility
\item often accompanied by skeletal muscle weakness
\item adults may suffer fatigue or arrhythmias
\item screening has shown that many subjects with low plasma carnitine remain asymptomatic
\end{itemize}
\item Metabolic Derangement
\label{sec:org511a3a3}
\begin{itemize}
\item organic cation/carnitine transporter(OCTN2) responsible for
carnitine uptake (Figure \ref{fig:orgffedb04})
\begin{itemize}
\item analysis of carnitine transport in different tissues suggests the
presence of heterogeneous transporters
\item liver and brain have a low-affinity (K\(_{\text{M}}\)=2-10 uM), high-capacity transporter
\item fibroblast, muscle, and heart cells have a high-affinity (K\(_{\text{M}}\)=5-10 uM), low-capacity system
\end{itemize}
\item defects \(\to\) primary carnitine deficiency with \(\uparrow\) renal loss of carnitine
\begin{itemize}
\item \(\downarrow\) plasma concentrations
\item \(\downarrow\) intracellular concentrations \(\to\) impair fatty acid
oxidation
\end{itemize}
\end{itemize}

\begin{figure}[htbp]
\centering
\includegraphics[width=0.6\textwidth]{./fao/figures/transporter.png}
\caption{\label{fig:orgffedb04}
Carnitine Transporter}
\end{figure}

\item Genetics
\label{sec:orgd518493}
\begin{itemize}
\item AR, OCTN2
\end{itemize}
\item Diagnostic Testing
\label{sec:org4dbf101}
\begin{itemize}
\item \(\Downarrow\) plasma total carnitine, \textless{} 5\% of normal
\item \(\uparrow\) urine free carnitine
\end{itemize}
\item Treatment
\label{sec:orgb726303}
\begin{itemize}
\item carnitine supplementation
\end{itemize}
\end{enumerate}
\item Carnitine Palmitoyltransferase I Deficiency
\label{sec:orgca8ca5f}
\begin{enumerate}
\item Clinical Presentation
\label{sec:orga013293}
\begin{itemize}
\item usually present by the age of 2 years with hypoketotic hypoglycaemia,
\begin{itemize}
\item induced by fasting or illness
\end{itemize}
\item accompanied by hepatomegaly, liver dysfunction and occasionally cholestasis
\begin{itemize}
\item may also be transient lipaemia and renal tubular acidosis
\end{itemize}
\end{itemize}
\item Metabolic Derangement
\label{sec:orgbd3b3fc}
\begin{itemize}
\item CPT1 is responsible for the formation of acyl carnitines
\begin{itemize}
\item catalyzes transfer of the acyl group of a long-chain fatty
acyl-CoA from coenzyme A to l-carnitine
\item allows for subsequent movement of the acyl carnitine from the
cytosol into the intermembrane space of mitochondria
\end{itemize}
\item \textbf{CPTIa} liver and kidney
\item \textbf{CPTIb}  muscle and heart
\item \textbf{CPTIc}  brain
\item only CPTIa deficiency has been identified
\item medium chain and short chain fatty acids pass directly into
mitochondria and do not require esterification with carnitine
\item CPT II is situated on the inner mitochondrial membrane, catalyzes
the regeneration of carnitine and the long chain fatty acyl CoAs,
which then undergo \(\beta\)-oxidation
\end{itemize}

\begin{figure}[htbp]
\centering
\includegraphics[width=0.6\textwidth]{./fao/figures/cpt1.png}
\caption{\label{fig:org57fc7f3}
CPT1}
\end{figure}

\item Genetics
\label{sec:org654825c}
\begin{itemize}
\item AR, CPT1A
\item CPTI deficiency is extremely common in the Inuit population of Canada and Greenland
\begin{itemize}
\item c.1436C>T, P479L
\end{itemize}
\item a few of these patients present with hypoglycaemia as neonates or young children
\begin{itemize}
\item most remain asymptomatic
\end{itemize}
\end{itemize}
\item Diagnostic Testing
\label{sec:org6bb6ae9}
\begin{itemize}
\item \(\uparrow\) total/free carnitine
\item \(\uparrow\) C0
\item \(\downarrow\) C16, C18, C18:1
\end{itemize}
\item Treatment
\label{sec:org68c4e07}
\begin{itemize}
\item prevent hypoglycaemia
\item low-fat diet
\item medium-chain triglycerides to provide \(\sim\) 1/3 total calories
\begin{itemize}
\item C6-C10 fatty acids do not require the carnitine shuttle for entry into the mitochondrion
\end{itemize}
\end{itemize}
\end{enumerate}
\item Carnitine Acylcarnitine Translocase Deficiency
\label{sec:org3b26b84}
\begin{enumerate}
\item Clinical Presentation
\label{sec:orgb2bf5a5}
\begin{itemize}
\item rare disorder usually presents in the neonatal period, with
death by 3 months of age
\begin{itemize}
\item severe hypoglycaemia and hyperammonaemia, cardiomyopathy,
atrioventricular block and ventricular arrhythmias
\end{itemize}
\item few more mildly affected patients present later with hypoglycaemic
encephalopathy
\begin{itemize}
\item precipitated by fasting or infections
\end{itemize}
\end{itemize}
\item Metabolic Derangement
\label{sec:orge9ba877}
\begin{itemize}
\item carnitine-acylcarnitine translocase, catalyzes the transfer of the
acylcarnitines across the inner mitochondrial membrane (Figure \ref{fig:org501fffa})
\item deficiency of carnitine acyl translocase leads to the accumulation
of the free fatty acids outside the mitochondrial matrix
\item long chain acylcarnitines and short chains are also found, because
translocase catalyzes the transport of short as well as long chain
acylcarnitines
\item \(\Uparrow\) long chain acyl carnitines during illness and fasting
induced lipolysis
\item \(\uparrow\) medium and short chain esters might reflect the acyl CoA products
of peroxisomal oxidation that would require transfer into the
mitochondria via the translocase for final oxidation
\item secondary deficiency of free carnitine would be expected to result
from the excretion over time of large amounts of esterified
carnitine
\end{itemize}

\begin{figure}[htbp]
\centering
\includegraphics[width=0.6\textwidth]{./fao/figures/translocase.png}
\caption{\label{fig:org501fffa}
Carnitine Translocase}
\end{figure}

\item Genetics
\label{sec:org3dc3c59}
\begin{itemize}
\item AR, SLC25A20
\end{itemize}
\item Diagnostic Testing
\label{sec:org5d4ba56}
\begin{itemize}
\item \(\Downarrow\) total carnitine
\item \(\downarrow\) C0
\item \(\Uparrow\) C16,18,C18:1
\end{itemize}
\item Treatment
\label{sec:orge85081e}
\begin{itemize}
\item prevent hypoglycaemia
\item low-fat diet
\end{itemize}
\end{enumerate}
\item Carnitine Palmitoyltransferase II Deficiency
\label{sec:org61666ee}
\begin{enumerate}
\item Clinical Presentation
\label{sec:org618eadf}
\begin{enumerate}
\item Neonatal
\label{sec:org3530772}
\begin{itemize}
\item severe neonatal onset CPT II deficiency is usually lethal
\item patients become comatose within a few days of birth
\begin{itemize}
\item hypoglycaemia and hyperammonaemia
\item may have cardiomyopathy, arrhythmias and congenital malformations,
principally renal cysts and neuronal migration defects
\end{itemize}
\item also an intermediate form of CPT II deficiency that causes
episodes of hypoglycaemia and liver dysfunction, sometimes
accompanied by cardiomyopathy and arrhythmias
\end{itemize}

\item Childhood
\label{sec:orge83a389}
\begin{itemize}
\item episodes may be brought on by infections or exercise
\item moderate or severe episodes there is myoglobinuria, \(\uparrow\) CK
\begin{itemize}
\item may lead to acute renal failure
\item CK often normalises between episodes but may remain moderately
elevated
\end{itemize}
\end{itemize}

\item Adolescence,  young adult
\label{sec:orgd7ee889}
\begin{itemize}
\item most common form is a partial deficiency that presents with
episodes of rhabdomyolysis
\begin{itemize}
\item usually precipitated by prolonged exercise
\item particularly in the cold or after fasting
\end{itemize}
\end{itemize}
\end{enumerate}

\item Metabolic Derangement
\label{sec:org28fce0d}
\begin{itemize}
\item CPTII is situated on the inner mitochondrial membrane, catalyzes the
regeneration of carnitine and the long chain fatty acyl CoAs, which
then undergo \(\beta\)-oxidation (Figure \ref{fig:org57fc7f3})
\end{itemize}
\item Genetics
\label{sec:org3412fd9}
\begin{itemize}
\item AR, CPT2
\end{itemize}
\item Diagnostic Testing
\label{sec:org7eae4d9}
\begin{itemize}
\item \(\downarrow\) total carnitine
\item \(\uparrow\) (C16 + C18)/C2
\end{itemize}

\item Treatment
\label{sec:org3da62f2}
\begin{itemize}
\item high-carbohydrate (70\%) and low-fat (<20\%) diet to provide fuel for glycolysis
\item carnitine to convert potentially toxic long-chain acyl-CoAs to
acylcarnitines
\end{itemize}
\end{enumerate}
\end{enumerate}
\subsection{\(\beta\)-Oxidation Defects}
\label{sec:orgddd1fcc}
\begin{enumerate}
\item Very-Long-Chain Acyl-CoA Dehydrogenase Deficiency
\label{sec:org9394c58}
\begin{enumerate}
\item Clinical Presentation
\label{sec:orgf923382}
\begin{enumerate}
\item Early infancy
\label{sec:org05dcc84}
\begin{itemize}
\item severely affected patients present in early infancy with
cardiomyopathy, in addition to the problems seen in milder patients
\end{itemize}
\item Childhood
\label{sec:orgfdb534d}
\begin{itemize}
\item patients present in childhood with hypoglycaemia but suffer exercise
or illness induced rhabdomyolysis or chronic weakness at a later age
\end{itemize}
\item Adolescence, Adult
\label{sec:org4a1bcd8}
\begin{itemize}
\item mildly affected patients present as adolescents or adults with
exercise-induced rhabdomyolysis
\end{itemize}
\end{enumerate}
\item Metabolic Derangement
\label{sec:org074c326}
\begin{itemize}
\item VLCAD is one of four mitochondrial acyl CoA dehydrogenases that
catalyze the initial steps in the \(\beta\)-oxidation of fatty acids
(Table \ref{tab:org3fb023e})
\item optimal substrate is c16, palmitoy-CoA
\item ACAD9 is responsible for production of C14:1-carnitine and
C12-carnitine in VLCAD deficiency
\end{itemize}
\begin{figure}[htbp]
\centering
\includegraphics[width=0.6\textwidth]{./fao/figures/vlcad.png}
\caption{\label{fig:orgb66d5bf}
VLCAD reaction}
\end{figure}

\item Genetics
\label{sec:orga5c2bfe}
\begin{itemize}
\item AR, ACADVL
\end{itemize}

\item Diagnostic Testing
\label{sec:orga478a75}
\begin{itemize}
\item \(\uparrow\) C14:1
\item \(\uparrow\) C14:1/C12:1
\item \(\uparrow\) UOA C3-C14 dicarboxylic acids
\end{itemize}
\item Treatment
\label{sec:org2924506}
\begin{itemize}
\item avoid fasting
\item more severe forms low-fat diet with MCT
\end{itemize}
\end{enumerate}

\item Mitochondrial Trifunctional Protein Deficiency
\label{sec:org8184c0c}
\begin{enumerate}
\item Clinical Presentation
\label{sec:orgf85772b}
\begin{itemize}
\item presentation of generalised MTP deficiency is heterogeneous
\item patients with severe deficiency present as neonates
\begin{itemize}
\item cardiomyopathy, respiratory distress, hypoglycaemia and liver dysfunction
\item most die within a few months, regardless of treatment
\end{itemize}
\item other patients resemble those with isolated LCHAD deficiency
\item milder neuromyopathic phenotype:
\begin{itemize}
\item exercise induced rhabdomyolysis and progressive peripheral
neuropathy
\item can present at any age from infancy to adulthood
\end{itemize}
\item mothers who are heterozygous for LCHAD or MTP deficiency have a high
risk of illness during pregnancies when carrying an affected fetus
\begin{itemize}
\item main problems are HELLP syndrome (Haemolysis, Elevated Liver
enzymes and Low Platelets) and acute fatty liver of pregnancy
(AFLP)
\end{itemize}
\end{itemize}
\item Metabolic Derangement
\label{sec:orge726de2}
\begin{itemize}
\item MTP a hetero-octamer composed of four \(\alpha\)-subunits and four
\(\beta\)-subunits
\item \(\alpha\)-subunit has long-chain enoyl-CoA hydratase (LCEH) and LCHAD
activities
\item \(\beta\)-subunit has long-chain ketoacyl-CoA thiolase (LCKAT) activity
\item patients may have isolated LCHAD deficiency or a generalised
deficiency of all three enzyme activities
\item MTP deficiency can result from mutations that affect the assembly of
and/or degradation of the heterooctomeric holoenzyme
\end{itemize}

\item Diagnostic Tests
\label{sec:orgf987491}
\begin{itemize}
\item see \ref{sec:org16b5ded}
\end{itemize}

\item Treatment
\label{sec:orga9a5dbd}
\begin{itemize}
\item see \ref{sec:org16b5ded}
\end{itemize}
\end{enumerate}

\item Long-Chain 3-Hydroxyacyl-CoA Dehydrogenase
\label{sec:org16b5ded}
\begin{enumerate}
\item Clinical Presentation
\label{sec:org290555d}
\begin{itemize}
\item isolated LCHAD deficiency usually presents acutely before 6 months of age
\begin{itemize}
\item hypoglycaemia, liver dysfunction, lactic acidosis
\item many have cardiomyopathy, some have hypoparathyroidism or ARDS
\end{itemize}
\item other patients present with chronic symptoms
\begin{itemize}
\item failure to thrive, hypotonia, occasionally cholestasis or cirrhosis
\end{itemize}
\item subsequently, episodes of rhabdomyolysis are common
\item many patients develop retinopathy, may start as early as 2 years of age
\item granular pigmentation followed by chorioretinal atrophy w deteriorating central vision
\item some patients develop cataracts
\end{itemize}
\item Metabolic Derangement
\label{sec:org101b95e}
\begin{itemize}
\item LCHAD is a component of MTP
\item bound to the inner mitochondrial membrane
\item activity is optimal for C12-C16
\item catalyzes dehydration of the 3-hydroxy group to a 3-keto group
(Figure \ref{fig:orgbb98097})
\end{itemize}
\begin{figure}[htbp]
\centering
\includegraphics[width=0.6\textwidth]{./fao/figures/lchad.png}
\caption{\label{fig:orgbb98097}
LCHAD reaction}
\end{figure}

\item Genetics
\label{sec:orgad27375}
\begin{itemize}
\item AR, HADHA
\end{itemize}

\item Diagnostic Tests
\label{sec:org7e11814}
\begin{itemize}
\item \(\uparrow\) lactate, 3-OH-palmitoyl-CoA inhibits PDH
\item \(\uparrow\) C14OH, C16OH, C18OH, C18:1OH
\item \(\uparrow\) UOA C6-C14 (hydroxy-)dicarboxylic acids
\end{itemize}

\item Treatment
\label{sec:orgb7a4903}
\begin{itemize}
\item avoid fasting
\item low fat diet with MCT
\end{itemize}
\end{enumerate}
\item Long-Chain Acyl-CoA Dehydrogenase Deficiency
\label{sec:org8eb4861}
\begin{itemize}
\item LCAD is one of four mitochondrial acyl CoA dehydrogenases that
catalyze the initial steps in the \(\beta\)-oxidation of fatty acids
(Table \ref{tab:org3fb023e})
\item no human disease-causing mutations have been identified
\item role  in  human  metabolism  is unclear
\item the substrate specificity of LCAD overlaps with that of
VLCAD and MCAD
\end{itemize}

\item Medium-Chain Acyl-CoA Dehydrogenase Deficiency
\label{sec:org630e261}
\begin{enumerate}
\item Clinical Presentation
\label{sec:org564ec63}
\begin{itemize}
\item most common FAOD with an incidence of approximately 1:10,000-20,000
in Europe,USA and Australia
\item before NBS, presented 4 months to 4 years
\begin{itemize}
\item acute hypoglycaemic encephalopathy and liver dysfunction, not always
\item some deteriorated rapidly and died
\end{itemize}
\item precipitated by prolonged fasting or infection with vomiting
\item some babies still present within 72 hours of birth before
newborn screening results are available
\begin{itemize}
\item hypoglycaemia and/or arrhythmias
\item breast-fed babies are at higher risk, due to the small supply of
breast milk at this stage
\end{itemize}
\item MCAD deficiency only presents clinically if exposed to an
appropriate environmental stress
\begin{itemize}
\item prior to NBS \textasciitilde{} 30-50\% remained asymptomatic
\end{itemize}
\item with NBS and preventative measures, hypoglycaemia is rare
\begin{itemize}
\item patients do not develop cardiomyopathy or myopathy and few present
initially as adults
\end{itemize}
\item healty MCAD deficient children > 1 year can fast for 12-14 hours without problems
\begin{itemize}
\item >14 hours \(\to\) hypoketotic hypoglycaemia
\end{itemize}
\item shorter fasts may cause problems in infancy
\item encephalopathy may occur without hypoglycaemia
\begin{itemize}
\item accumulation of FFA acids and carnitine/CoA esters
\end{itemize}
\end{itemize}
\item Metabolic Derangement
\label{sec:org5340c61}
\begin{itemize}
\item MCAD is one of four mitochondrial acyl CoA dehydrogenases that
catalyze the initial steps in the \(\beta\)-oxidation of fatty acids
(Table \ref{tab:org3fb023e})
\item MCAD accepts fatty acyl CoAs in which the acid chain length is 6–12
carbons in length
\end{itemize}

\begin{figure}[htbp]
\centering
\includegraphics[width=0.6\textwidth]{./fao/figures/acad.png}
\caption{\label{fig:orgcad8a8a}
Acyl-CoA Dehydrogenase Reaction}
\end{figure}

\item Genetics
\label{sec:orgfd4567b}
\begin{itemize}
\item AR, ACADM
\end{itemize}
\item Diagnostic Tests
\label{sec:org7c9c004}
\begin{itemize}
\item \(\uparrow\) C8, C6,
\item \(\uparrow\) C8/C10
\item \(\uparrow\) UOA C6-C10 dicarboxylic acids, suberylglycine, hexanolyglyine
\end{itemize}
\item Treatment
\label{sec:org3275050}
\begin{itemize}
\item avoid fasting
\item low fat diet in infants
\end{itemize}
\end{enumerate}

\item Short-Chain Acyl-CoA Dehydrogenase Deficiency
\label{sec:org76b8dad}
\begin{itemize}
\item non-disease
\begin{itemize}
\item previous association with symptoms due to ascertainment bias
\end{itemize}
\end{itemize}

\item 3-Hydroxyacyl-CoA Dehydrogenase Deficiency
\label{sec:org4036a01}
\begin{itemize}
\item HADH, previously called SCHAD deficiency, causes hyperinsulinaemic
hypoglycaemia
\item role in modulation of ATP production inhibition of GDH
\item see section Congenital Hyperinsulinema
\end{itemize}
\end{enumerate}
\subsection{Electron Transfer Defects}
\label{sec:org083cfae}
\begin{enumerate}
\item Multiple acyl-CoA dehydrogenase deficiency
\label{sec:org5a8a7ba}
\begin{itemize}
\item AKA glutaric aciduria type II
\item Electron transfer flavoprotein (ETF) and ETF ubiquinone
oxidoreductase (ETFQO) carry electrons to the respiratory chain from
multiple FAD-linked dehydrogenases
\item includes enzymes of amino acid, choline metabolism and acyl-CoA
dehydrogenases of \(\beta\)-oxidation
\item defects in ETF or ETFQO \(\to\)
\item GAII less often, a result of defects of riboflavin transport or
metabolism
\item ETF and ETFQO deficiencies \(\to\) wide range of clinical severity
\item severely affected patients present in the first few days of life
\begin{itemize}
\item hypoglycaemia, hyperammonaemia and acidosis
\item hypotonia and hepatomegaly
\end{itemize}
\item there is usually an odour of sweaty feet similar to that in isovaleric acidaemia
\item some patients have congenital anomalies
\begin{itemize}
\item large cystic kidneys, hypospadias and neuronal migration defects and facial dysmorphism
\begin{itemize}
\item low set ears, high forehead and midfacial hypoplasia
\end{itemize}
\end{itemize}
\item the malformations resemble those seen in CPT II deficiency but the pathogenesis is unknown
\item most patients with neonatal presentation die within a week of birth
\item others develop cardiomyopathy and die within a few months
\item less severe cases can present at any age from infancy to adulthood
\begin{itemize}
\item with hypoglycaemia, liver dysfunction and weakness
\item usually precipitated by an infection
\end{itemize}
\item cardiomyopathy is common in infants
\item rarer problems include stridor and leukodystrophy
\item mildly affected children may have recurrent bouts of vomiting
\item muscle weakness is the commonest presentation in adolescents and adults
\begin{itemize}
\item predominantly affects proximal muscles and may lead to scoliosis,
hypoventilation or an inability to lift the chin off the chest
\end{itemize}
\item weakness can worsen rapidly during infection or pregnancy, myoglobinuria is rare
\begin{itemize}
\item many milder defects respond to riboflavin
\end{itemize}
\end{itemize}
\end{enumerate}

\subsection{Common manifestations in FAODs}
\label{sec:orgfd05565}
\begin{itemize}
\item fasting hypoglycaemia is the classic metabolic disturbance in FAODs
\begin{itemize}
\item primarily due to increased peripheral glucose consumption
\item hepatic glucose output is also reduced under some conditions
\end{itemize}
\item the hypoglycaemia is hypoketotic
\begin{itemize}
\item ketone bodies can be synthesised
\begin{itemize}
\item medium-or short-chain FAODs or if there is high residual enzyme activity
\item plasma concentrations are lower than expected for hypoglycaemia or the plasma free fatty acid concentrations
\end{itemize}
\end{itemize}
\item hyperammonaemia occurs in some severe defects
\begin{itemize}
\item with normal or low glutamine concentrations
\item decreased acetyl-CoA production reducing the synthesis of N-acetylglutamate
\end{itemize}
\item lactic acidaemia is seen in long-chain FAODs(LCHAD and MTP deficiencies)
\begin{itemize}
\item inhibitory effects of metabolites on pyruvate metabolism
\end{itemize}
\item moderate hyperuricaemia - frequent finding during acute attacks
\item secondary hyperprolinaemia occurs in some babies with MAD deficiency
\item accumulating long-chain acylcarnitines may be responsible for
arrhythmias and may interfere with surfactant metabolism
\item in LCHAD and MTP deficiencies, long chain hydroxy-acylcarnitine
concentrations correlate with the severity of retinopathy and may
cause both this and the peripheral neuropathy
\end{itemize}

\begin{figure}[htbp]
\centering
\includegraphics[width=1.2\textwidth]{./fao/figures/Ch101f016.png}
\caption{\label{fig:org338b8f2}
Common manifestations in FAODs}
\end{figure}

\begin{itemize}
\item Green squares indicate that the feature is frequently seen in the disorder
\item Yellow squares represent an intermediate rate of occurrence
\item Red squares denote that it is uncommon
\end{itemize}
\section{Ketogenesis and Ketolysis}
\label{sec:orge11df7d}
\subsection{Introduction}
\label{sec:org4919fe2}

\begin{itemize}
\item During fasting, ketone bodies (KB) are an important fuel for many
tissues, including cardiac and skeletal muscle
\item particularly important for the brain, which cannot oxidise fatty acids
\item The principal KB, acetoacetate and 3-hydroxybutyrate, are
maintained in equilibrium by 3-hydroxybutyrate dehydrogenase
\item acetone is formed from acetoacetate non-enzymatically and eliminated in breath
\item formed in liver mitochondria from fatty acids and certain amino acids(e.g. leucine)
\item cross cell membranes by diffusion or facilitated by monocarboxylate transporter 1 (MCT1)
\begin{itemize}
\item MCT1 transport is important during catabolic stress, \(\uparrow\) flux
\end{itemize}
\item converted to acetyl-CoA in the mitochondria of extrahepatic tissues for use as fuel
\item One ketolytic enzyme, mitochondrial acetoacetyl-CoA thiolase (AKA
/beta-ketothiolase or T2), is also involved in isoleucine catabolism
\end{itemize}

\begin{figure}[htbp]
\centering
\includegraphics[width=0.9\textwidth]{./ketones/figures/ketones.png}
\caption{\label{fig:orgfa3a586}
Ketogenesis and Ketolysis}
\end{figure}

\begin{itemize}
\item Disorders of ketone body metabolism present either in the first few
days of life or later in childhood, during an infection or some
other metabolic stress.
\item the primary aim of treatment is to prevent decompensation.
\item Fasting is avoided and a high glucose intake is maintained at times
of metabolic stress, such as infections.
\end{itemize}

\subsection{Ketogenesis Defects}
\label{sec:org039cb4e}
\begin{itemize}
\item There are two defects of ketogenesis:
\begin{enumerate}
\item 3-hydroxy-3-methylglutaryl (HMG)-CoA lyase deficiency
\item HMG-CoA synthase deficiency
\end{enumerate}
\end{itemize}

\begin{enumerate}
\item Clinical Presentation
\label{sec:org8862cc6}
\begin{enumerate}
\item Mitochondrial 3-Hydroxy-3-Methylglutaryl-CoA (HMG-CoA) Synthase Deficiency
\label{sec:org02c3ebc}
\begin{itemize}
\item presents with hypoglycaemia
\begin{itemize}
\item often accompanied by coma and metabolic acidosis
\item precipitated by infections with vomiting or poor feeding in early
childhood (5 months to 6 years of age)
\end{itemize}
\item usually hepatomegaly, which resolves
\item Hyperammonaemia is rare, (surprisingly) ketonuria may be present
\item Most patients recover with intravenous glucose
\end{itemize}

\item HMG-CoA Lyase Deficiency
\label{sec:orgf656166}
\begin{itemize}
\item presents with hypoglycaemia, metabolic acidosis, vomiting and a
reduced level of consciousness
\item 30\% patients, the onset is within 5 days of birth, after a short symptom-free period
\item \textasciitilde{}70\% patients, symptoms are provoked by an infection in the first year
\item A few patients first present later, occasionally as adults
\item KB levels are inappropriately low blood lactate concentrations
may be markedly elevated, particularly in neonatal onset cases
\item often hyperammonaemia, hepatomegaly and abnormal liver function tests and
may develop pancreatitis or cardiomyopathy
\item neurological sequelae included:
\begin{itemize}
\item epilepsy, intellectual handicap, hemiplegia or cerebral visual loss
\item magnetic resonance imaging (MRI) shows diffuse mild signal changes
in the cerebral white matter on T 2 -weighted images with multiple
foci of more severe signal abnormality
\end{itemize}
\item myelination may be impaired because KB are a substrate for the
synthesis of myelin cholesterol.
\end{itemize}
\end{enumerate}

\item Metabolic Derangement
\label{sec:org3996e4b}
\begin{itemize}
\item KB are synthesised in hepatic mitochondria, primarily using
acetyl-CoA derived from fatty acid oxidation
\item HMG-CoA synthase catalyses the condensation of acetoacetyl-CoA and
acetyl-CoA to form HMG-CoA, which is cleaved by HMG-CoA lyase to
release acetyl-CoA and acetoacetate.
\item HMG-CoA can also be derived from leucine.
\item HMG-CoA synthase and HMG-CoA lyase deficiencies both impair
ketogenesis but HMG-CoA lyase deficiency also causes the
accumulation of intermediates of the leucine catabolic pathway.
\item During fasting, the lack of KB leads to excessive glucose
consumption and hypoglycaemia.
\end{itemize}

\item Genetics
\label{sec:orgf767c61}
\begin{itemize}
\item HMG-CoA synthase (HMGCS2) and HMG-CoA lyase (HMGCL) deficiencies are
AR
\end{itemize}

\item Diagnostic Tests
\label{sec:org2f90f82}
\begin{itemize}
\item Critical sample is important
\item If the plasma free fatty acid concentration is raised with an
inappropriately small rise in total KB it implies a defect of
ketogenesis or fatty acid oxidation 
\begin{itemize}
\item plasma FFA/total KB >2.5
\end{itemize}
\end{itemize}

\begin{enumerate}
\item HMG-CoA Synthase Deficiency
\label{sec:orgcdddbd9}
\begin{itemize}
\item During decompensation, urine contains saturated, unsaturated and
3-hydroxy-dicarboxylic acids, 5-hydroxyhexanoic acid and other
metabolites, of which 4-hydroxy-6-methyl-2-pyrone is the most
specific.
\item plasma acylcarnitines are normal when patients are well
\begin{itemize}
\item raised during illness.
\end{itemize}
\item The diagnosis is confirmed by mutation analysis.
\end{itemize}

\item HMG-CoA Lyase Deficiency
\label{sec:orgfe14a0b}
\begin{itemize}
\item Even when healthy, patients excrete increased quantities of
3-hydroxy-3-methylglutaric, 3-hydroxyisovaleric, 3-methyl-glutaconic
and 3-methylglutaric acids
\item 3-methyl-crotonylglycine may also be present
\item Blood acylcarnitine analysis shows raised
3-hydroxyisovalerylcarnitine (C5OH)
\item diagnosis is confirmed by mutation analysis or measuring HMG-CoA
lyase activity in leukocytes or cultured fibroblasts.
\end{itemize}
\end{enumerate}

\item Treatment
\label{sec:orgac376c6}
\begin{itemize}
\item avoid fasting and maintain  high carbohydrate intake during any
metabolic stress, such as infections
\item IV glucose is required if drinks containing if not oral intake or vomiting
\item IV sodium bicarbonate may be needed if there is severe acidosis in
HMG-CoA lyase deficiency
\item moderate protein restriction is usually recommended in HMG-CoA
lyase deficiency because of its role in leucine catabolism
\item HMG-CoA synthase deficiency has a good prognosis
\item HMG-CoA lyase deficiency have more encephalopathy as children or adults
\end{itemize}
\end{enumerate}

\subsection{Ketolysis Defects}
\label{sec:org17b761d}
\begin{itemize}
\item Ketone body utilisation is catalysed by:
\begin{enumerate}
\item succinyl-CoA:3-oxoacid CoA transferase (SCOT)
\item mitochondrial acetoacetyl-CoA thiolase (T2)
\end{enumerate}
\item Deficiencies of SCOT, T2 or MCT1 present with episodes of ketoacidosis.
\end{itemize}

\begin{enumerate}
\item Clinical Presentation
\label{sec:org32aeb85}
\begin{itemize}
\item Patients present with episodes of severe ketoacidosis in early childhood
\item A few patients have seizures or cardiomegaly at the time of presentation
\item Patients are healthy between episodes, with normal blood pH
\item Decompensation triggered by fasting or an infection with poor
feeding and vomiting
\item Blood glucose, lactate and ammonia concentrations are usually normal
\end{itemize}

\item Metabolic Derangement
\label{sec:org625868c}
\begin{itemize}
\item KB utilisation occurs in extrahepatic mitochondria, starting with
the transfer of coenzyme A from succinyl-CoA to acetoacetate,
catalysed by SCOT.
\item This forms acetoacetyl-CoA, which is converted to acetyl-CoA by T2.
\item The second reaction can also be catalysed to some extent by
medium-chain 3-ketoacyl-CoA thiolase (T1)
\begin{itemize}
\item may explain why T2 deficient patients do not have permanent
ketosis
\end{itemize}
\item SCOT is not expressed in liver and has no role other than
ketolysis.
\item T2 is expressed in liver, involved in ketogenesis and ketolysis
\item Patients with T2 deficiency present with ketoacidosis
\begin{itemize}
\item T2 more crucial in ketolysis than in ketogenesis
\end{itemize}
\item T2 also cleaves 2-methylacetoacetyl-CoA in the isoleucine
degradation pathway
\begin{itemize}
\item T2 deficiency \(\to\) \(\uparrow\) isoleucine-derived acyl-CoA esters
\item these may be responsible for neurodevelopmental abnormalities
\end{itemize}
\item ketoacidosis in patients with MCT1 deficiency \(\to\) transporters are
facilitate rapid entry of KB at times of stress.
\item MCT1 transporters are important for lactate transport
\begin{itemize}
\item expressed in the brain, particularly on oligodendroglia.
\end{itemize}
\item learning difficulties in MCT1 deficient patients may be due to the
absence of MCT1 in the brain
\end{itemize}

\item Genetics
\label{sec:org732f322}
\begin{itemize}
\item AR - SCOT (OXCT1), T2 (ACAT1) and MCT1 (SLC16A1)
\item Heterozygous SLC16A1 and OXCT1 mutations have, however, been found
in several patients investigated for ketoacidosis, suggesting that
they can cause problems if subjects are exposed to sufficient stress.

\item Heterozygous SLC16A1 mutations can also cause hyperinsulinism;
these patients have promoter mutations that prevent the normal
silencing of MCT1 expression in pancreatic \(\beta\)-cells
\end{itemize}

\item Diagnostic Tests
\label{sec:org7b14ba4}
\begin{enumerate}
\item SCOT \& MCT1 Deficiencies
\label{sec:org7f36eef}
\begin{itemize}
\item should be considered in a number of patients because ketoacidosis is relatively common
\item A plasma free fatty acid:
\begin{itemize}
\item plasma FFA/total KB <0.3 suggests a defect of ketolysis
\end{itemize}
\item Urine organic acids show \(\uparrow\) KB but no specific abnormalities.
\item Patients with severe SCOT deficiency have persistent ketonuria in
the fed state, but patients with a mild mutation do not
\item The diagnoses are now usually made by mutation analysis
\end{itemize}
\item T2 Deficiency
\label{sec:orgec13106}
\begin{itemize}
\item \(\uparrow\) urine 2-methylacetoacetate, 2-methyl-3-hydroxybutyric acid
and tiglylglycine.
\begin{itemize}
\item 2-methylacetoacetate is unstable
\end{itemize}
\item patient with mild mutations may only show abnormalities when they
are stressed (e.g isoleucine load)
\item 2-Methyl-3-hydroxybutyryl-CoA dehydrogenase deficiency causes a
similar pattern of organic acids but 2-methyacetoacetate is not excreted
\item diagnosis must be confirmed by mutation analysis or enzyme assay in fibroblasts.
\begin{itemize}
\item Assays are complicated by the presence of other thiolases that act
on acetoacetyl-CoA.
\end{itemize}
\item \(\uparrow\) 2-methyl-3-hydroxybutyrylcarnitine and
tiglylcarnitine on plasma acylcarnitine
\begin{itemize}
\item may be normal with mild mutations
\end{itemize}
\end{itemize}
\end{enumerate}

\item Treatment
\label{sec:org3e1ff72}
\begin{itemize}
\item same as ketogenesis defects above
\item T2 involved in isoleucine met, \(\therefore\) \(\downarrow\) protein
\end{itemize}
\end{enumerate}
\section{Creatine Deficiency}
\label{sec:orgd017769}
\subsection{Introduction}
\label{sec:org6f4247f}
\begin{itemize}
\item Creatine is synthesized by two enzymatic reactions:
\begin{enumerate}
\item L-arginine:glycine amidinotransferase (AGAT, GATM) catalyses the
transfer of an amidino group from arginine to glycine, yielding
guanidinoacetate
\item S-adenosyl-L-methionine:N-guanidinoacetatemethyltransferase
(GAMT, GAMT) catalyses the methylation of the amidino group in
the guanidino-acetate molecule, yielding creatine.
\end{enumerate}

\item Creatine synthesis occurs mainly, but not exclusively, in the kidney
and in the pancreas, which have high AGAT activity, and in the
liver, which has high GAMT activity.

\item From these organs of synthesis, creatine is transported via the
bloodstream to the organs of utilization, mainly muscle and brain
\begin{itemize}
\item taken up by a sodium and chloride dependent creatine transporter
(CRTR,SLC6A8).
\end{itemize}

\item Most brain creatine is transported from the blood via CRTR expressed
at blood-brain barrier.
\item A minor proportion is synthesized in the brain.

\item Part of intracellular creatine is reversibly converted into the
high-energy compound creatine-phosphate, by creatine kinase (CK).

\item Three cytosolic isoforms of CK (brain type BB-CK, muscle type MM-CK
and the MB-CK heterodimer), and two mitochondrial isoforms exist.

\item Creatine and creatine-phosphate are non-enzymatically converted into
creatinine, with a constant daily turnover of 1.5\% of body
creatine.

\item Creatinine is mainly excreted in urine and its daily excretion is
directly proportional to total body creatine, and in particular to
muscle mass
\end{itemize}

\begin{figure}[htbp]
\centering
\includegraphics[width=0.9\textwidth]{./creatine/figures/creatine.png}
\caption{\label{fig:org09f0204}
Metabolic pathway of creatine/creatine phosphate}
\end{figure}


\subsection{Creatine Deficiency Syndromes}
\label{sec:orgc7f839d}
\begin{itemize}
\item Creatine deficiency syndromes (CDS) are a group of inborn errors of
creatine synthesis:
\begin{itemize}
\item arginine:glycine amidinotransferase (AGAT) deficiency
\item guanidinoacetate methyltransferase (GAMT) deficiency
\end{itemize}
\item and transport:
\begin{itemize}
\item X-linked creatine transporter CRTR deficiency
\end{itemize}
\item global developmental delay/ intellectual disability along with
various neurological manifestations.

\item Secondary changes in creatine metabolism have been described in
disorders affecting arginine and ornithine metabolism including:
\begin{itemize}
\item ornithine aminotransferase (OAT) deficiency
\item urea cycle defects
\item hyperammonemia, hyperornithinemia, homocitrullinuria (HHH) syndrome
\item \(\Delta\)1-pyrroline-5-carboxylate synthetase (P5CS) deficiency
\item methylcobalamin synthesis defects
\item mitochondrial disease
\end{itemize}
\end{itemize}

\begin{enumerate}
\item Clinical Presentation
\label{sec:org6974ab3}
\begin{itemize}
\item global developmental delay (GDD) /intellectual disability (ID) with
prominent speech delay.
\item GDD or ID ranges from mild to severe and is characteristically
associated with challenging (hyperactive, autistic) behaviours.
\item Movement disorders and basal ganglia changes are additional features
in GAMT deficiency.
\item Myopathy is an additional feature of AGAT deficiency.
\item Epilepsy occurs in GAMT and CRTR deficiencies.
\end{itemize}

\item Metabolic Derangement
\label{sec:orgb1c4d92}
\begin{itemize}
\item Cerebral  creatine  deficiency is  caused  by  reduced synthesis  of
creatine in AGAT and GAMT deficiency  or by impaired uptake into the
brain in CRTR deficiency.
\item Low intracellular creatine and creatine-phosphate result in reduced
production of creatinine
\begin{itemize}
\item \(\therefore\) \low urine and plasma creatinine
\end{itemize}
\item Guanidinoacetate is neurotoxic \(\to\) CNS
\begin{itemize}
\item \(\downarrow\) AGAT deficiency
\item \(\uparrow\) GAMT deficiency
\end{itemize}
\end{itemize}

\item Genetics
\label{sec:org4c38df3}
\begin{itemize}
\item AR, AGAT (GATM) and GAMT (GAMT)
\item X-linked, CRTC (SLC6A8)
\end{itemize}
\item Diagnostic Tests
\label{sec:orgc8dfbcf}
\begin{itemize}
\item MRS to screen for cerebral creatine deficiency
\item urine guanidinoacetate, creatine and creatinine is an important
screening test for all CDS
\begin{itemize}
\item \(\downarrow\) AGAT, \(\le\) 10\% of LLN
\item \(\uparrow\) GAMT , 10x normal
\end{itemize}
\item CSF guanidinoacetate
\begin{itemize}
\item \(\uparrow\) GAMT, 100x normal
\end{itemize}
\item \(\uparrow\) urine creatinine/creatine in CRTR
\end{itemize}

\item Treatment
\label{sec:orgb742fb3}
\begin{itemize}
\item Normal neurodevelopmental outcome has been reported in early treated
patients with creatine synthesis defects.
\item AGAT and GAMT treated wtih creatine monohydrate
\item In GAMT deficiency, reduction of guanidinoacetate is achieved by
ornithine supplementation \textpm{} dietary arginine restriction.
\item CRTC no treatment
\item In CRTR deficiency, creatine, arginine and glycine supplementation
does not significantly improve outcome, although partial clinical
improvement has been reported in single patients.
\end{itemize}
\end{enumerate}
\end{document}