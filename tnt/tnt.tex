% Created 2019-11-14 Thu 15:53
% Intended LaTeX compiler: pdflatex
\documentclass{scrartcl}
\usepackage[utf8]{inputenc}
\usepackage[T1]{fontenc}
\usepackage{graphicx}
\usepackage{grffile}
\usepackage{longtable}
\usepackage{wrapfig}
\usepackage{rotating}
\usepackage[normalem]{ulem}
\usepackage{amsmath}
\usepackage{textcomp}
\usepackage{amssymb}
\usepackage{capt-of}
\usepackage{hyperref}
\hypersetup{colorlinks,linkcolor=black,urlcolor=blue}
\usepackage{textpos}
\usepackage{textgreek}
\usepackage[version=4]{mhchem}
\usepackage{chemfig}
\usepackage{siunitx}
\usepackage{gensymb}
\usepackage[usenames,dvipsnames]{xcolor}
\usepackage[T1]{fontenc}
\usepackage{lmodern}
\usepackage{verbatim}
\usepackage{tikz}
\usepackage{wasysym}
\usetikzlibrary{shapes.geometric,arrows,decorations.pathmorphing,backgrounds,positioning,fit,petri}
\usepackage{fancyhdr}
\pagestyle{fancy}
\author{Matthew Henderson, PhD, FCACB}
\date{\today}
\title{Thompson \& Thompson}
\hypersetup{
 pdfauthor={Matthew Henderson, PhD, FCACB},
 pdftitle={Thompson \& Thompson},
 pdfkeywords={},
 pdfsubject={},
 pdfcreator={Emacs 26.1 (Org mode 9.1.9)}, 
 pdflang={English}}
\begin{document}

\maketitle
\setcounter{tocdepth}{1}
\tableofcontents


\section{Chapter 2: Human Genome}
\label{sec:orgc7b5392}
\subsection{Mitosis}
\label{sec:org1f72700}
\begin{description}
\item[{Prophase}] condensation of the chromosomes
\begin{itemize}
\item formation of the mitotic spindle,
\item formation of a pair of centrosomes, from which microtubules
radiate and eventually take up positions at the poles of the cell.
\end{itemize}
\item[{Prometaphase}] nuclear membrane dissolves
\begin{itemize}
\item chromosomes disperse within the cell
\item attach, by their kinetochores, to microtubules of the mitotic
spindle.
\end{itemize}
\item[{Metaphase}] the chromosomes condensed and line up at the equatorial
plane.
\item[{Anaphase}] chromosomes separate at the centromere
\begin{itemize}
\item sister chromatids of each chromosome now become independent
daughter chromosomes, which move to opposite poles of the cell.
\end{itemize}
\item[{Telophase}] chromosomes begin to decondense
\begin{itemize}
\item nuclear membrane begins to re-form around each of the two daughter
nuclei
\item cytoplasm cleaves = cytokinesis
\end{itemize}
\end{description}

\begin{figure}[htbp]
\centering
\includegraphics[width=0.9\textwidth]{./figures/ch2_mitosis.png}
\caption{\label{fig:org6192e53}
Mitosis}
\end{figure}


\begin{figure}[htbp]
\centering
\includegraphics[width=0.9\textwidth]{./figures/ch2_karyotype.png}
\caption{\label{fig:orgeb4020f}
Karyotype}
\end{figure}

\subsection{Meiosis}
\label{sec:org4bec387}

\begin{itemize}
\item Meiosis I is also known as the reduction division because it is the
division in which the chromosome number is reduced by half through
the pairing of homologues in prophase and by their segregation to
different cells at anaphase of meiosis I.
\item the stage at which genetic recombination (also called meiotic crossing over) occurs.
\begin{itemize}
\item homologous segments of DNA are exchanged between non-sister
chromatids of each pair of homologous chromosomes, thus ensuring
that none of the gametes produced by meiosis will be identical to
another.
\end{itemize}
\item Theconceptual and practical consequences of recombination for many
\end{itemize}
aspects of human gen

\begin{figure}[htbp]
\centering
\includegraphics[width=0.9\textwidth]{./figures/ch2_meiosis.png}
\caption{\label{fig:org383edee}
Meiosis}
\end{figure}


\subsection{Homologous Recombination}
\label{sec:org7394649}

\begin{itemize}
\item the genetic content of each gamete is unique, because of random
assortment of the parental chromosomes to shuffle the combination of
sequence variants between chromosomes and because of homologous
recombination to shuffle the combination of sequence variants within
each and every chromosome.
\item This has significant consequences for patterns of genomic variation
among and between different populations around the globe and for
diagnosis and counseling of many common conditions with complex patterns of inheritance.
\item The amounts and patterns of meiotic recombination are determined by
sequence variants in specific genes and at specific “hot spots” and
differ between individuals, between the sexes, between families, and
between populations
\item Because recombination involves the physical intertwining of the two
homologues until the appropriate point during meiosis I, it is also
critical for ensuring proper chromosome segregation during
meiosis.
\item Failure to recombine properly can lead to chromosome missegregation
(nondisjunction) in meiosis I and is a frequent cause of pregnancy
loss and of chromosome abnormalities like Down syndrome
\item Although homologous recombination is normally precise, areas of
repetitive DNA in the genome and genes of variable copy number in
the population are prone to occasional unequal crossing over during
meiosis, leading to variations in clinically relevant traits such as
drug response, to common disorders such as the thalassemias or
autism, or to abnormalities of sexual differentiation
\item Although homologous recombination is a normal and essential part of
meiosis, it also occurs, albeit more rarely, in somatic
cells.
\item Anomalies in somatic recombination are one of the causes of genome
instability in cancer
\end{itemize}

\section{Chapter 3: Gene Structure and Function}
\label{sec:orgab86aaf}
\subsection{Allelic Imbalance in Gene Expression}
\label{sec:orga490707}
\begin{itemize}
\item Monoallelic gene expression
\begin{itemize}
\item Somatic rearrangement - T-cell receptors
\item random monoallelic expression
\item parent-of-origin imprinting
\end{itemize}
\item X chromosome inactivation
\begin{itemize}
\item random, X inactivation center
\item ncRNA called XIST
\end{itemize}
\end{itemize}

\section{Chapter 4: Human Genetic Diversity}
\label{sec:org92bf7a6}
\subsection{Inherited variation and polymorphism}
\label{sec:org45f28a6}
\begin{itemize}
\item SNP
\item Indels
\item microsatelite
\item mobile element insertion polymorphism
\begin{itemize}
\item retrotranspostion: Alu, LINE
\end{itemize}
\item CNVs
\begin{itemize}
\item related to indels and microsatellites but variation in
the number of copies of larger segments of the genome
\item 1000 bp to many hundreds of kilobase pairs.
\end{itemize}
\item Inversion polymorphism
\begin{itemize}
\item few base pairs up to several megabase pairs
\item can be present in either of two orientations in the genomes of different individuals
\end{itemize}
\end{itemize}
\begin{figure}[htbp]
\centering
\includegraphics[width=0.9\textwidth]{./figures/ch4_polymorphism.png}
\caption{\label{fig:org10c00fa}
Polymorphism}
\end{figure}


\subsection{Origins and Freq of Mutation types}
\label{sec:orgae9f44d}
\begin{itemize}
\item germline
\item somatic
\item Mutations:
\begin{itemize}
\item Chromosome number
\item Regional: affecting the structure or regional organization of chromosomes
\item Gene: base pair substitutions, insertions, and deletions
\end{itemize}
\end{itemize}

\begin{figure}[htbp]
\centering
\includegraphics[width=0.9\textwidth]{./figures/ch4_mutation.png}
\caption{\label{fig:org6e49342}
Mutation}
\end{figure}

\section{{\bfseries\sffamily TODO} Chapter 5: Cytogenetics}
\label{sec:org45fa635}
\section{{\bfseries\sffamily TODO} Chapter 6: Chromosomal and Genomic basis of Disease}
\label{sec:orgc20c9fa}
\section{{\bfseries\sffamily TODO} Chapter 7: Patterns of Single Gene Inheritance}
\label{sec:orgdab3e65}
\section{{\bfseries\sffamily TODO} Chapter 8: Complex Traits}
\label{sec:orgbcf4199}
\section{{\bfseries\sffamily TODO} Chapter 9: Genetic Variation in Populations}
\label{sec:org6244fd4}
\section{{\bfseries\sffamily TODO} Chapter 10: Identifying the Genetic Basis for Human Disease}
\label{sec:org65e63b3}
\section{{\bfseries\sffamily TODO} Chapter 11: The Molecular Basis of Genetic Disease}
\label{sec:org2df53e7}
\section{{\bfseries\sffamily TODO} Chapter 12: Molecular, Biochemical, and Cellular Basis of Genetic Disease}
\label{sec:org36b72cf}
\section{{\bfseries\sffamily TODO} Chapter 13: Treatment}
\label{sec:org463fff9}
\section{{\bfseries\sffamily TODO} Chapter 14: Developmental Genetics}
\label{sec:org7aaf750}
\section{{\bfseries\sffamily TODO} Chapter 15: Cancer}
\label{sec:org474464b}
\section{{\bfseries\sffamily TODO} Chapter 16: Risk}
\label{sec:org72a10d2}
\subsection{X linked}
\label{sec:org8f05ecd}
\begin{enumerate}
\item Transmission
\label{sec:org614b889}
\begin{itemize}
\item Daughters always inherit Dad's X
\item Moms transmit one X
\item Sons have phenotype
\item Daughters are carriers
\end{itemize}

\item Prior probability that female is a carrier of X linked lethal disorder
\label{sec:org4108b14}
\begin{itemize}
\item H is the population frequency of female carriers
\item There are there ways a female can be a carrier:
\begin{enumerate}
\item inherits from a carrier mother
\begin{itemize}
\item 1/2  H
\end{itemize}
\item receives a new mutant allele on X from mom
\begin{itemize}
\item \(\mu\)
\end{itemize}
\item receives a new mutant allele on X from dad
\begin{itemize}
\item \(\mu\)
\end{itemize}
\end{enumerate}
\end{itemize}
\[H = (1/2 \cdot H) + \mu + \mu \]
\[H = H/2 +2\mu \]
\[H = 4\mu \]

\begin{itemize}
\item The incidence of carrier females in next generation (C\(_{\text{n+1}}\)) will
be 1/2 the previous generation (C\(_{\text{n}}\)) plus the mutation rate in
females (\(\mu\)) plus the mutation rate in males (\(\nu\))
\end{itemize}

\[C_{n+1} = 1/2 \cdot C_n + \mu + \nu \]

\begin{itemize}
\item same as above
\end{itemize}

\begin{center}
\begin{tabular}{lll}
I-2 & C & NC\\
\hline
prior & 4\(\mu\) & 1\\
cond \footnotemark & 1/2 & \(\mu\)\\
joint & 2\(\mu\) & \(\mu\)\\
post & 2/3 & 1/3\\
\end{tabular}
\end{center}\footnotetext[1]{\label{org149dcef}Prop of affected son}


\begin{itemize}
\item 2/3 inherited from mom
\item 1/3 \emph{de novo}
\end{itemize}
\end{enumerate}
\subsection{AR}
\label{sec:org14ec1a6}
\begin{enumerate}
\item Carrier risk for unaffected sibs of patient w AR disease
\label{sec:org207f10d}
\begin{itemize}
\item 2/3
\end{itemize}
\end{enumerate}

\section{{\bfseries\sffamily TODO} Chapter 17: Prenatal}
\label{sec:org430a4f2}
\section{{\bfseries\sffamily TODO} Chapter 18: Application}
\label{sec:orgeb977ab}
\section{{\bfseries\sffamily TODO} Chapter 19: Ethics}
\label{sec:orgb34a9cd}
\end{document}