% Created 2020-01-09 Thu 17:05
% Intended LaTeX compiler: pdflatex
\documentclass{scrartcl}
\usepackage[utf8]{inputenc}
\usepackage[T1]{fontenc}
\usepackage{graphicx}
\usepackage{grffile}
\usepackage{longtable}
\usepackage{wrapfig}
\usepackage{rotating}
\usepackage[normalem]{ulem}
\usepackage{amsmath}
\usepackage{textcomp}
\usepackage{amssymb}
\usepackage{capt-of}
\usepackage{hyperref}
\hypersetup{colorlinks,linkcolor=black,urlcolor=blue}
\usepackage{textpos}
\usepackage{textgreek}
\usepackage[version=4]{mhchem}
\usepackage{chemfig}
\usepackage{siunitx}
\usepackage{gensymb}
\usepackage[usenames,dvipsnames]{xcolor}
\usepackage[T1]{fontenc}
\usepackage{lmodern}
\usepackage{verbatim}
\usepackage{tikz}
\usepackage{wasysym}
\usetikzlibrary{shapes.geometric,arrows,decorations.pathmorphing,backgrounds,positioning,fit,petri}
\usepackage{fancyhdr}
\pagestyle{fancy}
\author{Matthew Henderson, PhD, FCACB}
\date{\today}
\title{Thompson \& Thompson}
\hypersetup{
 pdfauthor={Matthew Henderson, PhD, FCACB},
 pdftitle={Thompson \& Thompson},
 pdfkeywords={},
 pdfsubject={},
 pdfcreator={Emacs 26.1 (Org mode 9.1.9)}, 
 pdflang={English}}
\begin{document}

\maketitle
\setcounter{tocdepth}{1}
\tableofcontents


\section{Chapter 2: Human Genome}
\label{sec:orge25c9a6}
\subsection{Mitosis}
\label{sec:org6b0a1d3}
\begin{description}
\item[{Prophase}] condensation of the chromosomes
\begin{itemize}
\item formation of the mitotic spindle,
\item formation of a pair of centrosomes, from which microtubules
radiate and eventually take up positions at the poles of the cell.
\end{itemize}
\item[{Prometaphase}] nuclear membrane dissolves
\begin{itemize}
\item chromosomes disperse within the cell
\item attach, by their kinetochores, to microtubules of the mitotic
spindle.
\end{itemize}
\item[{Metaphase}] the chromosomes condensed and line up at the equatorial
plane.
\item[{Anaphase}] chromosomes separate at the centromere
\begin{itemize}
\item sister chromatids of each chromosome now become independent
daughter chromosomes, which move to opposite poles of the cell.
\end{itemize}
\item[{Telophase}] chromosomes begin to decondense
\begin{itemize}
\item nuclear membrane begins to re-form around each of the two daughter
nuclei
\item cytoplasm cleaves = cytokinesis
\end{itemize}
\end{description}

\begin{figure}[htbp]
\centering
\includegraphics[width=0.9\textwidth]{./figures/ch2_mitosis.png}
\caption{\label{fig:orga08d130}
Mitosis}
\end{figure}


\begin{figure}[htbp]
\centering
\includegraphics[width=0.9\textwidth]{./figures/ch2_karyotype.png}
\caption{\label{fig:orga4e631c}
Karyotype}
\end{figure}

\subsection{Meiosis}
\label{sec:orgcdcb864}

\begin{itemize}
\item Meiosis I is also known as the reduction division because it is the
division in which the chromosome number is reduced by half through
the pairing of homologues in prophase and by their segregation to
different cells at anaphase of meiosis I.
\item the stage at which genetic recombination (also called meiotic crossing over) occurs.
\begin{itemize}
\item homologous segments of DNA are exchanged between non-sister
chromatids of each pair of homologous chromosomes, thus ensuring
that none of the gametes produced by meiosis will be identical to
another.
\end{itemize}
\item Theconceptual and practical consequences of recombination for many
\end{itemize}
aspects of human gen

\begin{figure}[htbp]
\centering
\includegraphics[width=0.9\textwidth]{./figures/ch2_meiosis.png}
\caption{\label{fig:org7adc6d2}
Meiosis}
\end{figure}


\subsection{Homologous Recombination}
\label{sec:orgc45615c}

\begin{itemize}
\item the genetic content of each gamete is unique, because of random
assortment of the parental chromosomes to shuffle the combination of
sequence variants between chromosomes and because of homologous
recombination to shuffle the combination of sequence variants within
each and every chromosome.
\item This has significant consequences for patterns of genomic variation
among and between different populations around the globe and for
diagnosis and counseling of many common conditions with complex patterns of inheritance.
\item The amounts and patterns of meiotic recombination are determined by
sequence variants in specific genes and at specific “hot spots” and
differ between individuals, between the sexes, between families, and
between populations
\item Because recombination involves the physical intertwining of the two
homologues until the appropriate point during meiosis I, it is also
critical for ensuring proper chromosome segregation during
meiosis.
\item Failure to recombine properly can lead to chromosome missegregation
(nondisjunction) in meiosis I and is a frequent cause of pregnancy
loss and of chromosome abnormalities like Down syndrome
\item Although homologous recombination is normally precise, areas of
repetitive DNA in the genome and genes of variable copy number in
the population are prone to occasional unequal crossing over during
meiosis, leading to variations in clinically relevant traits such as
drug response, to common disorders such as the thalassemias or
autism, or to abnormalities of sexual differentiation
\item Although homologous recombination is a normal and essential part of
meiosis, it also occurs, albeit more rarely, in somatic
cells.
\item Anomalies in somatic recombination are one of the causes of genome
instability in cancer
\end{itemize}

\section{Chapter 3: Gene Structure and Function}
\label{sec:org92aa602}
\subsection{Allelic Imbalance in Gene Expression}
\label{sec:org6e40ce0}
\begin{itemize}
\item Monoallelic gene expression
\begin{itemize}
\item Somatic rearrangement - T-cell receptors
\item random monoallelic expression
\item parent-of-origin imprinting
\end{itemize}
\item X chromosome inactivation
\begin{itemize}
\item random, X inactivation center
\item ncRNA called XIST
\end{itemize}
\end{itemize}

\section{Chapter 4: Human Genetic Diversity}
\label{sec:orga01d949}
\subsection{Inherited variation and polymorphism}
\label{sec:org6a715d2}
\begin{itemize}
\item SNP
\item Indels
\item microsatelite
\item mobile element insertion polymorphism
\begin{itemize}
\item retrotranspostion: Alu, LINE
\end{itemize}
\item CNVs
\begin{itemize}
\item related to indels and microsatellites but variation in
the number of copies of larger segments of the genome
\item 1000 bp to many hundreds of kilobase pairs.
\end{itemize}
\item Inversion polymorphism
\begin{itemize}
\item few base pairs up to several megabase pairs
\item can be present in either of two orientations in the genomes of different individuals
\end{itemize}
\end{itemize}
\begin{figure}[htbp]
\centering
\includegraphics[width=0.9\textwidth]{./figures/ch4_polymorphism.png}
\caption{\label{fig:orgf5611cb}
Polymorphism}
\end{figure}


\subsection{Origins and Freq of Mutation types}
\label{sec:org5138cea}
\begin{itemize}
\item germline
\item somatic
\item Mutations:
\begin{itemize}
\item Chromosome number
\item Regional: affecting the structure or regional organization of chromosomes
\item Gene: base pair substitutions, insertions, and deletions
\end{itemize}
\end{itemize}

\begin{figure}[htbp]
\centering
\includegraphics[width=0.9\textwidth]{./figures/ch4_mutation.png}
\caption{\label{fig:org35eedea}
Mutation}
\end{figure}

\section{Chapter 5: Cytogenetics}
\label{sec:org93075d1}
\subsection{Clinical Indications for Chromosome and Genome Analysis}
\label{sec:orgf10e070}
\begin{enumerate}
\item Problems of early growth and development
\label{sec:org9e826e0}
\begin{itemize}
\item failure to thrive, developmental delay,dysmorphic facies, multiple
malformations, short stature, ambiguous genitalia, and
intellectual disability are frequent findings in children with
chromosome abnormalities
\end{itemize}
\item Stillbirth and neonatal death
\label{sec:org23015fc}
\begin{itemize}
\item incidence of chromosome abnormalities is much higher among
stillbirths (\(\sim\)10\%) than among live births (\(\sim\)0.7\%)
\item also elevated among infants who die in the neonatal period (\textasciitilde{}10\%)
\item karyotyping (or other comprehensive ways of scanning the genome) is
essential for accurate genetic counseling
\end{itemize}
\item Fertility problems
\label{sec:orgb4336d4}
\begin{itemize}
\item chromosome studies are indicated for women presenting with
amenorrhea and for couples with a history of infertility or recurrent miscarriage
\item chromosome abnormality is seen in one or the other parent in 3\% to
6\% of cases in which there is infertility or two or more
miscarriages
\end{itemize}
\item Family history
\label{sec:org76c7b93}
\begin{itemize}
\item known or suspected chromosome or genome abnormality in a first
degree relative is an indication for chromosome and genome analysis
\end{itemize}
\item Neoplasia
\label{sec:org7b7eb49}
\begin{itemize}
\item virtually all cancers are associated with one or more chromosome
abnormalities
\item chromosome and genome evaluation in the tumor itself, or in bone
marrow in the case of hematological malignant neoplasms, can offer
diagnostic or prognostic information
\end{itemize}
\item Pregnancy
\label{sec:org0939088}
\begin{itemize}
\item a higher risk for chromosome abnormality in fetuses conceived by
women of increased age, typically defined as \textgreater{} 35 years
\item fetal chromosome and genome analysis should be offered as a routine
part of prenatal care in such pregnancies
\item NIPT is a screening approach for the most common chromosome
disorders and is now available to pregnant women of all ages
\end{itemize}
\end{enumerate}

\subsection{Chromosome Identification}
\label{sec:org9c1e978}
\begin{itemize}
\item G-banding (Giemsa) is the gold standard for detection and
characterization of structural and numerical genomic abnormalities
\begin{itemize}
\item both constitutional (postnatal or prenatal) and acquired (cancer)
\item detection of deletions and duplications \textgreater{} 5 to 10 Mb
\end{itemize}
\end{itemize}
than approximately 5 to 10 Mb anywhere in the genome
\begin{itemize}
\item three types of chromosomes:
\begin{description}
\item[{metacentric}] central centromere
\item[{submetacentric}] off center centromere
\item[{acrocentric}] centromere at one end
\begin{itemize}
\item 13,14,15,21,22
\end{itemize}
\end{description}
\end{itemize}

\subsection{Fluorescence In Situ Hybridization}
\label{sec:org8a879e6}
\begin{itemize}
\item detecting the presence or absence of a particular DNA sequence or
for evaluating the number or organization of a chromosome or
chromosomal region \emph{in situ}
\item uses ordered collections of recombinant DNA clones containing DNA
from around the entire genome
\item limited by the need to target a specific genomic region based
\end{itemize}
on a clinical diagnosis or suspicion

\subsection{Microarrays}
\label{sec:orgc6ad65b}
\begin{itemize}
\item comparative genome hybridization (CGH)
\begin{itemize}
\item detects relative copy number gains and losses genome-wide by
hybridizing two samples:
\begin{itemize}
\item control genome
\item patient
\end{itemize}
\item excess of sequences from one or the other genome indicates an
overrepresentation or underrepresentation of those sequences in the
patient genome relative to the control
\end{itemize}
\item SNP arrays
\begin{itemize}
\item relative representation and intensity of alleles in different
regions of the genome indicate if a chromosome or chromosomal
region is present at the appropriate dosage
\end{itemize}
\item probe spacing provides a resolution as high as 250 kb
\end{itemize}
\subsection{Chromosome Abnormalities}
\label{sec:org9571d28}
\begin{itemize}
\item numerical or structural
\item incidence of 1/154 live births
\item aneuploidy is most common
\begin{itemize}
\item associated with physical and/or mental abnormalities
\end{itemize}
\item structural abnormalities are also common
\end{itemize}

\begin{figure}[htbp]
\centering
\includegraphics[width=0.9\textwidth]{./tnt/figures/ch5_freq.png}
\caption{\label{fig:orga3b2c8a}
Incidence of chromosomal abnormalities}
\end{figure}

\begin{figure}[htbp]
\centering
\includegraphics[width=0.9\textwidth]{./figures/ch5_nom.png}
\caption{\label{fig:orgb191e25}
ISCN for common cytogenetic aberration}
\end{figure}

\begin{figure}[htbp]
\centering
\includegraphics[width=0.9\textwidth]{./figures/ch5_nom2.png}
\caption{\label{fig:org9d1f76e}
ISCN (continued)}
\end{figure}

\subsection{Gene Dosage, Balance and Imbalance}
\label{sec:orgc58b0b5}
\begin{itemize}
\item for chromosome and genomic disorders, it is the quantitative aspects
of gene expression that underlie disease, in contrast to single-gene
disorders, in which pathogenesis often reflects qualitative aspects
of a gene's function
\end{itemize}
\begin{enumerate}
\item monosomies are more deleterious than trisomies
\label{sec:org3bb34c7}
\begin{itemize}
\item complete monosomies are generally not viable, except for monosomy
for the X chromosome
\item complete trisomies are viable for chromosomes 13, 18, 21, X, and Y
\end{itemize}

\item phenotype in partial aneuploidy depends on a number of factors
\label{sec:orga76f7d9}
\begin{itemize}
\item size of the unbalanced segment
\item which regions of the genome are affected
\item which genes are involved
\item whether the imbalance is monosomic or trisomic
\end{itemize}
\item risk in cases of inversions depends on the location of the inversion with respect to the centromere and on the size of the inverted segment
\label{sec:org1a5d8f4}
\begin{itemize}
\item paracentric inversions do not involve the centromere
\begin{itemize}
\item very low risk for an abnormal phenotype in the next generation
\end{itemize}
\item pericentric inversions do involve the centromere
\begin{itemize}
\item risk for birth defects in offspring may be significant and
increases with the size of the inverted segment
\end{itemize}
\end{itemize}

\item mosaic karyotype involving any chromosome abnormality, all bets are off!
\label{sec:org6556891}
\begin{itemize}
\item the degree of mosaicism in relevant tissues or relevant stages of
development is generally unknown
\item there is uncertainty about the severity of the phenotype
\end{itemize}
\end{enumerate}

\subsection{Abnormalities of Chromosome Number}
\label{sec:org79d053f}
\begin{description}
\item[{heteroploid}] chromosome complement other than 46 is
\item[{euploid}] exact multiple of n
\item[{aneuploid}] non-multiple of n
\end{description}

\begin{enumerate}
\item Triploidy and tetraploidy
\label{sec:orgfd184c5}
\begin{itemize}
\item most result from fertilization of an egg by two sperm (dispermy)
\item also failure of one of the meiotic divisions in either sex,
resulting in a diploid egg or sperm
\item maternal source are aborted
\item paternal source \(\to\) degenerative placenta (parital hydatidiform
mole) w small fetus
\end{itemize}

\item Aneuploidy
\label{sec:orga0bb243}
\begin{itemize}
\item most common cause is meiotic nondisjunction in meiosis I or II (Figure \ref{fig:orgf315aa4})
\begin{description}
\item[{trisomy}] 21,18,13
\item[{monosomy}] X (Turner syndrome)
\end{description}
\end{itemize}

\begin{figure}[htbp]
\centering
\includegraphics[width=0.9\textwidth]{./figures/ch5_nondys.png}
\caption{\label{fig:orgf315aa4}
Nondisjunction}
\end{figure}
\end{enumerate}

\subsection{Abnormalities of Chromosome Structure}
\label{sec:org7ed4eee}
\begin{itemize}
\item present in 1/375 newborns
\item balances or unbalanced
\end{itemize}

\begin{figure}[htbp]
\centering
\includegraphics[width=0.9\textwidth]{./figures/ch5_struct.png}
\caption{\label{fig:orgd1d302f}
Structural rearrangements of chromosomes}
\end{figure}

\begin{enumerate}
\item Unbalanced rearrangements
\label{sec:org53c11b7}
\begin{itemize}
\item Delections and Duplications
\item Marker and Ring Chromosomes
\begin{itemize}
\item very small, unidentified chromosomes
\end{itemize}
\item Isochromosomes
\begin{itemize}
\item one arm is missing and the other duplicated in a mirror-image
\end{itemize}
\item Dicentric
\begin{itemize}
\item two chromosome segments, each with a centromere, fuse end to end
\end{itemize}
\end{itemize}
\item Balanced rearrangements
\label{sec:orgf98dd32}
\begin{itemize}
\item "balanced" depends on resolution
\item Translocations
\begin{itemize}
\item Reciprocal translocations
\item Robertsonian translocations
\item Insertions
\end{itemize}
\item Inversions
\begin{itemize}
\item paracentric - outside the centromere
\item pericentric - includes the centromere
\end{itemize}
\end{itemize}

\begin{figure}[htbp]
\centering
\includegraphics[width=0.9\textwidth]{./figures/ch5_trans.png}
\caption{\label{fig:orgcd49cac}
Balanced translocations}
\end{figure}
\end{enumerate}

\section{{\bfseries\sffamily STARTED} Chapter 6: Chromosomal and Genomic basis of Disease}
\label{sec:org0fbadb8}

\begin{itemize}
\item Disorders due to:
\begin{itemize}
\item abnormal chromosome segregation (nondisjunction)
\item recurrent chromosomal syndromes, involving
deletions or duplications at genomic hot spots
\item idiopathic chromosomal abnormalities, typically de novo
\item unbalanced familial chromosomal abnormalities
\item chromosomal and genomic events that reveal regions
of genomic imprinting
\end{itemize}
\end{itemize}

\begin{figure}[htbp]
\centering
\includegraphics[width=0.9\textwidth]{./figures/ch6_mech.png}
\caption{\label{fig:orge4ebca5}
Mechanisms of chromosome abnormalities and genomic imbalance}
\end{figure}

\subsection{Lessons From Genomic Disorders}
\label{sec:org4bc28a6}
\begin{itemize}
\item altered gene dosage for any extensive chromosomal or genomic region
is likely to result in a clinical abnormality, the phenotype of
which will, in principle, reflect haploinsufficiency for or
overexpression of one or more genes encoded within the region.
\begin{itemize}
\item in some cases, the clinical presentation appears to be accounted
for by dosage imbalance for just a single gene; in other
syndromes, however, the phenotype appears to reflect imbalance for
multiple genes across the region
\end{itemize}
\item the distribution of these duplication/deletion disorders is not random,
\begin{itemize}
\item segmental duplications in pericentromeric and subtelomeric
regions, predisposes particular regions to the unequal
recombination events that underlie these syndromes
\end{itemize}
\item patients carrying what appears to be the same chromosomal deletion
or duplication can present with a range of variable phenotypes
\end{itemize}

\subsection{Aneuploidy}
\label{sec:org9ae1599}
\begin{itemize}
\item common mutation in our species involves errors in chromosome segregation
\item only three well-defined nonmosaic chromosome disorders compatible
with postnatal survival in which there is an abnormal dose of an
entire autosome:
\begin{enumerate}
\item trisomy 21 (Down syndrome)
\item trisomy 18
\item trisomy 13
\end{enumerate}
\item also have the \(\downarrow\) genes among autosomes
\end{itemize}

\begin{enumerate}
\item Down Syndrome
\label{sec:orgc88aff6}
\begin{itemize}
\item see conditions
\end{itemize}
\begin{enumerate}
\item Robertsonian Translocation
\label{sec:org439d98b}
\begin{itemize}
\item \(\sim\)4\% of Down syndrome patients have 46 chromosomes
\item one of which is a Robertsonian translocation between chromosome
21q and the long arm of one of the other acrocentric chromosomes
(usually chromosome 14 or 22)
\item 46,XX or XY,rob(14;21)(q10;q10),+21
\end{itemize}

\begin{figure}[htbp]
\centering
\includegraphics[width=0.9\textwidth]{./figures/ch6_rtgam.png}
\caption{\label{fig:org326b091}
Chromosomes of gametes that theoretically can be produced by a carrier of a Robertsonian translocation, rob(14;21)}
\end{figure}
\end{enumerate}

\item Uniparental Disomy
\label{sec:org5f88841}
\begin{itemize}
\item nondisjunction \(\to\) both copies of a chromosome derive from the same
parent
\begin{itemize}
\item called uniparental disomy
\item defined as the presence of a disomic cell line containing two
chromosomes, or portions thereof, that are inherited from only one
parent
\end{itemize}
\item isodisomy if  two chromosomes are derived from identical sister chromatids
\item heterodisomy if if both homologues from one parent are present
\item common explanation for uniparental disomy is trisomy “rescue” due to
chromosome nondisjunction in cells of a trisomic conceptus to
restore a disomic state
\end{itemize}

\item Contiguous Gene Syndrome
\label{sec:orgb3f6714}
\begin{itemize}
\item segmental aneusomy is a form of genetic imbalance due to recurrent
subchromosomal or regional abnormalities
\begin{itemize}
\item typically detected by microarray
\item called contiguous gene syndrome
\end{itemize}
\end{itemize}
\end{enumerate}

\subsection{Idiopathic Chromosome Abnormalities}
\label{sec:org69db778}
\begin{itemize}
\item Autosomal deletion syndromes
\begin{itemize}
\item cri du chat syndrome, there is either a terminal or interstitial
deletion of part of the short arm of chromosome 5
\end{itemize}
\item Balanced translocations with developmental phenotypes
\end{itemize}

\subsection{Disorders Associated with Genomic Imprinting}
\label{sec:org90406a4}
\begin{itemize}
\item Prader-Willi
\item Angelman syndrome
\item Beckwith-Wiedemann syndrome
\end{itemize}

\begin{figure}[htbp]
\centering
\includegraphics[width=0.9\textwidth]{./figures/ch6_pw_as.png}
\caption{\label{fig:org0e9e868}
Mechanism causing Prader-Willi and Angelman Syndrome}
\end{figure}

\section{{\bfseries\sffamily TODO} Chapter 7: Patterns of Single Gene Inheritance}
\label{sec:org59ceb4c}
\section{{\bfseries\sffamily TODO} Chapter 8: Complex Traits}
\label{sec:orgb942b72}
\section{{\bfseries\sffamily TODO} Chapter 9: Genetic Variation in Populations}
\label{sec:orge7810b0}
\section{{\bfseries\sffamily TODO} Chapter 10: Identifying the Genetic Basis for Human Disease}
\label{sec:org3146a58}
\section{{\bfseries\sffamily TODO} Chapter 11: The Molecular Basis of Genetic Disease}
\label{sec:org8e19e33}
\section{{\bfseries\sffamily TODO} Chapter 12: Molecular, Biochemical, and Cellular Basis of Genetic Disease}
\label{sec:orgfd89473}
\section{{\bfseries\sffamily TODO} Chapter 13: Treatment}
\label{sec:orga82a1c1}
\section{{\bfseries\sffamily TODO} Chapter 14: Developmental Genetics}
\label{sec:org8fecf83}
\section{Chapter 15: Cancer}
\label{sec:org62bd80e}
\begin{itemize}
\item cancer is not a single disease but rather comes in many forms and
degrees of malignancy
\item there are three main classes of cancer:
\begin{description}
\item[{sarcomas}] tumor has arisen in mesenchymal tissue, such as bone,
muscle, or connective tissue, or in nervous system
tissue
\item[{carcinomas}] originate in epithelial tissue, such as the cells
lining the intestine, bronchi, or mammary ducts
\item[{hematopoietic and lymphoid}] leukemia and lymphoma which spread
throughout the bone marrow, lymphatic system, and peripheral
blood
\end{description}
\end{itemize}
\subsection{Genetic Basis of Cancer}
\label{sec:org3f3f1df}
\begin{itemize}
\item driver and passenger mutations
\item particular translocations are sometimes highly specific for certain
types of cancer and involve specific genes
\begin{itemize}
\item BCR-ABL translocation in chronic myelogenous leukemia
\end{itemize}
\item cancers can also show complex rearrangements in which chromosomes
break into numerous pieces and rejoin, forming novel and complex
combinations (a process known as “chromosome shattering”)
\item large genomic alterations involving many kilobases of DNA can form
the basis for loss of function or increased function of one or more
driver genes

\item genes in which mutations cause cancer are referred to as \textbf{driver}
\textbf{genes}, and the cancer causing mutations in these genes are \textbf{driver}
\textbf{mutations}
\item driver genes fall into two distinct categories
\begin{enumerate}
\item activated oncogenes
\item tumor suppressor genes (TSGs)
\end{enumerate}

\item an activated oncogene is a mutant allele of a proto-oncogene, a
class of normal cellular protein-coding genes that promotes growth
and survival of cells
\item oncogenes facilitate malignant transformation by stimulating
proliferation or inhibiting apoptosis
\item oncogenes encode proteins such as the following:
\begin{itemize}
\item proteins in signaling pathways for cell proliferation
\item transcription factors that control the expression of growth-promoting genes
\item inhibitors of programmed cell death machinery
\end{itemize}
\item A TSG is a gene in which loss of function through mutation or
epigenomic silencing directly removes normal regulatory controls on
cell growth or leads indirectly to such losses through an increased
mutation rate or aberrant gene expression
\item TSGs encode proteins involved in many aspects of cellular function including:
\begin{itemize}
\item maintenance of correct chromosome number and structure
\item DNA repair proteins
\item proteins involved in regulating the cell cycle, cellular
proliferation or contact inhibition
\end{itemize}

\item tumor initiation can be caused by different types of genetic
alterations:
\begin{itemize}
\item activating or gain-of-function mutations
\item ectopic and heterochronic mutations of protooncogenes
\item chromosome translocations that cause misexpression of genes or chimeric genes
\item LOF of both alleles, or a dominant negative mutation of one allele, of TSGs
\end{itemize}

\item tumor progression occurs as a result of accumulating additional
genetic damage,through mutations or epigenetic silencing, of driver
genes that encode the machinery that repairs damaged DNA and
maintains cytogenetic normality
\end{itemize}

\subsection{Cancer in Families}
\label{sec:orgb61d73a}

\begin{itemize}
\item germline mutation - inherited
\item second-hit
\begin{itemize}
\item somatic mutation
\item loss of heterozygozity around locus
\end{itemize}
\end{itemize}


\subsection{Sporadic Cancer}
\label{sec:org777688a}
\begin{itemize}
\item activation of oncogenes by point mutation
\item activation of oncogenes by chromosome translocation
\begin{itemize}
\item best-known example is the translocation between chromosomes 9 and
22 (Philadelphia chromosome) that is seen in CML
\item moves the protooncogene ABL1, a tyrosine kinase, from its normal
position on chromosome 9q to a gene of unknown function, BCR, on
chromosome 22q
\item results in the synthesis of a novel, chimeric protein, BCR-ABL1 w
enhanced tyrosine kinase activity
\end{itemize}
\end{itemize}


\begin{figure}[htbp]
\centering
\includegraphics[width=0.9\textwidth]{./figures/ch15_neo.png}
\caption{\label{fig:orgcb105af}
Characteristic Chromosome Translocations in Selected Human Malignant Neoplasms}
\end{figure}

\subsection{Cytogenetic Changes in Cancer}
\label{sec:orgfcb589b}

\begin{enumerate}
\item Aneuploidy and Aneusomy
\label{sec:orgf89ad37}
\begin{itemize}
\item when CML, with the 9;22 Philadelphia chromosome, evolves from the
typically indolent chronic phase to a severe, life-threatening blast
crisis, there may be several additional cytogenetic abnormalities,
including numerical or structural changes, such as a second copy of
the 9;22 translocation chromosome or an isochromosome for 17q

\item another cytogenetic aberration seen in many cancers is gene
amplification, a phenomenon in which many additional copies of a
segment of the genome are present in the cell
\begin{itemize}
\item \textbf{double minutes} (very small accessory chromosomes)
\item \textbf{homogeneously staining regions} that do not band normally and
contain multiple, amplified copies of a particular DNA segment
\end{itemize}
\item known to include extra copies of proto-oncogenes such as the genes
encoding Myc, Ras, and epithelial growth factor receptor, which
stimulate cell growth, block apoptosis, or both
\end{itemize}
\end{enumerate}

\subsection{Targeted Cancer Therapy}
\label{sec:orgdfcfc2a}
\begin{itemize}
\item The proof of principle for this approach was established with the
development of imatinib, a highly effective inhibitor of a number of
tyrosine kinases, including the ABL1 kinase in CML
\item Prolonged remissions of this disease have been seen, in some cases
with apparently indefinite postponement of the transformation into a
virulent acute leukemia (blast crisis) that so often meant the end
of a CML patient’s life
\item Additional kinase inhibitors have been developed to target other
activated oncogene driver genes in a variety of tumor types
\item initial results with targeted therapies, although very promising in
some cases, have not led to permanent cures in most patients because
tumors develop resistance to the targeted therapy
\item The outgrowth of resistant tumors because cancer cells are highly
mutable, and their genomes undergo recurrent mutation
\end{itemize}


\begin{figure}[htbp]
\centering
\includegraphics[width=0.9\textwidth]{./figures/ch15_target.png}
\caption{\label{fig:org260aba9}
Targeted cancer treatment}
\end{figure}

\section{{\bfseries\sffamily TODO} Chapter 16: Risk}
\label{sec:org80c61cb}
\subsection{X linked}
\label{sec:orga9f7c53}
\begin{enumerate}
\item Transmission
\label{sec:orgced7844}
\begin{itemize}
\item Daughters always inherit Dad's X
\item Moms transmit one X
\item Sons have phenotype
\item Daughters are carriers
\end{itemize}

\item Prior probability that female is a carrier of X linked lethal disorder
\label{sec:org4c8ccd3}
\begin{itemize}
\item H is the population frequency of female carriers
\item There are there ways a female can be a carrier:
\begin{enumerate}
\item inherits from a carrier mother
\begin{itemize}
\item 1/2  H
\end{itemize}
\item receives a new mutant allele on X from mom
\begin{itemize}
\item \(\mu\)
\end{itemize}
\item receives a new mutant allele on X from dad
\begin{itemize}
\item \(\mu\)
\end{itemize}
\end{enumerate}
\end{itemize}
\[H = (1/2 \cdot H) + \mu + \mu \]
\[H = H/2 +2\mu \]
\[H = 4\mu \]

\begin{itemize}
\item The incidence of carrier females in next generation (C\(_{\text{n+1}}\)) will
be 1/2 the previous generation (C\(_{\text{n}}\)) plus the mutation rate in
females (\(\mu\)) plus the mutation rate in males (\(\nu\))
\end{itemize}

\[C_{n+1} = 1/2 \cdot C_n + \mu + \nu \]

\begin{itemize}
\item same as above
\end{itemize}

\begin{center}
\begin{tabular}{lll}
I-2 & C & NC\\
\hline
prior & 4\(\mu\) & 1\\
cond \footnotemark & 1/2 & \(\mu\)\\
joint & 2\(\mu\) & \(\mu\)\\
post & 2/3 & 1/3\\
\end{tabular}
\end{center}\footnotetext[1]{\label{orgd0c8f0d}Prop of affected son}


\begin{itemize}
\item 2/3 inherited from mom
\item 1/3 \emph{de novo}
\end{itemize}
\end{enumerate}
\subsection{AR}
\label{sec:org6d102a5}
\begin{enumerate}
\item Carrier risk for unaffected sibs of patient w AR disease
\label{sec:org3eabb51}
\begin{itemize}
\item 2/3
\end{itemize}
\end{enumerate}

\section{{\bfseries\sffamily TODO} Chapter 17: Prenatal}
\label{sec:orgfd9a2b2}
\section{{\bfseries\sffamily TODO} Chapter 18: Application}
\label{sec:orgb740f49}
\section{{\bfseries\sffamily TODO} Chapter 19: Ethics}
\label{sec:org7eb5a26}




\section{1/3 Rule}
\label{sec:org5760a84}
If the reproductive fitness of a male affected with an X-linked
recessive disorder is low or nil, then in a population \textbf{one-third of}
\textbf{all affected X chromosomes will be removed from the gene pool every
generation}. If the incidence of the disease is constant, then
one-third of cases must be due to mutations arising \emph{de novo} in a
family.


\[
 p^2 + 2pq + q^2 = 1 
 \]
\[
 q^2 \sim 0
 \]

\begin{center}
\begin{tabular}{lll}
 & X & Y\\
\hline
X & XX & XY\\
\hline
x & xX & \textbf{xY}\\
\end{tabular}
\end{center}


\begin{enumerate}
\item The Haldane Hypothesis
\label{sec:orga8637d3}
\begin{itemize}
\item Applies to X-linked recessive traits
\begin{itemize}
\item A study of fertility rates in hemophillia
\end{itemize}

\item In a large population of 2N (N \male{} and N \female)
\item (1 - f)xN genes removed per generation
\begin{itemize}
\item x = proportion of affected males in the polulation
\item f = effective fertility
\end{itemize}

\item Each of N \female{} has 2X/cell, and each of N \male{} has 1X/cell
\item The mean mutation rate per X-chromosome per generation is: \footnote{Haldane JB. The rate of spontaneous mutation of a human gene. 1935. J Genet 2004;83:235-44.}
\end{itemize}

\[
 u = 1/3(1 - f)x  
 \]
\end{enumerate}
\end{document}