% Created 2019-11-18 Mon 08:59
% Intended LaTeX compiler: pdflatex
\documentclass{scrartcl}
\usepackage[utf8]{inputenc}
\usepackage[T1]{fontenc}
\usepackage{graphicx}
\usepackage{grffile}
\usepackage{longtable}
\usepackage{wrapfig}
\usepackage{rotating}
\usepackage[normalem]{ulem}
\usepackage{amsmath}
\usepackage{textcomp}
\usepackage{amssymb}
\usepackage{capt-of}
\usepackage{hyperref}
\hypersetup{colorlinks,linkcolor=black,urlcolor=blue}
\usepackage{textpos}
\usepackage{textgreek}
\usepackage[version=4]{mhchem}
\usepackage{chemfig}
\usepackage{siunitx}
\usepackage{gensymb}
\usepackage[usenames,dvipsnames]{xcolor}
\usepackage[T1]{fontenc}
\usepackage{lmodern}
\usepackage{verbatim}
\usepackage{tikz}
\usepackage{wasysym}
\usetikzlibrary{shapes.geometric,arrows,decorations.pathmorphing,backgrounds,positioning,fit,petri}
\usepackage{fancyhdr}
\pagestyle{fancy}
\author{Matthew Henderson, PhD, FCACB}
\date{\today}
\title{Methods}
\hypersetup{
 pdfauthor={Matthew Henderson, PhD, FCACB},
 pdftitle={Methods},
 pdfkeywords={},
 pdfsubject={},
 pdfcreator={Emacs 26.1 (Org mode 9.1.9)}, 
 pdflang={English}}
\begin{document}

\maketitle
\tableofcontents


\tikzstyle{chemical} = [rectangle, rounded corners, text width=5em, minimum height=1em,text centered, draw=black, fill=none]
\tikzstyle{hardware} = [rectangle, rounded corners, text width=5em, minimum height=1em,text centered, draw=black, fill=gray!30]
\tikzstyle{ms} = [rectangle, rounded corners, text width=5em, minimum height=1em,text centered, draw=orange, fill=none]
\tikzstyle{msw} = [rectangle, rounded corners, text width=7em, minimum height=1em,text centered, draw=orange, fill=none]
\tikzstyle{label} = [rectangle,text width=8em, minimum height=1em, text centered, draw=none, fill=none]
\tikzstyle{hl} = [rectangle, rounded corners, text width=5em, minimum height=1em,text centered, draw=black, fill=red!30]
\tikzstyle{box} = [rectangle, rounded corners, text width=5em, minimum height=5em,text centered, draw=black, fill=none]
\tikzstyle{arrow} = [thick,->,>=stealth]
\tikzstyle{hl-arrow} = [ultra thick,->,>=stealth,draw=red]


\section{Amino Acids}
\label{sec:org943bbdc}
\subsection{Introduction}
\label{sec:org0e506a8}
\begin{enumerate}
\item Amino Acids: A Very Short Introduction
\label{sec:org07f0e1a}
\begin{itemize}
\item Amino acids are mono or dicarboxylic acids with one or more amino groups.
\begin{itemize}
\item Zwitterion at ph 7.45
\end{itemize}

\item Proteinogenic amino acids (22)
\begin{itemize}
\item 21 amino acids naturally incorportated into polypeptides in humans
\item 20 genetically encoded
\item selenocysteine
\end{itemize}

\item Non-proteinogenic
\begin{itemize}
\item post-translational modification
\begin{itemize}
\item hydroxylation of proline \(\to\) hydroxyproline
\end{itemize}
\item Not found in proteins
\begin{itemize}
\item gamma-aminobutryic acid
\item ornithine, citrulline
\end{itemize}
\end{itemize}

\item 76 amino acids of biological interest in humans
\end{itemize}

\item Indications for Measurement of Amino Acids
\label{sec:orgfe82d25}
\begin{itemize}
\item Diagnosis of inborn errors of amino acid metabolism and transport
\item Diet monitoring in patients with known IEM
\item Nutritional assessment of patients with non-metabolic conditions [e.g. short bowel syndrome]
\item Signs and symptoms:
\begin{enumerate}
\item Lethargy, coma, seizures or vomiting in a neonate
\item Hyperammonaemia
\item Ketosis
\item Metabolic acidosis or lactic acidaemia
\item Alkalosis
\item Metabolic decompensation
\item Unexplained developmental delay or developmental regression
\item Polyuria, polydipsia and dehydration
\item Unexplained liver dysfunction
\item Unexplained neurological symptoms
\item Abnormal amino acid results on newborn screening programme
\item Previous sibling with similar clinical presentation
\item Clinical presentation specific to an amino acid disorder
\item Monitoring treatment and diet.
\end{enumerate}
\end{itemize}

\begin{table}[htbp]
\caption{\label{tab:org1e0a9a2}
Recommended Plasma AA Profile for Diagnosis of Amino Acid Disorders}
\centering
\begin{tabular}{lll}
Alanine & Glutamine & Ornithine\\
Alloisoleucine & Glycine & Phenylalanine\\
Arginine & Histidine & Proline\\
Argininosuccinic acid & Homocysteine \footnotemark & Serine\\
Cystine & Isoleucine & Sulphocysteine \footnotemark\\
Citrulline & Leucine & Taurine\\
Glutamic acid & Lysine & Threonine\\
Valine & Methionine & Tyrosine\\
\end{tabular}
\end{table}\footnotetext[1]{\label{org84acaf4}Plasma total homocysteine is not detected by routine methods, plasma free homocystine analysis has poor clinical sensitivity.}\footnotetext[2]{\label{orgc3de2fe}Sulphocysteine may not be detectable in plasma using routine methods}

\item Plasma
\label{sec:org6a26ffa}
\begin{itemize}
\item Patient prep
\begin{itemize}
\item Fasting (overnight preferred, 4 hours minimum). Infants and children should be
drawn just before next feeding (2-3 hours without TPN if possible).
\end{itemize}
\item Sample
\begin{itemize}
\item LiHeparin venous plasma
\item Prompt separation and deproteinisation is essential for accurate measurement of (free) sulphur containing amino acids.
\item Protein binding: cystine and homocystine
\item release of arginase from RBCs
\item store at -20\degree{}C to limit glutamine decomposition
\end{itemize}
\end{itemize}

\item Plasma - Pre-analytical
\label{sec:org7deb8af}
\begin{itemize}
\item Serum should not be used because there may be deamination (asparagine to
aspartic acid and glutamine to glutamic acid), loss of sulphur
containing amino acids and release of oligopeptides.
\item EDTA plasma is recommended in some labs as the specimen of
choice. The older literature reports ninhydrin positive artefacts in
EDTA plasma but modern tubes do not seem to have this problem.
\item Haemolysis will cause increases in serine, glycine, taurine,
phosphoethanolamine, aspartic acid, glutamic acid, ornithine and
decreased arginine.
\item Delayed separation or leucocyte and platelet contamination will
cause increased serine, glycine, taurine, phosphoethanolamine,
ornithine, glutamic acid and decreased arginine, homocystine,
cystine.
\item Phenylalanine and tyrosine increase if specimen separation is
delayed
\item Amino acids are more stable in frozen deproteinised plasma than
in frozen native plasma.
\item Capillary blood may be used with careful cleaning of the skin prior
to specimen collection provided the blood is flowing freely.
\item Free tryptophan may be lost when using sulphosalicylic acid as
deproteinising agent.
\begin{itemize}
\item trichloroacetic acid is the deproteinising agent of choice for
this amino acid.
\end{itemize}
\item Sodium metabisulphite, found in some intravenous preparations as a
preservative, can cause the conversion of cystine to sulphocysteine.
\end{itemize}

\item Urine
\label{sec:orgf6af4ee}
\begin{itemize}
\item Urine
\begin{itemize}
\item 24 hour or random urine
\item preservative free bottle
\item Specimen quality is checked by testing for nitrite and pH.
\begin{itemize}
\item Specimen deterioration causes:
\begin{itemize}
\item \(\downarrow\) serine
\item \(\uparrow\) \(\downarrow\) alanine
\item increased glycine
\item decarboxylation of glutamic acid \(\to\) \(\gamma\)-aminobutyric acid
\item breakdown of phosphoethanolamine \(\to\) ethanolamine and phosphate
\item breakdown of cystathionine \(\to\) homocystine
\item hydrolysis of peptides causing \(\uparrow\) proline
\end{itemize}
\end{itemize}
\item Fecal contamination causes increased proline, glutamic acid, branched chain amino acids but not hydroxyproline.
\item Faecal bacteria can produce \(\gamma\)-aminobutyric acid from glutamic acid and b-alanine from aspartic acid
\item reported in \textmu{}mol/g creatinine
\item Aminoaciduria due to overflow and amino acid transport defects.
\end{itemize}
\end{itemize}

\item Cerebral Spinal Fluid
\label{sec:orga696d89}
\begin{itemize}
\item free of blood contamination
\item which tube is used?
\item investigation of neurological disorders and essential for the
diagnosis of non-ketotic hyperglycinaemia.
\item CSF/Plasma ratio of amino acids is more informative than an isolated CSF sample.
\begin{itemize}
\item A paired plasma sample should be obtained within two hours.
\end{itemize}
\end{itemize}
\item Sample prep
\label{sec:orga0043af}
\begin{itemize}
\item Hydrophillic amino acids require deprotonization with acetonitrile or alcohol
\item Deproteination to release cysteine, homocysteine and tryptophan.
\end{itemize}
\end{enumerate}

\subsection{Amino Acid Analyser}
\label{sec:org2b8bf9d}
\begin{itemize}
\item Cation-exchange chromatography using a lithium buffer system
followed by post-column derivatization with ninhydrin
\item Samples are de-proteinized with sulfosalicylic acid prior to
injection
\item Utilizes a lithium-based cation-exchange column
\item Eluting amino acids undergo post column reaction with ninhydrin
\item Optical detection in the visible spectrum
\begin{itemize}
\item amino acids: 570nm
\item imino acids 440 nm
\end{itemize}
\end{itemize}

\begin{center}
\begin{tikzpicture}[node distance=9em]
% nodes
\node(column)[msw, right of=extraction]{Chromatography};
\node(derivatization)[msw, right of=column]{Ninhyrin Derivatization};
\node(detector)[ms, right of=derivatization]{UV Detector};
% arrows
\draw[arrow](column) -- (derivatization);
\draw[arrow](derivatization) -- (detector);
\end{tikzpicture}

\vspace{2em}

\schemedebug{false}
\schemestart
\chemfig[][scale=.33]{*6(=*5(-(=O)-(-[6]OH)(-[8]OH)-(=O)-)-=-=-)}
\+
\chemfig[][scale=.5]{{\color{red}R}-[::-60](<[::-60]NH_3^+)-[::60](=[::60]O)-[::-60]OH}
\arrow{-U>[][{\tiny \ce{2H2O}}]}
\chemfig[][scale=.33]{*6(=*5(-(=O)-(=N-[::-60](-[::-60]{\color{red}R})-[::60](=[::60]O)-[::-60]OH)-(=O)-)-=-=-)}
\arrow{->[][]}
\chemfig[][scale=.33]{*6(=*5(-(=O)-(-N=*5(-(=O)-(*6(-=-=-))--(=O)-))-(=O)-)-=-=-)}
\schemestop

\schemedebug{false}
\schemestart
\chemfig[][scale=.33]{*6(=*5(-(=O)-(-[6]OH)(-[8]OH)-(=O)-)-=-=-)}
\+
\chemfig[][scale=.33]{H-*5(N----(-COOH)-)}
\arrow{->[][]}
\chemfig[][scale=.33]{*6(=*5(-(=O)-(-*5(N-----))-(=O)-)-=-=-)}
\schemestop
\end{center}

\begin{figure}[htbp]
\centering
\includegraphics[width=0.5\textwidth]{./aa/figures/212.jpg}
\caption{\label{fig:org48aa3ff}
???}
\end{figure}

\begin{figure}[htbp]
\centering
\includegraphics[width=0.5\textwidth]{./figures/aaachrom.png}
\caption{\label{fig:orgb16ce07}
Amino Acids Analyser}
\end{figure}

\begin{enumerate}
\item Pros and Cons of Amino Acid Analysers
\label{sec:org85692c7}
\begin{enumerate}
\item Pros
\label{sec:org776127d}
\begin{itemize}
\item Standardized and established technology
\item Interpretive experience
\end{itemize}
\item Cons
\label{sec:org3f68393}
\begin{itemize}
\item Long run time (90 – 150 minutes)
\item Lack of analyte specificity (identification by retention time)
\begin{itemize}
\item antibiotics (ampicillin, amoxicillin, and gentamicin), acetaminophen
\end{itemize}
\item Co-eluting substances cannot be separated and distinguished on a standard IEC chromatogram
\begin{itemize}
\item Homocitrulline co-elutes with methionine
\item ASA co-elutes with leucine
\item Alloisoleucine co-elutes with cystathionine
\item Tryptophan co-elutes with histidine
\end{itemize}
\end{itemize}
\end{enumerate}
\end{enumerate}

\subsection{LC-MS/MS}
\label{sec:org3f929be}
\begin{enumerate}
\item LC-MS/MS schematic
\label{sec:orgf2ff6e9}
\begin{center}
\begin{tikzpicture}[node distance=7em]
% nodes
\node(ms1)[ms]{MS1: Mass Filter};
\node(cc)[ms, right of=ms1]{Collision cell};
\node(ms2)[ms, right of=cc]{MS2: Mass Filter};
\node(ion)[ms, below of=ms1,yshift=3em]{Ionization};
\node(lc)[msw, below of=ion,yshift=3em]{Chromatography};
\node(detector)[ms, below of=ms2, yshift=3em]{Detector};
% arrows
\draw[arrow](lc) -- (ion);
\draw[arrow](ion) -- (ms1);
\draw[arrow](ms1) -- (cc);
\draw[arrow](cc) -- (ms2);
\draw[arrow](ms2) -- (detector);
\end{tikzpicture}
\end{center}



\item LC-MS/MS sample prep
\label{sec:org99e9dc7}
\begin{itemize}
\item 10 \textmu{}L of sample is mixed with 990 \textmu{}L of IS in 0.5 mM Perfluoroheptanoic acid and centrifuge to deproteinize.
\item 200 \textmu{}L of supernatant is removed
\item 7.5 \textmu{}L is injected onto an octadecylsilyl (C18) stationary phase.
\end{itemize}


\begin{center}
\includegraphics[width=0.7\textwidth]{./aa/figures/outletmethod.png}
\end{center}

\item Ion-Pairing Chromatography
\label{sec:org2bde6f9}
\chemfig[][scale=.70]{CF_3-{(CF_2)_4}-CF_2-[::30](=[::60]O)-[::-60]OH}
\vspace{3em}
\ce{AA+ + PFHA-  <=> AA+ PFHA-}

\item LC- MS/MS transitions
\label{sec:org8590935}
\begin{itemize}
\item ESI in positive mode
\begin{itemize}
\item MRM
\end{itemize}
\end{itemize}

\begin{table}[htbp]
\caption{\label{tab:orgd6e7b5b}
AA Quantified by LC-MS/MS}
\centering
\begin{tabular}{lll}
phosphoserine & alanine & phenylalanine\\
taurine & citulline & aminoisobutyric\\
pphosphoethanolamine & 2-aminobutyric & \(\gamma\)-aminobutryic\\
aspartate & valine & ethanolamine\\
hydroxyproline & cystine & hydroxylysine\\
threonine & saccharopine & ornithine\\
serine & methionine & lysine\\
asparagine & alloisoleucine & 1-methylhistidine\\
glutamate & cystathionine & histidine\\
glutamine & isoleucine & tryptophan\\
sarcosine & leucine & 3-methylhistidine\\
aminoadipic & arginosuccinic acid & anserine\\
proline & tyrosine & carnosine\\
glycine & \(\beta\)-alanine & arginine\\
 &  & s-sulfocyteine*\\
\end{tabular}
\end{table}

\item Pros and Cons of LC-MS/MS
\label{sec:orgfe2e8de}
\begin{itemize}
\item As compared to FIA-MS/MS
\end{itemize}
\begin{enumerate}
\item Pros
\label{sec:orga322fd4}
\begin{itemize}
\item 43 vs 11 amino acids quantified
\begin{itemize}
\item Leu/Ile/Allo
\end{itemize}
\item Iso-baric compounds resolved
\begin{itemize}
\item Leucine, Isoleucine, Alloisoleucine
\end{itemize}
\end{itemize}
\item Cons
\label{sec:org3bb2e95}
\begin{itemize}
\item Too slow for NBS
\item Manual peak integration

\item As compared to AAA
\end{itemize}
\item Pros
\label{sec:orgdad92ba}
\begin{itemize}
\item \textasciitilde{} 30 min shorter analysis time
\item Analyte specificity
\begin{itemize}
\item Based on MRM rather than RT and ninhydrin reactivity
\begin{itemize}
\item gentamycin, acetaminophen, dopamine analogs
\end{itemize}
\item Co-eluting substances cannot be separated and distinguished on a
standard IEC chromatogram
\begin{itemize}
\item Homocitrulline co-elutes with methionine
\item ASA co-elutes with leucine
\item Alloisoleucine co-elutes with cystathionine
\item Tryptophan co-elutes with histidine
\end{itemize}
\end{itemize}
\item Long term reagent expense
\end{itemize}

\item Cons
\label{sec:org8061833}
\begin{itemize}
\item Upfront hardware expense
\item Manual peak integration
\item Lab developed test - not standardized
\item Changing LOQ with equipment age
\end{itemize}
\end{enumerate}
\end{enumerate}
\section{Acylcarnitines}
\label{sec:org0095074}
\subsection{Introduction}
\label{sec:org3f2c168}
\begin{itemize}
\item Carnitine (\(\beta\)-hydroxy-\(\gamma\)-N-trimethylaminobutyric acid) is
an endogenous quaternary ammonium compound synthesized from lysine
and methionine.
\item L-Carnitine has been described as a "conditionally essential"
nutrient for humans.
\item Populations with an exogenous carnitine requirement include:
\begin{itemize}
\item infants (premature and full-term),
\item patients on long-term parenteral nutrition,
\item perhaps children.
\end{itemize}
\item Exogenous carnitine is required to maintain "normal" (in the
epidemiologic sense) plasma or serum carnitine concentrations in
humans of all ages.
\item Primary function is to shuttle long chain fatty acids to the
mitochondrial matrix, for \(\beta\)-oxidation.
\item Acylcarnitines are markers for FAODs and OAs
\end{itemize}

\vspace{2em}

\chemname{\chemfig[][scale=.5]{H3C-N^{+}([2]-CH3)([6]-CH3)-CH2-C([2]-H)([6]-OH)-CH_2-C([1]=O)([7]-O^{-})}}{\tiny Carnitine}
\hspace{3em}
\chemname{\chemfig[][scale=.5]{H3C-N^{+}([2]-CH3)([6]-CH3)-CH2-C([2]-H)([6]-O-C([0]=O)-{\color{red}R})-CH_2-C([1]=O)([7]-O^{-})}}{\tiny Acylcarnitine}
%\chemname{\chemfig[][scale=.5]{H3C-N^{+}([2]-CH3)([6]-CH3)-CH2-C([2]-H)([6]-O-C([0]=O)-{\color{red}R})-CH_2-C([2]=O)-O-CH_2-CH_2-CH_2-CH_3}}{\tiny Acylcarnitine, butyl ester}

\subsection{Diagnostic FIA-MS/MS}
\label{sec:org1d35556}
\begin{enumerate}
\item Diagnostic FIA-MS/MS schematic
\label{sec:org69c3641}
\begin{center}
\begin{tikzpicture}[node distance=7em]
% nodes
\node(ms1)[ms]{MS1: Mass Filter};
\node(cc)[ms, right of=ms1]{Collision cell};
\node(ms2)[ms, right of=cc]{MS2: Mass Filter};
\node(ion)[ms, below of=ms1,yshift=3em]{Ionization};
\node(lc)[msw, below of=ion,yshift=3em]{Fused silica};
\node(detector)[ms, below of=ms2, yshift=3em]{Detector};
% arrows
\draw[arrow](lc) -- (ion);
\draw[arrow](ion) -- (ms1);
\draw[arrow](ms1) -- (cc);
\draw[arrow](cc) -- (ms2);
\draw[arrow](ms2) -- (detector);
\end{tikzpicture}
\end{center}
\begin{itemize}
\item ESI in positive mode
\end{itemize}
\item Sample prep
\label{sec:org5d968c8}
\begin{itemize}
\item Sample Type
\begin{itemize}
\item Plasma
\begin{itemize}
\item Diagnostic testing for FAODs and OAs
\item Monitoring
\end{itemize}
\item Urine
\begin{itemize}
\item Diagnosis of CUD
\end{itemize}
\end{itemize}
\item 20 \textmu{}L of sample is mixed with 400 \textmu{}L of IS in Methanol centrifuge to deproteinize.
\item supernatant is removed and 100 \textmu{}L of n-butanol-3M HCL is added
\item dried down
\item reconstituted with 200 \textmu{}L 80\% acetonitrile.
\item 7.5 \textmu{}L injection.
\end{itemize}

\begin{center}
\includegraphics[width=0.7\textwidth]{./ac/figures/outletmethod.pdf}
\end{center}

\item Fragmentation
\label{sec:org54bc688}
\definesubmol{x}{-[1,.6]-[7,.6]}
\centering
 \chemname{\chemfig[][scale=.33]{H_{3}C-N^{+}([2]-CH_3)([6]-CH_{3})-CH_2-C([2]-H)([6]-O-C([0]=O)-{\color{red}R})-CH_2-C([2]=O)-O-CH_2-CH_2-CH_2-CH_3}}{\tiny acylcarnitine, butyl ester}

\vspace{2.5em}

 \chemname{\chemfig[][scale=.33]{H_{3}C-N([1]-CH_3)([7]-CH_3)}}{\tiny trimethylamine}
\hspace{2em}
\chemname{\chemfig[][scale=.33]{{\color{red}R}-C([1]=O)([7]-OH)}}{\tiny carboxylic acid}
\hspace{2em}
 \chemname{\chemfig[][scale=.33]{H!x!x}}{\tiny butyl group}
\hspace{2em}
 \chemname{\chemfig[][scale=.33]{H_{2}C^{+}-HC=CH-C([1]=O)([7]-OH)}}{\tiny 85 m/z}
\item Precursor Ion Scan
\label{sec:org1df8e05}
\begin{center}
\begin{tikzpicture}
\node[box](ms1)[]{};
\node[label](ms1u)[above=of ms1,yshift=-3em]{MS1};
\node[label](ms1l)[below=of ms1,yshift=3em]{scanning};
\node[box](cc)[right= of ms1]{};
\node[label](ccu)[above=of cc,yshift=-3em]{Collision cell};
\node[label](ccl)[below=of cc,yshift=3em]{fragmentation};
\node[box](ms2)[right= of cc]{};
\node[label](ms2u)[above=of ms2,yshift=-3em]{MS2};
\node[label](ms2l)[below=of ms2,yshift=3em]{85 m/z};
\draw[->](ms1) -- (cc);
\draw[->](cc) -- (ms2);

%ms1
\draw [gray,->, decorate,decoration=snake] (-.8,0.5) -- (.8,0.5);
\draw [gray,->, decorate,decoration=snake] (-.8,0.25) -- (.8,0.25);
\draw [blue, ->,decorate,decoration=snake] (-.8, 0) -- (.8,0);
\draw [gray,->, decorate,decoration=snake] (-.8,-0.25) -- (.8,-0.25);
\draw [gray,->,decorate,decoration=snake] (-.8,-0.5) -- (.8,-0.5);

%cc
\draw [blue,->,decorate,decoration=snake] (2.1, 0) -- (2.4,0);
\fill (2.6,0) circle (0.1); 
\draw [gray,->,decorate,decoration=snake] (2.8, 0) -- (3.8,0.5);
\draw [red, ->,decorate,decoration=snake] (2.8, 0) -- (3.8,0);
\draw [gray,->,decorate,decoration=snake] (2.8, 0) -- (3.8,-0.5);

%ms2
\draw [red,->,decorate,decoration=snake] (5.1, 0) -- (6.8,0);
\end{tikzpicture}
\end{center}
\item MRM is used to detected dicarboxylic acylcarnitines
\label{sec:orgde6f16b}

\begin{itemize}
\item C0-Bu 218.1 Da \(\to\) 103 Da transition is optimal
\item All others benefit from the added sensitivity of MRM mode as compared to parent ion scan
\end{itemize}

\begin{table}[htbp]
\caption{\label{tab:org3805134}
MRM is used to detected selected acylcarnitines}
\centering
\begin{tabular}{ll}
Compound & Reaction\\
\hline
C0 & 218.10 > 103.00\\
C0 IS & 227.10 > 103.00\\
C2 & 260.20 > 85.00\\
C2 IS & 263.20 > 85.00\\
C3 & 274.20 > 85.00\\
C3 IS & 277.20 > 85.00\\
C3DC & 360.30 > 85.00\\
C4DC & 374.30 > 85.00\\
C5DC & 388.35 > 85.00\\
C5DC IS & 391.35 > 85.00\\
C6DC & 402.45 > 85.00\\
C8DC & 430.45 > 85.00\\
\end{tabular}
\end{table}

\begin{table}[htbp]
\caption{\label{tab:org9a40757}
Quantified Acylcarnitines}
\centering
\begin{tabular}{lll}
C0 & C8 & C16\\
C2 & C8:1 & C16:1\\
C3 & C10 & C16:1-OH\\
C3DC & C10:1 & C16-OH\\
C4 & C12 & C18\\
C4DC & C12:1 & C18:1\\
C5 & C14 & C18:1-OH\\
C5:1 & C14:1 & C18:2\\
C5DC & C14:2 & C18-OH\\
C5-OH & C14-OH & \\
C6 &  & \\
C6DC &  & \\
\end{tabular}
\end{table}

\item Overestimation of Free Carnitine
\label{sec:orgbb02005}

\begin{itemize}
\item Butylated acylcarnitines are converted to butylated carnitine in
n-butanol-3M HCl at 65\degree{}C. \footnote{Johnson, D. W. (1999). Inaccurate measurement of free
carnitine by the electrospray tandem mass spectrometry screening
method for blood spots. Journal of Inherited Metabolic Disease, 22(2),
201–202.\label{org9b21f28}}
\end{itemize}

\begin{center}
\begin{tabular}{lr}
Acyl Carnitine & Half-life (min)\\
\hline
C2 & 31\\
C10 & 125\\
C18 & 172\\
\end{tabular}
\end{center}

\begin{itemize}
\item 65\degree{}C for 15 min.
\item NSO uses 60\degree{}C for 20 minutes.
\item IMD uses 55\degree{}C for 20 minutes.

\item In a sample with low free carnitine and high acetylcarnitine.
\begin{itemize}
\item 30\% of the acetylcarnitine and smaller amounts of higher
molecular mass acylcarnitines are converted to carnitine
\item a low carnitine sample could appear to be normal.
\end{itemize}
\item "The free carnitine results obtained by this screening method on
blood spots with high levels of acylcarnitines should therefore be
used with caution." \textsuperscript{\ref{org9b21f28}}
\end{itemize}

\item Pros and Cons of Butanol  FIA-MSMS for Aceylcarnitines
\label{sec:org07603c8}
\begin{enumerate}
\item Pros
\label{sec:orgff0bedd}
\begin{itemize}
\item Speed
\item Sensitivity
\item Expertise
\item Amino acid measurement
\end{itemize}
\item Cons
\label{sec:org020e9a6}
\begin{itemize}
\item Isobaric compounds
\begin{itemize}
\item C5DC and C10-OH
\end{itemize}
\item Overestimation of CO due to hydrolysis
\end{itemize}
\end{enumerate}
\end{enumerate}


\subsection{{\bfseries\sffamily TODO} Free and Total Carnitine}
\label{sec:orgefc0742}
\begin{itemize}
\item Method?
\end{itemize}
\begin{enumerate}
\item Fractional Tubular Re-absorption of Carnitine
\label{sec:org329e524}

\begin{LaTeX}
\begin{equation*}
FTR_{carnitine}\% = \left( 1 -  \frac{carnitine_{urine} \cdot creatinine_{plasma}}{carnitine_{plasma} \cdot creatinine_{urine}}\right) \cdot 100
\end{equation*}
\end{LaTeX}

\begin{itemize}
\item normally >98\%, \(\Downarrow\) in CUD
\end{itemize}

\item Free/Total Carnitine
\label{sec:org0eec331}

\[
\frac{Free_{carnitine}}{Total_{carnitine}} = \frac{C_0}{\sum_{0}^{18} C_n}
\]

\begin{itemize}
\item \(\Downarrow\) in CUD, < 5-10\% of normal
\end{itemize}
\end{enumerate}
\section{Organic Acids}
\label{sec:org6a26c69}
\subsection{Background}
\label{sec:org5074331}
\begin{enumerate}
\item What are urine organic acids?
\label{sec:org2702143}
\begin{itemize}
\item Water soluble compounds containing \(\ge\) one carboxyl group(s) and
nonamino functional groups
\end{itemize}

\centering
\chemfig{X-C(-[2]X)(-[6]X)-C(-[2]X)(-[6]X)-C(-[7]OH)=[1]O}
\begin{enumerate}
\item Acylglycines
\label{sec:org0c71a6b}
\begin{itemize}
\item Acylglycines are also detected in UOA analysis
\begin{itemize}
\item conjugation of acyl-CoA species to glycine
\item catalysed by glycine N-acylase
\end{itemize}
\end{itemize}

\begin{table}[htbp]
\caption{\label{tab:org56ea60c}
Organic Acid Nomenclature}
\centering
\begin{tabular}{lll}
Length & Monocarboxylic acid & Dicarboxylic acid\\
\hline
C2 & Acetic & Oxalic\\
C3 & Propionic & Malonic\\
C4 & Butyric & Succinic\\
 & Isobutyric & \\
C5 & Valeric & Glutaric\\
 & Isovaleric & \\
 & 2-Methylbutyric & \\
C6 & Hexanoic (caprioc) & Adipic\\
C7 & Heptanoic (enanthic) & Pimelic\\
C8 & Octanoic (caprylic) & Suberic\\
C9 & Nonanoic (pelargonic) & Azelaic\\
C10 & Decanoic (capric) & Sebacic\\
\end{tabular}
\end{table}
\end{enumerate}


\item Functional Groups
\label{sec:org6a30285}
\centering
\chemfig{X-C(-[2]X)(-[6]X)-C(-[2]X)(-[6]X)-C(-[7]OH)=[1]O}
\begin{table}[htbp]
\caption{\label{tab:org4cbdcc9}
Organic Acid Functional Groups}
\centering
\begin{tabular}{ll}
Functional group & Formula\\
\hline
hydrogen & -H\\
keto & .= O\\
hydroxyl & -OH\\
carboxyl & -COOH\\
side chain & -(CH\(_2\))\(_n\)\\
\end{tabular}
\end{table}

\item Side Chains
\label{sec:org99b2da7}
\centering
\chemfig{X-C(-[2]X)(-[6]X)-C(-[2]X)(-[6]X)-C(-[7]OH)=[1]O}
\begin{table}[htbp]
\caption{\label{tab:orgb073dc5}
Organic Acid Side Chains}
\centering
\begin{tabular}{ll}
Side chain & Structure\\
\hline
Methyl & \chemfig{CH_3-}\\
Ethyl & \chemfig{CH_3-CH_2-}\\
Propyl & \chemfig{CH_3-CH_2-CH_2-}\\
Butyl & \chemfig{CH_3-CH_2-CH_2-CH_2-}\\
\end{tabular}
\end{table}


\item Where do they come from?
\label{sec:orgf363208}
\begin{enumerate}
\item Endogenous Sources
\label{sec:org9a2726a}
\begin{itemize}
\item Originate from the intermediate metabolism of all major groups of
organic cellular components
\begin{itemize}
\item amino acids
\item lipids
\item nucleotides
\item carbohydrates
\item nucleic acids
\item steroids
\end{itemize}
\end{itemize}

\item Exogenous
\label{sec:org94c94a1}
\begin{itemize}
\item food
\item environment
\item medications
\end{itemize}
\end{enumerate}

\item Urine organic acids detected in health
\label{sec:orgedc5ca0}

\begin{itemize}
\item Tricarboxylic acid cycle acids
\begin{itemize}
\item citric
\end{itemize}
\item hydroxyaliphatic acids
\begin{itemize}
\item 3-hydroxybutyric
\end{itemize}
\item aliphatic keto acids
\begin{itemize}
\item pyrvic
\end{itemize}
\item aliphatic acids
\begin{itemize}
\item oxalic
\end{itemize}
\item aldonic and deoxyaldonic acids (sugar acids)
\item aromatic acids
\begin{itemize}
\item hippuric
\end{itemize}
\end{itemize}

\item Abnormal Urine Organic acids profiles
\label{sec:orgf6ab0a0}
\begin{itemize}
\item Elevated concentration of normal metabolites
\begin{itemize}
\item fumaric acid in fumarase deficiency
\item adipic, suberic, and sebacic acids in MCADD
\item ketones in fasting
\begin{itemize}
\item 3-hydroxybutyric
\item acetoacetic
\end{itemize}
\end{itemize}

\item Pathological metabolites
\begin{itemize}
\item succinylacetone, methylcitric acid
\end{itemize}

\item Food, medications, environment
\begin{itemize}
\item ethosuximide
\item adipic acid
\item cresol
\item 2-furaldehyde
\end{itemize}
\end{itemize}
\end{enumerate}

\subsection{Urine Organic Acids by GC-MS}
\label{sec:org18c3d50}
\begin{enumerate}
\item Oximation
\label{sec:org218c746}
\begin{itemize}
\item Oximated with 10\% hydroxylamine-HCL
\begin{itemize}
\item avoids multiple TMS species due to keto-enol tautomerism
\end{itemize}
\end{itemize}

\centering
\schemedebug{false}
\schemestart
\chemname{\chemfig[][scale=.5]{R=[1](-[2]OH)-[7]R}}{\tiny enol}
\arrow{<=>}
\chemname{\chemfig[][scale=.5]{R-[1](=[2]O)-[7]R}}{\tiny ketone}
\+
\chemname{\chemfig[][scale=.5]{N(<:[::-160]H)(<[::-120]H)-O-[1]H}}{\tiny hydroxylamine}
\arrow{->}
\chemname{\chemfig[][scale=.5]{R-[1](=[2]N-[1]OH)-[7]R}}{\tiny ketoxime}
\schemestop
\item BSTFA Derivatisation
\label{sec:org15a2581}
\begin{itemize}
\item Acidified and extracted twice with ethyl ether
\item Derivatised with BSTFA (N,O-bis(trimethylsilyl)trifluoroacetamide) \footnote{Stalling DL, Gehrke CW, Zumwalt RW. A new silylation
reagent for amino acids bis(trimethylsilyl)trifluoroacetamide
(BSTFA). Biochemical and Biophysical Research Communications. 1968 May
23;31(4):616-22.}
\begin{itemize}
\item forms organic acid TMS esters
\end{itemize}
\end{itemize}

\centering
\schemedebug{false}
\schemestart
\chemname{\chemfig[][scale=.5]{F{_3}C-C(-[1]OTMS)=[7]NTMS}}{\tiny BSTFA}
\+
\chemname{\chemfig[][scale=.5]{R-C(=[1]O)-[7]OH}}{\tiny carboxylic acid}
\arrow{->}
\chemname{\chemfig[][scale=.5]{R-C(=[1]O)-[7]OTMS}}{\tiny TMS ester}
\+
\chemname{\chemfig[][scale=.5]{F{_3}C-C(=[1]O)-[7]NTMS}}{\tiny TMS amide}
\schemestop

\item Gas Chromatography
\label{sec:org09e31b2}

\item Mass-spectroscopy
\label{sec:org1e26425}

\item Reporting
\label{sec:org95e1b8c}
\end{enumerate}

\section{NBS}
\label{sec:org915ac96}
\subsection{Dried Blood Spot}
\label{sec:orgd580348}
\begin{itemize}
\item Collected from free flowing blood spotted onto filter paper
\item Newborn Screening
\item Monitoring Therapy/Diet
\item Each DBS is assume to contain 3.2 \textmu{}L of blood
\item The quantity of blood present in the paper varies by
\begin{itemize}
\item hematocrit
\item degree of saturation
\item the cotton fiber paper
\item the environment  when applying blood (humidity and temperature).
\end{itemize}
\item Because of these numerous factors, a dried blood spot is a highly
imprecise specimen compared with liquids such as urine, blood, and plasma.
\end{itemize}

\subsection{AAAC}
\label{sec:orga5d0b55}
\begin{enumerate}
\item FIA-MS/MS schematic
\label{sec:org4d41773}
\begin{center}
\begin{tikzpicture}[node distance=7em]
% nodes
\node(ms1)[ms]{MS1: Mass Filter};
\node(cc)[ms, right of=ms1]{Collision cell};
\node(ms2)[ms, right of=cc]{MS2: Mass Filter};
\node(ion)[ms, below of=ms1,yshift=3em]{Ionization};
\node(lc)[msw, below of=ion,yshift=3em]{Injection};
\node(detector)[ms, below of=ms2, yshift=3em]{Detector};
% arrows
\draw[arrow](lc) -- (ion);
\draw[arrow](ion) -- (ms1);
\draw[arrow](ms1) -- (cc);
\draw[arrow](cc) -- (ms2);
\draw[arrow](ms2) -- (detector);
\end{tikzpicture}
\end{center}

\item FIA-MS/MS sample
\label{sec:orgb14d1b1}

\begin{itemize}
\item Amino acids and Acylcarnitines in the DBS eluate are esterified as butyl esters with butanol-hydrogen chloride.
\item Solvent delivery is via HPLC with no chromatography, called flow injection analysis.
\item 10 \textmu{}L of sample extract is injected into a flowing stream operating at \textasciitilde{} 0.15 ml/min.

\item Typical injection rates between samples are 2 min, giving a potential 400-
to 600-sample capacity per instrument per day.
\begin{itemize}
\item volume is typically 200-400 specimens per instrument per day
\item maintenance issues, repeat specimen analysis.
\end{itemize}
\end{itemize}

\centering
\schemedebug{false}
\schemestart
\chemname{\chemfig[][scale=.33]{{\color{red}R}-[::-60](<[::-60]NH_3^+)-[::60](=[::60]O)-[::-60]OH}}{\tiny \textalpha{}-amino acid}
\+
\chemname{\chemfig[][scale=.33]{HO-[::30]-[::-60]-[::60]-[::-60]}}{\tiny n-butanol}
\arrow{-U>[][{\tiny \ce{H2O}}]}
\chemname{\chemfig[][scale=.33]{{\color{red}R}-[::-60](<[::-60]NH_3^+)-[::60](=[::60]O)-[::-60]O-[::60]-[::-60]-[::60]-[::-60]}}{\tiny AA butyl ester}
\schemestop

\definesubmol{x}{-[1,.6]-[7,.6]}
\definesubmol{y}{-[7,.6]-[1,.6]}
\definesubmol{d}{!y!y-[7,.6]{\color{red}COOH}}
\definesubmol{e}{!y!y}
\centering
\schemedebug{false}
\schemestart
\chemname{\chemfig[][scale=.33]{-N^{+}([2]-)([6]-)-[1]-[7]([6]-O-([5]=O)!e)-[1]-[7]([7]=O)([1]-O^{-})}}{\tiny C5-carnitine}
\+
\chemname{\chemfig[][scale=.33]{HO!x!x}}{\tiny n-butanol}
\arrow{-U>[][{\tiny \ce{H2O}}]}
\chemname{\chemfig[][scale=.33]{-N^{+}([2]-)([6]-)-[1]-[7]([6]-O-([5]=O)!e)-[1]-[7]([6]=O)-[1,.6]O!y!y}}{\tiny C5-carnitine, butyl ester}
\schemestop
\vspace{2em}
\schemedebug{false}
\schemestart
\chemname{\chemfig[][scale=.33]{-N^{+}([2]-)([6]-)-[1]-[7]([6]-O-([5]=O)!d)-[1]-[7]([7]=O)([1]-O^{-})}}{\tiny C6DC-carnitine}
\+
\chemname{\chemfig[][scale=.33]{HO!x!x}}{\tiny n-butanol}
\arrow{-U>[][{\tiny \ce{2H2O}}]}
\chemname{\chemfig[][scale=.33]{-N^{+}([2]-)([6]-)-[1]-[7]([6]-O-([5]=O)!e-[7,.6]O!x!x)-[1]-[7]([6]=O)-[1,.6]O!y!y}}{\tiny C6DC-carnitine, butyl ester}
\schemestop 

\begin{center}
\includegraphics[width=0.7\textwidth]{./nbs/figures/outletmethod.pdf}
\end{center}

\item AA NL Scan
\label{sec:org9298a5c}
\begin{itemize}
\item Electrospray ionization in positive mode
\item \(\alpha\)-amino acids fragment to produce the neutral butyl formate molecule (102 Da).
\item A neutral loss scan is used to identify parent molecules with a NL of 102 Da.
\item MRM is used to detected amino acids with basic functional groups
\begin{itemize}
\item arginine, ornithine and citrulline
\end{itemize}
\end{itemize}


\centering
\schemedebug{false}
\schemestart
\chemname{\chemfig[][scale=.33]{{\color{red}R}-[::-60](<[::-60]NH_3^+)-[::60](=[::60]O)-[::-60]O-[::60]-[::-60]-[::60]-[::-60]}}{\tiny AA butyl ester}
\arrow{->[{\tiny fragmentation}]}
\chemname{\chemfig[][scale=.33]{{\color{red}R}-[::60]=NH_2^{+}}}{\tiny fragment}
\+
\chemname{\chemfig[][scale=.33]{H-[::60](=[::60]O)-[::-60]O-[::60]-[::-60]-[::60]-[::-60]}}{\tiny butyl formate (102 Da)}
\schemestop
\item AA MRM
\label{sec:org8bbc854}
\begin{itemize}
\item Citrulline contains a labile amino group that fragments together with butyl formate.
\item CID results in net neutral fragmentation of butyl formate (102 Da) plus \ce{NH3} (17 Da)
\item \href{https://en.wikipedia.org/wiki/Selected\_reaction\_monitoring}{SRM} Citrulline-Bu 232.15 Da \(\to\) 113 Da , loss of 119 Da
\end{itemize}

\centering
\schemedebug{false}
\schemestart
\chemname{\chemfig[][scale=.33]{H_2N-[::30,,2,](=[::60]O)-[::-60]NH-[::60]-[::-60]-[::60]-[::-60](<[::-60]NH_3^+)-[::60](=[::60]O)-[::-60]OH}}{\tiny citrulline 175 Da}
\+
\chemname{\chemfig[][scale=.33]{HO-[::30]-[::-60]-[::60]-[::-60]}}{\tiny n-butanol 74 Da}
\arrow{-U>[][{\tiny \ce{H2O}}]}
\chemname{\chemfig[][scale=.33]{H_2N-[::30,,2,](=[::60]O)-[::-60]NH-[::60]-[::-60]-[::60]-[::-60](<[::-60]NH_3^+)-[::60](=[::60]O)-[::-60]O-[::60]-[::-60]-[::60]-[::-60]}}{\tiny 232 Da}
\schemestop

\centering
\schemedebug{false}
\schemestart
\chemname{\chemfig[][scale=.33]{H_2N-[::60]-[::-60]-[::60]-[::-60]-[::60]N=O=C}}{\tiny 113 Da}
\+
\chemname{\chemfig[][scale=.33]{H-[::60](=[::60]O)-[::-60]O-[::60]-[::-60]-[::60]-[::-60]}}{\tiny 102 Da}
\+
\chemname{\chemfig[][scale=.43]{NH_3}}{\tiny 17 Da}
\schemestop
\begin{itemize}
\item Its name is derived from citrullus, the Latin word for watermelon, from which it was first isolated in 1914 by Koga and Odake.
\end{itemize}

\begin{table}[htbp]
\caption{\label{tab:org530b8f5}
Quantified Amino Acids}
\centering
\begin{tabular}{ll}
Glycine & Tyrosine\\
Alanine & Ornithine\\
Valine & Citruline\\
Leucine & Arginine\\
Methionine & \color{blue}Succinylacetone\\
Phenylalanine & \\
\end{tabular}
\end{table}

\item AC Precursor Ion Scan
\label{sec:orgddd1b1a}
\begin{itemize}
\item Electrospray ionization in positive mode
\item Butylated acylcarnitines fragment to produce a characteristic ion with mass of 85 Da.
\item A precursor ion scan is used to identify molecules that fragment to form a 85 m/z molecule.
\end{itemize}

\definesubmol{x}{-[1,.6]-[7,.6]}
\centering
 \chemname{\chemfig[][scale=.33]{H_{3}C-N^{+}([2]-CH_3)([6]-CH_{3})-CH_2-C([2]-H)([6]-O-C([0]=O)-{\color{red}R})-CH_2-C([2]=O)-O-CH_2-CH_2-CH_2-CH_3}}{\tiny acylcarnitine, butyl ester}

\vspace{2.5em}

 \chemname{\chemfig[][scale=.33]{H_{3}C-N([1]-CH_3)([7]-CH_3)}}{\tiny trimethylamine}
\hspace{2em}
\chemname{\chemfig[][scale=.33]{{\color{red}R}-C([1]=O)([7]-OH)}}{\tiny carboxylic acid}
\hspace{2em}
 \chemname{\chemfig[][scale=.33]{H!x!x}}{\tiny butyl group}
\hspace{2em}
 \chemname{\chemfig[][scale=.33]{H_{2}C^{+}-HC=CH-C([1]=O)([7]-OH)}}{\tiny 85 m/z}

\begin{center}
\begin{tikzpicture}
\node[box](ms1)[]{};
\node[label](ms1u)[above=of ms1,yshift=-3em]{MS1};
\node[label](ms1l)[below=of ms1,yshift=3em]{scanning};
\node[box](cc)[right= of ms1]{};
\node[label](ccu)[above=of cc,yshift=-3em]{Collision cell};
\node[label](ccl)[below=of cc,yshift=3em]{fragmentation};
\node[box](ms2)[right= of cc]{};
\node[label](ms2u)[above=of ms2,yshift=-3em]{MS2};
\node[label](ms2l)[below=of ms2,yshift=3em]{85 m/z};
\draw[->](ms1) -- (cc);
\draw[->](cc) -- (ms2);

%ms1
\draw [gray,->, decorate,decoration=snake] (-.8,0.5) -- (.8,0.5);
\draw [gray,->, decorate,decoration=snake] (-.8,0.25) -- (.8,0.25);
\draw [blue, ->,decorate,decoration=snake] (-.8, 0) -- (.8,0);
\draw [gray,->, decorate,decoration=snake] (-.8,-0.25) -- (.8,-0.25);
\draw [gray,->,decorate,decoration=snake] (-.8,-0.5) -- (.8,-0.5);

%cc
\draw [blue,->,decorate,decoration=snake] (2.1, 0) -- (2.4,0);
\fill (2.6,0) circle (0.1); 
\draw [gray,->,decorate,decoration=snake] (2.8, 0) -- (3.8,0.5);
\draw [red, ->,decorate,decoration=snake] (2.8, 0) -- (3.8,0);
\draw [gray,->,decorate,decoration=snake] (2.8, 0) -- (3.8,-0.5);

%ms2
\draw [red,->,decorate,decoration=snake] (5.1, 0) -- (6.8,0);
\end{tikzpicture}
\end{center}

\item AC MRM
\label{sec:org2ef9bbc}

\begin{itemize}
\item C0-Bu 218.1 Da \(\to\) 103 Da transition is optimal
\item All others benefit from the added sensitivity of MRM mode as compared to parent ion scan
\end{itemize}

\begin{table}[htbp]
\caption{\label{tab:org79216fd}
MRM is used to detected selected acylcarnitines}
\centering
\begin{tabular}{ll}
Compound & Reaction\\
\hline
C0 & 218.10 > 103.00\\
C0 IS & 227.10 > 103.00\\
C2 & 260.20 > 85.00\\
C2 IS & 263.20 > 85.00\\
C3 & 274.20 > 85.00\\
C3 IS & 277.20 > 85.00\\
C3DC & 360.30 > 85.00\\
C4DC & 374.30 > 85.00\\
C5DC & 388.35 > 85.00\\
C5DC IS & 391.35 > 85.00\\
C6DC & 402.45 > 85.00\\
C8DC & 430.45 > 85.00\\
\end{tabular}
\end{table}

\begin{table}[htbp]
\caption{\label{tab:orgd579984}
Quantified Acylcarnitines}
\centering
\begin{tabular}{lll}
C0 & C8 & C16\\
C2 & C8:1 & C16:1\\
C3 & C10 & C16:1-OH\\
C3DC & C10:1 & C16-OH\\
C4 & C12 & C18\\
C4DC & C12:1 & C18:1\\
C5 & C14 & C18:1-OH\\
C5:1 & C14:1 & C18:2\\
C5DC & C14:2 & C18-OH\\
C5-OH & C14-OH & \\
C6 &  & \\
C6DC &  & \\
\end{tabular}
\end{table}

\item Why derivatize?
\label{sec:org8dabd45}

\begin{center}
\includegraphics[width=.9\linewidth]{./nbs/figures/ionization.png}
\end{center}

\item Overestimation of Free Carnitine
\label{sec:org314b2b4}
\begin{itemize}
\item Butylated acylcarnitines are converted to butylated carnitine in
n-butanol-3M HCl at 65\degree{}C. \footnote{Johnson, D. W. (1999). Inaccurate measurement of free
carnitine by the electrospray tandem mass spectrometry screening
method for blood spots. Journal of Inherited Metabolic Disease, 22(2),
201–202.\label{orgafec481}}
\end{itemize}

\begin{center}
\begin{tabular}{lr}
Acyl Carnitine & Half-life (min)\\
\hline
C2 & 31\\
C10 & 125\\
C18 & 172\\
\end{tabular}
\end{center}

\begin{itemize}
\item 65\degree{}C for 15 min.
\item NSO uses 60\degree{}C for 20 minutes.
\item IMD uses 55\degree{}C for 20 minutes.

\item In a sample with low free carnitine and high acetylcarnitine.
\begin{itemize}
\item 30\% of the acetylcarnitine and smaller amounts of higher
molecular mass acylcarnitines are converted to carnitine
\item a low carnitine sample could appear to be normal.
\end{itemize}
\item "The free carnitine results obtained by this screening method on
blood spots with high levels of acylcarnitines should therefore be
used with caution." \textsuperscript{\ref{orgafec481}}
\end{itemize}

\item Pros and Cons of FIA-MS/MS using DBS
\label{sec:org6ab4553}
\begin{itemize}
\item As compared to AAA and LC-MS/MS
\end{itemize}
\begin{enumerate}
\item Pros
\label{sec:org8bb9345}
\begin{itemize}
\item \textasciitilde{} 2 min analysis time
\item Analyte specificity
\item ACs and AAs quantified simultaneously
\end{itemize}

\item Cons
\label{sec:orgcaf9ed7}
\begin{itemize}
\item Variability in DBS sample as described above
\item Iso-baric compounds
\begin{itemize}
\item leucine, Isoleucine, Alloisoleucine
\item C5DC and C10-OH
\end{itemize}
\item Overestimation of CO due to hydrolysis
\item Fewer AA quantified
\begin{itemize}
\item homocystine (free)
\item glutamine
\end{itemize}
\item Overestimation of CO due to hydrolysis
\end{itemize}
\end{enumerate}
\end{enumerate}
\subsection{Multiplex DBS lysosomal Enzyme Assay}
\label{sec:org2ad8767}
\begin{itemize}
\item The DBS screening assay tests for:
\begin{itemize}
\item Gaucher
\item Krabbe
\item Niemann-Pick-A/B
\item Pompe
\item Fabry
\item MPS-I
\end{itemize}
\item a single 3-mm DBS punch, which is incubated in a single-assay
cocktail with all substrates and internal standards.
\item After incubation and liquid-liquid extraction, samples are analyzed by flow injection MS/MS.
\item All deuterated internal standards correspond to enzymatically generated products.
\end{itemize}
\subsection{Biotinidase}
\label{sec:org6b4fd56}
\subsection{GALT}
\label{sec:orgf200c58}

\section{{\bfseries\sffamily TODO} VLCFA}
\label{sec:orged9d035}
\section{{\bfseries\sffamily TODO} GAGs and Oligosacarides}
\label{sec:org92814f0}
\section{{\bfseries\sffamily TODO} Enzymes}
\label{sec:org7b62c7f}
\section{{\bfseries\sffamily TODO} WBC prep}
\label{sec:org17cbbca}
\end{document}