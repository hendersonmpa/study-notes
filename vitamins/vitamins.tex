% Created 2019-12-18 Wed 16:53
% Intended LaTeX compiler: pdflatex
\documentclass{scrartcl}
\usepackage[utf8]{inputenc}
\usepackage[T1]{fontenc}
\usepackage{graphicx}
\usepackage{grffile}
\usepackage{longtable}
\usepackage{wrapfig}
\usepackage{rotating}
\usepackage[normalem]{ulem}
\usepackage{amsmath}
\usepackage{textcomp}
\usepackage{amssymb}
\usepackage{capt-of}
\usepackage{hyperref}
\hypersetup{colorlinks,linkcolor=black,urlcolor=blue}
\usepackage{textpos}
\usepackage{textgreek}
\usepackage[version=4]{mhchem}
\usepackage{chemfig}
\usepackage{siunitx}
\usepackage{gensymb}
\usepackage[usenames,dvipsnames]{xcolor}
\usepackage[T1]{fontenc}
\usepackage{lmodern}
\usepackage{verbatim}
\usepackage{tikz}
\usepackage{wasysym}
\usetikzlibrary{shapes.geometric,arrows,decorations.pathmorphing,backgrounds,positioning,fit,petri}
\usepackage{fancyhdr}
\pagestyle{fancy}
\author{Matthew Henderson, PhD, FCACB}
\date{\today}
\title{Vitamins}
\hypersetup{
 pdfauthor={Matthew Henderson, PhD, FCACB},
 pdftitle={Vitamins},
 pdfkeywords={},
 pdfsubject={},
 pdfcreator={Emacs 26.1 (Org mode 9.1.9)}, 
 pdflang={English}}
\begin{document}

\maketitle
\setcounter{tocdepth}{2}
\tableofcontents


\section{Biotin}
\label{sec:orge406aa9}
\subsection{Introduction}
\label{sec:org12f7811}
\begin{itemize}
\item biotin is a water-soluble vitamin widely present in small amounts in
natural food-stuffs, in which it is mostly protein bound
\item classic role of biotin is to function as the coenzyme of five
important carboxylases involved in gluconeogenesis, fatty acid
synthesis and the catabolism of several amino acids
\item covalent binding of biotin to the inactive apocarboxylases,
catalysed by holocarboxylase synthetase (HCS), is required to
generate the active holocarboxylases
\item recycling of biotin first involves proteolytic degradation of the
holocarboxylases, yielding biotin bound to lysine (biocytin) or to
short biotinyl peptides
\item biotinidase releases biotin
\item transcription of a large number of genes, including those encoding
HCS and the biotin-dependent carboxylases , is regulated by biotin
in a process that requires biotinyl-5’-AMP, the intermediate of the
HCS reaction
\item SLC19A3 encoding the thiamine transporter hTHTR2, mutations cause
biotin-responsive basal ganglia disease
\end{itemize}

\begin{figure}[htbp]
\centering
\includegraphics[width=0.9\textwidth]{./biotin/figures/carboxylases.png}
\caption{\label{fig:org681a2fb}
Biotin dependant carboxylases}
\end{figure}


\begin{figure}[htbp]
\centering
\includegraphics[width=0.9\textwidth]{./biotin/figures/biotin.png}
\caption{\label{fig:orgdebc850}
Biotin cycle}
\end{figure}

\subsection{Biotin-responsive disorders}
\label{sec:org9cbd0fc}
\begin{itemize}
\item two inherited defects affecting the coenzyme function of biotin:
\begin{enumerate}
\item holocarboxylase synthetase (HCS) deficiency
\item biotinidase deficiency
\end{enumerate}
\item HCS deficiency \(\to\) impaired binding of biotin to apocarboxylases
\item biotinidase deficiency \(\to\) biotin depletion
\item both lead to multiple carboxylase deficiency (MCD) a deficiency of
all biotin-dependent carboxylases:
\begin{enumerate}
\item acetyl-CoA carboxylase (ACC)
\item propionyl-CoA carboxylase (PCC)
\item 3-methylcrotonyl carboxylase (MCC)
\item pyruvate carboxylase (PC)
\item urea carboxylase (UC)
\end{enumerate}
\end{itemize}

\subsubsection{Clinical Presentation}
\label{sec:org5d728bd}
\begin{itemize}
\item patients with HCS deficiency commonly present with typical MCD
\begin{itemize}
\item organic aciduria, neurological symptoms and skin
disease
\end{itemize}
\item patients with biotinidase deficiency show a less consistent clinical
picture
\item onset in biotinidase deficiency may be insidious and the
manifestation is usually very variable
\begin{itemize}
\item neurological symptoms often being prominent
\item without markedly abnormal organic acid excretion or metabolic
acidosis
\end{itemize}
\item later-onset forms of HCS deficiency cannot be clinically
distinguished from biotinidase deficiency

\item acquired biotin deficiency, which also causes MCD, is extremely rare
\begin{itemize}
\item excessive consumption of raw eggs - avidin
\end{itemize}

\item deficient activity of biotin-dependent carboxylases in both HCS and
biotinidase deficiencies results in accumulation of:
\begin{itemize}
\item lactic acid
\item derivatives of 3-methylcrotonyl-coenzyme A (CoA)
\item derivatives of propionyl-CoA
\end{itemize}
\end{itemize}

\begin{enumerate}
\item HCS Deficiency
\label{sec:org1ff2c94}
\begin{itemize}
\item age of onset varies widely,from a few hours after birth to 8 years
of age
\item \(\sim\) 50\% presented acutely in the first days of life with symptoms
very similar to those observed in other severe organic acidurias
\begin{itemize}
\item lethargy, hypotonia, vomiting, seizures and hypothermia
\end{itemize}
\item infection \(\to\) catabolism \(\to\) acute illness
\end{itemize}

\item Biotinidase Deficiency
\label{sec:orgfa9c057}
\begin{itemize}
\item gradual development of symptoms and episodes of remission
\begin{itemize}
\item may be related to increased free biotin in the diet
\end{itemize}
\item clinical picture has been reported as early as 7 weeks
\item neurological symptoms may occur much earlier, even in neonatal period:
\begin{itemize}
\item lethargy, muscular hypotonia, grand mal and myoclonic seizures, ataxia
\end{itemize}
\item the most frequent initial symptoms are neurological
\item many children also have developmental delay, hearing loss,
conjunctivitis and visual problems, including optic atrophy
\item skin rash and/or alopecia are hallmarks but may develop late or not
at all
\item metabolic acidosis and the characteristic organic aciduria of MCD
are frequently lacking in the early stages of the disease
\item plasma lactate and 3-hydroxyisovalerate may be only slightly
elevated,
\item CSF levels may be significantly higher
\end{itemize}
\end{enumerate}

\subsubsection{Metabolic Derangement}
\label{sec:org8969a7d}
\begin{enumerate}
\item HCS Deficiency
\label{sec:org9bd38c8}
\begin{itemize}
\item decreased affinity of the enzyme for biotin and/or a decreased
maximal velocity lead to reduced formation of the five
holocarboxylases from their corresponding inactive apocarboxylases
at physiological biotin concentrations
\item increased Km for Biotin
\begin{itemize}
\item normally 1-6 nmol/L, patients 9-12 nmol/L
\end{itemize}
\item abnormality of the K m values correlates well with the time of onset
and severity of illness
\begin{itemize}
\item \(\uparrow\) K\(_{\text{M}}\) \(\to\) early onset, severe disease
\end{itemize}

\item mutations outside the biotin-binding site in the N-terminal region
are associated with virtually normal K\(_{\text{M}}\) but decreased V\(_{\text{max}}\)
\begin{itemize}
\item most patients with V\(_{\text{max}}\) mutation respond to a higher biotin
dose and residual biochemical and clinical abnormalities persist
\item response likely due to \(\uparrow\) HLCS mRNA transcription
\end{itemize}
\end{itemize}

\begin{figure}[htbp]
\centering
\includegraphics[width=0.9\textwidth]{./mcd/figures/kinetics.png}
\caption[Kinetics]{\label{fig:orge1ae745}
Holocarboxylase Synthetase Kinetics}
\end{figure}

\item Biotinidase deficiency
\label{sec:org6e574b1}
\begin{itemize}
\item biotin cannot be released from biocytin and short biotinyl
peptides
\begin{itemize}
\item unable to recycle endogenous biotin and use protein-bound dietary biotin
\end{itemize}
\item biotin is lost in the urine, mainly as biocytin
\end{itemize}
\end{enumerate}

\subsubsection{Genetics}
\label{sec:org8055532}
\begin{description}
\item[{HCS}] AR , HLCS
\item[{Biotinidase}] AR, BTD
\begin{itemize}
\item one-third of the alleles, are c.98-104del7ins3 and p.R538C
\item \textasciitilde{} 50\% NBS positive are p.Q456H, the double-mutant allele p.A171T +
p.D444H, and p.D252G
\item almost all individuals with partial biotinidase deficiency have
the p.D444H mutation in combination with a mutation causing
profound biotinidase deficiency on the second allele
\end{itemize}
\end{description}

\subsubsection{Diagnostic Tests}
\label{sec:org70d52a4}
\begin{itemize}
\item A characteristic organic aciduria is the key feature of MCD.
\item unpleasant urine odour (cat’s urine) may even be suggestive of the
defect
\item MCD is reflected in elevated urinary and plasma concentrations of
organic acids as follows:
\begin{itemize}
\item Deficiency of MCC:
\begin{itemize}
\item \(\Uparrow\) urine 3-hydroxyisovaleric acid
\item \(\Uparrow\) plasma 3-hydroxyisovalerylcarnitine (C5-OH)
\item \(\uparrow\) urine 3-methylcrotonylglycine
\item \(\uparrow\) plasma tiglylcarnitine (C5:1)
\end{itemize}
\item Deficiency of PCC:
\begin{itemize}
\item \(\uparrow\) urine methylcitrate
\item \(\uparrow\) urine 3-hydroxypropionate
\item \(\uparrow\) urine propionylglycine
\item \(\uparrow\) urine tiglylglycine
\item \(\uparrow\) urine propionic acid
\item \(\uparrow\) plasma propionylcarnitine (C3)
\end{itemize}
\item Deficiency of PC:
\begin{itemize}
\item \(\Uparrow\) lactate
\item \(\downarrow\) pyruvate
\end{itemize}
\end{itemize}
\item \textbf{NB} a similar organic acid profile can occur in patients with
hyperammonemia due to carbonic anhydrase VA deficiency

\item above pattern seen in HCS during acute illness
\item biotindase deficiency often only \(\uparrow\) urine 3-hydroxyisovalerate
\item biotinidase activity in serum
\item confirm with molecular testing
\end{itemize}

\begin{enumerate}
\item Biotinidase Activity
\label{sec:org2fd2208}
\begin{itemize}
\item initially, most symptomatic children with biotinidase deficiency
were found to have 3\% of mean serum biotinidase activity of normal
individuals
\item three standard deviations above this mean, corresponding to 10\% of
mean normal activity, was taken as the threshold below which
individuals were considered to have profound biotinidase deficiency
\item with NBS for biotinidase deficiency babies were identified with about 25\% of mean normal activity
\begin{itemize}
\item \textasciitilde{} all have p.Asp444His variant as one of their alleles
\end{itemize}
\item this variant, together with a variant for profound deficiency on the
other allele, results in 10–30\% of mean normal biotinidase activity.
\item These children are considered to have partial biotinidase deficiency
\end{itemize}
\end{enumerate}

\subsubsection{Treatment and Prognosis}
\label{sec:orge1cdaa2}
\begin{itemize}
\item oral biotin, at pharmacological dose
\item initiate treatment prior to irreversible neurological damage
\begin{itemize}
\item deafness
\end{itemize}
\item treatment of partial biotinidase deficiency is recommended
\end{itemize}

\section{Cobalamin and Folate}
\label{sec:orgbf0ebac}
\subsection{Introduction}
\label{sec:orgbfaf472}
\subsubsection{Cobalamin - B\(_{\text{12}}\)}
\label{sec:org0622ada}
\begin{itemize}
\item Cobalamin (Cbl or vitamin B\(_{\text{12}}\)) is a cobalt-containing
water-soluble vitamin that is synthesised by lower organisms but not
by higher plants and animals
\item only source of Cbl in the human diet is animal products
\item Cbl is needed for only two reactions in man:
\begin{itemize}
\item MeCbl it is a cofactor of the cytoplasmic enzyme methionine synthase (MTR)
\end{itemize}
\end{itemize}
\ce\{HCY + MeCbl + 5-methylTHF ->[MTR] MET + B\(_{\text{12}}\) + THF\}
\begin{itemize}
\item AdoCbl is a cofactor of the mitochondrial enzyme methylmalonyl-coenzyme A mutase (MUT)
\end{itemize}
\ce{methylmalonyl-CoA ->[MUT + AdoCbl] succinyl-CoA}
\begin{itemize}
\item its metabolism involves complex absorption and transport systems and
multiple intracellular conversions.
\end{itemize}


\begin{figure}[htbp]
\centering
\includegraphics[width=0.9\textwidth]{./b12b9/figures/cbl.png}
\caption{\label{fig:orgdd238d7}
Cobalamin (Cbl) endocytosis and intracellular metabolism}
\end{figure}

\begin{itemize}
\item serum Cbl level is usually low in patients with disorders affecting
absorption and transport of Cbl
\begin{itemize}
\item with the exception of transcobalamin (TC) deficiency
\end{itemize}
\item patients with disorders of intracellular Cbl metabolism typically
have serum Cbl levels within the reference range
\begin{itemize}
\item levels may be reduced in the cblF and cblJ disorders
\end{itemize}
\item disorders of Cbl absorption,transport, MeCbl synthesis result in:
\begin{itemize}
\item homocystinuria, hyperhomocysteinaemia
\item megaloblastic anaemia
\item neurological disorders
\end{itemize}
\item disorders of AdoCbl synthesis result in:
\begin{itemize}
\item methylmalonic aciduria and acidemia (MMA) resulting in metabolic
acidosis
\end{itemize}
\item \(\uparrow\) urine MMA and plasma Hcy are also found in nutritional
vitamin B\(_{\text{12}}\) deficiency
\item severe vitamin B\(_{\text{12}}\) deficiency in newborn infants can occur in
breast fed infants born to vegan mothers or those with sub-clinical
pernicious anaemia can range from:
\begin{itemize}
\item elevation in serum concentration of propionylcarnitine detected by
newborn screening
\item severe neonatal encephalopathy
\end{itemize}
\item mother does not necessarily have a very low serum concentration of
vitamin B\(_{\text{12}}\)
\item IM vitamin B\(_{\text{12}}\) replacement therapy to normalize vitamin B\(_{\text{12}}\) serum
concentration reverses the metabolic abnormality
\end{itemize}

\subsubsection{Folate - B\(_{\text{9}}\)}
\label{sec:org9cb3118}
\begin{itemize}
\item folic acid (pteroylglutamic acid) is plentiful in foods such as
liver, leafy vegetables, legumes and some fruits
\item metabolism involves reduction to dihydrofolate (DHF) and
tetrahydrofolate (THF)
\begin{itemize}
\item followed by addition of a single-carbon unit, which is provided by
serine or histidine; this carbon unit occurs in various redox
states
\begin{itemize}
\item methyl, methylene, methenyl or formyl
\end{itemize}
\end{itemize}
\item transfer of this single-carbon unit is essential for the endogenous
formation of:
\begin{itemize}
\item methionine
\item thymidylate (dTMP)
\item formylglycineamide ribotide (FGAR) and
formylaminoimidazolecarboxamide ribotide (FAICAR) two
intermediates of purine synthesis
\end{itemize}
\item these reactions regenerate DHF and THF
\item the predominant folate derivative in blood and in cerebrospinal
fluid is 5-methylTHF
\begin{itemize}
\item product of the methylenetetrahydrofolate reductase (MTHFR) rxn
\end{itemize}
\end{itemize}

\begin{figure}[htbp]
\centering
\includegraphics[width=0.9\textwidth]{./b12b9/figures/folate.png}
\caption{\label{fig:orgba8bec9}
Folate metabolism}
\end{figure}

\subsection{Cobalamin}
\label{sec:orgc03bcf4}
\subsubsection{Absorption and Transport}
\label{sec:org3494dc0}
\begin{itemize}
\item absorption of dietary Cbl first involves binding to a glyco protein
(haptocorrin,R binder) in the saliva
\item haptocorrin is digested by proteases in the intestine
\begin{itemize}
\item releases Cbl to bind to intrinsic factor (IF)
\end{itemize}
\item IF-Cbl binds to receptor cubam and enters the enterocyte
\item Cbl enters the portal circulation bound to transcobalamin (TC)
\begin{itemize}
\item TC is the physiologically important circulating Cbl-binding
protein
\end{itemize}
\item inherited defects of several of these steps are known
\begin{itemize}
\item Hereditary Intrinsic Factor Deficiency
\item Defective Transport of Cobalamin by Enterocytes
\item Haptocorrin (R Binder) Deficiency
\item Transcobalamin Deficiency
\item Transcobalamin Receptor Deficiency
\end{itemize}
\item all are rare
\end{itemize}
\subsubsection{Cbl mutants (A-G, J, X)}
\label{sec:org208a39f}
\begin{itemize}
\item a number of disorders of intracellular metabolism of Cbl have been
classified as cbl mutants (A-G, J, X)
\begin{itemize}
\item based on the biochemical phenotype and somatic cell analysis
\end{itemize}
\item precise diagnosis of the inborn errors of Cbl metabolism requires
either tests in cultured fibroblasts or identification of causal
mutations
\item complementation analysis can be used to reliably assign a patient to
one of the known classes of inborn error if function of either
methylmalonyl-Coenzyme A (CoA) mutase or methionine synthase is
reduced in patient fibroblasts
\item the one exception is the cblX disorder which cannot be
differentiated from cblC by complementation analysis
\end{itemize}
\subsubsection{Combined AdoCbl and MeCbl Deficiencies}
\label{sec:org0d87332}
\begin{itemize}
\item \emph{cblF, cblJ, cblC, cblX, CblD}
\end{itemize}
\subsubsection{Cobalamin C}
\label{sec:orgc7410e3}
\begin{enumerate}
\item Clinical Presentation
\label{sec:org58ca562}
\begin{itemize}
\item most frequent inborn error of Cbl metabolism
\item many acutely ill in the 1st month of life
\item most were diagnosed within the 1st year
\item early-onset group shows feeding difficulties and lethargy
\begin{itemize}
\item followed by progressive neurological deterioration may include: 
\begin{itemize}
\item hypotonia, hypertonia or both, abnormal movements or seizures
and coma
\end{itemize}
\item severe pancytopenia or a non-regenerative anaemia may be present
\begin{itemize}
\item megaloblastic on bone marrow examination
\end{itemize}
\end{itemize}
\item a small number of cblC patients diagnosed \textgreater{} 1st year of life
\begin{itemize}
\item as late as 4th decade
\end{itemize}
\end{itemize}

\item Metabolic Derangement
\label{sec:orgec67d4e}
\begin{itemize}
\item \emph{cblC} is caused by defects in MMACHC
\item MMACHC binds Cbl and catalyses removal of upper axial ligands from
alkylcobalamins including the methyl group from MeCbl and the
adenosyl group from AdoCbl and from CNCbl
\end{itemize}

\item Genetics
\label{sec:orgc977564}
\begin{itemize}
\item AR, MMACHC
\item c.271dupA accounts for \(\ge\) 40\% of disease alleles in patient
populations of European origin
\end{itemize}

\item Diagnostic Tests
\label{sec:orgc0fd3b0}
\begin{itemize}
\item methylmalonic acidaemia and aciduria are the
biochemical hallmarks of this disease
\begin{itemize}
\item MMA \textless{} MUT deficiency
\item MMA \textgreater{} transport defects
\end{itemize}
\item \(\uparrow\) plasma total homocysteine
\item \(\downarrow\) to normal plasma methionine
\item \(\uparrow\) urine HCY
\end{itemize}

\item Treatment
\label{sec:orgae90442}
\begin{itemize}
\item parenteral OHCbl
\item oral betaine
\end{itemize}
\end{enumerate}

\subsubsection{Cobalamin X}
\label{sec:org07e3d40}
\begin{itemize}
\item \emph{cblX} is caused by mutations in HCFC1
\item same phenotype as \emph{cblC}
\begin{itemize}
\item encodes a transcription regulator that affects expression of a
number of genes, including MMACHC (\emph{cblC})
\end{itemize}
\item The metabolic consequences of mutations stem from decreased MMACHC
expression leading to decreased synthesis of both AdoCbl and MeCbl
\end{itemize}

\subsubsection{Adenosylcobalamin Deficiency}
\label{sec:org27ccd40}
\begin{itemize}
\item \emph{cblA and cblB}
\item characterised by methylmalonic aciduria (MMA)
\item often Cbl-responsive
\item phenotype resembles methylmalonyl-CoA mutase deficiency
\item treated with protein restriction and OHCbl
\end{itemize}

\subsubsection{Methylcobalamin Deficiency}
\label{sec:orgcc916c5}
\begin{itemize}
\item \emph{cblE and cblG}
\item megaloblastic anaemia and neurological disease
\end{itemize}
\subsection{Folate}
\label{sec:org1a0b582}
\begin{itemize}
\item there are a number of very rare disorders of folate absorption and metabolism
\item severe MTHFR deficiency is the most frequent
\end{itemize}
\subsubsection{Methylenetetrahydrofolate Reductase Deficiency}
\label{sec:orgff6d81e}
\begin{itemize}
\item severe form of this deficiency not the polymorphisms associated
common disease risk
\begin{itemize}
\item neural tube defects
\item cardiovascular disease
\end{itemize}
\end{itemize}
\begin{enumerate}
\item Clinical Presentation
\label{sec:org1a79660}
\begin{itemize}
\item most diagnosed in infancy
\item \textgreater{} 50\% present in the 1st year of life
\item common presentation is progressive encephalopathy with apnoea,
seizures and microcephaly
\item not associated with megaloblastic anaemia
\end{itemize}

\item Metabolic Derangement
\label{sec:org6a001a3}
\begin{itemize}
\item \(\downarrow\) methyl-THF
\item methyl-THF is the methyl donor for the conversion of homocysteine \(\to\) methionine
\begin{itemize}
\item \(\uparrow\) total plasma homocysteine
\item \(\downarrow\) methionine
\end{itemize}
\item the block in the conversion of methylene-THF to methyl-THF does not
result in the trapping of folates as methyl-THF and does not
interfere with the availability of reduced folates for purine and
pyrimidine synthesis in contrast to disorders at the level of
methionine synthase
\begin{itemize}
\item explains why patients do not have megaloblastic anaemia
\end{itemize}
\end{itemize}

\item Genetics
\label{sec:orge6ae1c3}
\begin{itemize}
\item AR, MTHFR
\end{itemize}

\item Diagnostic Tests
\label{sec:orga86ccc0}
\begin{itemize}
\item methyl-THF is the major circulating form of folate
\begin{itemize}
\item \(\therefore\) serum folate levels may sometimes be low
\end{itemize}
\item \(\Uparrow\) plasma homocysteine
\item \(\downarrow\) plasma methionine
\end{itemize}

\item Treatment
\label{sec:org984e12a}
\begin{itemize}
\item betaine (trimethylglycine)
\item betaine-homocysteine methyltransferase (BHMT) is betaine
(trimethylglycine)–dependent
\end{itemize}
\ce{trimethylglycine (methyl donor) + homocysteine (hydrogen donor) ->[BHMT] dimethylglycine (hydrogen receiver) + methionine (methyl receiver)}
\begin{itemize}
\item in the liver, BHMT catalyzes up to 50\% of homocysteine metabolism
\end{itemize}
\end{enumerate}

\section{Thiamine and Pyridoxine}
\label{sec:orgb59518f}
\end{document}