% Created 2020-01-11 Sat 11:10
% Intended LaTeX compiler: pdflatex
\documentclass{scrartcl}
\usepackage[utf8]{inputenc}
\usepackage[T1]{fontenc}
\usepackage{graphicx}
\usepackage{grffile}
\usepackage{longtable}
\usepackage{wrapfig}
\usepackage{rotating}
\usepackage[normalem]{ulem}
\usepackage{amsmath}
\usepackage{textcomp}
\usepackage{amssymb}
\usepackage{capt-of}
\usepackage{hyperref}
\hypersetup{colorlinks,linkcolor=black,urlcolor=blue}
\usepackage{textpos}
\usepackage{textgreek}
\usepackage[version=4]{mhchem}
\usepackage{chemfig}
\usepackage{siunitx}
\usepackage{gensymb}
\usepackage[usenames,dvipsnames]{xcolor}
\usepackage[T1]{fontenc}
\usepackage{lmodern}
\usepackage{verbatim}
\usepackage{tikz}
\usepackage{wasysym}
\usetikzlibrary{shapes.geometric,arrows,decorations.pathmorphing,backgrounds,positioning,fit,petri}
\usepackage{fancyhdr}
\pagestyle{fancy}
\author{Matthew Henderson, PhD, FCACB}
\date{\today}
\title{Vitamins}
\hypersetup{
 pdfauthor={Matthew Henderson, PhD, FCACB},
 pdftitle={Vitamins},
 pdfkeywords={},
 pdfsubject={},
 pdfcreator={Emacs 26.1 (Org mode 9.1.9)}, 
 pdflang={English}}
\begin{document}

\maketitle
\setcounter{tocdepth}{2}
\tableofcontents


\section{Biotin}
\label{sec:org206ae6a}
\subsection{Introduction}
\label{sec:orgd6677e1}
\begin{itemize}
\item biotin is a water-soluble vitamin widely present in small amounts in
natural food-stuffs, in which it is mostly protein bound
\item classic role of biotin is to function as the coenzyme of five
important carboxylases involved in gluconeogenesis, fatty acid
synthesis and the catabolism of several amino acids
\item covalent binding of biotin to the inactive apocarboxylases,
catalysed by holocarboxylase synthetase (HCS), is required to
generate the active holocarboxylases
\item recycling of biotin first involves proteolytic degradation of the
holocarboxylases, yielding biotin bound to lysine (biocytin) or to
short biotinyl peptides
\item biotinidase releases biotin
\item transcription of a large number of genes, including those encoding
HCS and the biotin-dependent carboxylases , is regulated by biotin
\item SLC19A3 encoding the thiamine transporter hTHTR2, mutations cause
biotin-responsive basal ganglia disease
\end{itemize}

\begin{figure}[htbp]
\centering
\includegraphics[width=0.9\textwidth]{./biotin/figures/carboxylases.png}
\caption{\label{fig:org6505cbb}
Biotin dependent carboxylases}
\end{figure}


\begin{figure}[htbp]
\centering
\includegraphics[width=0.9\textwidth]{./biotin/figures/biotin.png}
\caption{\label{fig:org93d48de}
Biotin cycle}
\end{figure}

\subsection{Biotin-responsive disorders}
\label{sec:orgbaa37d6}
\begin{itemize}
\item two inherited defects affecting the coenzyme function of biotin:
\begin{enumerate}
\item holocarboxylase synthetase (HCS) deficiency
\item biotinidase deficiency
\end{enumerate}
\item HCS deficiency \(\to\) impaired binding of biotin to apocarboxylases
\item biotinidase deficiency \(\to\) biotin depletion
\item both lead to multiple carboxylase deficiency (MCD) a deficiency of
all biotin-dependent carboxylases:
\begin{enumerate}
\item acetyl-CoA carboxylase (ACC)
\item propionyl-CoA carboxylase (PCC)
\item 3-methylcrotonyl carboxylase (MCC)
\item pyruvate carboxylase (PC)
\item urea carboxylase (UC)
\end{enumerate}
\end{itemize}

\subsubsection{Clinical Presentation}
\label{sec:orged09beb}
\begin{itemize}
\item patients with HCS deficiency commonly present with typical MCD
\begin{itemize}
\item organic aciduria, neurological symptoms and skin
disease
\end{itemize}
\item patients with biotinidase deficiency show a less consistent clinical
picture
\item onset in biotinidase deficiency may be insidious and the
manifestation is usually very variable
\begin{itemize}
\item neurological symptoms often being prominent
\item without markedly abnormal organic acid excretion or metabolic
acidosis
\end{itemize}
\item later-onset forms of HCS deficiency cannot be clinically
distinguished from biotinidase deficiency

\item acquired biotin deficiency, which also causes MCD, is extremely rare
\begin{itemize}
\item excessive consumption of raw eggs - avidin
\end{itemize}

\item deficient activity of biotin-dependent carboxylases in both HCS and
biotinidase deficiencies results in accumulation of:
\begin{itemize}
\item lactic acid
\item derivatives of 3-methylcrotonyl-CoA
\item derivatives of propionyl-CoA
\end{itemize}
\end{itemize}

\begin{enumerate}
\item HCS Deficiency
\label{sec:org09bb2a7}
\begin{itemize}
\item age of onset varies widely, from a few hours after birth to 8 years
of age
\item \(\sim\) 50\% presented acutely in the first days of life with symptoms
similar to other severe organic acidurias
\begin{itemize}
\item lethargy, hypotonia, vomiting, seizures and hypothermia
\end{itemize}
\item infection \(\to\) catabolism \(\to\) acute illness
\end{itemize}

\item Biotinidase Deficiency
\label{sec:orgbb50508}
\begin{itemize}
\item gradual development of symptoms and episodes of remission
\begin{itemize}
\item remission may be related to increased free biotin in the diet
\end{itemize}
\item clinical picture has been reported as early as 7 weeks
\item neurological symptoms may occur much earlier, even in neonatal period:
\begin{itemize}
\item lethargy, muscular hypotonia, grand mal and myoclonic seizures, ataxia
\end{itemize}
\item many children also have developmental delay, hearing loss,
conjunctivitis and visual problems, including optic atrophy
\item skin rash and/or alopecia are hallmarks but may develop late or not
at all
\item metabolic acidosis and the characteristic organic aciduria of MCD
often absent in the early stages of the disease
\begin{itemize}
\item plasma lactate and 3-hydroxyisovalerate may be only slightly
elevated
\item CSF levels may be significantly higher
\end{itemize}
\end{itemize}
\end{enumerate}

\subsubsection{Metabolic Derangement}
\label{sec:org9e45bd7}
\begin{enumerate}
\item HCS Deficiency
\label{sec:orgb5bf995}
\begin{itemize}
\item decreased affinity of the enzyme for biotin and/or a decreased
maximal velocity lead to reduced formation of the five
holocarboxylases from their corresponding inactive apocarboxylases
at physiological biotin concentrations
\item increased Km for Biotin
\begin{itemize}
\item normally 1-6 nmol/L, patients 9-12 nmol/L
\end{itemize}
\item abnormality of the K\(_{\text{M}}\) values correlates well with the time of onset
and severity of illness
\begin{itemize}
\item \(\uparrow\) K\(_{\text{M}}\) \(\to\) early onset, severe disease
\end{itemize}

\item mutations outside the biotin-binding site in the N-terminal region
are associated with virtually normal K\(_{\text{M}}\) but decreased V\(_{\text{max}}\)
\begin{itemize}
\item most patients with V\(_{\text{max}}\) mutation respond to a higher biotin
dose and residual biochemical and clinical abnormalities persist
\item response likely due to \(\uparrow\) HLCS mRNA transcription
\end{itemize}
\end{itemize}

\begin{center}
\includegraphics[width=.9\linewidth]{./figures/kinetics.png}
\end{center}

\begin{figure}[htbp]
\centering
\includegraphics[width=0.9\textwidth]{./biotin/figures/kinetics.png}
\caption[Kinetics]{\label{fig:orgb54b9be}
Holocarboxylase Synthetase Kinetics}
\end{figure}

\item Biotinidase deficiency
\label{sec:org0f902fc}
\begin{itemize}
\item biotin cannot be released from biocytin and short biotinyl
peptides
\begin{itemize}
\item unable to recycle endogenous biotin and use protein-bound dietary biotin
\end{itemize}
\item biotin is lost in the urine, mainly as biocytin
\end{itemize}
\end{enumerate}

\subsubsection{Genetics}
\label{sec:orgcfdd05a}
\begin{description}
\item[{HCS}] AR , HLCS
\item[{Biotinidase}] AR, BTD
\begin{itemize}
\item one-third of the alleles, are c.98-104del7ins3 and p.R538C
\item \textasciitilde{} 50\% NBS positive are p.Q456H, the double-mutant allele p.A171T +
p.D444H, and p.D252G
\item almost all individuals with partial biotinidase deficiency have
the p.D444H mutation in combination with a mutation causing
profound biotinidase deficiency on the second allele
\end{itemize}
\end{description}

\subsubsection{Diagnostic Tests}
\label{sec:orge873f53}
\begin{itemize}
\item A characteristic organic aciduria is the key feature of MCD.
\item unpleasant urine odour (cat’s urine) may even be suggestive of the
defect
\item MCD is reflected in elevated urinary and plasma concentrations of
organic acids as follows:
\begin{itemize}
\item Deficiency of MCC:
\begin{itemize}
\item \(\Uparrow\) urine 3-hydroxyisovaleric acid
\item \(\Uparrow\) plasma 3-hydroxyisovalerylcarnitine (C5-OH)
\item \(\uparrow\) urine 3-methylcrotonylglycine
\item \(\uparrow\) plasma tiglylcarnitine (C5:1)
\end{itemize}
\item Deficiency of PCC:
\begin{itemize}
\item \(\uparrow\) urine methylcitrate
\item \(\uparrow\) urine 3-hydroxypropionate
\item \(\uparrow\) urine propionylglycine
\item \(\uparrow\) urine tiglylglycine
\item \(\uparrow\) urine propionic acid
\item \(\uparrow\) plasma propionylcarnitine (C3)
\end{itemize}
\item Deficiency of PC:
\begin{itemize}
\item \(\Uparrow\) lactate
\item \(\downarrow\) pyruvate
\end{itemize}
\end{itemize}
\item above pattern seen in HCS during acute illness
\item \textbf{NB} a similar organic acid profile can occur in patients with
hyperammonemia due to carbonic anhydrase VA deficiency
\end{itemize}


\begin{itemize}
\item biotindase deficiency often only \(\uparrow\) urine 3-hydroxyisovalerate
\item \(\downarrow\) biotinidase activity in serum
\item confirm with molecular testing
\end{itemize}

\begin{enumerate}
\item Biotinidase Activity
\label{sec:org69974da}
\begin{itemize}
\item most symptomatic children with biotinidase deficiency were found to
have 3\% of mean serum biotinidase activity of normal individuals
\begin{description}
\item[{profound deficiency}] \textless{} 10\% of mean normal activity
\item[{partial deficinecy}] 10-30\% of mean normal activity
\end{description}
\end{itemize}
\end{enumerate}

\subsubsection{Treatment and Prognosis}
\label{sec:org9d8d9cf}
\begin{itemize}
\item oral biotin, at pharmacological dose
\item initiate treatment prior to irreversible neurological damage
\begin{itemize}
\item deafness
\end{itemize}
\item treatment of partial biotinidase deficiency is recommended
\end{itemize}

\section{Cobalamin and Folate}
\label{sec:org058de45}
\subsection{Cobalamin}
\label{sec:org588be19}
\begin{itemize}
\item Cobalamin (Cbl or vitamin B\(_{\text{12}}\)) is a cobalt-containing
water-soluble vitamin that is synthesised by lower organisms but not
by higher plants and animals
\item only source of Cbl in the human diet is animal products
\item Cbl is needed for only two reactions:
\begin{itemize}
\item MeCbl it is a cofactor of the cytoplasmic enzyme methionine synthase (MTR)
\end{itemize}
\end{itemize}
\ce{HCY + MeCbl + 5-methylTHF ->[MTR] MET + B_12 + THF}
\begin{itemize}
\item AdoCbl is a cofactor of the mitochondrial enzyme methylmalonyl-coenzyme A mutase (MUT)
\end{itemize}
\ce{methylmalonyl-CoA ->[MUT + AdoCbl] succinyl-CoA}
\begin{itemize}
\item its metabolism involves complex absorption and transport systems and
multiple intracellular conversions.
\end{itemize}


\begin{figure}[htbp]
\centering
\includegraphics[width=0.9\textwidth]{./b12b9/figures/cbl.png}
\caption{\label{fig:org4c46c12}
Cobalamin (Cbl) endocytosis and intracellular metabolism}
\end{figure}

\begin{itemize}
\item serum Cbl level is usually low in patients with disorders affecting
absorption and transport of Cbl
\begin{itemize}
\item with the exception of transcobalamin (TC) deficiency
\end{itemize}
\item patients with disorders of intracellular Cbl metabolism typically
have serum Cbl levels within the reference range
\begin{itemize}
\item levels may be reduced in the cblF and cblJ disorders
\end{itemize}
\item disorders of Cbl absorption, transport, MeCbl synthesis result in:
\begin{itemize}
\item homocystinuria, hyperhomocysteinaemia
\item megaloblastic anaemia
\item neurological disorders
\end{itemize}
\item disorders of AdoCbl synthesis result in:
\begin{itemize}
\item methylmalonic aciduria and acidemia (MMA) \(\to\) metabolic
acidosis
\end{itemize}
\item \(\uparrow\) urine MMA and plasma HCY are also found in nutritional
vitamin B\(_{\text{12}}\) deficiency
\item severe vitamin B\(_{\text{12}}\) deficiency in newborn infants can occur in
breast fed infants born to vegan mothers or those with sub-clinical
pernicious anaemia can range from:
\begin{itemize}
\item elevation in serum concentration of propionylcarnitine detected by
newborn screening
\item severe neonatal encephalopathy
\end{itemize}
\item mother does not necessarily have a very low serum concentration of
vitamin B\(_{\text{12}}\)
\item IM vitamin B\(_{\text{12}}\) replacement therapy to normalize vitamin B\(_{\text{12}}\) serum
concentration reverses the metabolic abnormality
\end{itemize}

\subsubsection{Absorption and Transport}
\label{sec:orgcfc82c4}
\begin{itemize}
\item absorption of dietary Cbl first involves binding to a glyco protein
(haptocorrin,R binder) in the saliva
\item haptocorrin is digested by proteases in the intestine
\begin{itemize}
\item releases Cbl to bind to intrinsic factor (IF)
\end{itemize}
\item IF-Cbl binds to receptor cubam and enters the enterocyte
\item Cbl enters the portal circulation bound to transcobalamin (TC)
\begin{itemize}
\item TC is the physiologically important circulating Cbl-binding
protein
\end{itemize}
\item inherited defects of several of these steps are known
\begin{itemize}
\item Hereditary Intrinsic Factor Deficiency
\item Defective Transport of Cobalamin by Enterocytes
\item Haptocorrin (R Binder) Deficiency
\item Transcobalamin Deficiency
\item Transcobalamin Receptor Deficiency
\end{itemize}
\item all are rare
\end{itemize}
\subsubsection{Cbl mutants (A-G, J, X)}
\label{sec:orga58de39}
\begin{itemize}
\item a number of disorders of intracellular metabolism of Cbl have been
classified as cbl mutants (A-G, J, X)
\begin{itemize}
\item based on the biochemical phenotype and somatic cell analysis
\end{itemize}
\item precise diagnosis of the inborn errors of Cbl metabolism requires
either tests in cultured fibroblasts or identification of causal
mutations
\item complementation analysis can be used to reliably assign a patient to
one of the known classes of inborn error if function of either
methylmalonyl-Coenzyme A (CoA) mutase or methionine synthase is
reduced in patient fibroblasts
\item the one exception is the cblX disorder which cannot be
differentiated from cblC by complementation analysis
\end{itemize}
\subsubsection{Combined AdoCbl and MeCbl Deficiencies}
\label{sec:org25aabb9}
\begin{itemize}
\item \emph{cblF, cblJ, cblC, cblX, CblD}
\end{itemize}
\subsubsection{Cobalamin C}
\label{sec:org38e7800}
\begin{enumerate}
\item Clinical Presentation
\label{sec:org1e74088}
\begin{itemize}
\item most frequent inborn error of Cbl metabolism
\item many acutely ill in the 1st month of life
\item most were diagnosed within the 1st year
\item early-onset group shows feeding difficulties and lethargy
\begin{itemize}
\item followed by progressive neurological deterioration may include: 
\begin{itemize}
\item hypotonia, hypertonia or both, abnormal movements or seizures
and coma
\end{itemize}
\item severe pancytopenia or a non-regenerative anaemia may be present
\begin{itemize}
\item megaloblastic on bone marrow examination
\end{itemize}
\end{itemize}
\item a small number of cblC patients diagnosed \textgreater{} 1st year of life
\begin{itemize}
\item as late as 4th decade
\end{itemize}
\end{itemize}

\item Metabolic Derangement
\label{sec:orgb0ce07a}
\begin{itemize}
\item \emph{cblC} is caused by defects in MMACHC
\item MMACHC binds Cbl and catalyses removal of upper axial ligands from
alkylcobalamins including the methyl group from MeCbl and the
adenosyl group from AdoCbl and from CNCbl
\end{itemize}

\item Genetics
\label{sec:orgfbe6e8b}
\begin{itemize}
\item AR, MMACHC
\item c.271dupA accounts for \(\ge\) 40\% of disease alleles in patient
populations of European origin
\end{itemize}

\item Diagnostic Tests
\label{sec:org9c38bfb}
\begin{itemize}
\item methylmalonic acidaemia and aciduria are the
biochemical hallmarks of this disease
\begin{itemize}
\item MMA \textless{} MUT deficiency
\item MMA \textgreater{} transport defects
\end{itemize}
\item \(\uparrow\) plasma total homocysteine
\item \(\downarrow\) to normal plasma methionine
\item \(\uparrow\) urine HCY
\end{itemize}

\item Treatment
\label{sec:org5c26ad8}
\begin{itemize}
\item parenteral OHCbl
\item oral betaine (trimethylglycine)
\item betaine-homocysteine methyltransferase (BHMT) is betaine dependent
\end{itemize}
\ce{trimethylglycine + homocysteine ->[BHMT] dimethylglycine + methionine}
\begin{itemize}
\item in the liver, BHMT catalyzes up to 50\% of homocysteine metabolism
\end{itemize}
\end{enumerate}
\subsubsection{Cobalamin X}
\label{sec:org41b36b5}
\begin{itemize}
\item \emph{cblX} is caused by mutations in HCFC1
\item same phenotype as \emph{cblC}
\begin{itemize}
\item encodes a transcription regulator that affects expression of a
number of genes, including MMACHC (\emph{cblC})
\end{itemize}
\item The metabolic consequences of mutations stem from decreased MMACHC
expression leading to decreased synthesis of both AdoCbl and MeCbl
\end{itemize}

\subsubsection{Adenosylcobalamin Deficiency}
\label{sec:orgd0cff4c}
\begin{itemize}
\item \emph{cblA and cblB}
\item characterised by methylmalonic aciduria (MMA)
\item often Cbl-responsive
\item phenotype resembles methylmalonyl-CoA mutase deficiency
\item treated with protein restriction and OHCbl
\end{itemize}

\subsubsection{Methylcobalamin Deficiency}
\label{sec:orgc4cfd98}
\begin{itemize}
\item \emph{cblE and cblG}
\item megaloblastic anaemia and neurological disease
\end{itemize}
\subsection{Folate}
\label{sec:org9919074}
\begin{itemize}
\item folic acid (pteroylglutamic acid) is plentiful in foods such as
liver, leafy vegetables, legumes and some fruits
\item metabolism involves reduction to dihydrofolate (DHF) and
tetrahydrofolate (THF)
\begin{itemize}
\item followed by addition of a single-carbon unit, which is provided by
serine or histidine; this carbon unit occurs in various redox
states
\begin{itemize}
\item methyl, methylene, methenyl or formyl
\end{itemize}
\end{itemize}
\item transfer of this single-carbon unit is essential for the endogenous
formation of:
\begin{itemize}
\item methionine
\item thymidylate (dTMP)
\item formylglycineamide ribotide (FGAR) and
formylaminoimidazolecarboxamide ribotide (FAICAR) two
intermediates of purine synthesis
\end{itemize}
\item these reactions regenerate DHF and THF
\item the predominant folate derivative in blood and in cerebrospinal
fluid is 5-methyltetrahydrofolate
\begin{itemize}
\item product of the methylenetetrahydrofolate reductase (MTHFR) rxn
\end{itemize}
\end{itemize}

\begin{figure}[htbp]
\centering
\includegraphics[width=1.0\textwidth]{./b12b9/figures/folate.png}
\caption{\label{fig:orgc8831d7}
Folate metabolism:1 methionine synthase; 2 methylenetetrahydrofolate reductase; 3 methylenetetrahydrofolate dehydrogenase; 4 methenyltetrahydrofolate cyclohydrolase: 5 formyltetrahydrofolate synthetase; 6 dihydrofolate reductase; 7 glutamate formiminotransferase; 8 formiminotetrahydrofolate cyclodeaminase}
\end{figure}

\begin{itemize}
\item there are a number of very rare disorders of folate absorption and metabolism
\item severe MTHFR deficiency is the most frequent
\end{itemize}
\subsubsection{Methylenetetrahydrofolate Reductase Deficiency}
\label{sec:org8aad070}
\begin{itemize}
\item severe form of this deficiency not the polymorphisms associated
common disease risk
\begin{itemize}
\item neural tube defects
\item cardiovascular disease
\end{itemize}
\end{itemize}
\begin{enumerate}
\item Clinical Presentation
\label{sec:org959cde4}
\begin{itemize}
\item most diagnosed in infancy
\item \textgreater{} 50\% present in the 1st year of life
\item common presentation is progressive encephalopathy with apnoea,
seizures and microcephaly
\item not associated with megaloblastic anaemia
\end{itemize}

\item Metabolic Derangement
\label{sec:orgf99d972}
\begin{itemize}
\item \(\downarrow\) methyl-THF
\item methyl-THF is the methyl donor for the conversion of homocysteine \(\to\) methionine
\begin{itemize}
\item \(\uparrow\) total plasma homocysteine
\item \(\downarrow\) methionine
\end{itemize}
\item the block in the conversion of methylene-THF to methyl-THF does not
result in the trapping of folates as methyl-THF
\begin{itemize}
\item \(\therefore\) does not \(\downarrow\) reduced folates for purine and
pyrimidine synthesis
\item contrast to disorders at the level of methionine synthase
\item explains why patients do not have megaloblastic anaemia
\end{itemize}
\end{itemize}

\item Genetics
\label{sec:org0fa9df8}
\begin{itemize}
\item AR, MTHFR
\end{itemize}

\item Diagnostic Tests
\label{sec:org58baa3a}
\begin{itemize}
\item methyl-THF is the major circulating form of folate
\begin{itemize}
\item \(\therefore\) serum folate levels may sometimes be low
\end{itemize}
\item \(\Uparrow\) plasma homocysteine
\item \(\downarrow\) plasma methionine
\end{itemize}

\item Treatment
\label{sec:org49ad0dc}
\begin{itemize}
\item betaine (trimethylglycine)
\item betaine-homocysteine methyltransferase (BHMT) is betaine dependent
\end{itemize}
\ce{trimethylglycine + homocysteine ->[BHMT] dimethylglycine + methionine}
\begin{itemize}
\item in the liver, BHMT catalyzes up to 50\% of homocysteine metabolism
\end{itemize}
\end{enumerate}

\section{Thiamine and Pyridoxine}
\label{sec:org1531171}
\subsection{Thiamine}
\label{sec:org07208a2}
\begin{itemize}
\item thiamine (vitamin B\(_{\text{1}}\)) is water soluble
\item transported across cell membranes by THTR1 and THTR2
\begin{itemize}
\item encoded by the SLC19A2 and SLC19A3
\end{itemize}
\item both are widely expressed in the body
\item differ in kinetic properties and level of expression in different tissues
\item THTR2 is the major transporter in the upper small intestine where
dietary thiamine is absorbed at the luminal surface
\item THTR1 predominates at the basal surface
\item active cofactor of thiamine, TPP, is formed in the cytoplasm by the
enzyme thiamine pyrophosphokinase
\item TPP cofactor is attached directly to the transketolase and 2-hydroxyacyl CoA lyase apoproteins
\item a TPP transporter (MTPPT) in the inner mitochondrial membrane
delivers the cofactor to the \(\alpha\)-ketoacid dehydrogenases in the
mitochondrial matrix
\end{itemize}

\begin{figure}[htbp]
\centering
\includegraphics[width=0.6\textwidth]{./b1b6/figures/thiamine.png}
\caption{\label{fig:orga2d055a}
Thiamine transport}
\end{figure}

\subsubsection{Defective Transport}
\label{sec:org256bb13}
\begin{itemize}
\item THTR1 or 2 deficiency
\end{itemize}
\subsubsection{TPP biosynthesis}
\label{sec:org1802fac}
\begin{itemize}
\item Thiamine pyrophosphokinase (TPK1) deficiency
\item \(\uparrow\) blood and CSF lactate
\item \(\uparrow\) urine \(\alpha\)-ketoglutarate
\item \(\downarrow\) TPP
\end{itemize}
\subsubsection{Mitochondrial transport}
\label{sec:org5a24883}
\begin{itemize}
\item mitochondrial TPP transporter deficiency - amish lethal microcephaly
\item \(\uparrow\) urine \(\alpha\)-ketoglutarate
\end{itemize}
\subsubsection{Binding to apo-enzyme}
\label{sec:orgd584905}
\begin{itemize}
\item PDHC
\begin{itemize}
\item \(\uparrow\) lactate
\item normal L/P ratio
\end{itemize}
\item BCKAD
\begin{itemize}
\item \(\uparrow\) BCAAs
\end{itemize}
\end{itemize}

\subsection{Pyridoxine}
\label{sec:orgabf3f78}
\begin{itemize}
\item pyridoxine (vitamin B\(_{\text{6}}\)) is water soluble
\item broad availability from various food sources
\item three vitamers and their phosphorylated esters are absorbed in the
small intestine:
\begin{enumerate}
\item pyridoxal
\item pyridoxamine
\item pyridoxine
\end{enumerate}

\item phosphorylated forms undergo dephosphorylation for cellular uptake
by tissue non-specific alkaline phosphatase (TNSAP)
\begin{itemize}
\item transport mechanism of B\(_{\text{6}}\) vitamers across cell membranes not
elucidated
\end{itemize}
\item within cells vitamers are rephosphorylated by kinases and further oxidised to the
active cofactor pyridoxal 5’-phosphate (PLP) by pyridox(am)ine
5’-phosphate oxidase(PNPO)
\item liver seems to be the most important organ of PLP formation
\begin{itemize}
\item PNPO is expressed in various cell types including neurons
\end{itemize}
\item PLP is one of the most abundant cofactors and participates in over
140 reactions in amino acid and neurotransmitter metabolism
\item systemic vitamin B\(_{\text{6}}\) deficiency causes seizures, failure to thrive
and anemia
\item nutritional vitamin B\(_{\text{6}}\) deficiency is rarely seen
\begin{itemize}
\item usually occurs together with other vitamin deficiencies in
malnutrition or in association with severe chronic disease
\end{itemize}
\end{itemize}


\begin{figure}[htbp]
\centering
\includegraphics[width=0.9\textwidth]{./b1b6/figures/pyridoxine.png}
\caption{\label{fig:orgca7ad02}
Pyridoxine metabolism}
\end{figure}

\begin{itemize}
\item several mechanisms that lead to an increased requirement for
pyridoxine and/or PLP:
\begin{enumerate}
\item inborn errors affecting the pathways of B\(_{\text{6}}\) vitamer metabolism
\begin{itemize}
\item PNPO and alkaline phosphatase defects
\end{itemize}
\item inborn errors that lead to accumulation of small molecules that
react with PLP and inactivate it
\begin{itemize}
\item hyperprolinemia type II
\item antiquitin deficiency (pyridoxine dependent epilepsy)
\end{itemize}
\item inborn errors affecting specific PLP dependent enzymes
\begin{itemize}
\item X-linked sideroblastic anemia
\item classical homocystinuria
\item gyrate atrophy of the choroid
\end{itemize}
\item drugs as D-penicillamine or isozianid thataffect the metabolism of
B 6 vitamers or react with PLP
\item coeliac disease(malabsorption) or renal dialysis(loss)
\end{enumerate}
\item inborn errors of metabolism \(\to\) \(\downarrow\) PLP
\begin{itemize}
\item by inactivation, reduced formation, or reduced cellular uptake of PLP
\item all autosomal recessive
\item seizures are a hallmark of the disease
\begin{itemize}
\item no or incomplete response to common anticonvulsants
\item good response to pyridoxine or PLP
\end{itemize}
\end{itemize}
\end{itemize}

\subsubsection{Antiquitin deficiency}
\label{sec:orgcfddf29}
\begin{itemize}
\item presents in neonates with myoclonic and tonic seizures or
status epilepticus
\item onset up to 3 years observed
\item antiquitin (ALDH7A1) encodes for \(\alpha\)-aminoadipic semialdehyde dehydrogenase
\begin{itemize}
\item involved in lysine degradation
\item deficiency results in \(\uparrow\) \(\alpha\)-aminoadipic acid semialdehyde (AASA)
\begin{itemize}
\item AASA is in equilibrium with piperideine-6-carboxylate (PC6)
\item PC6 inactivates PLP
\end{itemize}
\end{itemize}
\item simultaneous determination of sulfocysteine is crucial to exclude
molybdenum cofactor and sulfite oxidase deficiency causing secondary
inhibition of antiquitin
\item most common form of pyridoxine dependent epilepsy (PDE)
\item \(\uparrow\) urine AASA (\(\alpha\)-aminoadipic acid)
\item \(\uparrow\) urine P6C (piperideine-6-carboxylate)
\item \(\uparrow\) urine P5C (pyroline-5-carboxylate)
\item pipecolic acid in plasma, the first described biomarker of PDE, is
less specific as it can also be found in peroxisomal disease and has
been found normal in older patients while on pyridoxine
\item treated with pyridoxine
\end{itemize}

\begin{figure}[htbp]
\centering
\includegraphics[width=0.9\textwidth]{./b1b6/figures/lysine_deg.png}
\caption{\label{fig:orgaf25576}
Lysine degradation and antiquitin deficiency (blue bar)}
\end{figure}

\subsubsection{Hyperprolinemia Type II}
\label{sec:org58f81bf}
\begin{itemize}
\item attenuated phenotype
\item \textasciitilde{} 50\% present with seizures
\item \(\uparrow\) inactivating compound P5C due to deficiency of pyrroline-5-carboxylate dehydrogenase
\item \(\Uparrow\) plasma proline
\item \(\uparrow\) urine P5C
\end{itemize}

\subsubsection{PNPO deficiency}
\label{sec:org128b3d1}
\begin{itemize}
\item clinically indistinguishable from antiquitin deficiency
\item severe (systemic) PLP deficiency and impaired function of PLP
dependent enzymes
\item \(\uparrow\) urine vanillactate
\item \(\uparrow\) pyridoxamine
\item \(\uparrow\) pyridoxamine/pyridoxic acids
\item treat with oral PLP
\end{itemize}

\subsubsection{Congenital Hypophosphatasia}
\label{sec:org811a671}
\begin{itemize}
\item severe form \(\to\) neonatal seizures
\item osteomalacia
\item Tissue Non Specific Alkaline Phosphatase (TNSAP) deficiency
\begin{itemize}
\item impaired dephosphorylation of PLP for cellular uptake
\end{itemize}
\item \(\Downarrow\) plasma alkaline phosphatase
\item \(\uparrow\) serum calcium
\item \(\downarrow\) serum phosphate
\item \(\uparrow\) plasma phosphoethanolamine
\end{itemize}
\end{document}