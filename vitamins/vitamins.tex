% Created 2020-08-04 Tue 09:58
% Intended LaTeX compiler: pdflatex
\documentclass[12pt]{scrartcl}
\usepackage[utf8]{inputenc}
\usepackage[T1]{fontenc}
\usepackage{graphicx}
\usepackage{grffile}
\usepackage{longtable}
\usepackage{wrapfig}
\usepackage{rotating}
\usepackage[normalem]{ulem}
\usepackage{amsmath}
\usepackage{textcomp}
\usepackage{amssymb}
\usepackage{capt-of}
\usepackage{hyperref}
\hypersetup{colorlinks,linkcolor=black,urlcolor=blue}
\usepackage{textpos}
\usepackage{textgreek}
\usepackage[version=4]{mhchem}
\usepackage{chemfig}
\usepackage{siunitx}
\usepackage{gensymb}
\usepackage[usenames,dvipsnames]{xcolor}
\usepackage{lmodern}
\usepackage{verbatim}
\usepackage{tikz}
\usepackage{wasysym}
\usetikzlibrary{shapes.geometric,arrows,decorations.pathmorphing,backgrounds,positioning,fit,petri}
\usepackage[automark, autooneside=false, headsepline]{scrlayer-scrpage}
\clearpairofpagestyles
\ihead{\leftmark}% section on the inner (oneside: right) side
\ohead{\rightmark}% subsection on the outer (oneside: left) side
\addtokomafont{pagehead}{\upshape}% header upshape instead of italic
\ofoot*{\pagemark}% the pagenumber in the center of the foot, also on plain pages
\pagestyle{scrheadings}
\author{Matthew Henderson, PhD, FCACB}
\date{\today}
\title{Vitamins}
\hypersetup{
 pdfauthor={Matthew Henderson, PhD, FCACB},
 pdftitle={Vitamins},
 pdfkeywords={},
 pdfsubject={},
 pdfcreator={Emacs 26.3 (Org mode 9.3.7)}, 
 pdflang={English}}
\begin{document}

\maketitle
\setcounter{tocdepth}{2}
\tableofcontents


\setchemfig{atom style={scale=0.75}}

\section{Biotin}
\label{sec:org395f75c}
\subsection{Introduction}
\label{sec:orgce6fce4}
\begin{itemize}
\item biotin is a water-soluble vitamin widely present in small amounts in
natural food-stuffs, in which it is mostly protein bound
\item coenzyme for five important carboxylases involved in
gluconeogenesis, fatty acid synthesis and the catabolism of several
amino acids
\begin{itemize}
\item acetyl-CoA carboxylase (ACC)
\end{itemize}
\ce{acetyl-CoA ->[ACC] malonyl-CoA} 
\begin{itemize}
\item propionyl-CoA carboxylase (PCC)
\end{itemize}
\ce{propionyl-CoA ->[PCC] methlymalonyl-CoA}
\begin{itemize}
\item 3-methylcrotonyl carboxylase (MCC)
\end{itemize}
\ce{3-methycrotonyl-CoA ->[MCC] 3-methylglutaconyl-CoA}
\begin{itemize}
\item pyruvate carboxylase (PC)
\end{itemize}
\ce{pyruvate + ATP + CO2 ->[PC] oxaloacetate + ADP + Pi}
\begin{itemize}
\item urea carboxylase (UC)
\end{itemize}
\ce{ATP + urea + HCO3- ->[UC] ADP + Pi + urea-1-carboxylate}

\item covalent binding of biotin to the inactive apocarboxylases
catalysed by holocarboxylase synthetase (HCS) is required to
generate the active holocarboxylases
\item recycling of biotin first involves proteolytic degradation of the
holocarboxylases, yielding biotin bound to lysine (biocytin) or to
short biotinyl peptides
\begin{itemize}
\item biotinidase releases biotin
\end{itemize}
\item transcription of a large number of genes, including those encoding
HCS and the biotin-dependent carboxylases are regulated by biotin
\item mutations in SLC19A3 (thiamine transporter hTHTR2) cause
biotin-responsive basal ganglia disease
\end{itemize}

\begin{figure}[htbp]
\centering
\includegraphics[width=0.9\textwidth]{biotin/figures/carboxylases.png}
\caption{\label{fig:orge72a953}Biotin Dependent Carboxylases}
\end{figure}

\begin{figure}[htbp]
\centering
\includegraphics[width=0.9\textwidth]{biotin/figures/biotin.png}
\caption{\label{fig:org2525176}Biotin Cycle}
\end{figure}

\begin{figure}[htbp]
\centering
\includegraphics[width=0.9\textwidth]{biotin/figures/Slide25.png}
\caption{\label{fig:orgf5dffdd}Biotin Cycle}
\end{figure}

\subsection{Biotin-Responsive Disorders}
\label{sec:orgdbed711}
\begin{itemize}
\item two inherited defects affecting the coenzyme function of biotin:
\begin{enumerate}
\item \textbf{holocarboxylase synthetase (HCS)} deficiency
\begin{itemize}
\item impaired binding of biotin to apocarboxylases
\end{itemize}
\item \textbf{biotinidase} deficiency
\begin{itemize}
\item biotin depletion
\end{itemize}
\end{enumerate}

\item both result in multiple carboxylase deficiency (MCD) a deficiency of
all biotin-dependent carboxylases:
\begin{enumerate}
\item acetyl-CoA carboxylase (ACC)
\item propionyl-CoA carboxylase (PCC)
\item 3-methylcrotonyl carboxylase (MCC)
\item pyruvate carboxylase (PC)
\item urea carboxylase (UC)
\end{enumerate}
\end{itemize}

\subsubsection{Clinical Presentation}
\label{sec:orga4a04fb}
\begin{itemize}
\item patients with HCS deficiency commonly present with typical MCD
\begin{itemize}
\item organic aciduria, neurological symptoms and skin
disease
\end{itemize}
\item patients with biotinidase deficiency show a less consistent clinical
picture
\item onset in biotinidase deficiency may be insidious and the
manifestation is usually very variable
\begin{itemize}
\item neurological symptoms often being prominent
\item without markedly abnormal organic acid excretion or metabolic
acidosis
\end{itemize}
\item later-onset forms of HCS deficiency cannot be clinically
distinguished from biotinidase deficiency

\item acquired biotin deficiency, which also causes MCD, is extremely rare
\begin{itemize}
\item excessive consumption of raw eggs - avidin
\end{itemize}
\end{itemize}
\begin{enumerate}
\item HCS Deficiency
\label{sec:orgb242a93}
\begin{itemize}
\item age of onset varies widely from a few hours after birth to 8 years
of age
\item \(\sim\) 50\% presented acutely in the first days of life with symptoms
similar to other severe organic acidurias
\begin{itemize}
\item lethargy, hypotonia, vomiting, seizures and hypothermia
\end{itemize}
\item infection \(\to\) catabolism \(\to\) acute illness
\end{itemize}

\item Biotinidase Deficiency
\label{sec:org6d7b5ae}
\begin{itemize}
\item gradual development of symptoms and episodes of remission
\begin{itemize}
\item remission may be related to increased free biotin in the diet
\end{itemize}
\item clinical picture has been reported as early as 7 weeks
\item neurological symptoms may occur much earlier, even in neonatal period
\begin{itemize}
\item lethargy, muscular hypotonia, grand mal and myoclonic seizures, ataxia
\end{itemize}
\item many children also have developmental delay, hearing loss,
conjunctivitis and visual problems including optic atrophy
\item skin rash and/or alopecia are hallmarks but may develop late or not
at all
\item metabolic acidosis and the characteristic organic aciduria of MCD
often absent in the early stages of the disease
\begin{itemize}
\item plasma lactate and 3-hydroxyisovalerate may be only slightly
elevated
\item CSF levels may be significantly higher
\end{itemize}
\end{itemize}
\end{enumerate}

\subsubsection{Metabolic Derangement}
\label{sec:orgfdc59ba}
\begin{itemize}
\item deficient activity of biotin-dependent carboxylases in both HCS and
biotinidase deficiencies results in accumulation of:
\begin{itemize}
\item lactic acid
\item derivatives of 3-methylcrotonyl-CoA
\item derivatives of propionyl-CoA
\end{itemize}
\end{itemize}
\begin{enumerate}
\item HCS Deficiency
\label{sec:org0c9a702}
\begin{itemize}
\item decreased affinity of the enzyme for biotin and/or a decreased
maximal velocity lead to reduced formation of the five
holocarboxylases from their corresponding inactive apocarboxylases
at physiological biotin concentrations
\begin{itemize}
\item increased K\textsubscript{M} for biotin
\begin{itemize}
\item normally 1-6 nmol/L, patients 9-12 nmol/L
\end{itemize}
\item abnormality of the K\textsubscript{M} values correlates well with the time of onset
and severity of illness
\begin{itemize}
\item \(\uparrow\) K\textsubscript{M} \(\to\) early onset, severe disease
\end{itemize}
\end{itemize}

\item mutations outside the biotin-binding site are associated with
virtually normal K\textsubscript{M} but decreased V\textsubscript{max}
\begin{itemize}
\item most patients with V\textsubscript{max} mutation respond to a higher biotin
dose and residual biochemical and clinical abnormalities persist
\item response likely due to \(\uparrow\) HLCS mRNA transcription
\end{itemize}
\end{itemize}

\item Biotinidase Deficiency
\label{sec:orgffcef41}
\begin{itemize}
\item biotin cannot be released from biocytin and short biotinyl
peptides
\begin{itemize}
\item unable to recycle endogenous biotin and use protein-bound dietary biotin
\end{itemize}
\item biotin is lost in the urine, mainly as biocytin
\end{itemize}
\end{enumerate}

\subsubsection{Genetics}
\label{sec:org9b4b646}
\begin{description}
\item[{HCS}] AR HLCS
\item[{Biotinidase}] AR BTD
\end{description}

\subsubsection{Diagnostic Tests}
\label{sec:orge659a5a}
\begin{itemize}
\item characteristic organic aciduria is the key feature of MCD
\item unpleasant urine odour (cat’s urine) may even be suggestive of the
defect
\item MCD is reflected in elevated urinary and plasma concentrations of
organic acids as follows:
\begin{itemize}
\item \(\downarrow\) MCC activity:
\begin{itemize}
\item \(\Uparrow\) urine 3-hydroxyisovaleric acid
\item \(\uparrow\) urine 3-methylcrotonylglycine
\item \(\Uparrow\) plasma 3-hydroxyisovaleryl-carnitine (C5-OH)
\item \(\uparrow\) plasma tiglyl-carnitine (C5:1)
\end{itemize}
\item \(\downarrow\) PCC activity:
\begin{itemize}
\item \(\uparrow\) urine methylcitrate
\item \(\uparrow\) urine 3-hydroxypropionate
\item \(\uparrow\) urine propionylglycine
\item \(\uparrow\) urine tiglylglycine
\item \(\uparrow\) urine propionic acid
\item \(\uparrow\) plasma propionyl-carnitine (C3)
\end{itemize}
\item \(\downarrow\) PC activity:
\begin{itemize}
\item \(\Uparrow\) lactate
\item \(\downarrow\) pyruvate
\end{itemize}
\end{itemize}
\item above pattern seen in HCS during acute illness
\item \textbf{NB} a similar organic acid profile can occur in patients with
hyperammonemia due to carbonic anhydrase VA deficiency (Section Urea Cycle)
\begin{itemize}
\item supplies bicarbonate to carboxylases
\end{itemize}
\item biotindase deficiency often only \(\uparrow\) urine 3-hydroxyisovalerate
\item \(\downarrow\) biotinidase activity in serum
\item confirm with molecular testing

\item biotinidase activity measurement
\begin{itemize}
\item most symptomatic children with biotinidase deficiency were found to
have 3\% of mean serum biotinidase activity of normal individuals
\begin{description}
\item[{profound deficiency}] \textless{} 10\% of mean normal activity
\item[{partial deficinecy}] 10-30\% of mean normal activity
\end{description}
\end{itemize}
\end{itemize}

\subsubsection{Treatment and Prognosis}
\label{sec:orgc0dfe32}
\begin{itemize}
\item oral pharmacological dose of biotin
\item initiate treatment prior to irreversible neurological damage
\begin{itemize}
\item deafness
\end{itemize}
\item treatment of partial biotinidase deficiency is recommended
\end{itemize}

\section{Cobalamin and Folate}
\label{sec:orgaa3874f}
\subsection{Cobalamin}
\label{sec:org69da5e2}
\begin{itemize}
\item cobalamin (Cbl or vitamin B\textsubscript{12}) is a cobalt-containing
water-soluble vitamin that is synthesised by lower organisms but not
by higher plants and animals
\item only source of Cbl in the human diet is animal products
\item \textbf{Cbl is needed for only two reactions}
\begin{enumerate}
\item \textbf{MeCbl} is a cofactor of the cytoplasmic enzyme methionine synthase (MS)
\begin{itemize}
\item \ce{homocysteine + methyl-THF ->[MS + MeCbl] methionine + THF}
\end{itemize}
\item \textbf{AdoCbl} is a cofactor of the mitochondrial enzyme methylmalonyl-CoA mutase (MUT)
\begin{itemize}
\item \ce{methylmalonyl-CoA ->[MUT + AdoCbl] succinyl-CoA}
\end{itemize}
\end{enumerate}
\item Cbl metabolism involves complex absorption and transport systems and
multiple intracellular conversions
\end{itemize}

\begin{figure}[htbp]
\centering
\includegraphics[width=0.9\textwidth]{b12b9/figures/cbl.png}
\caption{\label{fig:org27d1890}Cobalamin Transport and Metabolism}
\end{figure}

\begin{figure}[htbp]
\centering
\includegraphics[width=0.9\textwidth]{b12b9/figures/Slide24.png}
\caption{\label{fig:orgb8afbd3}Cobalamin Absorption, Transport and Metabolism}
\end{figure}

\begin{itemize}
\item serum Cbl level is usually low in patients with disorders affecting
absorption and transport of Cbl
\begin{itemize}
\item with the exception of transcobalamin (TC) deficiency
\end{itemize}
\item patients with disorders of intracellular Cbl metabolism typically
have serum Cbl levels within the reference range
\begin{itemize}
\item levels may be reduced in the CblF and CblJ disorders
\end{itemize}
\item disorders of Cbl absorption, transport and MeCbl synthesis result in:
\begin{itemize}
\item homocystinuria, homocysteinaemia
\item megaloblastic anaemia
\item neurological disorders
\end{itemize}
\item disorders of AdoCbl synthesis result in:
\begin{itemize}
\item methylmalonic aciduria and acidemia (MMA) \(\to\) metabolic
acidosis
\end{itemize}
\item \(\uparrow\) urine MMA and plasma HCY are also found in nutritional
vitamin B\textsubscript{12} deficiency
\item severe vitamin B\textsubscript{12} deficiency in newborn infants can occur in
breast fed infants born to vegan mothers or those with sub-clinical
pernicious anaemia can range from:
\begin{itemize}
\item elevation in serum concentration of propionylcarnitine detected by
newborn screening
\item severe neonatal encephalopathy
\end{itemize}
\item mother does not necessarily have a very low serum concentration of
vitamin B\textsubscript{12}
\item IM vitamin B\textsubscript{12} replacement therapy to normalize vitamin B\textsubscript{12} serum
concentration reverses the metabolic abnormality
\end{itemize}

\subsubsection{Absorption and Transport}
\label{sec:org59b0ab5}
\begin{itemize}
\item absorption of dietary Cbl first involves binding to a glycoprotein
(haptocorrin, R binder) in the saliva
\item haptocorrin is digested by proteases in the intestine
\begin{itemize}
\item releases Cbl to bind to intrinsic factor (IF)
\end{itemize}
\item IF-Cbl binds to receptor cubam and enters the enterocyte
\item Cbl enters the portal circulation bound to transcobalamin (TC)
\begin{itemize}
\item TC is the physiologically important circulating Cbl binding
protein
\end{itemize}
\item inherited defects of several of these steps are known
\begin{itemize}
\item Hereditary Intrinsic Factor Deficiency
\item Defective Transport of Cobalamin by Enterocytes
\item Haptocorrin (R Binder) Deficiency
\item Transcobalamin Deficiency
\item Transcobalamin Receptor Deficiency
\end{itemize}
\item all are rare
\end{itemize}
\subsubsection{Cbl Complementation Groups}
\label{sec:org6e4b0c4}
\begin{itemize}
\item a number of disorders of intracellular metabolism of Cbl have been
classified as Cbl complementation groups (A-G, J, X)
\begin{itemize}
\item based on the biochemical phenotype and somatic cell analysis
\end{itemize}
\item precise diagnosis of the inborn errors of Cbl metabolism requires
either tests in cultured fibroblasts or identification of causal
mutations
\item complementation analysis can be used to reliably assign a patient to
one of the known classes of inborn error if function of either
methylmalonyl-Coenzyme A (CoA) mutase or methionine synthase
activity is reduced in patient fibroblasts
\item the one exception is the CblX disorder which cannot be
differentiated from CblC by complementation analysis
\end{itemize}
\subsubsection{Combined AdoCbl and MeCbl Deficiencies}
\label{sec:orgd0ca393}
\begin{itemize}
\item CblF, CblJ, CblC, CblX, CblD
\begin{itemize}
\item F, J, D and X are very rare
\end{itemize}
\end{itemize}
\subsubsection{Cobalamin C}
\label{sec:orgc9a9e82}
\begin{enumerate}
\item Clinical Presentation
\label{sec:orgaec5a38}
\begin{itemize}
\item most frequent inborn error of Cbl metabolism
\item many acutely ill in the 1st month of life
\item most were diagnosed within the 1st year
\item early-onset group shows feeding difficulties and lethargy
\begin{itemize}
\item followed by progressive neurological deterioration may include: 
\begin{itemize}
\item hypotonia, hypertonia or both, abnormal movements or seizures
and coma
\end{itemize}
\item severe pancytopenia or a non-regenerative anaemia may be present
\begin{itemize}
\item megaloblastic on bone marrow examination
\end{itemize}
\end{itemize}
\item a small number of CblC patients diagnosed \textgreater{} 1st year of life
\begin{itemize}
\item as late as 4th decade
\end{itemize}
\end{itemize}

\item Metabolic Derangement
\label{sec:org7c8c3f4}
\begin{itemize}
\item CblC is before the branch point \(\therefore\) both MUT and MS activity affected
\begin{itemize}
\item \(\uparrow\) MMA
\item \(\uparrow\) HCY
\end{itemize}
\end{itemize}

\item Genetics
\label{sec:org325a081}
\begin{itemize}
\item AR MMACHC
\end{itemize}

\item Diagnostic Tests
\label{sec:orgc98e907}
\begin{itemize}
\item methylmalonic acidaemia and aciduria are the
biochemical hallmarks of this disease
\begin{itemize}
\item MMA \textless{} MUT deficiency
\item MMA \textgreater{} transport defects
\end{itemize}
\item \(\uparrow\) plasma total homocysteine
\item \(\downarrow\) to normal plasma methionine
\item \(\uparrow\) urine homocysteine
\end{itemize}

\item Treatment
\label{sec:orgfc5657a}
\begin{itemize}
\item parenteral OHCbl
\item oral betaine (trimethylglycine)
\begin{itemize}
\item betaine-homocysteine methyltransferase (BHMT) is betaine dependent
\end{itemize}
\end{itemize}

\ce{trimethylglycine + homocysteine ->[BHMT] dimethylglycine + methionine}

\begin{itemize}
\item in the liver BHMT catalyzes up to 50\% of homocysteine metabolism
\item betaine treatment \(\to\) \(\uparrow\) sarcosine (methlyglycine) in plasma amino acids
\end{itemize}
\end{enumerate}

\subsubsection{Cobalamin X}
\label{sec:org1298c14}
\begin{itemize}
\item CblX is caused by mutations in HCFC1
\begin{itemize}
\item X-linked
\item encodes a \textbf{transcription regulator} that affects expression of a
number of genes, including MMACHC (CblC)
\end{itemize}
\item same phenotype as CblC
\item metabolic consequences of mutations stem from decreased MMACHC
expression leading to decreased synthesis of both AdoCbl and MeCbl
\end{itemize}

\subsubsection{Adenosylcobalamin Deficiency}
\label{sec:org3ce31f7}
\begin{itemize}
\item CblA and CblB
\item deficient MUT activity
\begin{itemize}
\item characterized by methylmalonic aciduria (MMA)
\end{itemize}
\item phenotype resembles methylmalonyl-CoA mutase deficiency
\item treated with protein restriction and OHCbl
\end{itemize}

\subsubsection{Methylcobalamin Deficiency}
\label{sec:orgd1c55f3}
\begin{itemize}
\item CblE and CblG
\item deficient MS activity
\begin{itemize}
\item \(\uparrow\) homocysteine
\item \down arrow methionine
\end{itemize}
\item megaloblastic anaemia and neurological disease
\begin{itemize}
\item accumulation of methyl-THF causes depletion of THF required for
purine and pyrimidine synthesis
\end{itemize}
\end{itemize}

\subsection{Folate}
\label{sec:org41c8f17}
\begin{itemize}
\item folic acid (pteroylglutamic acid) is plentiful in foods such as
liver, leafy vegetables, legumes and some fruits
\item metabolism involves reduction to dihydrofolate (DHF) and
tetrahydrofolate (THF)
\begin{itemize}
\item followed by addition of a single-carbon unit, which is provided by
serine or histidine this carbon unit occurs in various redox
states
\begin{itemize}
\item methyl, methylene, methenyl or formyl
\end{itemize}
\end{itemize}
\item transfer of this single-carbon unit is essential for the endogenous
formation of:
\begin{itemize}
\item methionine (methionine synthase)
\item thymidylate (dTMP)
\item formylglycineamide ribotide (FGAR) and
formylaminoimidazolecarboxamide ribotide (FAICAR) two
intermediates of purine synthesis
\end{itemize}
\item these reactions regenerate DHF and THF
\item the predominant folate derivative in blood and in cerebrospinal
fluid is 5-methyltetrahydrofolate
\begin{itemize}
\item product of the methylenetetrahydrofolate reductase (MTHFR) rxn
\end{itemize}
\item there are a number of very rare disorders of folate absorption and metabolism
\item severe MTHFR deficiency is the most frequent
\end{itemize}

\begin{figure}[htbp]
\centering
\includegraphics[width=1.0\textwidth]{b12b9/figures/folate.png}
\caption{\label{fig:orgd9b118e}Folate Metabolism:1 methionine synthase; 2 methylenetetrahydrofolate reductase; 3 methylenetetrahydrofolate dehydrogenase; 4 methenyltetrahydrofolate cyclohydrolase: 5 formyltetrahydrofolate synthetase; 6 dihydrofolate reductase; 7 glutamate formiminotransferase; 8 formiminotetrahydrofolate cyclodeaminase}
\end{figure}

\subsubsection{Methylenetetrahydrofolate Reductase Deficiency}
\label{sec:org475c38f}
\begin{itemize}
\item disambiguation
\begin{itemize}
\item severe form of this deficiency
\item not the C677T polymorphism associated \(\uparrow\) risk of common
disease
\begin{itemize}
\item \(\uparrow\) neural tube defects in maternal hyperhomocysteinaemia
\item \(\uparrow\) cardiovascular disease in 3rd or 4th decade of life
\end{itemize}
\end{itemize}
\end{itemize}
\begin{enumerate}
\item Clinical Presentation
\label{sec:org6cac9db}
\begin{itemize}
\item most diagnosed in infancy
\item \textgreater{} 50\% present in the 1st year of life
\item common presentation is progressive encephalopathy with apnoea,
seizures and microcephaly
\item not associated with megaloblastic anaemia
\end{itemize}

\item Metabolic Derangement
\label{sec:org535a2a8}
\begin{itemize}
\item \textbf{methylenetetrahydrofolate reductase} deficiency
\end{itemize}
\ce{5,10-methylene-THF ->[MTHFR] 5-methyl-THF}
\begin{itemize}
\item \(\downarrow\) methyl-THF
\item methyl-THF is the methyl donor for the conversion of homocysteine
\(\to\) methionine by methionine synthase
\begin{itemize}
\item \(\uparrow\) total plasma homocysteine
\item \(\downarrow\) methionine
\end{itemize}
\item the block in the conversion of 5,10-methylene-THF to methyl-THF does
not result in the trapping of folates as methyl-THF
\begin{itemize}
\item \(\therefore\) does not \(\downarrow\) reduced THF for purine and
pyrimidine synthesis
\item contrast to disorders at the level of methionine synthase
\item explains why patients do not have megaloblastic anaemia
\end{itemize}
\end{itemize}

\item Genetics
\label{sec:org8eb75f5}
\begin{itemize}
\item AR MTHFR
\end{itemize}

\item Diagnostic Tests
\label{sec:orgfba1959}
\begin{itemize}
\item methyl-THF is the major circulating form of folate
\begin{itemize}
\item \(\therefore\) serum folate levels may sometimes be low
\end{itemize}
\item \(\Uparrow\) plasma homocysteine
\item \(\downarrow\) plasma methionine
\end{itemize}

\item Treatment
\label{sec:org68fa6ee}
\begin{itemize}
\item oral betaine (trimethylglycine)
\item betaine-homocysteine methyltransferase (BHMT) is betaine dependent
\end{itemize}

\ce{trimethylglycine + homocysteine ->[BHMT] dimethylglycine + methionine}

\begin{itemize}
\item in the liver BHMT catalyzes up to 50\% of homocysteine metabolism
\item betaine treatment \(\to\) \(\uparrow\) sarcosine (methlyglycine) in plasma amino acids
\end{itemize}
\end{enumerate}

\section{Thiamine and Pyridoxine}
\label{sec:orga55013d}
\subsection{Thiamine}
\label{sec:org6bf891c}
\begin{itemize}
\item thiamine (vitamin B\textsubscript{1}) is water soluble
\item transported across cell membranes by THTR1 and THTR2
\begin{itemize}
\item encoded by the SLC19A2 and SLC19A3
\end{itemize}
\item both are widely expressed in the body
\item differ in kinetic properties and level of expression in different tissues
\item THTR2 is the major transporter in the upper small intestine where
dietary thiamine is absorbed at the luminal surface
\item THTR1 predominates at the basal surface
\item cofactor thiamine pyrophospate (TPP) is formed in the cytoplasm by
the enzyme thiamine pyrophosphokinase
\begin{itemize}
\item mitochondrial TPP transporter (MTPPT) in the inner mitochondrial membrane
delivers the cofactor to the \(\alpha\)-ketoacid dehydrogenases in the
mitochondrial matrix
\begin{itemize}
\item PDH, KGDH and BCKDH
\end{itemize}
\end{itemize}
\item TPP cofactor is attached directly to the transketolase and 2-hydroxyacyl CoA lyase apoproteins
\end{itemize}


\begin{figure}[htbp]
\centering
\includegraphics[width=0.6\textwidth]{b1b6/figures/thiamine.png}
\caption{\label{fig:orgbb8594b}Thiamine Transport}
\end{figure}

\subsubsection{Defective Transport}
\label{sec:org2f0de4c}
\begin{itemize}
\item THTR1 or 2 deficiency
\end{itemize}
\subsubsection{TPP biosynthesis}
\label{sec:orgc9df1f9}
\begin{itemize}
\item \textbf{thiamine pyrophosphokinase (TPK1)} deficiency
\item \(\uparrow\) blood and CSF lactate
\item \(\uparrow\) urine \(\alpha\)-ketoglutarate
\item \(\downarrow\) TPP
\end{itemize}
\subsubsection{Mitochondrial transport}
\label{sec:orga1264a6}
\begin{itemize}
\item \textbf{mitochondrial TPP transporter (MTTPT)} deficiency - amish lethal
microcephaly
\item \(\uparrow\) urine \(\alpha\)-ketoglutarate
\end{itemize}
\subsubsection{Binding to Apo-Enzymes}
\label{sec:org5443dfa}
\begin{itemize}
\item PDH and KGDH
\begin{itemize}
\item \(\uparrow\) lactate
\item normal L/P ratio
\end{itemize}
\item BCKDH
\begin{itemize}
\item \(\uparrow\) BCAAs
\end{itemize}
\end{itemize}

\subsection{Pyridoxine}
\label{sec:orgd567f00}
\begin{itemize}
\item pyridoxine (vitamin B\textsubscript{6}) is water soluble
\item broad availability from various food sources
\item three vitamers and their phosphorylated esters are absorbed in the
small intestine:
\begin{enumerate}
\item pyridoxal - aldehyde group
\item pyridoxamine - amine group
\item pyridoxine - hydroxyl group
\end{enumerate}
\end{itemize}

\begin{center}
\chemnameinit{}
\chemname{\chemfig{R-[1]C(=[2]O)-[7]H}}{aldehyde}
\chemnameinit{}
\chemname{\chemfig{R-NH_2}}{amine}
\chemnameinit{}
\chemname{\chemfig{R-OH}}{hydroxyl}
\end{center}

\begin{itemize}
\item phosphorylated forms undergo dephosphorylation by tissue
non-specific alkaline phosphatase (TNSAP) prior to cellular uptake
\begin{itemize}
\item transport mechanism of B\textsubscript{6} vitamers across cell membranes not
elucidated
\end{itemize}
\item within cells vitamers are rephosphorylated by kinases and further oxidised to the
active cofactor pyridoxal 5’-phosphate (PLP) by pyridox(am)ine
5’-phosphate oxidase (PNPO)
\item liver seems to be the most important organ of PLP formation
\begin{itemize}
\item PNPO is expressed in various cell types including neurons
\end{itemize}
\item PLP is one of the most abundant cofactors and participates in over
140 reactions in amino acid and neurotransmitter metabolism
\item functions include:
\begin{itemize}
\item neurotransmitter biosynthesis
\begin{itemize}
\item serotonin, dopamine, epinephrine, norepinephrine, GABA
\end{itemize}
\item amino acids metabolism
\begin{itemize}
\item proline, transamination (i.e. alanine), serine, tryptophan
\end{itemize}
\item glucose metabolism
\begin{itemize}
\item transamination of glycogenic amino acids
\item glycogenolysis - glycogen phosphorylase
\end{itemize}
\item lipid metabolism
\item heme synthesis - ALA synthase
\end{itemize}
\item systemic vitamin B\textsubscript{6} deficiency causes seizures, failure to thrive
and anemia
\item nutritional vitamin B\textsubscript{6} deficiency is rarely seen
\begin{itemize}
\item usually occurs together with other vitamin deficiencies in
malnutrition or in association with severe chronic disease
\end{itemize}
\end{itemize}


\begin{figure}[htbp]
\centering
\includegraphics[width=0.9\textwidth]{b1b6/figures/pyridoxine.png}
\caption{\label{fig:org3bae01b}Pyridoxine Metabolism}
\end{figure}

\begin{figure}[htbp]
\centering
\includegraphics[width=0.9\textwidth]{b1b6/figures/Slide26.png}
\caption{\label{fig:org07e2ed7}Pyridoxine Metabolism}
\end{figure}

\begin{itemize}
\item several mechanisms that lead to an increased requirement for
pyridoxine and/or PLP:
\begin{enumerate}
\item IEMs affecting the pathways of B\textsubscript{6} vitamer metabolism
\begin{itemize}
\item PNPO and alkaline phosphatase defects
\end{itemize}
\item IEMs that lead to accumulation of small molecules that
react with PLP and sequester it
\begin{itemize}
\item P5CDH deficiency (hyperprolinemia type II)
\item antiquitin deficiency (pyridoxine dependent epilepsy)
\end{itemize}
\item IEMs affecting specific PLP dependent enzymes
\begin{itemize}
\item ALAS deficiency (X-linked sideroblastic anemia)
\item CBS deficiency (classical homocystinuria)
\item OAT deficiency (gyrate atrophy of the choroid retina)
\end{itemize}
\item drugs that affect the metabolism of B\textsubscript{6} vitamers or react with PLP
\begin{itemize}
\item D-penicillamine and isozianid
\end{itemize}
\item coeliac disease (malabsorption) or renal dialysis (loss)
\end{enumerate}
\item IEMs \(\to\) \(\downarrow\) PLP
\begin{itemize}
\item via inactivation, reduced formation, or reduced cellular uptake of
PLP
\item all autosomal recessive
\item seizures are a hallmark of the disease
\begin{itemize}
\item no or incomplete response to common anticonvulsants
\item good response to pyridoxine or PLP
\end{itemize}
\end{itemize}
\end{itemize}

\subsubsection{Antiquitin deficiency}
\label{sec:orgf8cbcb7}
\begin{itemize}
\item AKA: pyridoxine responsive seizures
\item presents in neonates with myoclonic and tonic seizures or
status epilepticus
\item onset up to 3 years observed
\item antiquitin (ALDH7A1) encodes \textbf{\(\alpha\)-aminoadipic semialdehyde dehydrogenase}
\begin{itemize}
\item involved in lysine degradation (Figure \ref{fig:org141ea58})
\end{itemize}
\end{itemize}
\ce{AASA ->[antiquitin] AAA}
\begin{itemize}
\item deficiency results in \(\uparrow\) \(\alpha\)-aminoadipic acid semialdehyde (AASA)
\begin{itemize}
\item AASA is in equilibrium with piperideine-6-carboxylate (P6C)
\item P6C inactivates PLP
\end{itemize}
\end{itemize}
\begin{itemize}
\item \textbf{simultaneous determination of S-sulfocysteine is crucial to exclude}
\textbf{molybdenum cofactor and sulfite oxidase deficiency causing secondary}
\textbf{inhibition of antiquitin}
\begin{itemize}
\item \(\uparrow\) S-sulfocysteine inhibits antiquitin
\end{itemize}
\item most common form of pyridoxine dependent epilepsy (PDE)
\item \(\uparrow\) urine AASA (\(\alpha\)-aminoadipic acid)
\item \(\uparrow\) urine P6C (piperideine-6-carboxylate)
\item \(\uparrow\) pipecolic acid in plasma, the first described biomarker of PDE
\begin{itemize}
\item less specific as it can also be found in peroxisomal disease
\end{itemize}
\item treat with pyridoxine
\end{itemize}

\begin{figure}[htbp]
\centering
\includegraphics[width=0.9\textwidth]{b1b6/figures/lysine_deg.png}
\caption{\label{fig:org141ea58}Lysine Degradation and Antiquitin Deficiency (blue bar)}
\end{figure}

\subsubsection{Hyperprolinemia Type II}
\label{sec:org7104358}
\begin{itemize}
\item see Ornithine and Proline
\end{itemize}

\subsubsection{PNPO deficiency}
\label{sec:org53835f3}
\begin{itemize}
\item \textbf{pyridox(am)ine 5’-phosphate oxidase (PNPO)} deficiency
\item AKA: pyridoxal phosphate responsive seizures
\begin{itemize}
\item does not respond to pyridoxine
\end{itemize}
\item clinically indistinguishable from antiquitin deficiency
\item severe (systemic) PLP deficiency and impaired function of PLP
dependent enzymes
\item \(\uparrow\) urine vanillactate
\item \(\uparrow\) pyridoxamine
\item \(\uparrow\) pyridoxamine/pyridoxic acids
\item treat with oral PLP
\end{itemize}

\subsubsection{Congenital Hypophosphatasia}
\label{sec:orgbefd9c4}
\begin{itemize}
\item severe form \(\to\) neonatal seizures
\item osteomalacia
\item \textbf{tissue non-specific alkaline phosphatase (TNSAP)} deficiency
\begin{itemize}
\item impaired dephosphorylation of PLP for cellular uptake
\end{itemize}
\item \(\Downarrow\) plasma alkaline phosphatase
\item \(\uparrow\) serum calcium
\item \(\downarrow\) serum phosphate
\item \(\uparrow\) plasma phosphoethanolamine
\item AR ALPL
\begin{itemize}
\item some mutations are AD
\end{itemize}
\end{itemize}

\subsubsection{X-Linked Hypophosphatasia}
\label{sec:org32e5071}
\begin{itemize}
\item AKA: X-Linked Hypophosphatemic Rickets
\item ranges from isolated hypophosphatemia to severe lower-extremity bowing
\item presents first \lte 2 years when lower-extremity bowing becomes
evident with the onset of weight bearing
\begin{itemize}
\item sometimes not until adulthood
\end{itemize}
\item \textbf{phosphate-regulating endopeptidase} deficiency
\begin{itemize}
\item involved in bone and tooth mineralization and renal phosphate retention
\end{itemize}
\item \(\uparrow\) serum calcium
\item \(\downarrow\) serum phosphate
\item \(\uparrow\) 1,25 hydroxy-vitamin D
\item \(\Uparrow\) plasma alkaline phosphatase in children
\item XLD PHEX
\end{itemize}
\end{document}