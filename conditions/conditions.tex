% Created 2019-11-09 Sat 13:06
% Intended LaTeX compiler: pdflatex
\documentclass{scrartcl}
\usepackage[utf8]{inputenc}
\usepackage[T1]{fontenc}
\usepackage{graphicx}
\usepackage{grffile}
\usepackage{longtable}
\usepackage{wrapfig}
\usepackage{rotating}
\usepackage[normalem]{ulem}
\usepackage{amsmath}
\usepackage{textcomp}
\usepackage{amssymb}
\usepackage{capt-of}
\usepackage{hyperref}
\hypersetup{colorlinks,linkcolor=black,urlcolor=blue}
\usepackage{textpos}
\usepackage{textgreek}
\usepackage[version=4]{mhchem}
\usepackage{chemfig}
\usepackage{siunitx}
\usepackage{gensymb}
\usepackage[usenames,dvipsnames]{xcolor}
\usepackage[T1]{fontenc}
\usepackage{lmodern}
\usepackage{verbatim}
\usepackage{tikz}
\usepackage{wasysym}
\usetikzlibrary{shapes.geometric,arrows,decorations.pathmorphing,backgrounds,positioning,fit,petri}
\usepackage{fancyhdr}
\pagestyle{fancy}
\author{Matthew Henderson, PhD, FCACB}
\date{\today}
\title{Genetic Conditions}
\hypersetup{
 pdfauthor={Matthew Henderson, PhD, FCACB},
 pdftitle={Genetic Conditions},
 pdfkeywords={},
 pdfsubject={},
 pdfcreator={Emacs 26.1 (Org mode 9.1.9)}, 
 pdflang={English}}
\begin{document}

\maketitle
\setcounter{tocdepth}{1}
\tableofcontents


\section{Cancer Genetics}
\label{sec:org073f639}
\subsection{{\bfseries\sffamily TODO} Tuberous sclerosis}
\label{sec:orgaf02498}
\subsection{{\bfseries\sffamily TODO} Hereditary breast cancer}
\label{sec:org7dc40fd}
\subsection{{\bfseries\sffamily TODO} Chronic myelogenous leukemia (CML)}
\label{sec:orgad12360}
\subsection{{\bfseries\sffamily TODO} Familial adenomatous polyposis}
\label{sec:orgbf2f388}
\subsection{{\bfseries\sffamily TODO} Hereditary non-polyposis colon cancer (HNPCC)}
\label{sec:orgaad55fc}
\subsection{{\bfseries\sffamily TODO} Li-Fraumeni syndrome}
\label{sec:org564c5b8}
\subsection{{\bfseries\sffamily TODO} Retinoblastoma}
\label{sec:orga0420d2}

\section{Clinical Genetics}
\label{sec:org58a53c4}
\subsection{{\bfseries\sffamily TODO} Duchenne/Becker Muscular Dystrophy}
\label{sec:org04358a3}
\subsection{{\bfseries\sffamily TODO} Osteogenesis Imperfecta}
\label{sec:org815fb87}
\subsection{{\bfseries\sffamily TODO} Long QT syndrome}
\label{sec:org51ea0df}
\subsection{{\bfseries\sffamily TODO} Marfan syndrome}
\label{sec:org402a879}
\subsection{{\bfseries\sffamily TODO} Neurofibromatosis type I}
\label{sec:org0e2db14}
\subsection{{\bfseries\sffamily TODO} Neurofibromatosis type II}
\label{sec:org1e172f2}
\subsection{{\bfseries\sffamily TODO} Smith-Lemli-Opitz syndrome}
\label{sec:org1966ae8}
\subsection{{\bfseries\sffamily TODO} Noonan syndrome}
\label{sec:org15a7a74}
\subsection{{\bfseries\sffamily TODO} Charge syndrome}
\label{sec:org6348a80}
\subsection{{\bfseries\sffamily TODO} FGFR-related disorders (Craniosynostosis\ldots{})}
\label{sec:org8ebbc51}
\subsection{{\bfseries\sffamily TODO} Factor V Leiden thrombophilia}
\label{sec:orgd0b894d}
\subsection{{\bfseries\sffamily TODO} G6PD deficiency}
\label{sec:org1255564}
\subsection{{\bfseries\sffamily TODO} Hemoglobinopathies (Sickle cell anemia, Alpha / Beta Thalassemia)}
\label{sec:org1ba5652}
\subsection{{\bfseries\sffamily TODO} Hemophilia A / B}
\label{sec:org1b97765}
\subsection{{\bfseries\sffamily TODO} Hemochromatosis}
\label{sec:org40ec1ad}
\subsection{{\bfseries\sffamily TODO} SRY translocation}
\label{sec:orga24a6a3}
\subsection{{\bfseries\sffamily TODO} Turner syndrome}
\label{sec:org9a9a811}
\subsection{{\bfseries\sffamily TODO} Androgen insensitivity syndrome}
\label{sec:orgb1de541}
\subsection{{\bfseries\sffamily TODO} 21-Hydroxylase deficiency}
\label{sec:org8715378}

\section{Complex Inheritance}
\label{sec:org7ab568b}
\subsection{{\bfseries\sffamily TODO} Alzheimer disease}
\label{sec:org611a7ec}
\subsection{{\bfseries\sffamily TODO} Congenital hearing loss}
\label{sec:orgdcf5e4f}
\subsection{{\bfseries\sffamily TODO} Diabetes mellitus (insulin vs non-insulin dependent)}
\label{sec:orgbca7914}

\section{Cytogenetics}
\label{sec:orgcc743d4}
\subsection{{\bfseries\sffamily TODO} Cri du chat syndrome}
\label{sec:org34ad5ba}
\subsection{{\bfseries\sffamily TODO} Pallister-Killian syndrome}
\label{sec:orgb56719d}
\subsection{{\bfseries\sffamily TODO} Triploidy}
\label{sec:orge87b290}
\subsection{{\bfseries\sffamily TODO} Trisomy 13}
\label{sec:org7c13401}
\subsection{{\bfseries\sffamily TODO} Trisomy 18}
\label{sec:orgc64a84b}
\subsection{{\bfseries\sffamily TODO} Trisomy 21}
\label{sec:org0f73f43}
\subsection{{\bfseries\sffamily TODO} Klinefelter syndrome}
\label{sec:org5fd341f}
\subsection{{\bfseries\sffamily TODO} Fanconi anemia}
\label{sec:orgf98c5b2}
\subsection{{\bfseries\sffamily TODO} Ataxia-telangiectasia}
\label{sec:orgcc76d88}
\subsection{{\bfseries\sffamily TODO} Williams syndrome}
\label{sec:orgaf3c42e}
\subsection{{\bfseries\sffamily TODO} 22q11 deletion syndrome}
\label{sec:org2907742}

\section{Metabolics}
\label{sec:org4aca792}
\subsection{{\bfseries\sffamily TODO} Acute intermittent porphyria}
\label{sec:org7ce0950}
\subsection{{\bfseries\sffamily TODO} Alpha-1 antitrypsin deficiency}
\label{sec:orgccd6c50}
\subsection{{\bfseries\sffamily TODO} Canavan disease}
\label{sec:org5eb25bb}
\subsection{{\bfseries\sffamily TODO} Gaucher disease}
\label{sec:org19f63a8}
\subsection{{\bfseries\sffamily TODO} Homocystinuria (CBS deficiency)}
\label{sec:orgea8dd15}
\subsection{{\bfseries\sffamily TODO} Hurler syndrome}
\label{sec:orgdb6c032}
\subsection{{\bfseries\sffamily TODO} I cell disease}
\label{sec:org313d5e3}
\subsection{{\bfseries\sffamily TODO} MCAD deficiency}
\label{sec:orge07da43}
\subsection{{\bfseries\sffamily TODO} Mitochondrial DNA mutations}
\label{sec:org399df10}
\begin{itemize}
\item (MERRF/MELAS/LHON, mtDNA deletion syndromes)
\end{itemize}
\subsection{{\bfseries\sffamily TODO} Ornithine transcarbamylase (OTC) deficiency}
\label{sec:orgece3d55}
\subsection{{\bfseries\sffamily TODO} Peroxisome biogenesis disorder (Zellweger)}
\label{sec:org902ad67}
\subsection{{\bfseries\sffamily TODO} Phenylketonuria}
\label{sec:org23c2ec4}
\subsection{{\bfseries\sffamily TODO} Pompe disease}
\label{sec:orgae5c993}
\subsection{{\bfseries\sffamily TODO} Tay Sachs Disease}
\label{sec:orgaad311b}
\subsection{{\bfseries\sffamily TODO} Tyrosinemia type I}
\label{sec:org289af5c}
\subsection{{\bfseries\sffamily TODO} X-linked Adrenoleukodystrophy}
\label{sec:org976f77c}
\subsection{{\bfseries\sffamily TODO} Wilson disease}
\label{sec:orge52b9ba}
\subsection{{\bfseries\sffamily TODO} Hyperprolinemia type I}
\label{sec:org144ed82}

\section{Molecular Genetics}
\label{sec:orgc929732}
\subsection{{\bfseries\sffamily TODO} Cystic fibrosis}
\label{sec:org6df3261}
\subsection{{\bfseries\sffamily TODO} Achondroplasia}
\label{sec:org98c684a}
\subsection{{\bfseries\sffamily TODO} Huntington disease}
\label{sec:orga10850d}
\subsection{Fragile X}
\label{sec:orged92039}
\begin{enumerate}
\item Clinical Characteristics
\label{sec:orgca53e47}
\begin{itemize}
\item Fragile X syndrome occurs in individuals with an FMR1 full mutation
or other loss-of-function variant.
\item Males: moderate intellectual disability
\item Females: mild intellectual disability
\item FMR1 pathogenic variants are complex alterations involving non-classic
gene-disrupting alterations (trinucleotide repeat expansion) and
abnormal gene methylation,
\begin{itemize}
\item \(\therefore\) affected individuals occasionally have an atypical presentation with an IQ above 70,
\begin{itemize}
\item the traditional  demarcation denoting intellectual disability.
\end{itemize}
\end{itemize}
\item Males with an FMR1 full mutation accompanied by aberrant methylation may have a characteristic appearance:
\begin{itemize}
\item large head, long face, prominent forehead and chin, protruding ears
\item connective tissue findings (joint laxity), and large testes after puberty.
\item Behavioral abnormalities, sometimes including autism spectrum disorder, are common.
\end{itemize}
\item fragile X-associated tremor/ataxia syndrome, and FMR1-related
primary ovarian insufficiency are less severe forms due to smaller
repeats
\end{itemize}
\item Diagnostic Testing
\label{sec:org63f5ad9}
\begin{itemize}
\item alteration in FMR1.
\item \textgreater{} 99\% of individuals with fragile X syndrome have:
\begin{itemize}
\item lof variant of FMR1 caused by an increased number of CGG
trinucleotide repeats (typically >200)
\item accompanied by aberrant methylation of FMR1
\end{itemize}
\item Other pathogenic variants include:
\begin{itemize}
\item deletions and single-nucleotide variants.
\end{itemize}
\end{itemize}
\item Genetic Counseling
\label{sec:org6841b45}
\begin{itemize}
\item All mothers of individuals with an FMR1 full mutation (expansion
>200 CGG trinucleotide repeats and abnormal methylation) are
carriers of an FMR1 pathogenic variant.
\item Mothers and their female relatives who are premutation carriers are
at increased risk for FXTAS and POI;
\item those with a full mutation may have findings of fragile X syndrome.
\item All are at increased risk of having offspring with fragile X syndrome, FXTAS, and POI.
\item Males with premutations are at increased risk for FXTAS.
\item Males with FXTAS will transmit their FMR1 premutation expansion to none of their sons and to all of their daughters, who will be premutation carriers.
\item Carrier testing for at-risk relatives and prenatal testing for
pregnancies at increased risk are possible if the diagnosis of an
FMR1-related disorder has been confirmed in a family member.
\end{itemize}
\end{enumerate}
\subsection{Friedreich's ataxia}
\label{sec:org85b398b}
\begin{enumerate}
\item Clinical Characteristics
\label{sec:orga32a56c}
\begin{itemize}
\item characterized by slowly progressive ataxia with onset usually before
age 25 years (mean 10-15 yrs).
\item FRDA is typically associated with dysarthria, muscle weakness,
spasticity particularly in the lower limbs, scoliosis, bladder
dysfunction, absent lower-limb reflexes, and loss of position and
vibration sense.
\item \textasciitilde{}2/3 have cardiomyopathy
\item \textasciitilde{}30\% have diabetes mellitus,
\item \textasciitilde{}25\% have an "atypical" presentation with later onset or retained
tendon reflexes.
\end{itemize}
\item Diagnostic Testing
\label{sec:org3c38999}
\begin{itemize}
\item established in a proband by detection of biallelic pathogenic
variants in FXN.
\item An abnormally expanded GAA repeat in intron 1 of FXN observed on
both alleles in \textasciitilde{}96\% with FRDA
\item remaining are compound heterozygotes for abnormally expanded GAA
repeat in the disease-causing range on one allele and another
intragenic pathogenic variant on the other allele.
\end{itemize}

\item Genetic Counseling
\label{sec:org0052ffb}
\begin{itemize}
\item autosomal recessive
\item Each sib has a 25\% chance of being affected
\begin{itemize}
\item 50\% chance of being an asymptomatic carrier
\item 25\% chance of having no pathogenic variant.
\end{itemize}
\item Carrier testing of at-risk relatives, prenatal testing for
pregnancies at increased risk, and pre-implantation genetic diagnosis
are possible if both FXN pathogenic variants have been identified in
an affected family member.
\end{itemize}
\end{enumerate}

\subsection{Myotonic dystrophy type I}
\label{sec:org2e28197}
\begin{enumerate}
\item Clinical Characteristics
\label{sec:org05c7702}
\begin{itemize}
\item multisystem disorder that affects skeletal and smooth muscle as well
as the eye, heart, endocrine system, and central nervous system.

\item clinical findings, from mild to severe:

\item[{Mild DM1}] cataract and mild myotonia (sustained muscle
contraction) life span is normal

\item[{Classic DM1}] muscle weakness and wasting, myotonia, cataract, and
often cardiac conduction abnormalities; adults may
become physically disabled and may have a shortened
life span.

\item[{Congenital DM1}] is characterized by hypotonia and severe
generalized weakness at birth, often with
respiratory insufficiency and early death;
intellectual disability is common.
\end{itemize}
\item Diagnostic Testing
\label{sec:org7d67d63}
\begin{itemize}
\item caused by expansion of a CTG trinucleotide repeat in the noncoding region of DMPK.
\item molecular genetic testing of DMPK.
\item CTG repeat length exceeding 34 repeats is abnormal.
\item Molecular genetic testing detects pathogenic variants in nearly 100\%
of affected individuals.
\end{itemize}

\item Genetic Counseling
\label{sec:org5bbb822}
\begin{itemize}
\item autosomal dominant
\item Offspring of an affected individual have a 50\% chance of inheriting the expanded allele.
\item Pathogenic alleles may expand in length during gametogenesis
\begin{itemize}
\item \(\to\) transmission of longer trinucleotide repeat alleles
\item \(\to\) earlier onset and more severe disease the parent
\end{itemize}
\item Prenatal testing is possible for pregnancies at increased risk when
the diagnosis of DM1 has been confirmed by molecular genetic testing
in an affected family member.
\end{itemize}
\end{enumerate}
\subsection{Angelman syndrome}
\label{sec:org15c619a}
\begin{enumerate}
\item Clinical Characteristics
\label{sec:org4720ac5}
\begin{itemize}
\item Severe developmental delay or intellectual disability, severe speech
impairment, gait ataxia and/or tremulousness of the limbs
\item unique behavior with an inappropriate happy demeanor that includes
frequent laughing, smiling, and excitability.
\item Microcephaly and seizures are also common.
\item Developmental delays are first noted at around age six months
\item clinical features of AS do not become manifest until after age one year
\begin{itemize}
\item can take several years before the correct clinical diagnosis is obvious.
\end{itemize}
\end{itemize}

\item Diagnostic Testing
\label{sec:org2e56439}

\begin{figure}[htbp]
\centering
\includegraphics[width=0.6\textwidth]{./figures/aspws.jpg}
\caption{\label{fig:org9385033}
AS and PWS}
\end{figure}

\begin{itemize}
\item molecular genetic testing deficient expression or function of the
maternally inherited UBE3A allele.
\item parent-specific DNA methylation imprints in the 15q11.2-q13 chromosome region detects approximately 80\%
\begin{itemize}
\item including deletion, uniparental disomy (UPD), imprinting defect (ID)
\end{itemize}
\item \textless{} 1\% have a cytogenetically visible chromosome rearrangement (i.e., translocation or inversion).
\item UBE3A sequence analysis detects pathogenic variants \textasciitilde{}11\% of individuals.
\item molecular genetic testing (methylation analysis and UBE3A sequence
analysis) \textasciitilde{}90\% of individuals.
\item Remaining 10\% with classic phenotypic features of AS have the
disorder as a result of an as-yet unidentified genetic mechanism
\begin{itemize}
\item not amenable to diagnostic testing
\end{itemize}
\end{itemize}
\item Genetic Counseling
\label{sec:org474f639}
\begin{itemize}
\item caused by disruption of maternally imprinted UBE3A located within
the 15q11.2-q13 Angelman syndrome/Prader-Willi syndrome region.
\item The risk to sibs of a proband depends on the genetic mechanism
leading to the loss of UBE3A function
\begin{itemize}
\item typically less than 1\% risk for probands with a deletion or UPD
\item as high as 50\% for probands with an ID or a pathogenic variant of UBE3A.
\end{itemize}
\item Members of the mother's extended family are also at increased risk
when an ID or a UBE3A pathogenic variant is present.
\item Cytogenetically visible chromosome rearrangements may be inherited,usually \emph{de novo}.
\item Prenatal testing is possible when the underlying genetic mechanism
is a deletion, UPD, an ID, a UBE3A pathogenic variant, or a
chromosome rearrangement.
\end{itemize}
\end{enumerate}
\subsection{Beckwith-Wiedemann syndrome}
\label{sec:orgbef3e11}
\begin{enumerate}
\item Clinical Characteristics
\label{sec:orge6d8540}
\begin{itemize}
\item growth disorder variably characterized by neonatal hypoglycemia,
macrosomia, macroglossia, hemihyperplasia, omphalocele, embryonal
tumors (e.g., Wilms tumor, hepatoblastoma, neuroblastoma, and
rhabdomyosarcoma), visceromegaly, adrenocortical cytomegaly, renal
abnormalities (e.g., medullary dysplasia, nephrocalcinosis,
medullary sponge kidney, and nephromegaly), and ear creases/pits.

\item a clinical spectrum, may have many of these features or only one or two.

\item Early death may occur from complications of prematurity,
hypoglycemia, cardiomyopathy, macroglossia, or tumors.
\end{itemize}

\item Diagnostic Testing
\label{sec:orge07cd84}

\begin{itemize}
\item Cytogenetically detectable abnormalities involving chromosome 11p15
are found in 1\% or fewer of affected individuals.

\item Molecular genetic testing can identify epigenetic and genomic
alterations of chromosome 11p15 in individuals with BWS:
\begin{itemize}
\item Loss of methylation on the maternal chromosome at imprinting
center 2 (IC2) in 50\% of affected individuals;
\item Paternal uniparental disomy for chromosome 11p15 in 20\%
\item Gain of methylation on the maternal chromosome at imprinting
center 1 (IC1) in 5\%.
\end{itemize}
\item Methylation alterations associated with deletions or duplications in
this region have high heritability.

\item Sequence analysis of CDKN1C identifies a heterozygous maternally
inherited pathogenic variant in approximately 40\% of familial cases
and 5\%-10\% of cases with no family history of BWS.
\end{itemize}

\item Genetic Counseling
\label{sec:org1fe5f2d}
\begin{itemize}
\item associated with abnormal regulation of gene transcription in two
imprinted domains on chromosome 11p15.5.
\item Most individuals with BWS are reported to have normal chromosome
studies or karyotypes.
\item \textasciitilde{}85\% of individuals with BWS have no family history of BWS
\item \textasciitilde{}15\% have a family history consistent with parent-of-origin
autosomal dominant transmission.
\item Children of subfertile parents conceived by assisted reproductive
technology (ART) may be at increased risk for imprinting disorders,
including BWS.
\item Identification of the underlying genetic mechanism causing BWS
permits better estimation of recurrence risk.
\item Prenatal screening for pregnancies in the general population that
identifies findings suggestive of a diagnosis of BWS may lead to the
consideration of
\begin{itemize}
\item chromosome analysis, chromosomal microarray, and/or molecular genetic testing.
\end{itemize}
\item prenatal testing by chromosome analysis for families with an
inherited chromosome abnormality or by molecular genetic testing for
families in which the molecular mechanism of BWS has been defined
\end{itemize}
\end{enumerate}

\subsection{Prader-Willi syndrome}
\label{sec:org530536e}
\begin{enumerate}
\item Clinical Characteristics
\label{sec:org414d6e1}
\begin{itemize}
\item severe hypotonia and feeding difficulties in early infancy
\item later infancy or early childhood by excessive eating
\item gradual development of morbid obesity
\item Motor milestones and language development are delayed.
\item All individuals have some degree of cognitive impairment.
\item A distinctive behavioral phenotype (with temper tantrums, stubbornness, manipulative behavior, and obsessive-compulsive characteristics) is common.
\item Hypogonadism is present in both males and females and manifests as genital hypoplasia, incomplete pubertal development, and, in most, infertility.
\item Short stature is common (if not treated with growth hormone);
\item characteristic facial features, strabismus, and scoliosis are often present.
\end{itemize}

\item Diagnostic Testing
\label{sec:orga247bc0}
\begin{itemize}
\item DNA methylation testing to detect abnormal parent-specific
imprinting within the Prader-Willi critical region (PWCR) on
chromosome 15
\item testing determines whether the region is maternally inherited only
\begin{itemize}
\item the paternally contributed region is absent
\item detects more than 99\% of affected individuals
\item DNA methylation-specific testing is important to confirm the
diagnosis of PWS in all individuals,
\end{itemize}
\end{itemize}

\item Genetic Counseling
\label{sec:orgbd6aff6}
\begin{itemize}
\item PWS is caused by an absence of expression of imprinted genes in the
paternally derived PWS/Angelman syndrome (AS) region (15q11.2-q13)
of chromosome 15:
\begin{itemize}
\item paternal deletion, maternal uniparental disomy 15 and rarely an imprinting defect.
\end{itemize}
\item The risk to the sibs depends on the genetic mechanism.
\begin{itemize}
\item \textless{} 1\% if the affected child has a deletion or uniparental disomy
\item up to 50\% if the affected child has an imprinting defect
\item up to 25\% if a parental chromosome translocation is present
\end{itemize}
\item Prenatal testing is possible for pregnancies at increased risk if
the underlying genetic mechanism is known.
\end{itemize}
\end{enumerate}
\subsection{{\bfseries\sffamily TODO} Russell-Silver syndrome}
\label{sec:org80e2972}

\section{Neurogenetics}
\label{sec:org9f3ef91}
\subsection{{\bfseries\sffamily TODO} Rett syndrome}
\label{sec:org4ed4534}
\subsection{{\bfseries\sffamily TODO} Charcot-Marie-Tooth disease type IA}
\label{sec:org8b4189c}
\subsection{{\bfseries\sffamily TODO} Hereditary neuropathy with pressure palsies}
\label{sec:org7bfd936}
\subsection{{\bfseries\sffamily TODO} Spinal muscular atrophy}
\label{sec:orgbcd607b}
\end{document}