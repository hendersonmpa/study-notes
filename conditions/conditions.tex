% Created 2020-07-06 Mon 13:54
% Intended LaTeX compiler: pdflatex
\documentclass[12pt]{scrartcl}
\usepackage[utf8]{inputenc}
\usepackage[T1]{fontenc}
\usepackage{graphicx}
\usepackage{grffile}
\usepackage{longtable}
\usepackage{wrapfig}
\usepackage{rotating}
\usepackage[normalem]{ulem}
\usepackage{amsmath}
\usepackage{textcomp}
\usepackage{amssymb}
\usepackage{capt-of}
\usepackage{hyperref}
\usepackage{dejavu}
\hypersetup{colorlinks,linkcolor=black,urlcolor=blue}
\usepackage{textpos}
\usepackage{textgreek}
\usepackage[version=4]{mhchem}
\usepackage{chemfig}
\usepackage{siunitx}
\usepackage{gensymb}
\usepackage[usenames,dvipsnames]{xcolor}
\usepackage[T1]{fontenc}
\usepackage{lmodern}
\usepackage{verbatim}
\usepackage{tikz}
\usepackage{wasysym}
\usetikzlibrary{shapes.geometric,arrows,decorations.pathmorphing,backgrounds,positioning,fit,petri}
\usepackage[automark, autooneside=false, headsepline, footsepline]{scrlayer-scrpage}
\clearpairofpagestyles
\ihead{\leftmark}% section on the inner (oneside: right) side
\ohead{\rightmark}% subsection on the outer (oneside: left) side
\addtokomafont{pagehead}{\upshape}% header upshape instead of italic
\ofoot*{\pagemark}% the pagenumber in the center of the foot, also on plain pages
\pagestyle{scrheadings}
\author{Matthew Henderson, PhD, FCACB}
\date{\today}
\title{Genetic Conditions}
\hypersetup{
 pdfauthor={Matthew Henderson, PhD, FCACB},
 pdftitle={Genetic Conditions},
 pdfkeywords={},
 pdfsubject={},
 pdfcreator={Emacs 26.3 (Org mode 9.3.7)}, 
 pdflang={English}}
\begin{document}

\maketitle
\setcounter{tocdepth}{2}
\tableofcontents


\section{Cancer Genetics}
\label{sec:orgf2e9ad6}
\subsection{Tuberous sclerosis}
\label{sec:orgf524037}
\begin{enumerate}
\item Clinical Characteristics
\label{sec:org71f0135}
\begin{itemize}
\item TSC involves abnormalities of the:
\begin{description}
\item[{skin}] hypomelanotic macules, confetti skin lesions, facial
angiofibromas, shagreen patches, fibrous cephalic plaques,
ungual fibromas
\item[{brain}] subependymal nodules, cortical dysplasias, and
subependymal giant cell astrocytomas [SEGAs], seizures,
intellectual disability / developmental delay,
psychiatric illness
\item[{kidney}] angiomyolipomas, cysts, renal cell carcinomas
\item[{heart}] rhabdomyomas, arrhythmias
\item[{lungs}] lymphangioleiomyomatosis [LAM], multifocal micronodular
pneumonocyte hyperplasia.
\end{description}
\item CNS tumors are the leading cause of morbidity and mortality
\item renal disease is the second leading cause of early death.
\end{itemize}

\item Diagnosis
\label{sec:org33ca527}
\begin{itemize}
\item TSC is established in a proband with one of the following:
\begin{itemize}
\item Two major clinical features
\item One major clinical feature and two or more minor features
\item Identification of a heterozygous pathogenic variant in TSC1 or
TSC2 by molecular genetic testing
\end{itemize}
\end{itemize}

\item Genetic Counseling
\label{sec:orgd3da134}
\begin{itemize}
\item AD
\item Two thirds of affected individuals have TSC as the result of a \emph{de novo} pathogenic variant.
\item The offspring of an affected individual are at a 50\% risk of inheriting the pathogenic variant.
\item If the pathogenic variant has been identified in an affected family
member, prenatal testing for pregnancies at increased risk and
preimplantation genetic diagnosis are possible.
\end{itemize}

\item Pathophysiology
\label{sec:orgc3ea3df}
\begin{itemize}
\item Tuberin (TSC1) and hamartin (TSC2) are key regulators of the
AKT/mTOR signaling pathway and to participate in several other
signaling pathways including the MAPK, AMPK, b-catenin, calmodulin,
CDK, autophagy, and cell cycle pathways
\end{itemize}
\end{enumerate}

\subsection{Hereditary breast cancer}
\label{sec:org82c3f31}
\begin{enumerate}
\item Clinical Characteristics
\label{sec:org63f89e8}
\begin{itemize}
\item BRCA1- and BRCA2-associated hereditary breast and ovarian cancer
syndrome (HBOC) is characterized by an increased risk for female and
male:
\begin{itemize}
\item breast cancer
\item ovarian cancer (includes fallopian tube and primary peritoneal cancers)
\item to a lesser extent other cancers: prostate cancer, pancreatic
cancer, and melanoma primarily in individuals with a BRCA2
pathogenic variant
\end{itemize}
\item The exact cancer risks differ slightly depending on whether HBOC is
caused by a BRCA1 or BRCA2 pathogenic variant
\end{itemize}
\item Diagnosis
\label{sec:org59bd565}
\begin{itemize}
\item The diagnosis of BRCA1 and BRCA2 HBOC is established in a proband by
identification of a heterozygous germline pathogenic variant in
BRCA1 or BRCA2 on molecular genetic testing.
\end{itemize}

\item Genetic Counseling
\label{sec:org66fc4ba}
\begin{itemize}
\item Germline pathogenic variants in BRCA1 and BRCA2 are inherited in an
autosomal dominant manner
\item The vast majority of individuals with a BRCA1 or BRCA2 pathogenic
variant have inherited it from a parent
\item Because of incomplete penetrance, variable age of cancer
development, cancer risk reduction resulting from prophylactic
surgery, or early death, not all individuals with a BRCA1 or BRCA2
pathogenic variant have a parent affected with cancer
\item Offspring of an individual with a BRCA1 or BRCA2 germline pathogenic variant have a 50\% chance of inheriting the variant
\item Prenatal testing is possible for pregnancies at increased risk if the cancer-predisposing variant in the family is known
\item requests for prenatal diagnosis of adult-onset diseases are uncommon and require careful genetic counseling
\end{itemize}

\item Pathophysiology
\label{sec:orgc38a621}
\begin{itemize}
\item BRCA1 interacts with several proteins involved in cellular pathways,
including cell-cycle progression, gene transcription regulation, DNA
damage response, and ubiquitination
\item BRCA2 appears to be involved in the DNA repair process.
\end{itemize}
\end{enumerate}
\subsection{Chronic myelogenous leukemia (CML)}
\label{sec:org247f881}
\begin{enumerate}
\item Clinical Characteristics
\label{sec:orgc0c0fb9}
\begin{itemize}
\item CML is a cancer of the white blood cells
\item It is a form of leukemia characterized by the increased and
unregulated growth of myeloid cells in the bone marrow and the
accumulation of these cells in the blood
\item CML is a clonal bone marrow stem cell disorder in which a
proliferation of mature granulocytes (neutrophils, eosinophils and
basophils) and their precursors is found
\item It is a type of myeloproliferative neoplasm associated with a
characteristic chromosomal translocation called the Philadelphia
chromosome
\end{itemize}

\item Diagnosis
\label{sec:orgfa65450}
\begin{itemize}
\item CML is often suspected on the basis of a complete blood count,
\begin{itemize}
\item increased granulocytes of all types, typically including mature myeloid cells.
\item Basophils and eosinophils are almost universally increased; this feature may help differentiate CML from a leukemoid reaction.
\end{itemize}
\item A bone marrow biopsy is often performed as part of the evaluation for CML
\item CML is diagnosed by cytogenetics that detects the translocation t(9;22)(q34;q11.2) which involves the ABL1 gene in chromosome 9 and the BCR gene in chromosome 22.
\item As a result of this translocation, the chromosome looks smaller than
its homologue chromosome, and this appearance is known as the
Philadelphia chromosome chromosomal abnormality.
\begin{itemize}
\item can be detected by routine cytogenetics
\item involved genes BCR-ABL1 can be detected by FISH, as well as by
PCR.
\end{itemize}
\end{itemize}

\item Pathophysiology
\label{sec:org0271326}
\begin{itemize}
\item Chromosomal translocation where parts of two chromosomes (the 9th
and 22nd) switch places.
\item part of the BCR ("breakpoint cluster region") gene from chromosome
22 is fused with the ABL gene on chromosome 9.
\item abl carries a tyrosine kinase, \therfore the bcr-abl fusion gene
product is also a tyrosine kinase
\item The fused BCR-ABL protein interacts with the interleukin 3beta(c) receptor subunit.
\item The BCR-ABL transcript is continuously active and does not require activation by other cellular messaging proteins.
\item In turn, BCR-ABL activates a cascade of proteins that control the cell cycle, speeding up cell division.
\end{itemize}
\end{enumerate}
\subsection{Familial adenomatous polyposis}
\label{sec:org1a385a4}
\begin{enumerate}
\item Clinical Characteristics
\label{sec:org8d13e47}
\begin{itemize}
\item FAP is a colon cancer predisposition syndrome in which hundreds to
thousands of adenomatous colonic polyps develop, beginning, on
average, at age 16 years (range 7-36 years).
\item By age 35 years, 95\% of individuals with FAP have polyps; without
colectomy, colon cancer is inevitable.
\item The mean age of colon cancer diagnosis in untreated individuals is
39 years (range 34-43 years).
\item Extracolonic manifestations are variably present and include:
\begin{itemize}
\item polyps of the gastric fundus and duodenum, osteomas, dental anomalies,
\item congenital hypertrophy of the retinal pigment epithelium (CHRPE)
\item soft tissue tumors, desmoid tumors, and associated cancers
\end{itemize}
\end{itemize}

\item Diagnosis
\label{sec:org3644444}
\begin{itemize}
\item suspected in an individual with suggestive personal and/or family
history features and confirmed by identification of a heterozygous
germline pathogenic variant in APC.
\end{itemize}

\item Genetic Counseling
\label{sec:orgbffa723}
\begin{itemize}
\item AD
\item \textasciitilde{}75\%-80\% of individuals with an APC-associated polyposis condition
have an affected parent
\item Offspring of an affected individual are at a 50\% risk of inheriting
the pathogenic variant in APC.
\item Prenatal testing and preimplantation genetic diagnosis are possible
if a pathogenic variant has been identified in an affected family
member.
\end{itemize}
\item Pathophysiology
\label{sec:org8378fc2}
\begin{itemize}
\item The APC protein product is a tumor suppressor.
\item APC appears to prevent accumulation of cytosolic beta-catenin and
maintain normal apoptosis and may also decrease cell proliferation,
probably through its regulation of beta-catenin.
\end{itemize}
\end{enumerate}
\subsection{Hereditary non-polyposis colon cancer (HNPCC)}
\label{sec:orga2b4eee}
\begin{itemize}
\item AKA: Lynch Syndrome
\end{itemize}
\begin{enumerate}
\item Clinical Characteristics
\label{sec:org19aece9}
\begin{itemize}
\item increased risk for colorectal cancer (CRC) and cancers of the
endometrium, stomach, ovary, small bowel, hepatobiliary tract,
urinary tract, brain, and skin.

\item In individuals with Lynch syndrome the following lifetime risks for
cancer are seen:
\begin{description}
\item[{CRC}] 52\%-82\% (mean age at diagnosis 44-61 years)
\item[{Endometrial cancer in females}] 25\%-60\% (mean age at diagnosis 48-62 years)
\item[{Gastric cancer}] 6\%-13\% (mean age at diagnosis 56 years)
\item[{Ovarian cancer}] 4\%-12\% (mean age at diagnosis 42.5 years; \textasciitilde{}30\% are diagnosed < age 40 years).
\end{description}

\item The risk for other Lynch syndrome-related cancers is lower, though
substantially increased over general population rates
\end{itemize}

\item Diagnosis
\label{sec:org84398d5}
\begin{itemize}
\item Lynch syndrome is established in a proband by identification of a
germline heterozygous pathogenic variant in MLH1, MSH2, MSH6, or
PMS2 or an EPCAM deletion on molecular genetic testing.
\end{itemize}

\item Genetic Counseling
\label{sec:org1a1db82}
\begin{itemize}
\item AD
\item The majority of individuals diagnosed with Lynch syndrome have
inherited the condition from a parent.
\item because of incomplete penetrance, variable age of cancer
development, cancer risk reduction as a result of screening or
prophylactic surgery, or early death, not all individuals with a
pathogenic variant in one of the genes associated with Lynch
syndrome have a parent who had cancer.
\item Each child of an individual with Lynch syndrome has a 50\% chance of
inheriting the pathogenic variant.
\item Prenatal diagnosis for pregnancies at increased risk is possible if
the pathogenic variant in the family is known.
\end{itemize}

\item Pathophysiology
\label{sec:org01a09eb}
\begin{itemize}
\item EPCAM 2p21	Epithelial cell adhesion molecule
\begin{itemize}
\item EPCAM is not a mismatch repair gene, recurrent germline deletions
of the 3' region result in silencing of the adjacent downstream
MSH2 by hypermethylation
\item The adjacent MSH2 allele itself is not mutated
\item Sequence analysis of EPCAM is not appropriate for diagnosis of
Lynch syndrome
\end{itemize}
\item MLH1	3p22​.2	DNA mismatch repair protein
\item MSH2	2p21-p16 DNA mismatch repair protein
\item MSH6	2p16​.3	DNA mismatch repair protein
\item PMS2	7p22​.1	Mismatch repair endonuclease
\end{itemize}
\end{enumerate}

\subsection{Li-Fraumeni syndrome}
\label{sec:org56e9208}
\begin{enumerate}
\item Clinical Characteristics
\label{sec:orgff9eae9}
\begin{itemize}
\item LFS is a cancer predisposition syndrome associated with the
development of the following classic tumors:
\begin{itemize}
\item soft tissue sarcoma
\item osteosarcoma
\item pre-menopausal breast cancer
\item brain tumors
\item adrenocortical carcinoma (ACC)
\item leukemias.
\end{itemize}
\item In addition, a variety of other neoplasms may occur.
\item LFS-related cancers often occur in childhood or young adulthood and
survivors have an increased risk for multiple primary cancers.
\end{itemize}

\item Diagnosis
\label{sec:orge96b876}
\begin{itemize}
\item LFS is diagnosed in individuals meeting established clinical
criteria or in those who have a germline pathogenic variant in TP53
regardless of family cancer history.
\item At least 70\% of individuals diagnosed clinically have an
identifiable germline pathogenic variant in TP53, the only gene so
far identified in which pathogenic variants are definitively
associated with LFS.
\end{itemize}
\item Genetic Counseling
\label{sec:org7e7c3da}
\begin{itemize}
\item AD
\item 7-20\% \emph{de novo} germline TP53 pathogenic variant
\item Offspring of an affected individual have a 50\% chance of inheriting
the pathogenic variant.
\item Predisposition testing for at-risk family members and prenatal
testing for pregnancies at increased risk are possible if the
heritable pathogenic variant in the family has been identified.
\end{itemize}

\item Pathophysiology
\label{sec:orgff8f744}
\begin{itemize}
\item TP53 has been called "the guardian of the genome" and its protein
plays major roles in both the regulation of cell growth and the
maintenance of homeostasis
\item The loss of this important tumor suppressor gene decreases the
likelihood that cells with genetic errors will be flagged for DNA
repair or apoptosis. These DNA-damaged cells can go on to further
proliferate, which can lead to a colony of abnormal cells and
eventually a malignant tumor.
\end{itemize}
\end{enumerate}

\subsection{Retinoblastoma}
\label{sec:org1ff6d42}
\begin{enumerate}
\item Clinical Characteristics
\label{sec:org7fef0a4}
\begin{itemize}
\item Retinoblastoma is a malignant tumor of the developing retina that occurs in children, usually before age five years
\item Retinoblastoma develops from cells that have cancer-predisposing variants in both copies of RB1
\item Retinoblastoma may be unifocal or multifocal.
\item \(\sim\) 60\% of affected individuals have unilateral retinoblastoma with a mean age of diagnosis of 24 months
\item \(\sim\) 40\% have bilateral retinoblastoma with a mean age of diagnosis of 15 months
\item Heritable retinoblastoma is an autosomal dominant susceptibility for retinoblastoma
\item Individuals with heritable retinoblastoma are also at increased risk of developing non-ocular tumors
\end{itemize}

\item Diagnosis
\label{sec:org439561c}
\begin{itemize}
\item established by examination of the fundus of the eye using indirect
ophthalmoscopy
\item Imaging studies can be used to support the diagnosis and stage the tumor
\item The diagnosis of heritable retinoblastoma is established in a
proband with:
\begin{itemize}
\item retinoblastoma or retinoma and a family history of retinoblastoma or
\item identification of a heterozygous germline pathogenic variant in RB1.
\end{itemize}

\item The following staging has been recommended for individuals with
retinoblastoma and/or risk of heritable retinoblastoma to include
"H" to describe the genetic risk for an individual to have a
germline pathogenic variant in RB1:

\begin{description}
\item[{HX}] Unknown or insufficient evidence of a constitutional
(germline) RB1 pathogenic variant

\item[{H0}] Normal RB1 alleles in blood tested with demonstrated
high-sensitivity assays

\item[{H0*}] Normal RB1 in blood with <1\% residual risk for mosaicism

\item[{H1}] Bilateral retinoblastoma, trilateral retinoblastoma
(retinoblastoma with intracranial CNS midline embryonic
tumor), family history of retinoblastoma, or RB1 pathogenic
variant identified in blood
\end{description}
\end{itemize}

\item Genetic Counseling
\label{sec:org51fe082}
\begin{itemize}
\item AD
\item Individuals with heritable retinoblastoma (H1) have a heterozygous
\emph{de novo} or inherited germline RB1 pathogenic variant.
\item Offspring of H1 individuals have a 50\% chance of inheriting the
pathogenic variant.
\item Prenatal testing for pregnancies at increased risk is possible if
the RB1 pathogenic variant has been identified in an affected family
member.
\end{itemize}

\item Pathophysiology
\label{sec:org3f7c67f}
\begin{itemize}
\item RB1 encodes a ubiquitously expressed nuclear protein that is
involved in cell cycle regulation (G1 to S transition)
\item The RB protein is phosphorylated by members of the cyclin-dependent kinase
(cdk) system prior to the entry into S-phase
\item On phosphorylation, the binding activity of the pocket domain is
lost, resulting in the release of cellular proteins.
\end{itemize}
\end{enumerate}

\section{Clinical Genetics}
\label{sec:org306bcce}
\subsection{Duchenne/Becker Muscular Dystrophy}
\label{sec:orgd0c16c0}
\begin{enumerate}
\item Clinical Characteristics
\label{sec:orgc594826}
\begin{itemize}
\item The dystrophinopathies cover a spectrum of X-linked muscle disease
ranging from mild to severe that includes Duchenne muscular
dystrophy (DMD), Becker muscular dystrophy (BMD), and DMD-associated
dilated cardiomyopathy (DCM)
\item The mild end of the spectrum includes the phenotypes of asymptomatic
increase in serum concentration of creatine phosphokinase (CK) and
muscle cramps with myoglobinuria
\item The severe end of the spectrum includes progressive muscle diseases
that are classified as Duchenne/Becker muscular dystrophy when
skeletal muscle is primarily affected and as DMD-associated dilated
cardiomyopathy (DCM) when the heart is primarily affected
\end{itemize}

\begin{enumerate}
\item DMD
\label{sec:org8c314e8}
\begin{itemize}
\item usually presents in early childhood with delayed motor milestones
including delays in walking independently and standing up from a
supine position
\item proximal weakness causes a waddling gait and difficulty climbing
stairs, running, jumping, and standing up from a squatting
position
\item rapidly progressive, with affected children being
wheelchair dependent by age 12 years
\item few survive beyond the third decade, with respiratory complications and
progressive cardiomyopathy being common causes of death
\end{itemize}

\item BMD
\label{sec:org148f6f4}
\begin{itemize}
\item characterized by later-onset skeletal muscle weakness
\item heart failure is a common cause of morbidity and the most common
cause of death in BMD
\item mean age of death is in the mid-40s
\end{itemize}
\item DCM
\label{sec:org44c689a}
\begin{itemize}
\item characterized by left ventricular dilation and congestive heart
failure
\item females heterozygous for a DMD pathogenic variant are at increased
risk for DCM
\end{itemize}
\end{enumerate}
\item Diagnostic Testing
\label{sec:org4f736a9}
\begin{itemize}
\item \(\Uparrow\) CK
\item DMD molecular
\begin{itemize}
\item \male with hemizygous pathogenic variant
\item \female with heterozygous pathogenic variant
\end{itemize}
\end{itemize}
\item Genetic Counseling
\label{sec:org82a60f3}
\begin{itemize}
\item X-linked
\item daughters who inherit the pathogenic variant are heterozygous and
may have a range of clinical manifestations
\end{itemize}
\end{enumerate}
\subsection{Osteogenesis Imperfecta}
\label{sec:org4767c62}
\begin{enumerate}
\item Clinical Characteristics
\label{sec:org67cfa0e}
\begin{itemize}
\item characterized by fractures with minimal or absent trauma, variable
dentinogenesis imperfecta
\item four types
\begin{itemize}
\item classic non-deforming OI with blue sclerae
\item perinatally lethal OI
\item progressively deforming OI
\item common variable OI with normal sclerae
\end{itemize}
\end{itemize}

\item Diagnostic Testing
\label{sec:org789cece}
\begin{itemize}
\item identification of a heterozygous pathogenic or likely pathogenic
variant in COL1A1 or COL1A2
\end{itemize}
\item Genetic Counseling
\label{sec:org6088ed8}
\begin{itemize}
\item AD
\end{itemize}
\end{enumerate}
\subsection{Long QT Syndrome}
\label{sec:orgd5a1602}
\begin{enumerate}
\item Clinical Characteristics
\label{sec:orgb297953}
\begin{itemize}
\item cardiac electrophysiologic disorder, characterized by QT
prolongation and T-wave abnormalities on the ECG that are associated
with tachyarrhythmias, typically the ventricular tachycardia torsade
de pointes (TdP)
\item TdP is usually self-terminating, thus causing a syncopal event, the
most common symptom in individuals with LQTS
\item such cardiac events typically occur during exercise and emotional
stress, less frequently during sleep, and usually without warning
\item in some instances, TdP degenerates to ventricular fibrillation and
causes aborted cardiac arrest (if the individual is defibrillated)
or sudden death
\item approximately 50\% of untreated individuals with a pathogenic variant
in one of the genes associated with LQTS have symptoms, usually one
to a few syncopal events
\item cardiac events may occur from infancy through middle age, they are
most common from the preteen years through the 20s
\end{itemize}
\item Diagnostic Testing
\label{sec:orga8f6d02}
\begin{itemize}
\item established by prolongation of the QTc interval in the absence of
specific conditions known to lengthen it
\item diagnostic variants in one or more of the 15 genes known to be
associated with LQTS
\begin{itemize}
\item KCNH2 (LQT2), KCNQ1 (LQT1), and SCN5A (LQT3) are the most common
\end{itemize}
\end{itemize}
\item Genetic Counseling
\label{sec:org2017029}
\begin{itemize}
\item AD
\end{itemize}
\end{enumerate}
\subsection{Marfan Syndrome}
\label{sec:orga6082de}
\begin{enumerate}
\item Clinical Characteristics
\label{sec:org84b66d3}
\begin{itemize}
\item a systemic disorder of connective tissue with a high degree of
clinical variability, comprises a broad phenotypic continuum ranging
from mild to severe and rapidly progressive neonatal multiorgan
disease
\item cardinal manifestations involve the ocular, skeletal, and
cardiovascular systems
\item normal life expectancy with proper management
\end{itemize}
\item Diagnostic Testing
\label{sec:org88c8cf9}
\begin{itemize}
\item one of the following sets of findings:

\item FBN1 pathogenic variant known to be associated with Marfan syndrome
AND one of the following:
\begin{itemize}
\item aortic root enlargement
\item ectopia lentis
\end{itemize}
\item aortic root enlargement and ectopia lentis OR a
defined combination of features throughout the body
\end{itemize}

\item Genetic Counseling
\label{sec:org9375f55}
\begin{itemize}
\item AD
\item \(\sim\) 75\% of individuals with Marfan syndrome have an affected
parent
\item \(\sim\) 25\% have a \emph{de novo} FBN1 pathogenic variant
\end{itemize}
\end{enumerate}
\subsection{Neurofibromatosis Type I}
\label{sec:org90af521}
\begin{enumerate}
\item Clinical Characteristics
\label{sec:org74c1099}
\begin{itemize}
\item characterized by multiple café au lait spots, axillary and inguinal freckling, multiple cutaneous neurofibromas, iris Lisch nodules, and choroidal freckling
\item \textasciitilde{} 50\% have plexiform neurofibromas, but most are internal and not suspected clinically
\item learning disabilities in \textasciitilde{} 50\%
\item Less common manifestations include optic nerve and other central
nervous system gliomas, malignant peripheral nerve sheath tumors,
scoliosis, tibial dysplasia, and vasculopathy
\end{itemize}
\item Diagnostic Testing
\label{sec:org8fbf169}
\begin{itemize}
\item usually based on clinical findings
\item heterozygous pathogenic variants in NF1 are responsible for neurofibromatosis 1
\item molecular genetic testing of NF1 is rarely needed for diagnosis
\end{itemize}
\item Genetic Counseling
\label{sec:orgf45720f}
\begin{itemize}
\item AD, NF1
\item 50\% due to \emph{de novo} NF1 pathogenic variant
\end{itemize}
\end{enumerate}
\subsection{Neurofibromatosis Type II}
\label{sec:org7fc17ad}
\begin{enumerate}
\item Clinical Characteristics
\label{sec:org79bda67}
\begin{itemize}
\item bilateral vestibular schwannomas with associated symptoms of tinnitus, hearing loss, and balance dysfunction
\item average age of onset is 18 to 24 years
\item almost all affected individuals develop bilateral vestibular
schwannomas by age 30 years
\item NF2 is considered an adult-onset disease
\end{itemize}
\item Diagnostic Testing
\label{sec:org54f183d}
\begin{itemize}
\item consensus diagnostic criteria and/or by identification of a
heterozygous pathogenic variant in NF2 on molecular genetic testing
\end{itemize}

\item Genetic Counseling
\label{sec:org5ec57c9}
\begin{itemize}
\item AD, NF2
\item 50\% with affected parent
\item 50\% due to \emph{de novo} NF2 pathogenic variant
\item mosaic also possible
\end{itemize}
\end{enumerate}
\subsection{Smith-Lemli-Opitz Syndrome}
\label{sec:org9017ae0}
\begin{enumerate}
\item Clinical Characteristics
\label{sec:org221a1e0}
\begin{itemize}
\item a congenital multiple-anomaly/cognitive impairment syndrome caused
by an abnormality in cholesterol metabolism resulting from
deficiency of the enzyme 7-dehydrocholesterol (7-DHC) reductase
\item characterized by prenatal and postnatal growth restriction,
microcephaly, moderate-to-severe intellectual disability, and
multiple major and minor malformations
\item distinctive facial features, cleft palate, cardiac defects,
underdeveloped external genitalia in males, postaxial polydactyly,
and 2-3 syndactyly of the toes
\item wide clinical spectrum
\end{itemize}
\item Diagnostic Testing
\label{sec:orgf1c4a53}
\begin{itemize}
\item suggestive clinical features
\item elevated 7-dehydrocholesterol level
\item identification of biallelic pathogenic variants in DHCR7
\item serum concentration of cholesterol is usually low
\begin{itemize}
\item it may be in the normal range in approximately 10\% of affected
individuals \(\therefore\) unreliable for screening and diagnosis
\end{itemize}
\end{itemize}
\item Genetic Counseling
\label{sec:org8983fa2}
\begin{itemize}
\item AR DHCR7
\end{itemize}
\end{enumerate}
\subsection{Noonan Syndrome}
\label{sec:orgf260c5a}
\begin{enumerate}
\item Clinical Characteristics
\label{sec:org39f428e}
\begin{itemize}
\item lentigines, hypertrophic cardiomyopathy, short stature, pectus
deformity, and dysmorphic facial features, including widely spaced
eyes and ptosis
\item multiple lentigines present as dispersed flat, black-brown macules,
mostly on the face, neck and upper part of the trunk
\begin{itemize}
\item do not appear until age four to five years but then increase to
the thousands by puberty
\end{itemize}
\end{itemize}
\item Diagnostic Testing
\label{sec:org94d465d}
\begin{itemize}
\item clinical findings or, if clinical findings are insufficient, by
identification of a heterozygous pathogenic variant in one of four
genes
\begin{itemize}
\item PTPN11, RAF1, BRAF, and MAP2K1
\end{itemize}
\end{itemize}
\item Genetic Counseling
\label{sec:orgc5e09c4}
\begin{itemize}
\item AD
\end{itemize}
\end{enumerate}
\subsection{Charge Syndrome}
\label{sec:org73e391d}
\begin{enumerate}
\item Clinical Characteristics
\label{sec:org9b73b63}
\begin{itemize}
\item CHARGE is a mnemonic for coloboma, heart defects, choanal atresia,
retarded growth and development, genital abnormalities, and ear
anomalies
\item neonates with CHARGE syndrome often have multiple life-threatening
medical conditions
\item feeding difficulties are a major cause of
morbidity in all age groups
\end{itemize}
\item Diagnostic Testing
\label{sec:org7ddcf49}
\begin{itemize}
\item clinical findings and temporal bone imaging
\item CHD7 encodes chromodomain helicase DNA binding protein
\end{itemize}
\item Genetic Counseling
\label{sec:org78f2d2f}
\begin{itemize}
\item AD CHD7
\end{itemize}
\end{enumerate}
\subsection{FGFR Craniosynostosis}
\label{sec:orgc98e6e8}
\begin{enumerate}
\item Clinical Characteristics
\label{sec:org869e6ce}
\begin{itemize}
\item pectrum of severity ranges from severe prenatal multisuture
craniosynostosis with feeding and airway issues to isolated
unicoronal craniosynostosis
\end{itemize}
\item Diagnostic Testing
\label{sec:org8dc0a62}
\begin{itemize}
\item a craniosynostosis multigene panel that includes
\begin{itemize}
\item FGFR1, FGFR2, FGFR3, TCF12, and TWIST1
\end{itemize}
\end{itemize}
\item Genetic Counseling
\label{sec:orgb56e76c}
\begin{itemize}
\item AD
\end{itemize}
\end{enumerate}
\subsection{Factor V Leiden Thrombophilia}
\label{sec:orge1e89d3}
\begin{enumerate}
\item Clinical Characteristics
\label{sec:org76f486b}
\begin{itemize}
\item characterized by a poor anticoagulant response to activated protein
C (APC) and an increased risk for venous thromboembolism (VTE)
\item DVT is the most common VTE, with the legs being the most common
site
\end{itemize}
\item Diagnostic Testing
\label{sec:orgb92ba04}
\begin{itemize}
\item a history of first and recurrent venous thromboembolism (VTE)
manifest as deep vein thrombosis (DVT) or pulmonary embolism (PE),
especially in women with a history of VTE during pregnancy or in
association with use of estrogen-containing contraceptives
\item a family history of recurrent thrombosis

\item identification of a heterozygous or homozygous c.1691G>A variant in F5
(the factor V Leiden variant)
\begin{itemize}
\item in conjunction with coagulation tests
\end{itemize}
\end{itemize}
\item Genetic Counseling
\label{sec:org4ff6b1f}
\begin{itemize}
\item AD F5
\end{itemize}
\end{enumerate}
\subsection{G6PD deficiency}
\label{sec:org2497a34}
\begin{enumerate}
\item Clinical Characteristics
\label{sec:org111ade2}
\begin{itemize}
\item hemolytic anemia
\begin{itemize}
\item results in paleness, yellowing of the skin and whites of the
eyes (jaundice), dark urine, fatigue, shortness of breath, and a
rapid heart rate
\end{itemize}
\item G6PD converts glucose-6-phosphate into
6-phosphoglucono-\(\delta\)-lactone and is the rate-limiting enzyme of
the PP pathway pathway that supplies reducing energy to cells by
maintaining the level NADPH
\item NADPH maintains the supply of reduced glutathione in the cells that
is used to mop up free radicals that cause oxidative damage
\end{itemize}
\item Diagnostic Testing
\label{sec:org9f076d5}
\begin{itemize}
\item CBC, Heinz bodies on film
\item \(\uparrow\) LDH in hemolysis
\item \(\downarrow\) haptoglobin
\item Beutler spot test
\end{itemize}
\item Genetic Counseling
\label{sec:orgf598c75}
\begin{itemize}
\item X-linked recessive G6PD
\end{itemize}
\end{enumerate}

\subsection{Sickle Cell Disease}
\label{sec:org1b1ac48}
\begin{enumerate}
\item Clinical Characteristics
\label{sec:org46f8421}
\begin{itemize}
\item characterized by intermittent vaso-occlusive events and chronic
hemolytic anemia
\item vaso-occlusive events result in tissue ischemia leading to acute and
chronic pain as well as organ damage that can affect any organ
system, including the bones, spleen, liver, brain, lungs, kidneys,
and joints
\item dactylitis (pain and/or swelling of the hands or feet) is often the
earliest manifestation
\end{itemize}
\item Diagnostic Testing
\label{sec:orgd988b94}
\begin{itemize}
\item encompasses a group of disorders characterized by the presence of at
least one hemoglobin S allele (HbS; p.Glu6Val in HBB) and a second
HBB pathogenic variant resulting in abnormal hemoglobin
polymerization
\item Hb S/S (homozygous p.Glu6Val in HBB) accounts for 60-70\% of SCD in
the United States
\item other forms of SCD result from coinheritance of HbS with other
abnormal \(\beta\)-globin chain variants
\begin{itemize}
\item the most common forms being
\begin{itemize}
\item sickle-hemoglobin C disease (Hb S/C)
\item two types of sickle \(\beta\)-thalassemia
\begin{itemize}
\item Hb S/\(\beta\)\^{}+-thalassemia
\item Hb S/\(\beta\)\textsuperscript{0}-thalassemia
\end{itemize}
\end{itemize}
\item rarer forms result from coinheritance of other Hb variants such as
D-Punjab, O-Arab, and E
\end{itemize}
\item homozygous hemoglobin S alleles
\begin{itemize}
\item sickle cell disease (Hb S/S)
\end{itemize}
\item coinheritance of one hemoglobin S allele and a second HBB pathogenic variant
\begin{itemize}
\item sickle-hemoglobin C disease (Hb S/C)
\item sickle \(\beta\)-thalassemia (Hb S/\(\beta\)\^{}+-thalassemia and Hb S/\(\beta\)\textsuperscript{0}-thalassemia)
\item sickle-hemoglobin D, O, and E disease (or other beta globin chain variants)
\end{itemize}
\end{itemize}
\item Genetic Counseling
\label{sec:org298e4c2}
\begin{itemize}
\item AR HBB
\end{itemize}
\end{enumerate}
\subsection{\(\alpha\)-Thalassemia}
\label{sec:orga7d9cfb}
\begin{enumerate}
\item Clinical Characteristics
\label{sec:orgdae8791}
\begin{itemize}
\item two clinically significant forms:
\begin{itemize}
\item hemoglobin Bart hydrops fetalis (Hb Bart) syndrome, caused by
deletion of all four \(\alpha\)-globin genes
\begin{itemize}
\item characterized by fetal onset of generalized edema, pleural and
pericardial effusions, and severe hypochromic anemia, in the
absence of ABO or Rh blood group incompatibility
\item additional clinical features include marked hepatosplenomegaly,
extramedullary erythropoiesis, hydrocephalus, and cardiac and
urogenital defects
\item death usually occurs in the neonatal period
\end{itemize}
\item hemoglobin H (HbH) disease, most frequently caused by deletion of
three \(\alpha\)-globin genes 
\begin{itemize}
\item characterized by microcytic hypochromic hemolytic anemia,
splenomegaly, mild jaundice, and sometimes thalassemia-like bone
changes
\item individuals with HbH disease may develop gallstones and
experience acute episodes of hemolysis in response to oxidant
drugs and infections
\end{itemize}
\end{itemize}
\end{itemize}
\item Diagnostic Testing
\label{sec:org39e3da3}
\begin{itemize}
\item diagnosis of Hb Bart syndrome is established in a fetus with the
characteristic radiographic and laboratory features
\begin{itemize}
\item identification of biallelic pathogenic variants in HBA1 and HBA2
that result in deletion or inactivation of all four \(\alpha\)-globin
alleles confirms the diagnosis
\end{itemize}
\item diagnosis of HbH disease is established in a proband with the
characteristic laboratory and clinical features
\begin{itemize}
\item identification of biallelic pathogenic variants in HBA1 and HBA2
that result in deletion or inactivation of three \(\alpha\)-globin
alleles confirms the diagnosis
\end{itemize}
\end{itemize}
\item Genetic Counseling
\label{sec:orgbf6d418}
\begin{itemize}
\item AR
\end{itemize}

\begin{center}
\begin{tabular}{ll}
Phenotype & Genotype\\
\hline
Hb Bart & loss of 4 \(\alpha\)-globin genes\\
HbH disease & loss of 3 \(\alpha\)-globin genes\\
\(\alpha\)-thalassemia trait & loss of 2 \(\alpha\)-globin genes\\
 & in cis (--/\(\alpha \alpha\), \(\alpha\)0 carrier) or in trans (-\(\alpha\) /-\(\alpha\))\\
\(\alpha\)-thalassemia silent carrier & loss of 1 \(\alpha\)-globin gene (-\(\alpha\)/\(\alpha \alpha\), \(\alpha\)+ carrier)\\
\end{tabular}
\end{center}
\end{enumerate}
\subsection{\(\beta\)-Thalassemia}
\label{sec:org75b5873}
\begin{enumerate}
\item Clinical Characteristics
\label{sec:org9b2fc2b}
\begin{itemize}
\item characterized by reduced synthesis of the hemoglobin beta chain that
results in microcytic hypochromic anemia, an abnormal peripheral
blood smear with nucleated red blood cells, and reduced amounts of
hemoglobin A (HbA) on hemoglobin analysis
\item individuals with thalassemia major have severe anemia and
hepatosplenomegaly
\item usually come to medical attention within the first two years of
life
\item without treatment, affected children have severe failure to thrive
and shortened life expectancy
\item treatment with a regular transfusion program and chelation therapy,
aimed at reducing transfusion iron overload, allows for normal
growth and development and may improve the overall prognosis
\end{itemize}
\item Diagnostic Testing
\label{sec:org5c87130}
\begin{itemize}
\item red blood cell indices that reveal microcytic hypochromic anemia,
nucleated red blood cells on peripheral blood smear
\item hemoglobin analysis that reveals decreased amounts of HbA and
increased amounts of hemoglobin F (HbF) after age 12 months
\item clinical severity of anemia
\item identification of biallelic pathogenic variants in HBB
\end{itemize}
\item Genetic Counseling
\label{sec:org6c26215}
\begin{itemize}
\item AR HBB
\begin{itemize}
\item carrier thalassemia minor
\end{itemize}
\end{itemize}
\end{enumerate}
\subsection{Hemophilia A}
\label{sec:org3bc12c5}
\begin{enumerate}
\item Clinical Characteristics
\label{sec:org0b8603a}
\begin{itemize}
\item deficiency in factor VIII clotting activity that results in
prolonged oozing after injuries, tooth extractions, or surgery, and
delayed or recurrent bleeding prior to complete wound healing.

\item age of diagnosis and frequency of bleeding episodes are related to
the level of factor VIII clotting activity.

\begin{description}
\item[{severe hemophilia A}] 2 - 5 spontaneous bleeding episodes each month
\begin{itemize}
\item are usually diagnosed during the first 2 years of life following
bleeding from minor injuries.
\item spontaneous joint bleeds or deep-muscle hematomas,
\item prolonged bleeding or excessive pain and swelling from minor
injuries, surgery, and tooth extractions.
\end{itemize}

\item[{moderate hemophilia A}] seldom have spontaneous bleeding;
\begin{itemize}
\item prolonged or delayed oozing after relatively minor trauma
\item usually diagnosed before age 5 or 6
\end{itemize}

\item[{mild hemophilia A }] do not have spontaneous bleeding episodes;
\begin{itemize}
\item without pre- and postoperative treatment, abnormal bleeding occurs with surgery
or tooth extractions
\item often not diagnosed until later in life
\end{itemize}
\end{description}
\end{itemize}
\item Diagnostic Testing
\label{sec:orgc97ecf7}
\begin{itemize}
\item low factor VIII clotting activity in the presence of a normal,
functional von Willebrand factor level
\item a hemizygous F8 pathogenic variant in a male proband confirms the
diagnosis.
\item a heterozygous F8 pathogenic variant in a symptomatic female
confirms the diagnosis.
\end{itemize}
\item Genetic Counseling
\label{sec:orgbcd0663}
\begin{itemize}
\item X-linked, F8
\item risk to sibs of a proband depends on the carrier status of the mother.
\item Carrier females have a 50\% chance of transmitting the F8 pathogenic
variant in each pregnancy:
\begin{itemize}
\item sons who inherit the pathogenic variant will be affected
\item daughters who inherit the pathogenic variant are carriers.
\end{itemize}
\item Affected males transmit the pathogenic variant to all of their
daughters and none of their sons.
\item Carrier testing for at-risk family members and prenatal testing for
pregnancies at increased risk are possible if the F8 pathogenic
variant has been identified or if informative intragenic linked
markers have been identified.
\end{itemize}
\end{enumerate}
\subsection{Hemophilia B}
\label{sec:org024ff52}
\begin{enumerate}
\item Clinical Characteristics
\label{sec:orgb669602}
\begin{itemize}
\item deficiency in factor IX clotting
\item same as Hemophilia A (section \ref{sec:org3bc12c5})
\end{itemize}
\item Diagnostic Testing
\label{sec:orga393cc7}
\begin{itemize}
\item low factor IX clotting activity
\item hemizygous F9 pathogenic variant in a male proband confirms the
diagnosis.
\item heterozygous F9 pathogenic variant on in a symptomatic female
confirms the diagnosis.
\end{itemize}
\item Genetic Counseling
\label{sec:org7433e6b}
\begin{itemize}
\item X-linked, F9
\item same as Hemophilia A (section \ref{sec:org3bc12c5})
\end{itemize}
\end{enumerate}
\subsection{Hemochromatosis}
\label{sec:orga93cf39}
\begin{enumerate}
\item Clinical Characteristics
\label{sec:org38629fe}
\begin{itemize}
\item inappropriately high absorption of iron by the small intestinal
mucosa.

\item The phenotypic spectrum of HFE hemochromatosis includes:

\begin{description}
\item[{Clinical HFE hemochromatosis}] manifestations of end-organ damage secondary to iron overload are present
\begin{itemize}
\item excessive storage of iron in the liver, skin, pancreas, heart, joints, and anterior pituitary gland.
\item early symptoms include: abdominal pain, weakness, lethargy, weight loss, arthralgias, diabetes mellitus; and increased risk of cirrhosis
\end{itemize}
\item[{Biochemical HFE hemochromatosis}] \(\uparrow\) transferrin-iron saturation, and the only evidence of iron overload is \(\uparrow\) serum ferritin
\item[{Non-expressing p.Cys282Tyr homozygotes}] neither clinical manifestations of HFE hemochromatosis nor iron overload are present
\end{description}
\end{itemize}

\item Diagnostic Testing
\label{sec:org9fdd7fe}
\begin{itemize}
\item biallelic HFE pathogenic variants on molecular genetic testing.
\end{itemize}
\item Genetic Counseling
\label{sec:orgae95b71}
\begin{itemize}
\item AR, HFE
\begin{description}
\item[{Risk to sibs}] when both parents of a person with hemochromatosis
are heterozygous for an HFE p.Cys282Tyr variant,
the risk to sibs of inheriting two HFE p.Cys282Tyr
variants is 25\%.
\begin{itemize}
\item Because the HFE p.Cys282Tyr heterozygote prevalence in persons
of European origin is high (11\%, or 1/9), some parents of HFE
p.Cys282Tyr homozygotes have two abnormal HFE alleles.
\item If one parent is heterozygous and the other parent homozygous
for two abnormal HFE alleles, the risk to each sib of inheriting
two HFE pathogenic alleles is 50\%.
\end{itemize}
\item[{Risk to offspring}] Offspring of an individual with HFE
hemochromatosis inherit one HFE p.Cys282Tyr variant from the
parent with HFE hemochromatosis.
\begin{itemize}
\item Because the chance that the other parent is a heterozygote for
HFE p.Cys282Tyr is 1/9, the risk that the offspring will inherit
two HFE p.Cys282Tyr variants is approximately 5\%.
\end{itemize}
\item[{Prenatal testing}] not usually performed because HFE
hemochromatosis is an adult-onset, treatable disorder with low
clinical penetrance.
\end{description}
\end{itemize}
\end{enumerate}

\subsection{SRY translocation}
\label{sec:orgb218e19}
\begin{itemize}
\item Nonsyndromic 46,XX Testicular Disorders of Sex Development
\end{itemize}
\begin{enumerate}
\item Clinical Characteristics
\label{sec:orgc9ea555}
\begin{itemize}
\item characterized by the presence of a 46,XX karyotype
\begin{itemize}
\item male external genitalia ranging from normal to ambiguous
\item two testicles
\item azoospermia
\item absence of müllerian structures
\end{itemize}
\item \(\sim\) 85\% of individuals with nonsyndromic 46,XX testicular DSD
present after puberty with normal pubic hair and normal penile size
but small testes, gynecomastia, and sterility resulting from
azoospermia
\item \(\sim\) 15\% of individuals with nonsyndromic 46,XX testicular DSD
present at birth with ambiguous genitalia
\item gender role and gender identity are reported as \male
\item untreated \males with 46,XX testicular DSD experience the
consequences of testosterone deficiency
\end{itemize}
\item Diagnostic Testing
\label{sec:org26fd42e}
\begin{itemize}
\item clinical findings, endocrine testing, and cytogenetic testing
\item endocrine studies usually show hypergonadotropic hypogonadism
secondary to testicular failure
\item cytogenetic studies at the 550-band level demonstrate a 46,XX
karyotype
\item SRY, the gene that encodes the sex-determining region Y protein, is
the principal gene known to be associated with 46,XX testicular
DSD
\item \(\sim\) 80\% of individuals with nonsyndromic 46,XX testicular DSD are
SRY positive as shown by use of FISH or chromosomal microarray (CMA)
\item rearrangements in or around SOX9 and SOX3 detected by CMA, or rarely
karyotype, have recently been reported in a few cases
\begin{itemize}
\item at least one more unknown gene at another locus is implicated
\end{itemize}
\end{itemize}
\item Genetic Counseling
\label{sec:org2f19186}
\begin{itemize}
\item SRY-positive 46,XX testicular DSD is generally not inherited because
it results from \emph{de novo} abnormal interchange between the Y
chromosome and the X chromosome, resulting in the presence of SRY on
the X chromosome and infertility
\end{itemize}
\end{enumerate}
\subsection{Turner syndrome}
\label{sec:org5279b0b}
\begin{enumerate}
\item Clinical Characteristics
\label{sec:org7a370f9}
\begin{itemize}
\item 45,X, \female is partly or completely missing an X chromosome.
\item signs and symptoms vary, often a short and webbed neck, low-set
ears, low hairline at the back of the neck, short stature, and
swollen hands and feet are seen at birth.
\item develop menstrual periods and breasts only with hormone treatment,
and are unable to have children without reproductive technology
\item heart defects, diabetes, and low thyroid hormone occur more
frequently
\item most people with TS have normal intelligence
\item vision and hearing problems occur more often
\end{itemize}
\item Diagnostic Testing
\label{sec:org6d9435c}
\begin{itemize}
\item amniocentesis or chorionic villus sampling
\item abnormal ultrasound findings (i.e., heart defect, kidney
abnormality, cystic hygroma, ascites)
\item \(\uparrow\) risk of Turner syndrome may also be indicated by abnormal
triple or quadruple maternal serum screen
\item later diagnosis is via karyotype
\end{itemize}
\item Genetic Counseling
\label{sec:org69cedc4}
\begin{itemize}
\item functional X usually from mother
\item usually sporadic
\item exceptions:
\begin{itemize}
\item presence of a balanced translocation of the X chromosome in a parent
\item mother has 45,X mosaicism restricted to her germ cells
\end{itemize}
\end{itemize}
\end{enumerate}
\subsection{Androgen Insensitivity Syndrome}
\label{sec:orgdb47115}
\begin{enumerate}
\item Clinical Characteristics
\label{sec:orgd7921fd}
\begin{itemize}
\item characterized by evidence of feminization of the external genitalia
at birth, abnormal secondary sexual development in puberty, and
infertility in individuals with a 46,XY karyotype
\item spectrum of defects in androgen action with three broad phenotypes:
\begin{itemize}
\item complete androgen insensitivity syndrome with typical
female external genitalia
\item partial androgen insensitivity syndrome with predominantly
female, predominantly male, or ambiguous external genitalia
\item mild androgen insensitivity syndrome with typical male external
genitalia
\end{itemize}
\end{itemize}
\item Diagnostic Testing
\label{sec:orgac3cb09}
\begin{itemize}
\item established in an individual with a 46,XY karyotype who has:
\begin{itemize}
\item undermasculinization of the external genitalia
\item impaired spermatogenesis with otherwise normal testes
\item absent or rudimentary müllerian structures
\item evidence of normal or increased synthesis of testosterone and its
normal conversion to dihydrotestosterone
\item normal or increased luteinizing hormone (LH) production by the
pituitary gland
\item and/or a hemizygous pathogenic variant in AR
\end{itemize}
\end{itemize}
\item Genetic Counseling
\label{sec:orge82e7d7}
\begin{itemize}
\item X-linked
\item affected 46,XY individuals are almost always infertile
\end{itemize}
\end{enumerate}
\subsection{21-Hydroxylase deficiency}
\label{sec:orgcd343dd}
\begin{enumerate}
\item Clinical Characteristics
\label{sec:org8be88c2}
\begin{itemize}
\item most common cause of congenital adrenal hyperplasia
\begin{itemize}
\item a family of autosomal recessive disorders involving impaired
synthesis of cortisol from cholesterol by the adrenal cortex
\end{itemize}
\item excessive adrenal androgen biosynthesis results in virilization in
all individuals and salt wasting in some individuals
\begin{description}
\item[{classic form}] severe enzyme deficiency and prenatal onset of virilization
\begin{description}
\item[{salt-wasting form}] aldosterone  production is inadequate \(\ge\) 75\% of classic
\item[{simple virilizing form}] \(\sim\) 25\% of classic
\end{description}
\item[{non-classic form}] mild enzyme deficiency and postnatal onset
\end{description}
\item newborns with salt-wasting 21-OHD CAH are at risk for
life-threatening salt-wasting crises
\item individuals with the non-classic form of 21-OHD CAH present
postnatally with signs of hyperandrogenism
\item \female with the non-classic form are not virilized at birth
\end{itemize}
\item Diagnostic Testing
\label{sec:org36147fb}
\begin{itemize}
\item classic 21-OHD CAH is established in newborns with characteristic clinical features
\begin{itemize}
\item \(\uparrow\) serum 17-OHP
\item \(\uparrow\) adrenal androgens
\end{itemize}
\item non-classic 21-OHD is established by comparison of baseline serum
17-OHP and ACTH-stimulated serum 17-OHP or early morning elevated
17-OHP
\item identification of biallelic pathogenic variants in CYP21A2 confirms
the clinical diagnosis and allows for family studies
\end{itemize}
\item Genetic Counseling
\label{sec:org5224816}
\begin{itemize}
\item AR CYP21A2
\end{itemize}
\end{enumerate}

\section{Complex Inheritance}
\label{sec:orgba07f12}
\subsection{Alzheimer disease}
\label{sec:org3da6c6e}
\begin{enumerate}
\item Clinical Characteristics
\label{sec:orgf819cd3}
\begin{itemize}
\item characterized by dementia that typically begins with subtle and
poorly recognized failure of memory and slowly becomes more severe
and, eventually, incapacitating
\end{itemize}
\item Diagnostic Testing
\label{sec:org256d040}
\begin{itemize}
\item \(\sim\) 20 suseptibility genes including APOE e4 allele
\item early-onset familial alzheimer disease (EOFAD)
\begin{itemize}
\item pathogenic variantAPP, PSEN1 \& 2
\end{itemize}
\end{itemize}
\item Genetic Counseling
\label{sec:org30f68d4}
\begin{itemize}
\item EOFAD in which an individual has a pathogenic variant in APP, PSEN1,
or PSEN2 represents an autosomal dominant disease
\end{itemize}
\end{enumerate}
\subsection{Congenital hearing loss}
\label{sec:org263c556}
\begin{enumerate}
\item Clinical Characteristics
\label{sec:org24250b7}
\begin{itemize}
\item hearing loss present at birth
\end{itemize}
\item Diagnostic Testing
\label{sec:orgd7f50b6}
\begin{itemize}
\item congenital CMV infection
\item GJB2 at exon 1 and 2
\item GJB6 deletions
\item DFNB1
\end{itemize}
\item Genetic Counseling
\label{sec:org34b8921}
\begin{itemize}
\item GJB2, GJB6, SLC26A4
\end{itemize}
\end{enumerate}
\subsection{Diabetes Mellitus Type I}
\label{sec:orgc205d84}
\begin{enumerate}
\item Clinical Characteristics
\label{sec:orge442b0c}
\begin{itemize}
\item very little or no insulin is produced by the pancreas
\item progressive \(\beta\) cell destruction leads to dysfunction in the
neighboring \(\alpha\) cells which secrete glucagon, exacerbating
excursions away from euglycemia in both directions; overproduction
of glucagon after meals causes sharper hyperglycemia, and failure to
stimulate glucagon upon incipient hypoglycemia prevents a
glucagon-mediated rescue of glucose levels
\end{itemize}
\item Diagnostic Testing
\label{sec:orgd258a42}
\begin{itemize}
\item causes of type 1 diabetes are unknown, although several risk factors
have been identified
\item the risk of developing type 1 diabetes is increased by certain
variants of the HLA-DQA1, HLA-DQB1, and HLA-DRB1 genes
\item human leukocyte antigen (HLA) complex helps the immune system
distinguish the body's own proteins from proteins made by foreign
invaders such as viruses and bacteria
\item type 1 diabetes is generally considered to be an autoimmune
disorder
\end{itemize}

\item Genetic Counseling
\label{sec:orgbe494f3}
\begin{itemize}
\item HLA-DQA1, HLA-DQB1, and HLA-DRB1
\end{itemize}
\end{enumerate}

\section{Cytogenetics}
\label{sec:org8fc36b8}
\subsection{Cri du chat syndrome}
\label{sec:org06d1cff}
\begin{enumerate}
\item Clinical Characteristics
\label{sec:org0116b28}
\begin{itemize}
\item AKA 5p- (5p minus) syndrome
\item infants often have a high-pitched cry that sounds like that of a
cat
\item characterized by intellectual disability and delayed development,
microcephaly, low birth weight, hypotonia
\item distinctive facial features, including widely set eyes
(hypertelorism), low-set ears, a small jaw, and a rounded face.
\item some born with a heart defect
\end{itemize}

\item Diagnostic Testing
\label{sec:org3d3f6af}
\begin{itemize}
\item CMA, karyotype for a 5p deletion
\item size of the deletion varies among affected individuals
\item larger deletions \(\to\) more severe intellectual disability and
developmental delay than smaller deletions
\end{itemize}

\item Genetic Counseling
\label{sec:orgda68077}
\begin{itemize}
\item most cases are not inherited, result of a \emph{de novo} deletion
\item \textasciitilde{} 10 percent of people with cri-du-chat syndrome inherit the
chromosome abnormality from an unaffected parent with a balanced translocation
\end{itemize}
\end{enumerate}
\subsection{Pallister-Killian syndrome}
\label{sec:org7301e43}
\begin{enumerate}
\item Clinical Characteristics
\label{sec:org6f1b051}
\begin{itemize}
\item hypotonia in infancy and early childhood
\item intellectual disability, distinctive facial features, sparse hair,
areas of unusual pigmentation, and other birth defects
\end{itemize}

\item Diagnostic Testing
\label{sec:orgddc32fb}
\begin{itemize}
\item CMA, karyotype for mosaic isochromosome 12p (i(12p))
\end{itemize}

\item Genetic Counseling
\label{sec:orgd20a190}
\begin{itemize}
\item not inherited
\end{itemize}
\end{enumerate}

\subsection{Triploidy}
\label{sec:orge4ff2ad}
\begin{enumerate}
\item Clinical Characteristics
\label{sec:org40b7f9d}
Triploidy can result from either two sperm fertilizing one egg
(polyspermy) (60\%) or from one sperm fertilizing an egg with two
copies of every chromosome (40\%)

\begin{itemize}
\item Many organ systems are affected by triploidy, but the central
nervous system and skeleton are the most severely affected:
\begin{itemize}
\item holoprosencephaly, hydrocephalus, ventriculomegaly, Arnold–Chiari
malformation, agenesis of the corpus callosum and neural tube
defects
\end{itemize}
\item Skeletal manifestations include cleft lip/palate, hypertelorism,
club foot and syndactyly of fingers three and four
\item Congenital heart defects, hydronephrosis, omphalocele and
meningocele (spina bifida) are also common
\item IUGR
\end{itemize}

\item Diagnostic Testing
\label{sec:orga396580}
\begin{itemize}
\item \(\uparrow\) AFP
\item RAD, karyotype
\end{itemize}
\item Genetic Counseling
\label{sec:org3a2cd8b}
\begin{itemize}
\item not inherited
\end{itemize}
\end{enumerate}
\subsection{Trisomy 13}
\label{sec:orgfde1fb9}
\begin{itemize}
\item AKA Patau syndrome
\end{itemize}
\begin{enumerate}
\item Clinical Characteristics
\label{sec:org66d448d}
\begin{itemize}
\item severe intellectual disability and physical abnormalities in many parts of the body
\item heart defects, brain or spinal cord abnormalities
\item very small or poorly developed eyes (microphthalmia)
\item extra fingers or toes, cleft lip \textpm{} cleft palate, hypotonia
\item often die within their first days or weeks of life
\end{itemize}

\item Diagnostic Testing
\label{sec:org76ec3dc}
\begin{itemize}
\item CMD, RAD, karyotype
\end{itemize}

\item Genetic Counseling
\label{sec:orgb57cf20}
\begin{itemize}
\item 1 in 16,000 newborns
\item women of any age can have a child with trisomy 13
\begin{itemize}
\item increases with maternal age
\end{itemize}
\end{itemize}
\end{enumerate}
\subsection{Trisomy 18}
\label{sec:org54f99cd}
\begin{itemize}
\item AKA Edwards syndrome
\end{itemize}
\begin{enumerate}
\item Clinical Characteristics
\label{sec:org05b7e22}
\begin{itemize}
\item IUGR and LBW
\item heart defects and abnormalities of other organs that develop before birth
\item small, abnormally shaped head; a small jaw and mouth; and clenched fists with overlapping fingers
\item often die before birth or within their first month
\item 5-10\% live past their first year, and these children often have
severe intellectual disability
\end{itemize}

\item Diagnostic Testing
\label{sec:orge2da0bc}
\begin{itemize}
\item CMD, RAD, karyotype
\end{itemize}

\item Genetic Counseling
\label{sec:org7aac302}
\begin{itemize}
\item 1 in 5000 live-born infants
\item women of any age can have a child with trisomy 18
\begin{itemize}
\item increases with maternal age
\end{itemize}
\end{itemize}
\end{enumerate}

\subsection{Trisomy 21}
\label{sec:orga102043}
\begin{itemize}
\item AKA Down syndrome
\end{itemize}
\begin{enumerate}
\item Clinical Characteristics
\label{sec:org8fb9313}
\begin{itemize}
\item intellectual disability, a characteristic facial appearance, hypotonia in infancy
\begin{itemize}
\item intellectual disability is usually mild to moderate
\end{itemize}
\item \textasciitilde{}50\% have heart defects
\end{itemize}
\item Diagnostic Testing
\label{sec:org19c930e}
\begin{itemize}
\item CMD, RAD, karyotype
\end{itemize}

\item Genetic Counseling
\label{sec:orgcbfcf39}
\begin{itemize}
\item 1 in 800 newborns
\item women of any age can have a child with trisomy 21
\begin{itemize}
\item increases with maternal age
\end{itemize}
\end{itemize}
\end{enumerate}
\subsection{Klinefelter syndrome}
\label{sec:org47d3e24}
\begin{enumerate}
\item Clinical Characteristics
\label{sec:orgc7458de}
\begin{itemize}
\item boys and men, affects physical and intellectual development
\item taller than average and infertile
\item signs and symptoms of Klinefelter syndrome vary among boys and men with this condition
\item reduced testosterone
\end{itemize}
\item Diagnostic Testing
\label{sec:orga3ac91c}
\begin{itemize}
\item 47,XXY karyotype
\item mosaic Klinefelter syndrome 46,XY/47,XXY
\end{itemize}

\item Genetic Counseling
\label{sec:org05ec645}
\begin{itemize}
\item not inherited
\end{itemize}
\end{enumerate}
\subsection{Fanconi anemia}
\label{sec:org576938e}
\begin{enumerate}
\item Clinical Characteristics
\label{sec:org141a6ba}
\begin{itemize}
\item physical abnormalities, bone marrow failure, and increased risk for
malignancy
\item physical abnormalities, present in approximately 75\% of affected individuals, include one or more of the following:
\begin{itemize}
\item short stature, abnormal skin pigmentation, skeletal malformations
of the upper and lower limbs, microcephaly, and ophthalmic and
genitourinary tract anomalies
\end{itemize}
\end{itemize}

\item Diagnostic Testing
\label{sec:org8533b47}
\begin{itemize}
\item established in a proband with increased chromosome breakage and
radial forms on cytogenetic testing of lymphocytes with
diepoxybutane (DEB) and mitomycin C (MMC)

\item diagnosis is confirmed by identification of one of the following:
\begin{itemize}
\item biallelic pathogenic variants in one of the 19 genes known to
cause autosomal recessive FA
\item heterozygous pathogenic variant in RAD51, known to cause autosomal dominant FA
\item hemizygous pathogenic variant in FANCB, known to cause X-linked FA
\end{itemize}
\end{itemize}

\item Genetic Counseling
\label{sec:org4bb149b}
\begin{itemize}
\item AR, AD (RAD51) or X-linked (FANCB)
\end{itemize}
\end{enumerate}

\subsection{Ataxia-telangiectasia}
\label{sec:orga186808}
\begin{enumerate}
\item Clinical Characteristics
\label{sec:org13c765a}
\begin{itemize}
\item progressive cerebellar ataxia beginning between ages one and four
years, oculomotor apraxia, choreoathetosis, telangiectasias of the
conjunctivae, immunodeficiency, frequent infections, and an
increased risk for malignancy, particularly leukemia and lymphoma
\end{itemize}
\item Diagnostic Testing
\label{sec:org7bdd86e}
\begin{itemize}
\item diagnosis is established by the presence of biallelic (homozygous or
compound heterozygous) ATM pathogenic variants or (when available)
by immunoblotting to test for absent or reduced ATM protein
\end{itemize}

\item Genetic Counseling
\label{sec:org9a761b5}
\begin{itemize}
\item AR, ATM
\end{itemize}
\end{enumerate}
\subsection{Williams syndrome}
\label{sec:orgb896eb7}
\begin{enumerate}
\item Clinical Characteristics
\label{sec:org0c18fa4}
\begin{itemize}
\item cardiovascular disease :elastin arteriopathy, peripheral pulmonary
stenosis, supravalvar aortic stenosis, hypertension
\item distinctive facies, connective tissue abnormalities, intellectual
disability (usually mild)
\item a specific cognitive profile, unique personality characteristics
\item growth abnormalities, and endocrine abnormalities (hypercalcemia,
hypercalciuria, hypothyroidism, and early puberty)
\item hypotonia and hyperextensible joints can result in delayed
attainment of motor milestones
\end{itemize}

\item Diagnostic Testing
\label{sec:orgd7ae378}
\begin{itemize}
\item clinical diagnostic criteria
\item diagnosis requires detection of a recurrent 7q11.23 contiguous gene deletion of the Williams-Beuren syndrome critical region (WBSCR) that encompasses the elastin gene (ELN)
\begin{itemize}
\item can be detected using FISH and/or deletion/duplication testing
\end{itemize}
\end{itemize}
\item Genetic Counseling
\label{sec:orgcf76c7e}
\begin{itemize}
\item AD
\item most \emph{de novo}
\end{itemize}
\end{enumerate}

\subsection{22q11 deletion syndrome}
\label{sec:org486c67e}
\begin{enumerate}
\item Clinical Characteristics
\label{sec:org1f60219}
\begin{itemize}
\item a contiguous gene deletion syndrome
\item included phenotypes:
\begin{itemize}
\item DiGeorge syndrome
\item Velocardiofacial syndrome
\item Conotruncal anomaly face syndrome
\item Autosomal dominant Opitz G/BBB syndrome
\item Sedlackova syndrome
\item Cayler cardiofacial syndrome
\end{itemize}

\item congenital heart disease (74\%)
\begin{itemize}
\item tetralogy of Fallot, interrupted aortic arch, ventricular septal defect, and truncus arteriosus
\end{itemize}
\item palatal abnormalities (69\%)
\begin{itemize}
\item velopharyngeal incompetence, submucosal cleft palate, bifid uvula, and cleft palate
\end{itemize}
\item facial features (majority of northern European)
\item learning difficulties (70\%-90\%)
\item immune deficiency (77\%)
\end{itemize}

\item Diagnostic Testing
\label{sec:org248236f}
\begin{itemize}
\item submicroscopic deletion of chromosome 22 by FISH, MLPA, CMA
\end{itemize}

\item Genetic Counseling
\label{sec:orgd6e0f4f}
\begin{itemize}
\item AD
\item \textasciitilde{} 93\% \emph{de novo} deletion of 22q11.2
\item \textasciitilde{} 7\% inherited the 22q11.2 deletion
\end{itemize}
\end{enumerate}
\section{Metabolics}
\label{sec:orga3bf0f6}
\begin{itemize}
\item Acute intermittent porphyria
\item Alpha-1 antitrypsin deficiency
\item Canavan disease
\item Gaucher disease
\item Homocystinuria (CBS deficiency)
\item Hurler syndrome
\item I cell disease
\item MCAD deficiency
\item Mitochondrial DNA mutations
\begin{itemize}
\item MERRF/MELAS/LHON, mtDNA deletion syndromes
\end{itemize}
\item Ornithine transcarbamylase (OTC) deficiency
\item Peroxisome biogenesis disorder (Zellweger)
\item Phenylketonuria
\item Pompe disease
\item Tay Sachs Disease
\item Tyrosinemia type I
\item X-linked Adrenoleukodystrophy
\item Wilson disease
\item Hyperprolinemia type I
\end{itemize}
\section{Molecular Genetics}
\label{sec:org1fe7561}
\subsection{Cystic fibrosis}
\label{sec:org3ed91a7}
\begin{enumerate}
\item Clinical Characteristics
\label{sec:org03fee17}
\begin{itemize}
\item Multisystem disease affecting epithelia of the respiratory tract, exocrine pancreas, intestine, hepatobiliary system, and exocrine sweat glands.
\item Morbidities include progressive obstructive lung disease with bronchiectasis, frequent hospitalizations for pulmonary disease, pancreatic insufficiency and malnutrition, recurrent sinusitis and bronchitis, and male infertility.
\item Pulmonary disease is the major cause of morbidity and mortality in CF.
\item Meconium ileus occurs at birth in 15\%-20\% of newborns with CF.
\item More than 95\% of males with CF are infertile.
\end{itemize}

\item Diagnostic Testing
\label{sec:org0db11a5}
\begin{itemize}
\item one or more characteristic phenotypic features and evidence of an abnormality in CFTR function
\begin{itemize}
\item 2 elevated sweat chloride values or
\item biallelic CFTR pathogenic variants or
\item transepithelial nasal potential difference measurement characteristic of CF
\end{itemize}
\item Diagnosis of CF is established in an infant with:
\begin{itemize}
\item elevated NBS IRT and
\item identification of biallelic CFTR pathogenic variants or
\item an elevated sweat chloride.
\begin{itemize}
\item quantitative pilocarpine iontophoresis sweat chloride values (>60 mmol/L in infants age > 6 months)
\end{itemize}
\end{itemize}
\end{itemize}
\item Genetic Counseling
\label{sec:orga7b7d54}
\begin{itemize}
\item AR , CFTR
\item At conception, each sib of an affected individual with CF have:
\begin{itemize}
\item a 25\% chance of being affected
\item a 50\% chance of being an asymptomatic carrier
\item a 25\% chance of being unaffected and not a carrier.
\end{itemize}
\item Carrier testing for at-risk relatives and prenatal testing for pregnancies at increased risk are possible if the CFTR pathogenic variants in the family are known.
\end{itemize}
\end{enumerate}
\subsection{Achondroplasia}
\label{sec:org724bc6a}
\begin{enumerate}
\item Clinical Characteristics
\label{sec:org7066d84}
\begin{itemize}
\item Most common cause of disproportionate short stature.
\item Affected individuals have rhizomelic shortening of the limbs,
macrocephaly, and characteristic facial features with frontal
bossing and midface retrusion.
\item In infancy, hypotonia is typical, and acquisition of developmental
motor milestones is often both aberrant in pattern and delayed
\item Intelligence and life span are usually near normal, although
craniocervical junction compression increases the risk of death in
infancy.
\item Additional complications include obstructive sleep apnea, middle ear
dysfunction, kyphosis, and spinal stenosis.
\end{itemize}

\item Diagnostic Testing
\label{sec:org48e8d7c}
\begin{itemize}
\item Diagnosed by characteristic clinical and radiographic findings in
most affected individuals.
\item In individuals in whom there is diagnostic uncertainty or atypical
findings, identification of a heterozygous pathogenic variant in
FGFR3 can establish the diagnosis.
\end{itemize}

\item Genetic Counseling
\label{sec:org6f1f4ac}
\begin{itemize}
\item AD, FGFR3
\item \textasciitilde{}80\% of individuals with achondroplasia have parents with average
stature and have achondroplasia as the result of a \emph{de novo}
pathogenic variant.
\begin{itemize}
\item these parents have a very low risk of having another child with
achondroplasia.
\end{itemize}
\item An individual with achondroplasia who has a reproductive partner
with average stature is at 50\% risk in each pregnancy of having a
child with achondroplasia.
\item When both parents have achondroplasia the risk to their offspring of
having:
\begin{itemize}
\item average stature is 25\%
\item achondroplasia is 50\%
\item homozygous achondroplasia (a lethal condition) is 25\%.
\end{itemize}

\item If the proband and the proband's reproductive partner are affected
with different dominantly inherited skeletal dysplasias, genetic
counseling becomes more complicated because of the risk of
inheriting two dominant skeletal dysplasias.
\end{itemize}
\end{enumerate}
\subsection{Huntington disease}
\label{sec:org7b1e79a}
\begin{enumerate}
\item Clinical Characteristics
\label{sec:orgbdd2aec}
\begin{itemize}
\item a progressive disorder of motor, cognitive, and psychiatric
disturbances.
\item mean age of onset is 35 to 44 years and the median survival time is
15 to 18 years after onset.
\end{itemize}
\item Diagnostic Testing
\label{sec:org761452d}
\begin{itemize}
\item diagnosis of HD rests on:
\begin{itemize}
\item positive family history
\item characteristic clinical findings
\item detection of an expansion of 36 or more CAG trinucleotide repeats in HTT.
\end{itemize}

\item All individuals with HD have an expansion in the number of CAG
trinucleotide repeats that encode glutamine amino acids in exon 1 of
HTT.
\begin{description}
\item[{Normal alleles}] 26 or fewer CAG trinucleotide repeats
\item[{Intermediate alleles}] 27-35 CAG trinucleotide repeats
\begin{itemize}
\item An individual with an allele in this range is not at risk of
developing symptoms of HD but, because of instability in the CAG
tract, may be at risk of having a child with an allele in the
HD-causing range
\end{itemize}
\item[{HD-causing alleles}] \(\ge\) 36 CAG trinucleotide repeats
\begin{itemize}
\item Persons who have an HD-causing allele are considered at risk of
developing HD in their lifetime.
\item HD-causing alleles are further classified as:
\begin{description}
\item[{Reduced-penetrance HD-causing alleles}] 36-39 CAG
\item[{Full-penetrance HD-causing alleles}] \(\ge\) 40 CAG
\end{description}
\end{itemize}
\end{description}
\end{itemize}
\item Genetic Counseling
\label{sec:org0e81a7e}
\begin{itemize}
\item AD, HTT
\item Offspring of an individual with a pathogenic variant have a 50\% chance of inheriting the disease-causing allele.
\item Predictive testing in asymptomatic adults at risk is available but requires careful thought (including pre- and post-test genetic counseling) as there is currently no cure for the disorder.
\begin{itemize}
\item asymptomatic individuals at risk may be eligible to participate in clinical trials.
\end{itemize}
\item Predictive testing is not considered appropriate for asymptomatic at-risk individuals younger than age 18 years.
\item Prenatal testing by molecular genetic testing is possible.
\end{itemize}
\end{enumerate}
\subsection{Fragile X}
\label{sec:org10917fc}
\begin{enumerate}
\item Clinical Characteristics
\label{sec:orgeca6b10}
\begin{itemize}
\item Fragile X syndrome occurs in individuals with an FMR1 full mutation
or other loss-of-function variant.
\item Males: moderate intellectual disability
\item Females: mild intellectual disability
\item FMR1 pathogenic variants are complex alterations involving non-classic
gene-disrupting alterations (trinucleotide repeat expansion) and
abnormal gene methylation,
\begin{itemize}
\item \(\therefore\) affected individuals occasionally have an atypical presentation with an IQ above 70,
\begin{itemize}
\item the traditional  demarcation denoting intellectual disability.
\end{itemize}
\end{itemize}
\item Males with an FMR1 full mutation accompanied by aberrant methylation may have a characteristic appearance:
\begin{itemize}
\item large head, long face, prominent forehead and chin, protruding ears
\item connective tissue findings (joint laxity), and large testes after puberty.
\item Behavioral abnormalities, sometimes including autism spectrum disorder, are common.
\end{itemize}
\item fragile X-associated tremor/ataxia syndrome, and FMR1-related
primary ovarian insufficiency are less severe forms due to smaller
repeats
\end{itemize}
\item Diagnostic Testing
\label{sec:org94b72a7}
\begin{itemize}
\item alteration in FMR1.
\item \textgreater{} 99\% of individuals with fragile X syndrome have:
\begin{itemize}
\item lof variant of FMR1 caused by an increased number of CGG
trinucleotide repeats (typically >200)
\item accompanied by aberrant methylation of FMR1
\end{itemize}
\item Other pathogenic variants include:
\begin{itemize}
\item deletions and single-nucleotide variants.
\end{itemize}
\end{itemize}
\item Genetic Counseling
\label{sec:orge23fa0f}
\begin{itemize}
\item All mothers of individuals with an FMR1 full mutation (expansion
>200 CGG trinucleotide repeats and abnormal methylation) are
carriers of an FMR1 pathogenic variant.
\item Mothers and their female relatives who are premutation carriers are
at increased risk for FXTAS and POI;
\item those with a full mutation may have findings of fragile X syndrome.
\item All are at increased risk of having offspring with fragile X syndrome, FXTAS, and POI.
\item Males with premutations are at increased risk for FXTAS.
\item Males with FXTAS will transmit their FMR1 premutation expansion to none of their sons and to all of their daughters, who will be premutation carriers.
\item Carrier testing for at-risk relatives and prenatal testing for
pregnancies at increased risk are possible if the diagnosis of an
FMR1-related disorder has been confirmed in a family member.
\end{itemize}
\end{enumerate}
\subsection{Friedreich's ataxia}
\label{sec:orgbf63bc2}
\begin{enumerate}
\item Clinical Characteristics
\label{sec:orga90a0c5}
\begin{itemize}
\item characterized by slowly progressive ataxia with onset usually before
age 25 years (mean 10-15 yrs).
\item FRDA is typically associated with dysarthria, muscle weakness,
spasticity particularly in the lower limbs, scoliosis, bladder
dysfunction, absent lower-limb reflexes, and loss of position and
vibration sense.
\begin{itemize}
\item \textasciitilde{}2/3 have cardiomyopathy
\item \textasciitilde{}30\% have diabetes mellitus,
\item \textasciitilde{}25\% have an "atypical" presentation with later onset or retained
tendon reflexes.
\end{itemize}
\end{itemize}
\item Diagnostic Testing
\label{sec:org95d7017}
\begin{itemize}
\item established in a proband by detection of biallelic pathogenic
variants in FXN.
\item An abnormally expanded GAA repeat in intron 1 of FXN observed on
both alleles in \textasciitilde{}96\% with FRDA
\item remaining are compound heterozygotes for abnormally expanded GAA
repeat in the disease-causing range on one allele and another
intragenic pathogenic variant on the other allele.
\end{itemize}


\begin{itemize}
\item Four classes of alleles are recognized for the GAA repeat sequence in intron 1 of FXN
\begin{description}
\item[{Normal alleles}] 5-33 GAA repeats
\item[{Mutable normal (premutation) alleles}] 34-65 GAA repeats
\item[{Borderline alleles}] 44-66 GAA repeats. The shortest repeat length associated with disease
\item[{Full-penetrance (disease-causing expanded) alleles}] 66-1,300 GAA repeats
\end{description}
\end{itemize}

Rare alleles of variant structure. In contrast to the alleles discussed above in which the GAA trinucleotides are perfect repeats, in rare pathogenic alleles the GAA repeats are not in perfect tandem order but rather are interrupted by other nucleotides. Such "interrupted FXN alleles" differ in length and types of nucleotides in the interruption, but they are typically close to the 3' end of the GAA repeat tract (see Molecular Genetics).
\item Genetic Counseling
\label{sec:org73e82bc}
\begin{itemize}
\item AR, FXN
\item Each sib has a 25\% chance of being affected
\begin{itemize}
\item 50\% chance of being an asymptomatic carrier
\item 25\% chance of having no pathogenic variant.
\end{itemize}
\item Carrier testing of at-risk relatives, prenatal testing for
pregnancies at increased risk, and pre-implantation genetic diagnosis
are possible if both FXN pathogenic variants have been identified in
an affected family member.
\end{itemize}
\end{enumerate}

\subsection{Myotonic dystrophy type I}
\label{sec:org52aa0d5}
\begin{enumerate}
\item Clinical Characteristics
\label{sec:org494fb35}
\begin{itemize}
\item multisystem disorder that affects skeletal and smooth muscle as well
as the eye, heart, endocrine system, and central nervous system.

\item clinical findings, from mild to severe:

\begin{description}
\item[{Mild DM1}] cataract and mild myotonia (sustained muscle
contraction) life span is normal

\item[{Classic DM1}] muscle weakness and wasting, myotonia, cataract,
and often cardiac conduction abnormalities; adults
may become physically disabled and may have a
shortened life span.

\item[{Congenital DM1}] hypotonia and severe generalized weakness at
birth, often with respiratory insufficiency and
early death; intellectual disability is common.
\end{description}
\end{itemize}

\item Diagnostic Testing
\label{sec:org7edeff8}
\begin{itemize}
\item caused by expansion of a CTG trinucleotide repeat in the noncoding region of DMPK.
\item molecular genetic testing of DMPK.
\item CTG repeat length exceeding 34 repeats is abnormal.
\item Molecular genetic testing detects pathogenic variants in nearly 100\%
of affected individuals.

\begin{description}
\item[{Normal alleles}] 5-34 CTG repeats
\item[{Mutable normal (premutation) alleles}] 35-49 CTG repeats
\item[{Full-penetrance alleles}] \(\ge\) 50 CTG repeats
\end{description}
\end{itemize}

\item Genetic Counseling
\label{sec:org79c3542}
\begin{itemize}
\item AD, DMPK
\item Offspring of an affected individual have a 50\% chance of inheriting the expanded allele.
\item Pathogenic alleles may expand in length during gametogenesis
\begin{itemize}
\item \(\to\) transmission of longer trinucleotide repeat alleles
\item \(\to\) earlier onset and more severe disease the parent
\end{itemize}
\item Prenatal testing is possible for pregnancies at increased risk when
the diagnosis of DM1 has been confirmed by molecular genetic testing
in an affected family member.
\end{itemize}
\end{enumerate}
\subsection{Angelman syndrome}
\label{sec:orgd41ea79}
\begin{enumerate}
\item Clinical Characteristics
\label{sec:orgd5ae366}
\begin{itemize}
\item Severe developmental delay or intellectual disability, severe speech
impairment, gait ataxia and/or tremulousness of the limbs
\item unique behavior with an inappropriate happy demeanor that includes
frequent laughing, smiling, and excitability.
\item Microcephaly and seizures are also common.
\item Developmental delays are first noted at around age six months
\item clinical features of AS do not become manifest until after age one year
\begin{itemize}
\item can take several years before the correct clinical diagnosis is obvious.
\end{itemize}
\end{itemize}

\item Diagnostic Testing
\label{sec:org41fdc60}

\begin{figure}[htbp]
\centering
\includegraphics[width=0.6\textwidth]{./figures/aspws.jpg}
\caption{\label{fig:org120a1a1}AS and PWS}
\end{figure}

\begin{itemize}
\item molecular genetic testing deficient expression or function of the
maternally inherited UBE3A allele.
\item parent-specific DNA methylation imprints in the 15q11.2-q13 chromosome region detects approximately 80\%
\begin{itemize}
\item including deletion, uniparental disomy (UPD), imprinting defect (ID)
\end{itemize}
\item \textless{} 1\% have a cytogenetically visible chromosome rearrangement (i.e., translocation or inversion).
\item UBE3A sequence analysis detects pathogenic variants \textasciitilde{}11\% of individuals.
\item molecular genetic testing (methylation analysis and UBE3A sequence
analysis) \textasciitilde{}90\% of individuals.
\item Remaining 10\% with classic phenotypic features of AS have the
disorder as a result of an as-yet unidentified genetic mechanism
\begin{itemize}
\item not amenable to diagnostic testing
\end{itemize}
\end{itemize}
\item Genetic Counseling
\label{sec:org0e41598}
\begin{itemize}
\item caused by disruption of maternally imprinted UBE3A located within
the 15q11.2-q13 Angelman syndrome/Prader-Willi syndrome region.
\item The risk to sibs of a proband depends on the genetic mechanism
leading to the loss of UBE3A function
\begin{itemize}
\item typically less than 1\% risk for probands with a deletion or UPD
\item as high as 50\% for probands with an ID or a pathogenic variant of UBE3A.
\end{itemize}
\item Members of the mother's extended family are also at increased risk
when an ID or a UBE3A pathogenic variant is present.
\item Cytogepnetically visible chromosome rearrangements may be inherited,usually \emph{de novo}.
\item Prenatal testing is possible when the underlying genetic mechanism
is a deletion, UPD, an ID, a UBE3A pathogenic variant, or a
chromosome rearrangement.
\end{itemize}
\end{enumerate}
\subsection{Beckwith-Wiedemann syndrome}
\label{sec:org9fd1516}
\begin{enumerate}
\item Clinical Characteristics
\label{sec:org0f73f9a}
\begin{itemize}
\item growth disorder variably characterized by neonatal hypoglycemia,
macrosomia, macroglossia, hemihyperplasia, omphalocele, embryonal
tumors (e.g., Wilms tumor, hepatoblastoma, neuroblastoma, and
rhabdomyosarcoma), visceromegaly, adrenocortical cytomegaly, renal
abnormalities (e.g., medullary dysplasia, nephrocalcinosis,
medullary sponge kidney, and nephromegaly), and ear creases/pits.

\item a clinical spectrum, may have many of these features or only one or two.

\item Early death may occur from complications of prematurity,
hypoglycemia, cardiomyopathy, macroglossia, or tumors.
\end{itemize}

\item Diagnostic Testing
\label{sec:org0dd9439}

\begin{itemize}
\item Cytogenetically detectable abnormalities involving chromosome 11p15
are found in 1\% or fewer of affected individuals.

\item Molecular genetic testing can identify epigenetic and genomic
alterations of chromosome 11p15 in individuals with BWS:
\begin{itemize}
\item Loss of methylation on the maternal chromosome at imprinting
center 2 (IC2) in 50\% of affected individuals;
\item Paternal uniparental disomy for chromosome 11p15 in 20\%
\item Gain of methylation on the maternal chromosome at imprinting
center 1 (IC1) in 5\%.
\end{itemize}
\item Methylation alterations associated with deletions or duplications in
this region have high heritability.

\item Sequence analysis of CDKN1C identifies a heterozygous maternally
inherited pathogenic variant in approximately 40\% of familial cases
and 5\%-10\% of cases with no family history of BWS.
\end{itemize}

\begin{figure}[htbp]
\centering
\includegraphics[width=0.5\textwidth]{./figures/bws.png}
\caption{\label{fig:orgcb37974}BWS Chromosome 11}
\end{figure}

\item Genetic Counseling
\label{sec:org15656a3}
\begin{itemize}
\item associated with abnormal regulation of gene transcription in two
imprinted domains on chromosome 11p15.5.
\item Most individuals with BWS are reported to have normal chromosome
studies or karyotypes.
\item \textasciitilde{}85\% of individuals with BWS have no family history of BWS
\item \textasciitilde{}15\% have a family history consistent with parent-of-origin
autosomal dominant transmission.
\item Children of subfertile parents conceived by assisted reproductive
technology (ART) may be at increased risk for imprinting disorders,
including BWS.
\item Identification of the underlying genetic mechanism causing BWS
permits better estimation of recurrence risk.
\item Prenatal screening for pregnancies in the general population that
identifies findings suggestive of a diagnosis of BWS may lead to the
consideration of
\begin{itemize}
\item chromosome analysis, chromosomal microarray, and/or molecular genetic testing.
\end{itemize}
\item prenatal testing by chromosome analysis for families with an
inherited chromosome abnormality or by molecular genetic testing for
families in which the molecular mechanism of BWS has been defined
\end{itemize}
\end{enumerate}

\subsection{Prader-Willi syndrome}
\label{sec:org83ecb85}
\begin{enumerate}
\item Clinical Characteristics
\label{sec:orgfca23be}
\begin{itemize}
\item severe hypotonia and feeding difficulties in early infancy
\item later infancy or early childhood by excessive eating
\item gradual development of morbid obesity
\item Motor milestones and language development are delayed.
\item All individuals have some degree of cognitive impairment.
\item A distinctive behavioral phenotype (with temper tantrums, stubbornness, manipulative behavior, and obsessive-compulsive characteristics) is common.
\item Hypogonadism is present in both males and females and manifests as genital hypoplasia, incomplete pubertal development, and, in most, infertility.
\item Short stature is common (if not treated with growth hormone);
\item characteristic facial features, strabismus, and scoliosis are often present.
\end{itemize}

\item Diagnostic Testing
\label{sec:org634bd3d}
\begin{itemize}
\item DNA methylation testing to detect abnormal parent-specific
imprinting within the Prader-Willi critical region (PWCR) on
chromosome 15
\item testing determines whether the region is maternally inherited only
\begin{itemize}
\item the paternally contributed region is absent
\item detects more than 99\% of affected individuals
\item DNA methylation-specific testing is important to confirm the
diagnosis of PWS in all individuals,
\end{itemize}
\end{itemize}

\item Genetic Counseling
\label{sec:org91f2536}
\begin{itemize}
\item PWS is caused by an absence of expression of imprinted genes in the
paternally derived PWS/Angelman syndrome (AS) region (15q11.2-q13)
of chromosome 15:
\begin{itemize}
\item paternal deletion, maternal uniparental disomy 15 and rarely an imprinting defect.
\end{itemize}
\item The risk to the sibs depends on the genetic mechanism.
\begin{itemize}
\item \textless{} 1\% if the affected child has a deletion or uniparental disomy
\item up to 50\% if the affected child has an imprinting defect
\item up to 25\% if a parental chromosome translocation is present
\end{itemize}
\item Prenatal testing is possible for pregnancies at increased risk if
the underlying genetic mechanism is known.
\end{itemize}
\end{enumerate}
\subsection{Russell-Silver syndrome}
\label{sec:orgcf05e36}
\begin{enumerate}
\item Clinical Characteristics
\label{sec:org01263b7}
\begin{itemize}
\item asymmetric gestational growth restriction resulting in affected
individuals being born small for gestational age, with relative
macrocephaly at birth (head circumference \(\le\)1.5 SD above birth
weight and/or length), prominent forehead usually with frontal
bossing, and frequently body asymmetry.
\item This is followed by postnatal growth failure, and in some cases progressive limb length discrepancy and feeding difficulties.
\item Additional clinical features include triangular facies, fifth-finger clinodactyly, and micrognathia with narrow chin.
\item The average adult height in untreated individuals is \textasciitilde{}3.1\textpm{}1.4 SD below the mean.
\end{itemize}
\item Diagnostic Testing
\label{sec:org122cf52}
\begin{itemize}
\item a genetically heterogeneous condition.
\item Genetic testing confirms clinical diagnosis in approximately 60\% of
affected individuals.
\begin{itemize}
\item hypomethylation of the imprinted control region 1 (ICR1) at
11p15.5 causes SRS in 35\%-50\% of individuals
\item maternal uniparental disomy (mUPD7) causes SRS in 7\%-10\% of individuals.
\item a small number of individuals with SRS who have duplications,
deletions or translocations involving the imprinting centers at
11p15.5 or duplications, deletions, or translocations involving
chromosome 7.
\item rarely, affected individuals with pathogenic variants in CDKN1C,
IGF2, PLAG1, and HMGA2 have been described.
\item approximately 40\% of individuals who meet NH-CSS clinical criteria
for SRS have negative molecular and/or cytogenetic testing.
\end{itemize}
\end{itemize}

\begin{figure}[htbp]
\centering
\includegraphics[width=0.5\textwidth]{./figures/rss.png}
\caption{\label{fig:org2e1b0b2}11p Duplication in RSS}
\end{figure}

\item Genetic Counseling
\label{sec:orgff6265d}
\begin{itemize}
\item SRS has multiple etiologies and typically has a low recurrence
risk.
\item In most families, a proband with SRS represents a simplex case (a single affected family member) and has
SRS as the result of an apparent \emph{de novo} epigenetic or genetic alteration
\begin{itemize}
\item loss of paternal methylation at the 11p15 ICR1 H19/IGF2 imprinting center 1 or
\item maternal uniparental disomy for chromosome 7.
\end{itemize}
\item SRS may also occur as the result of a genetic alteration associated with up to a 50\% recurrence risk
\begin{itemize}
\item copy number variant on chromosome 7 or 11 or
\item an intragenic pathogenic  variant in CDKN1C, IGF2, PLAG2, or HMGA2
\end{itemize}
\item Accurate assessment of SRS recurrence therefore requires
identification of the causative genetic mechanism in the proband.
\end{itemize}
\end{enumerate}

\section{Neurogenetics}
\label{sec:org528874a}
\subsection{Rett syndrome}
\label{sec:org07d2fae}
\begin{enumerate}
\item Clinical Characteristics
\label{sec:org39fdd80}
\begin{description}
\item[{\female Classic Rett syndrome}] a progressive neurodevelopmental
disorder primarily affecting girls, is characterized by apparently
normal psychomotor development during the first six to 18 months of
life, followed by a short period of developmental stagnation, then
rapid regression in language and motor skills, followed by long-term
stability
\item[{\male Severe neonatal-onset encephalopathy}] the most common
phenotype in affected males, is characterized by a relentless
clinical course that follows a metabolic-degenerative type of
pattern, abnormal tone, involuntary movements, severe seizures, and
breathing abnormalities. Death often occurs before age two years.
\end{description}
\item Diagnostic Testing
\label{sec:org5557c06}
\begin{itemize}
\item the diagnosis of a MECP2 disorder is established by molecular
genetic testing in a female proband with suggestive findings and a
heterozygous MECP2 pathogenic variant, and in a male proband with
suggestive findings and a hemizygous MECP2 pathogenic variant.
\end{itemize}
\item Genetic Counseling
\label{sec:org1b752c0}
\begin{itemize}
\item XL MECP2
\item \textgreater{} 99\% are simplex cases (i.e., a single occurrence in a family),
resulting from a \emph{de novo} pathogenic variant or possibly from
inheritance of the pathogenic variant from a parent who has germline
mosaicism
\end{itemize}
\end{enumerate}
\subsection{Charcot-Marie-Tooth Disease Type IA}
\label{sec:org2258a72}
AKA Hereditary Neuropathy with Pressure Palsies
\begin{enumerate}
\item Clinical Characteristics
\label{sec:org4cbbc0a}
\begin{itemize}
\item characterized by repeated focal pressure neuropathies such as carpal
tunnel syndrome and peroneal palsy with foot drop
\item first attack usually occurs in the second or third decade
\item recovery from acute neuropathy is often complete; when recovery is
not complete, the resulting disability is usually mild
\item some affected individuals also have signs of a mild to moderate
peripheral neuropathy
\end{itemize}
\item Diagnostic Testing
\label{sec:org7b4c9eb}
\begin{itemize}
\item diagnosis of HNPP is established in an adult with recurrent focal
compression neuropathies who has a family history consistent with
autosomal dominant inheritance
\item PMP22 is the only gene known to be associated with HNPP
\item a contiguous gene deletion of chromosome 17p11.2 that includes PMP22
is present in approximately 80\% of affected individuals; the
remaining 20\% have a pathogenic variant in PMP22
\end{itemize}
\item Genetic Counseling
\label{sec:orga86e9cf}
\begin{itemize}
\item AD PMP22
\end{itemize}
\end{enumerate}
\subsection{Spinal muscular atrophy}
\label{sec:org85fbe29}
\begin{enumerate}
\item Clinical Characteristics
\label{sec:orgc31b368}
\begin{itemize}
\item characterized by muscle weakness and atrophy resulting from
progressive degeneration and irreversible loss of the anterior horn
cells in the spinal cord (i.e., lower motor neurons) and the brain
stem nuclei
\item onset of weakness ranges from before birth to adulthood
\item weakness is symmetric, proximal \textgreater{} distal, and progressive
\item used to be supportive care only now RNA therapy
\end{itemize}

\item Diagnostic Testing
\label{sec:org1126838}
\begin{itemize}
\item established in a proband with a history of motor difficulties or
regression, proximal muscle weakness, reduced/absent deep tendon
reflexes, evidence of motor unit disease
\item identification of biallelic pathogenic variants in SMN1 on
molecular genetic testing
\item increases in SMN2 copy number often modify the phenotype
\end{itemize}

\begin{figure}[htbp]
\centering
\includegraphics[width=0.9\textwidth]{./figures/sma.png}
\caption{\label{fig:orge6b15f4}Algorithm for SMA}
\end{figure}

\item Genetic Counseling
\label{sec:org31a0f62}
-AR SMN1
\end{enumerate}
\end{document}