% Created 2020-01-27 Mon 09:26
% Intended LaTeX compiler: pdflatex
\documentclass{scrartcl}
\usepackage[utf8]{inputenc}
\usepackage[T1]{fontenc}
\usepackage{graphicx}
\usepackage{grffile}
\usepackage{longtable}
\usepackage{wrapfig}
\usepackage{rotating}
\usepackage[normalem]{ulem}
\usepackage{amsmath}
\usepackage{textcomp}
\usepackage{amssymb}
\usepackage{capt-of}
\usepackage{hyperref}
\hypersetup{colorlinks,linkcolor=black,urlcolor=blue}
\usepackage{textpos}
\usepackage{textgreek}
\usepackage[version=4]{mhchem}
\usepackage{chemfig}
\usepackage{siunitx}
\usepackage{gensymb}
\usepackage[usenames,dvipsnames]{xcolor}
\usepackage[T1]{fontenc}
\usepackage{lmodern}
\usepackage{verbatim}
\usepackage{tikz}
\usepackage{wasysym}
\usetikzlibrary{shapes.geometric,arrows,decorations.pathmorphing,backgrounds,positioning,fit,petri}
\usepackage{fancyhdr}
\pagestyle{fancy}
\author{Matthew Henderson, PhD, FCACB}
\date{\today}
\title{Genetic Conditions}
\hypersetup{
 pdfauthor={Matthew Henderson, PhD, FCACB},
 pdftitle={Genetic Conditions},
 pdfkeywords={},
 pdfsubject={},
 pdfcreator={Emacs 26.1 (Org mode 9.1.9)}, 
 pdflang={English}}
\begin{document}

\maketitle
\setcounter{tocdepth}{2}
\tableofcontents


\section{Cancer Genetics}
\label{sec:org71e0dfc}
\subsection{Tuberous sclerosis}
\label{sec:org9649681}
\begin{enumerate}
\item Clinical Characteristics
\label{sec:org16673f7}
\begin{itemize}
\item TSC involves abnormalities of the:
\begin{description}
\item[{skin}] hypomelanotic macules, confetti skin lesions, facial
angiofibromas, shagreen patches, fibrous cephalic plaques,
ungual fibromas
\item[{brain}] subependymal nodules, cortical dysplasias, and
subependymal giant cell astrocytomas [SEGAs], seizures,
intellectual disability / developmental delay,
psychiatric illness
\item[{kidney}] angiomyolipomas, cysts, renal cell carcinomas
\item[{heart}] rhabdomyomas, arrhythmias
\item[{lungs}] lymphangioleiomyomatosis [LAM], multifocal micronodular
pneumonocyte hyperplasia.
\end{description}
\item CNS tumors are the leading cause of morbidity and mortality
\item renal disease is the second leading cause of early death.
\end{itemize}

\item Diagnosis
\label{sec:orgd4d18b0}
\begin{itemize}
\item TSC is established in a proband with one of the following:
\begin{itemize}
\item Two major clinical features
\item One major clinical feature and two or more minor features
\item Identification of a heterozygous pathogenic variant in TSC1 or
TSC2 by molecular genetic testing
\end{itemize}
\end{itemize}

\item Genetic Counseling
\label{sec:orgfaa7df1}
\begin{itemize}
\item AD
\item Two thirds of affected individuals have TSC as the result of a \emph{de novo} pathogenic variant.
\item The offspring of an affected individual are at a 50\% risk of inheriting the pathogenic variant.
\item If the pathogenic variant has been identified in an affected family
member, prenatal testing for pregnancies at increased risk and
preimplantation genetic diagnosis are possible.
\end{itemize}

\item Pathophysiology
\label{sec:orge84af04}
\begin{itemize}
\item Tuberin (TSC1) and hamartin (TSC2) are key regulators of the
AKT/mTOR signaling pathway and to participate in several other
signaling pathways including the MAPK, AMPK, b-catenin, calmodulin,
CDK, autophagy, and cell cycle pathways
\end{itemize}
\end{enumerate}

\subsection{Hereditary breast cancer}
\label{sec:org6ca8dd1}
\begin{enumerate}
\item Clinical Characteristics
\label{sec:org871cf61}
\begin{itemize}
\item BRCA1- and BRCA2-associated hereditary breast and ovarian cancer
syndrome (HBOC) is characterized by an increased risk for female and
male:
\begin{itemize}
\item breast cancer
\item ovarian cancer (includes fallopian tube and primary peritoneal cancers)
\item to a lesser extent other cancers: prostate cancer, pancreatic
cancer, and melanoma primarily in individuals with a BRCA2
pathogenic variant
\end{itemize}
\item The exact cancer risks differ slightly depending on whether HBOC is
caused by a BRCA1 or BRCA2 pathogenic variant
\end{itemize}
\item Diagnosis
\label{sec:org9f07885}
\begin{itemize}
\item The diagnosis of BRCA1 and BRCA2 HBOC is established in a proband by
identification of a heterozygous germline pathogenic variant in
BRCA1 or BRCA2 on molecular genetic testing.
\end{itemize}

\item Genetic Counseling
\label{sec:org13475a4}
\begin{itemize}
\item Germline pathogenic variants in BRCA1 and BRCA2 are inherited in an
autosomal dominant manner
\item The vast majority of individuals with a BRCA1 or BRCA2 pathogenic
variant have inherited it from a parent
\item Because of incomplete penetrance, variable age of cancer
development, cancer risk reduction resulting from prophylactic
surgery, or early death, not all individuals with a BRCA1 or BRCA2
pathogenic variant have a parent affected with cancer
\item Offspring of an individual with a BRCA1 or BRCA2 germline pathogenic variant have a 50\% chance of inheriting the variant
\item Prenatal testing is possible for pregnancies at increased risk if the cancer-predisposing variant in the family is known
\item requests for prenatal diagnosis of adult-onset diseases are uncommon and require careful genetic counseling
\end{itemize}

\item Pathophysiology
\label{sec:org1f903a9}
\begin{itemize}
\item BRCA1 interacts with several proteins involved in cellular pathways,
including cell-cycle progression, gene transcription regulation, DNA
damage response, and ubiquitination
\item BRCA2 appears to be involved in the DNA repair process.
\end{itemize}
\end{enumerate}
\subsection{Chronic myelogenous leukemia (CML)}
\label{sec:org28ae780}
\begin{enumerate}
\item Clinical Characteristics
\label{sec:org27f60d1}
\begin{itemize}
\item CML is a cancer of the white blood cells
\item It is a form of leukemia characterized by the increased and
unregulated growth of myeloid cells in the bone marrow and the
accumulation of these cells in the blood
\item CML is a clonal bone marrow stem cell disorder in which a
proliferation of mature granulocytes (neutrophils, eosinophils and
basophils) and their precursors is found
\item It is a type of myeloproliferative neoplasm associated with a
characteristic chromosomal translocation called the Philadelphia
chromosome
\end{itemize}

\item Diagnosis
\label{sec:orgafe9b9f}
\begin{itemize}
\item CML is often suspected on the basis of a complete blood count,
\begin{itemize}
\item increased granulocytes of all types, typically including mature myeloid cells.
\item Basophils and eosinophils are almost universally increased; this feature may help differentiate CML from a leukemoid reaction.
\end{itemize}
\item A bone marrow biopsy is often performed as part of the evaluation for CML
\item CML is diagnosed by cytogenetics that detects the translocation t(9;22)(q34;q11.2) which involves the ABL1 gene in chromosome 9 and the BCR gene in chromosome 22.
\item As a result of this translocation, the chromosome looks smaller than
its homologue chromosome, and this appearance is known as the
Philadelphia chromosome chromosomal abnormality.
\begin{itemize}
\item can be detected by routine cytogenetics
\item involved genes BCR-ABL1 can be detected by FISH, as well as by
PCR.
\end{itemize}
\end{itemize}

\item Pathophysiology
\label{sec:orgafd6660}
\begin{itemize}
\item Chromosomal translocation where parts of two chromosomes (the 9th
and 22nd) switch places.
\item part of the BCR ("breakpoint cluster region") gene from chromosome
22 is fused with the ABL gene on chromosome 9.
\item abl carries a tyrosine kinase, \therfore the bcr-abl fusion gene
product is also a tyrosine kinase
\item The fused BCR-ABL protein interacts with the interleukin 3beta(c) receptor subunit.
\item The BCR-ABL transcript is continuously active and does not require activation by other cellular messaging proteins.
\item In turn, BCR-ABL activates a cascade of proteins that control the cell cycle, speeding up cell division.
\end{itemize}
\end{enumerate}
\subsection{Familial adenomatous polyposis}
\label{sec:orgc94ce62}
\begin{enumerate}
\item Clinical Characteristics
\label{sec:orgc27bff0}
\begin{itemize}
\item FAP is a colon cancer predisposition syndrome in which hundreds to
thousands of adenomatous colonic polyps develop, beginning, on
average, at age 16 years (range 7-36 years).
\item By age 35 years, 95\% of individuals with FAP have polyps; without
colectomy, colon cancer is inevitable.
\item The mean age of colon cancer diagnosis in untreated individuals is
39 years (range 34-43 years).
\item Extracolonic manifestations are variably present and include:
\begin{itemize}
\item polyps of the gastric fundus and duodenum, osteomas, dental anomalies,
\item congenital hypertrophy of the retinal pigment epithelium (CHRPE)
\item soft tissue tumors, desmoid tumors, and associated cancers
\end{itemize}
\end{itemize}

\item Diagnosis
\label{sec:org3970e1d}
\begin{itemize}
\item suspected in an individual with suggestive personal and/or family
history features and confirmed by identification of a heterozygous
germline pathogenic variant in APC.
\end{itemize}

\item Genetic Counseling
\label{sec:org91ef9c9}
\begin{itemize}
\item AD
\item \textasciitilde{}75\%-80\% of individuals with an APC-associated polyposis condition
have an affected parent
\item Offspring of an affected individual are at a 50\% risk of inheriting
the pathogenic variant in APC.
\item Prenatal testing and preimplantation genetic diagnosis are possible
if a pathogenic variant has been identified in an affected family
member.
\end{itemize}
\item Pathophysiology
\label{sec:org4c4d908}
\begin{itemize}
\item The APC protein product is a tumor suppressor.
\item APC appears to prevent accumulation of cytosolic beta-catenin and
maintain normal apoptosis and may also decrease cell proliferation,
probably through its regulation of beta-catenin.
\end{itemize}
\end{enumerate}
\subsection{Hereditary non-polyposis colon cancer (HNPCC)}
\label{sec:orge3da0f7}
\begin{itemize}
\item AKA: Lynch Syndrome
\end{itemize}
\begin{enumerate}
\item Clinical Characteristics
\label{sec:orgf1cd81e}
\begin{itemize}
\item increased risk for colorectal cancer (CRC) and cancers of the
endometrium, stomach, ovary, small bowel, hepatobiliary tract,
urinary tract, brain, and skin.

\item In individuals with Lynch syndrome the following lifetime risks for
cancer are seen:
\begin{description}
\item[{CRC}] 52\%-82\% (mean age at diagnosis 44-61 years)
\item[{Endometrial cancer in females}] 25\%-60\% (mean age at diagnosis 48-62 years)
\item[{Gastric cancer}] 6\%-13\% (mean age at diagnosis 56 years)
\item[{Ovarian cancer}] 4\%-12\% (mean age at diagnosis 42.5 years; \textasciitilde{}30\% are diagnosed < age 40 years).
\end{description}

\item The risk for other Lynch syndrome-related cancers is lower, though
substantially increased over general population rates
\end{itemize}

\item Diagnosis
\label{sec:org2059ec1}
\begin{itemize}
\item Lynch syndrome is established in a proband by identification of a
germline heterozygous pathogenic variant in MLH1, MSH2, MSH6, or
PMS2 or an EPCAM deletion on molecular genetic testing.
\end{itemize}

\item Genetic Counseling
\label{sec:orgc1433d8}
\begin{itemize}
\item AD
\item The majority of individuals diagnosed with Lynch syndrome have
inherited the condition from a parent.
\item because of incomplete penetrance, variable age of cancer
development, cancer risk reduction as a result of screening or
prophylactic surgery, or early death, not all individuals with a
pathogenic variant in one of the genes associated with Lynch
syndrome have a parent who had cancer.
\item Each child of an individual with Lynch syndrome has a 50\% chance of
inheriting the pathogenic variant.
\item Prenatal diagnosis for pregnancies at increased risk is possible if
the pathogenic variant in the family is known.
\end{itemize}

\item Pathophysiology
\label{sec:org334f519}
\begin{itemize}
\item EPCAM 2p21	Epithelial cell adhesion molecule
\begin{itemize}
\item EPCAM is not a mismatch repair gene, recurrent germline deletions
of the 3' region result in silencing of the adjacent downstream
MSH2 by hypermethylation
\item The adjacent MSH2 allele itself is not mutated
\item Sequence analysis of EPCAM is not appropriate for diagnosis of
Lynch syndrome
\end{itemize}
\item MLH1	3p22​.2	DNA mismatch repair protein
\item MSH2	2p21-p16 DNA mismatch repair protein
\item MSH6	2p16​.3	DNA mismatch repair protein
\item PMS2	7p22​.1	Mismatch repair endonuclease
\end{itemize}
\end{enumerate}

\subsection{Li-Fraumeni syndrome}
\label{sec:orgb64f1b1}
\begin{enumerate}
\item Clinical Characteristics
\label{sec:orged448ed}
\begin{itemize}
\item LFS is a cancer predisposition syndrome associated with the
development of the following classic tumors:
\begin{itemize}
\item soft tissue sarcoma
\item osteosarcoma
\item pre-menopausal breast cancer
\item brain tumors
\item adrenocortical carcinoma (ACC)
\item leukemias.
\end{itemize}
\item In addition, a variety of other neoplasms may occur.
\item LFS-related cancers often occur in childhood or young adulthood and
survivors have an increased risk for multiple primary cancers.
\end{itemize}

\item Diagnosis
\label{sec:org7c65a1e}
\begin{itemize}
\item LFS is diagnosed in individuals meeting established clinical
criteria or in those who have a germline pathogenic variant in TP53
regardless of family cancer history.
\item At least 70\% of individuals diagnosed clinically have an
identifiable germline pathogenic variant in TP53, the only gene so
far identified in which pathogenic variants are definitively
associated with LFS.
\end{itemize}
\item Genetic Counseling
\label{sec:org2bb2d8c}
\begin{itemize}
\item AD
\item 7-20\% \emph{de novo} germline TP53 pathogenic variant
\item Offspring of an affected individual have a 50\% chance of inheriting
the pathogenic variant.
\item Predisposition testing for at-risk family members and prenatal
testing for pregnancies at increased risk are possible if the
heritable pathogenic variant in the family has been identified.
\end{itemize}

\item Pathophysiology
\label{sec:org51803d1}
\begin{itemize}
\item TP53 has been called "the guardian of the genome" and its protein
plays major roles in both the regulation of cell growth and the
maintenance of homeostasis
\item The loss of this important tumor suppressor gene decreases the
likelihood that cells with genetic errors will be flagged for DNA
repair or apoptosis. These DNA-damaged cells can go on to further
proliferate, which can lead to a colony of abnormal cells and
eventually a malignant tumor.
\end{itemize}
\end{enumerate}

\subsection{Retinoblastoma}
\label{sec:org1af59a5}
\begin{enumerate}
\item Clinical Characteristics
\label{sec:orgc20226d}
\begin{itemize}
\item Retinoblastoma is a malignant tumor of the developing retina that occurs in children, usually before age five years
\item Retinoblastoma develops from cells that have cancer-predisposing variants in both copies of RB1
\item Retinoblastoma may be unifocal or multifocal.
\item \(\sim\) 60\% of affected individuals have unilateral retinoblastoma with a mean age of diagnosis of 24 months
\item \(\sim\) 40\% have bilateral retinoblastoma with a mean age of diagnosis of 15 months
\item Heritable retinoblastoma is an autosomal dominant susceptibility for retinoblastoma
\item Individuals with heritable retinoblastoma are also at increased risk of developing non-ocular tumors
\end{itemize}

\item Diagnosis
\label{sec:org7ba1f44}
\begin{itemize}
\item established by examination of the fundus of the eye using indirect
ophthalmoscopy
\item Imaging studies can be used to support the diagnosis and stage the tumor
\item The diagnosis of heritable retinoblastoma is established in a
proband with:
\begin{itemize}
\item retinoblastoma or retinoma and a family history of retinoblastoma or
\item identification of a heterozygous germline pathogenic variant in RB1.
\end{itemize}

\item The following staging has been recommended for individuals with
retinoblastoma and/or risk of heritable retinoblastoma to include
"H" to describe the genetic risk for an individual to have a
germline pathogenic variant in RB1:

\begin{description}
\item[{HX}] Unknown or insufficient evidence of a constitutional
(germline) RB1 pathogenic variant

\item[{H0}] Normal RB1 alleles in blood tested with demonstrated
high-sensitivity assays

\item[{H0*}] Normal RB1 in blood with <1\% residual risk for mosaicism

\item[{H1}] Bilateral retinoblastoma, trilateral retinoblastoma
(retinoblastoma with intracranial CNS midline embryonic
tumor), family history of retinoblastoma, or RB1 pathogenic
variant identified in blood
\end{description}
\end{itemize}

\item Genetic Counseling
\label{sec:org51f0a20}
\begin{itemize}
\item AD
\item Individuals with heritable retinoblastoma (H1) have a heterozygous
\emph{de novo} or inherited germline RB1 pathogenic variant.
\item Offspring of H1 individuals have a 50\% chance of inheriting the
pathogenic variant.
\item Prenatal testing for pregnancies at increased risk is possible if
the RB1 pathogenic variant has been identified in an affected family
member.
\end{itemize}

\item Pathophysiology
\label{sec:orge372a7f}
\begin{itemize}
\item RB1 encodes a ubiquitously expressed nuclear protein that is
involved in cell cycle regulation (G1 to S transition)
\item The RB protein is phosphorylated by members of the cyclin-dependent kinase
(cdk) system prior to the entry into S-phase
\item On phosphorylation, the binding activity of the pocket domain is
lost, resulting in the release of cellular proteins.
\end{itemize}
\end{enumerate}

\section{Clinical Genetics}
\label{sec:org1c2687f}
\subsection{{\bfseries\sffamily TODO} Duchenne/Becker Muscular Dystrophy}
\label{sec:org5a23433}
\subsection{{\bfseries\sffamily TODO} Osteogenesis Imperfecta}
\label{sec:org980f489}
\subsection{{\bfseries\sffamily TODO} Long QT syndrome}
\label{sec:orgcd9a753}
\subsection{{\bfseries\sffamily TODO} Marfan syndrome}
\label{sec:orgad2240c}
\subsection{Neurofibromatosis type I}
\label{sec:org339a046}
\begin{enumerate}
\item Clinical Characteristics
\label{sec:org4c1f4d3}
\begin{itemize}
\item characterized by multiple café au lait spots, axillary and inguinal freckling, multiple cutaneous neurofibromas, iris Lisch nodules, and choroidal freckling
\item \textasciitilde{} 50\% have plexiform neurofibromas, but most are internal and not suspected clinically
\item learning disabilities in \textasciitilde{} 50\%
\item Less common manifestations include optic nerve and other central
nervous system gliomas, malignant peripheral nerve sheath tumors,
scoliosis, tibial dysplasia, and vasculopathy
\end{itemize}
\item Diagnostic Testing
\label{sec:org26b7316}
\begin{itemize}
\item usually based on clinical findings
\item heterozygous pathogenic variants in NF1 are responsible for neurofibromatosis 1
\item molecular genetic testing of NF1 is rarely needed for diagnosis
\end{itemize}
\item Genetic Counseling
\label{sec:org4ffb404}
\begin{itemize}
\item AD, NF1
\item 50\% due to \emph{de novo} NF1 pathogenic variant
\end{itemize}
\end{enumerate}
\subsection{Neurofibromatosis type II}
\label{sec:orga78ee8b}
\begin{enumerate}
\item Clinical Characteristics
\label{sec:orgdf461c6}
\begin{itemize}
\item bilateral vestibular schwannomas with associated symptoms of tinnitus, hearing loss, and balance dysfunction
\item average age of onset is 18 to 24 years
\item almost all affected individuals develop bilateral vestibular
schwannomas by age 30 years
\item NF2 is considered an adult-onset disease
\end{itemize}
\item Diagnostic Testing
\label{sec:orgdd00a7e}
\begin{itemize}
\item consensus diagnostic criteria and/or by identification of a
heterozygous pathogenic variant in NF2 on molecular genetic testing
\end{itemize}

\item Genetic Counseling
\label{sec:org0507dbc}
\begin{itemize}
\item AD, NF2
\item 50\% with affected parent
\item 50\% due to \emph{de novo} NF2 pathogenic variant
\item mosaic also possible
\end{itemize}
\end{enumerate}
\subsection{{\bfseries\sffamily TODO} Smith-Lemli-Opitz syndrome}
\label{sec:orgc2ff53f}
\subsection{{\bfseries\sffamily TODO} Noonan syndrome}
\label{sec:orgb82ae5d}
\subsection{{\bfseries\sffamily TODO} Charge syndrome}
\label{sec:orgeff74ce}
\subsection{{\bfseries\sffamily TODO} FGFR-related disorders (Craniosynostosis\ldots{})}
\label{sec:orgc0f9a29}
\subsection{{\bfseries\sffamily TODO} Factor V Leiden thrombophilia}
\label{sec:orgb6f5283}
\subsection{{\bfseries\sffamily TODO} G6PD deficiency}
\label{sec:org44f8e8e}
\subsection{{\bfseries\sffamily TODO} Hemoglobinopathies (Sickle cell anemia, Alpha / Beta Thalassemia)}
\label{sec:org87800c0}
\subsection{Hemophilia A}
\label{sec:org70afa21}
\begin{enumerate}
\item Clinical Characteristics
\label{sec:orga11b298}
\begin{itemize}
\item deficiency in factor VIII clotting activity that results in
prolonged oozing after injuries, tooth extractions, or surgery, and
delayed or recurrent bleeding prior to complete wound healing.

\item age of diagnosis and frequency of bleeding episodes are related to
the level of factor VIII clotting activity.

\begin{description}
\item[{severe hemophilia A}] 2 - 5 spontaneous bleeding episodes each month
\begin{itemize}
\item are usually diagnosed during the first 2 years of life following
bleeding from minor injuries.
\item spontaneous joint bleeds or deep-muscle hematomas,
\item prolonged bleeding or excessive pain and swelling from minor
injuries, surgery, and tooth extractions.
\end{itemize}

\item[{moderate hemophilia A}] seldom have spontaneous bleeding;
\begin{itemize}
\item prolonged or delayed oozing after relatively minor trauma
\item usually diagnosed before age 5 or 6
\end{itemize}

\item[{mild hemophilia A }] do not have spontaneous bleeding episodes;
\begin{itemize}
\item without pre- and postoperative treatment, abnormal bleeding occurs with surgery
or tooth extractions
\item often not diagnosed until later in life
\end{itemize}
\end{description}
\end{itemize}
\item Diagnostic Testing
\label{sec:org5231361}
\begin{itemize}
\item low factor VIII clotting activity in the presence of a normal,
functional von Willebrand factor level
\item a hemizygous F8 pathogenic variant in a male proband confirms the
diagnosis.
\item a heterozygous F8 pathogenic variant in a symptomatic female
confirms the diagnosis.
\end{itemize}
\item Genetic Counseling
\label{sec:orgc3db320}
\begin{itemize}
\item X-linked, F8
\item risk to sibs of a proband depends on the carrier status of the mother.
\item Carrier females have a 50\% chance of transmitting the F8 pathogenic
variant in each pregnancy:
\begin{itemize}
\item sons who inherit the pathogenic variant will be affected
\item daughters who inherit the pathogenic variant are carriers.
\end{itemize}
\item Affected males transmit the pathogenic variant to all of their
daughters and none of their sons.
\item Carrier testing for at-risk family members and prenatal testing for
pregnancies at increased risk are possible if the F8 pathogenic
variant has been identified or if informative intragenic linked
markers have been identified.
\end{itemize}
\end{enumerate}
\subsection{Hemophilia B}
\label{sec:orgfa07f1d}
\begin{enumerate}
\item Clinical Characteristics
\label{sec:org39af866}
\begin{itemize}
\item deficiency in factor IX clotting
\item same as Hemophilia A (section \ref{sec:org70afa21})
\end{itemize}
\item Diagnostic Testing
\label{sec:orgfe5c2e7}
\begin{itemize}
\item low factor IX clotting activity
\item hemizygous F9 pathogenic variant in a male proband confirms the
diagnosis.
\item heterozygous F9 pathogenic variant on in a symptomatic female
confirms the diagnosis.
\end{itemize}
\item Genetic Counseling
\label{sec:orgfd6ebf3}
\begin{itemize}
\item X-linked, F9
\item same as Hemophilia A (section \ref{sec:org70afa21})
\end{itemize}
\end{enumerate}
\subsection{Hemochromatosis}
\label{sec:orga6b1a5d}
\begin{enumerate}
\item Clinical Characteristics
\label{sec:org0c43363}
\begin{itemize}
\item inappropriately high absorption of iron by the small intestinal
mucosa.

\item The phenotypic spectrum of HFE hemochromatosis includes:

\begin{description}
\item[{Clinical HFE hemochromatosis}] manifestations of end-organ damage secondary to iron overload are present
\begin{itemize}
\item excessive storage of iron in the liver, skin, pancreas, heart, joints, and anterior pituitary gland.
\item early symptoms include: abdominal pain, weakness, lethargy, weight loss, arthralgias, diabetes mellitus; and increased risk of cirrhosis
\end{itemize}
\item[{Biochemical HFE hemochromatosis}] \(\uparrow\) transferrin-iron saturation, and the only evidence of iron overload is \(\uparrow\) serum ferritin
\item[{Non-expressing p.Cys282Tyr homozygotes}] neither clinical manifestations of HFE hemochromatosis nor iron overload are present
\end{description}
\end{itemize}

\item Diagnostic Testing
\label{sec:org17c1c13}
\begin{itemize}
\item biallelic HFE pathogenic variants on molecular genetic testing.
\end{itemize}
\item Genetic Counseling
\label{sec:org4faece5}
\begin{itemize}
\item AR, HFE
\begin{description}
\item[{Risk to sibs}] when both parents of a person with hemochromatosis
are heterozygous for an HFE p.Cys282Tyr variant,
the risk to sibs of inheriting two HFE p.Cys282Tyr
variants is 25\%.
\begin{itemize}
\item Because the HFE p.Cys282Tyr heterozygote prevalence in persons
of European origin is high (11\%, or 1/9), some parents of HFE
p.Cys282Tyr homozygotes have two abnormal HFE alleles.
\item If one parent is heterozygous and the other parent homozygous
for two abnormal HFE alleles, the risk to each sib of inheriting
two HFE pathogenic alleles is 50\%.
\end{itemize}
\item[{Risk to offspring}] Offspring of an individual with HFE
hemochromatosis inherit one HFE p.Cys282Tyr variant from the
parent with HFE hemochromatosis.
\begin{itemize}
\item Because the chance that the other parent is a heterozygote for
HFE p.Cys282Tyr is 1/9, the risk that the offspring will inherit
two HFE p.Cys282Tyr variants is approximately 5\%.
\end{itemize}
\item[{Prenatal testing}] not usually performed because HFE
hemochromatosis is an adult-onset, treatable disorder with low
clinical penetrance.
\end{description}
\end{itemize}
\end{enumerate}

\subsection{{\bfseries\sffamily TODO} SRY translocation}
\label{sec:org92ff2bf}
\subsection{{\bfseries\sffamily TODO} Turner syndrome}
\label{sec:orgbd3452b}
\subsection{{\bfseries\sffamily TODO} Androgen insensitivity syndrome}
\label{sec:org06d4caf}
\subsection{{\bfseries\sffamily TODO} 21-Hydroxylase deficiency}
\label{sec:org8a2b38d}

\section{Complex Inheritance}
\label{sec:org171b177}
\subsection{{\bfseries\sffamily TODO} Alzheimer disease}
\label{sec:orge910dff}
\subsection{{\bfseries\sffamily TODO} Congenital hearing loss}
\label{sec:org6bbecfb}
\subsection{{\bfseries\sffamily TODO} Diabetes mellitus (insulin vs non-insulin dependent)}
\label{sec:org9b6a59f}

\section{Cytogenetics}
\label{sec:orga63fa3b}
\subsection{Cri du chat syndrome}
\label{sec:org8c5ada5}
\begin{enumerate}
\item Clinical Characteristics
\label{sec:orgeab46a6}
\begin{itemize}
\item AKA 5p- (5p minus) syndrome
\item infants often have a high-pitched cry that sounds like that of a
cat
\item characterized by intellectual disability and delayed development,
microcephaly, low birth weight, hypotonia
\item distinctive facial features, including widely set eyes
(hypertelorism), low-set ears, a small jaw, and a rounded face.
\item some born with a heart defect
\end{itemize}

\item Diagnostic Testing
\label{sec:orge8d35ae}
\begin{itemize}
\item CMA, karyotype for a 5p deletion
\item size of the deletion varies among affected individuals
\item larger deletions \(\to\) more severe intellectual disability and
developmental delay than smaller deletions
\end{itemize}

\item Genetic Counseling
\label{sec:org6166704}
\begin{itemize}
\item most cases are not inherited, result of a \emph{de novo} deletion
\item \textasciitilde{} 10 percent of people with cri-du-chat syndrome inherit the
chromosome abnormality from an unaffected parent with a balanced translocation
\end{itemize}
\end{enumerate}
\subsection{Pallister-Killian syndrome}
\label{sec:orgd21df72}
\begin{enumerate}
\item Clinical Characteristics
\label{sec:orge8df80a}
\begin{itemize}
\item hypotonia in infancy and early childhood
\item intellectual disability, distinctive facial features, sparse hair,
areas of unusual pigmentation, and other birth defects
\end{itemize}

\item Diagnostic Testing
\label{sec:org15c6a24}
\begin{itemize}
\item CMA, karyotype for mosaic isochromosome 12p (i(12p))
\end{itemize}

\item Genetic Counseling
\label{sec:orgc3fd29f}
\begin{itemize}
\item not inherited
\end{itemize}
\end{enumerate}

\subsection{Triploidy}
\label{sec:org1253b21}
\begin{enumerate}
\item Clinical Characteristics
\label{sec:org1c34097}
Triploidy can result from either two sperm fertilizing one egg
(polyspermy) (60\%) or from one sperm fertilizing an egg with two
copies of every chromosome (40\%)

\begin{itemize}
\item Many organ systems are affected by triploidy, but the central
nervous system and skeleton are the most severely affected:
\begin{itemize}
\item holoprosencephaly, hydrocephalus, ventriculomegaly, Arnold–Chiari
malformation, agenesis of the corpus callosum and neural tube
defects
\end{itemize}
\item Skeletal manifestations include cleft lip/palate, hypertelorism,
club foot and syndactyly of fingers three and four
\item Congenital heart defects, hydronephrosis, omphalocele and
meningocele (spina bifida) are also common
\item IUGR
\end{itemize}

\item Diagnostic Testing
\label{sec:org8f4e873}
\begin{itemize}
\item \(\uparrow\) AFP
\item RAD, karyotype
\end{itemize}
\item Genetic Counseling
\label{sec:org18af5c9}
\begin{itemize}
\item not inherited
\end{itemize}
\end{enumerate}
\subsection{Trisomy 13}
\label{sec:org4173bd1}
\begin{itemize}
\item AKA Patau syndrome
\end{itemize}
\begin{enumerate}
\item Clinical Characteristics
\label{sec:org6126ba7}
\begin{itemize}
\item severe intellectual disability and physical abnormalities in many parts of the body
\item heart defects, brain or spinal cord abnormalities
\item very small or poorly developed eyes (microphthalmia)
\item extra fingers or toes, cleft lip \textpm{} cleft palate, hypotonia
\item often die within their first days or weeks of life
\end{itemize}

\item Diagnostic Testing
\label{sec:orgff29cea}
\begin{itemize}
\item CMD, RAD, karyotype
\end{itemize}

\item Genetic Counseling
\label{sec:orgbb22910}
\begin{itemize}
\item 1 in 16,000 newborns
\item women of any age can have a child with trisomy 13
\begin{itemize}
\item increases with maternal age
\end{itemize}
\end{itemize}
\end{enumerate}
\subsection{Trisomy 18}
\label{sec:org382fd2d}
\begin{itemize}
\item AKA Edwards syndrome
\end{itemize}
\begin{enumerate}
\item Clinical Characteristics
\label{sec:orgb2729fd}
\begin{itemize}
\item IUGR and LBW
\item heart defects and abnormalities of other organs that develop before birth
\item small, abnormally shaped head; a small jaw and mouth; and clenched fists with overlapping fingers
\item often die before birth or within their first month
\item 5-10\% live past their first year, and these children often have
severe intellectual disability
\end{itemize}

\item Diagnostic Testing
\label{sec:org40dc35f}
\begin{itemize}
\item CMD, RAD, karyotype
\end{itemize}

\item Genetic Counseling
\label{sec:org2cf052f}
\begin{itemize}
\item 1 in 5000 live-born infants
\item women of any age can have a child with trisomy 18
\begin{itemize}
\item increases with maternal age
\end{itemize}
\end{itemize}
\end{enumerate}

\subsection{Trisomy 21}
\label{sec:org6b82867}
\begin{itemize}
\item AKA Down syndrome
\end{itemize}
\begin{enumerate}
\item Clinical Characteristics
\label{sec:org1dfd2ec}
\begin{itemize}
\item intellectual disability, a characteristic facial appearance, hypotonia in infancy
\begin{itemize}
\item intellectual disability is usually mild to moderate
\end{itemize}
\item \textasciitilde{}50\% have heart defects
\end{itemize}
\item Diagnostic Testing
\label{sec:org5b76f35}
\begin{itemize}
\item CMD, RAD, karyotype
\end{itemize}

\item Genetic Counseling
\label{sec:orge762a5d}
\begin{itemize}
\item 1 in 800 newborns
\item women of any age can have a child with trisomy 21
\begin{itemize}
\item increases with maternal age
\end{itemize}
\end{itemize}
\end{enumerate}
\subsection{Klinefelter syndrome}
\label{sec:org2426455}
\begin{enumerate}
\item Clinical Characteristics
\label{sec:orgfc1036a}
\begin{itemize}
\item boys and men, affects physical and intellectual development
\item taller than average and infertile
\item signs and symptoms of Klinefelter syndrome vary among boys and men with this condition
\item reduced testosterone
\end{itemize}
\item Diagnostic Testing
\label{sec:org33b1dfe}
\begin{itemize}
\item 47,XXY karyotype
\item mosaic Klinefelter syndrome 46,XY/47,XXY
\end{itemize}

\item Genetic Counseling
\label{sec:org6c661ef}
\begin{itemize}
\item not inherited
\end{itemize}
\end{enumerate}
\subsection{Fanconi anemia}
\label{sec:org4f8ca3b}
\begin{enumerate}
\item Clinical Characteristics
\label{sec:orgdc646a4}
\begin{itemize}
\item physical abnormalities, bone marrow failure, and increased risk for
malignancy
\item physical abnormalities, present in approximately 75\% of affected individuals, include one or more of the following:
\begin{itemize}
\item short stature, abnormal skin pigmentation, skeletal malformations
of the upper and lower limbs, microcephaly, and ophthalmic and
genitourinary tract anomalies
\end{itemize}
\end{itemize}

\item Diagnostic Testing
\label{sec:org3f456dd}
\begin{itemize}
\item established in a proband with increased chromosome breakage and
radial forms on cytogenetic testing of lymphocytes with
diepoxybutane (DEB) and mitomycin C (MMC)

\item diagnosis is confirmed by identification of one of the following:
\begin{itemize}
\item biallelic pathogenic variants in one of the 19 genes known to
cause autosomal recessive FA
\item heterozygous pathogenic variant in RAD51, known to cause autosomal dominant FA
\item hemizygous pathogenic variant in FANCB, known to cause X-linked FA
\end{itemize}
\end{itemize}

\item Genetic Counseling
\label{sec:org3df86f6}
\begin{itemize}
\item AR, AD (RAD51) or X-linked (FANCB)
\end{itemize}
\end{enumerate}

\subsection{Ataxia-telangiectasia}
\label{sec:org52784da}
\begin{enumerate}
\item Clinical Characteristics
\label{sec:org30713b8}
\begin{itemize}
\item progressive cerebellar ataxia beginning between ages one and four
years, oculomotor apraxia, choreoathetosis, telangiectasias of the
conjunctivae, immunodeficiency, frequent infections, and an
increased risk for malignancy, particularly leukemia and lymphoma
\end{itemize}
\item Diagnostic Testing
\label{sec:orgda0181e}
\begin{itemize}
\item diagnosis is established by the presence of biallelic (homozygous or
compound heterozygous) ATM pathogenic variants or (when available)
by immunoblotting to test for absent or reduced ATM protein
\end{itemize}

\item Genetic Counseling
\label{sec:org783f92b}
\begin{itemize}
\item AR, ATM
\end{itemize}
\end{enumerate}
\subsection{Williams syndrome}
\label{sec:org8da309f}
\begin{enumerate}
\item Clinical Characteristics
\label{sec:orgde0c4b5}
\begin{itemize}
\item cardiovascular disease :elastin arteriopathy, peripheral pulmonary
stenosis, supravalvar aortic stenosis, hypertension
\item distinctive facies, connective tissue abnormalities, intellectual
disability (usually mild)
\item a specific cognitive profile, unique personality characteristics
\item growth abnormalities, and endocrine abnormalities (hypercalcemia,
hypercalciuria, hypothyroidism, and early puberty)
\item hypotonia and hyperextensible joints can result in delayed
attainment of motor milestones
\end{itemize}

\item Diagnostic Testing
\label{sec:org7037208}
\begin{itemize}
\item clinical diagnostic criteria
\item diagnosis requires detection of a recurrent 7q11.23 contiguous gene deletion of the Williams-Beuren syndrome critical region (WBSCR) that encompasses the elastin gene (ELN)
\begin{itemize}
\item can be detected using FISH and/or deletion/duplication testing
\end{itemize}
\end{itemize}
\item Genetic Counseling
\label{sec:orgbb165a7}
\begin{itemize}
\item AD
\item most \emph{de novo}
\end{itemize}
\end{enumerate}

\subsection{22q11 deletion syndrome}
\label{sec:orgcc5b1f8}
\begin{enumerate}
\item Clinical Characteristics
\label{sec:org249b678}
\begin{itemize}
\item a contiguous gene deletion syndrome
\item included phenotypes:
\begin{itemize}
\item DiGeorge syndrome
\item Velocardiofacial syndrome
\item Conotruncal anomaly face syndrome
\item Autosomal dominant Opitz G/BBB syndrome
\item Sedlackova syndrome
\item Cayler cardiofacial syndrome
\end{itemize}

\item congenital heart disease (74\%)
\begin{itemize}
\item tetralogy of Fallot, interrupted aortic arch, ventricular septal defect, and truncus arteriosus
\end{itemize}
\item palatal abnormalities (69\%)
\begin{itemize}
\item velopharyngeal incompetence, submucosal cleft palate, bifid uvula, and cleft palate
\end{itemize}
\item facial features (majority of northern European)
\item learning difficulties (70\%-90\%)
\item immune deficiency (77\%)
\end{itemize}

\item Diagnostic Testing
\label{sec:org1c94b7b}
\begin{itemize}
\item submicroscopic deletion of chromosome 22 by FISH, MLPA, CMA
\end{itemize}

\item Genetic Counseling
\label{sec:org92926ff}
\begin{itemize}
\item AD
\item \textasciitilde{} 93\% \emph{de novo} deletion of 22q11.2
\item \textasciitilde{} 7\% inherited the 22q11.2 deletion
\end{itemize}
\end{enumerate}

\section{Metabolics}
\label{sec:org289f9f3}
\subsection{{\bfseries\sffamily TODO} Acute intermittent porphyria}
\label{sec:org45140e5}
\subsection{{\bfseries\sffamily TODO} Alpha-1 antitrypsin deficiency}
\label{sec:orgcedb790}
\subsection{{\bfseries\sffamily TODO} Canavan disease}
\label{sec:org06bed88}
\subsection{{\bfseries\sffamily TODO} Gaucher disease}
\label{sec:org0c633cc}
\subsection{{\bfseries\sffamily TODO} Homocystinuria (CBS deficiency)}
\label{sec:org98e5a9c}
\subsection{{\bfseries\sffamily TODO} Hurler syndrome}
\label{sec:orga79809a}
\subsection{{\bfseries\sffamily TODO} I cell disease}
\label{sec:orgffeeca3}
\subsection{{\bfseries\sffamily TODO} MCAD deficiency}
\label{sec:org4a1e6ed}
\subsection{{\bfseries\sffamily TODO} Mitochondrial DNA mutations}
\label{sec:org7c43670}
\begin{itemize}
\item (MERRF/MELAS/LHON, mtDNA deletion syndromes)
\end{itemize}
\subsection{{\bfseries\sffamily TODO} Ornithine transcarbamylase (OTC) deficiency}
\label{sec:org63898b5}
\subsection{{\bfseries\sffamily TODO} Peroxisome biogenesis disorder (Zellweger)}
\label{sec:org041fec7}
\subsection{{\bfseries\sffamily TODO} Phenylketonuria}
\label{sec:org4c7a48c}
\subsection{{\bfseries\sffamily TODO} Pompe disease}
\label{sec:org54adfab}
\subsection{{\bfseries\sffamily TODO} Tay Sachs Disease}
\label{sec:org19f813b}
\subsection{{\bfseries\sffamily TODO} Tyrosinemia type I}
\label{sec:orgfcbb64a}
\subsection{{\bfseries\sffamily TODO} X-linked Adrenoleukodystrophy}
\label{sec:org435fd6a}
\subsection{{\bfseries\sffamily TODO} Wilson disease}
\label{sec:org6a6e624}
\subsection{{\bfseries\sffamily TODO} Hyperprolinemia type I}
\label{sec:org265245a}

\section{Molecular Genetics}
\label{sec:org358f356}
\subsection{Cystic fibrosis}
\label{sec:org144dead}
\begin{enumerate}
\item Clinical Characteristics
\label{sec:orgb958f2b}
\begin{itemize}
\item Multisystem disease affecting epithelia of the respiratory tract, exocrine pancreas, intestine, hepatobiliary system, and exocrine sweat glands.
\item Morbidities include progressive obstructive lung disease with bronchiectasis, frequent hospitalizations for pulmonary disease, pancreatic insufficiency and malnutrition, recurrent sinusitis and bronchitis, and male infertility.
\item Pulmonary disease is the major cause of morbidity and mortality in CF.
\item Meconium ileus occurs at birth in 15\%-20\% of newborns with CF.
\item More than 95\% of males with CF are infertile.
\end{itemize}

\item Diagnostic Testing
\label{sec:org92462c6}
\begin{itemize}
\item one or more characteristic phenotypic features and evidence of an abnormality in CFTR function
\begin{itemize}
\item 2 elevated sweat chloride values or
\item biallelic CFTR pathogenic variants or
\item transepithelial nasal potential difference measurement characteristic of CF
\end{itemize}
\item Diagnosis of CF is established in an infant with:
\begin{itemize}
\item elevated NBS IRT and
\item identification of biallelic CFTR pathogenic variants or
\item an elevated sweat chloride.
\begin{itemize}
\item quantitative pilocarpine iontophoresis sweat chloride values (>60 mmol/L in infants age > 6 months)
\end{itemize}
\end{itemize}
\end{itemize}
\item Genetic Counseling
\label{sec:org6ae38e1}
\begin{itemize}
\item AR , CFTR
\item At conception, each sib of an affected individual with CF have:
\begin{itemize}
\item a 25\% chance of being affected
\item a 50\% chance of being an asymptomatic carrier
\item a 25\% chance of being unaffected and not a carrier.
\end{itemize}
\item Carrier testing for at-risk relatives and prenatal testing for pregnancies at increased risk are possible if the CFTR pathogenic variants in the family are known.
\end{itemize}
\end{enumerate}
\subsection{Achondroplasia}
\label{sec:org6b1c79f}
\begin{enumerate}
\item Clinical Characteristics
\label{sec:org7bdb711}
\begin{itemize}
\item Most common cause of disproportionate short stature.
\item Affected individuals have rhizomelic shortening of the limbs,
macrocephaly, and characteristic facial features with frontal
bossing and midface retrusion.
\item In infancy, hypotonia is typical, and acquisition of developmental
motor milestones is often both aberrant in pattern and delayed
\item Intelligence and life span are usually near normal, although
craniocervical junction compression increases the risk of death in
infancy.
\item Additional complications include obstructive sleep apnea, middle ear
dysfunction, kyphosis, and spinal stenosis.
\end{itemize}

\item Diagnostic Testing
\label{sec:orge763aee}
\begin{itemize}
\item Diagnosed by characteristic clinical and radiographic findings in
most affected individuals.
\item In individuals in whom there is diagnostic uncertainty or atypical
findings, identification of a heterozygous pathogenic variant in
FGFR3 can establish the diagnosis.
\end{itemize}

\item Genetic Counseling
\label{sec:orgd0ef914}
\begin{itemize}
\item AD, FGFR3
\item \textasciitilde{}80\% of individuals with achondroplasia have parents with average
stature and have achondroplasia as the result of a \emph{de novo}
pathogenic variant.
\begin{itemize}
\item these parents have a very low risk of having another child with
achondroplasia.
\end{itemize}
\item An individual with achondroplasia who has a reproductive partner
with average stature is at 50\% risk in each pregnancy of having a
child with achondroplasia.
\item When both parents have achondroplasia the risk to their offspring of
having:
\begin{itemize}
\item average stature is 25\%
\item achondroplasia is 50\%
\item homozygous achondroplasia (a lethal condition) is 25\%.
\end{itemize}

\item If the proband and the proband's reproductive partner are affected
with different dominantly inherited skeletal dysplasias, genetic
counseling becomes more complicated because of the risk of
inheriting two dominant skeletal dysplasias.
\end{itemize}
\end{enumerate}
\subsection{Huntington disease}
\label{sec:org30c31f0}
\begin{enumerate}
\item Clinical Characteristics
\label{sec:org5898cbc}
\begin{itemize}
\item a progressive disorder of motor, cognitive, and psychiatric
disturbances.
\item mean age of onset is 35 to 44 years and the median survival time is
15 to 18 years after onset.
\end{itemize}
\item Diagnostic Testing
\label{sec:orgdae304a}
\begin{itemize}
\item diagnosis of HD rests on:
\begin{itemize}
\item positive family history
\item characteristic clinical findings
\item detection of an expansion of 36 or more CAG trinucleotide repeats in HTT.
\end{itemize}

\item All individuals with HD have an expansion in the number of CAG
trinucleotide repeats that encode glutamine amino acids in exon 1 of
HTT.
\begin{description}
\item[{Normal alleles}] 26 or fewer CAG trinucleotide repeats
\item[{Intermediate alleles}] 27-35 CAG trinucleotide repeats
\begin{itemize}
\item An individual with an allele in this range is not at risk of
developing symptoms of HD but, because of instability in the CAG
tract, may be at risk of having a child with an allele in the
HD-causing range
\end{itemize}
\item[{HD-causing alleles}] \(\ge\) 36 CAG trinucleotide repeats
\begin{itemize}
\item Persons who have an HD-causing allele are considered at risk of
developing HD in their lifetime.
\item HD-causing alleles are further classified as:
\begin{description}
\item[{Reduced-penetrance HD-causing alleles}] 36-39 CAG
\item[{Full-penetrance HD-causing alleles}] \(\ge\) 40 CAG
\end{description}
\end{itemize}
\end{description}
\end{itemize}
\item Genetic Counseling
\label{sec:orgce632f2}
\begin{itemize}
\item AD, HTT
\item Offspring of an individual with a pathogenic variant have a 50\% chance of inheriting the disease-causing allele.
\item Predictive testing in asymptomatic adults at risk is available but requires careful thought (including pre- and post-test genetic counseling) as there is currently no cure for the disorder.
\begin{itemize}
\item asymptomatic individuals at risk may be eligible to participate in clinical trials.
\end{itemize}
\item Predictive testing is not considered appropriate for asymptomatic at-risk individuals younger than age 18 years.
\item Prenatal testing by molecular genetic testing is possible.
\end{itemize}
\end{enumerate}
\subsection{Fragile X}
\label{sec:org644a32e}
\begin{enumerate}
\item Clinical Characteristics
\label{sec:orga402d77}
\begin{itemize}
\item Fragile X syndrome occurs in individuals with an FMR1 full mutation
or other loss-of-function variant.
\item Males: moderate intellectual disability
\item Females: mild intellectual disability
\item FMR1 pathogenic variants are complex alterations involving non-classic
gene-disrupting alterations (trinucleotide repeat expansion) and
abnormal gene methylation,
\begin{itemize}
\item \(\therefore\) affected individuals occasionally have an atypical presentation with an IQ above 70,
\begin{itemize}
\item the traditional  demarcation denoting intellectual disability.
\end{itemize}
\end{itemize}
\item Males with an FMR1 full mutation accompanied by aberrant methylation may have a characteristic appearance:
\begin{itemize}
\item large head, long face, prominent forehead and chin, protruding ears
\item connective tissue findings (joint laxity), and large testes after puberty.
\item Behavioral abnormalities, sometimes including autism spectrum disorder, are common.
\end{itemize}
\item fragile X-associated tremor/ataxia syndrome, and FMR1-related
primary ovarian insufficiency are less severe forms due to smaller
repeats
\end{itemize}
\item Diagnostic Testing
\label{sec:org16f4932}
\begin{itemize}
\item alteration in FMR1.
\item \textgreater{} 99\% of individuals with fragile X syndrome have:
\begin{itemize}
\item lof variant of FMR1 caused by an increased number of CGG
trinucleotide repeats (typically >200)
\item accompanied by aberrant methylation of FMR1
\end{itemize}
\item Other pathogenic variants include:
\begin{itemize}
\item deletions and single-nucleotide variants.
\end{itemize}
\end{itemize}
\item Genetic Counseling
\label{sec:orgcaa03b0}
\begin{itemize}
\item All mothers of individuals with an FMR1 full mutation (expansion
>200 CGG trinucleotide repeats and abnormal methylation) are
carriers of an FMR1 pathogenic variant.
\item Mothers and their female relatives who are premutation carriers are
at increased risk for FXTAS and POI;
\item those with a full mutation may have findings of fragile X syndrome.
\item All are at increased risk of having offspring with fragile X syndrome, FXTAS, and POI.
\item Males with premutations are at increased risk for FXTAS.
\item Males with FXTAS will transmit their FMR1 premutation expansion to none of their sons and to all of their daughters, who will be premutation carriers.
\item Carrier testing for at-risk relatives and prenatal testing for
pregnancies at increased risk are possible if the diagnosis of an
FMR1-related disorder has been confirmed in a family member.
\end{itemize}
\end{enumerate}
\subsection{Friedreich's ataxia}
\label{sec:org75010a1}
\begin{enumerate}
\item Clinical Characteristics
\label{sec:orgd4cef57}
\begin{itemize}
\item characterized by slowly progressive ataxia with onset usually before
age 25 years (mean 10-15 yrs).
\item FRDA is typically associated with dysarthria, muscle weakness,
spasticity particularly in the lower limbs, scoliosis, bladder
dysfunction, absent lower-limb reflexes, and loss of position and
vibration sense.
\begin{itemize}
\item \textasciitilde{}2/3 have cardiomyopathy
\item \textasciitilde{}30\% have diabetes mellitus,
\item \textasciitilde{}25\% have an "atypical" presentation with later onset or retained
tendon reflexes.
\end{itemize}
\end{itemize}
\item Diagnostic Testing
\label{sec:org4f89213}
\begin{itemize}
\item established in a proband by detection of biallelic pathogenic
variants in FXN.
\item An abnormally expanded GAA repeat in intron 1 of FXN observed on
both alleles in \textasciitilde{}96\% with FRDA
\item remaining are compound heterozygotes for abnormally expanded GAA
repeat in the disease-causing range on one allele and another
intragenic pathogenic variant on the other allele.
\end{itemize}


\begin{itemize}
\item Four classes of alleles are recognized for the GAA repeat sequence in intron 1 of FXN
\begin{description}
\item[{Normal alleles}] 5-33 GAA repeats
\item[{Mutable normal (premutation) alleles}] 34-65 GAA repeats
\item[{Borderline alleles}] 44-66 GAA repeats. The shortest repeat length associated with disease
\item[{Full-penetrance (disease-causing expanded) alleles}] 66-1,300 GAA repeats
\end{description}
\end{itemize}

Rare alleles of variant structure. In contrast to the alleles discussed above in which the GAA trinucleotides are perfect repeats, in rare pathogenic alleles the GAA repeats are not in perfect tandem order but rather are interrupted by other nucleotides. Such "interrupted FXN alleles" differ in length and types of nucleotides in the interruption, but they are typically close to the 3' end of the GAA repeat tract (see Molecular Genetics).
\item Genetic Counseling
\label{sec:org6225207}
\begin{itemize}
\item AR, FXN
\item Each sib has a 25\% chance of being affected
\begin{itemize}
\item 50\% chance of being an asymptomatic carrier
\item 25\% chance of having no pathogenic variant.
\end{itemize}
\item Carrier testing of at-risk relatives, prenatal testing for
pregnancies at increased risk, and pre-implantation genetic diagnosis
are possible if both FXN pathogenic variants have been identified in
an affected family member.
\end{itemize}
\end{enumerate}

\subsection{Myotonic dystrophy type I}
\label{sec:orgfe7c586}
\begin{enumerate}
\item Clinical Characteristics
\label{sec:orgcac532e}
\begin{itemize}
\item multisystem disorder that affects skeletal and smooth muscle as well
as the eye, heart, endocrine system, and central nervous system.

\item clinical findings, from mild to severe:

\begin{description}
\item[{Mild DM1}] cataract and mild myotonia (sustained muscle
contraction) life span is normal

\item[{Classic DM1}] muscle weakness and wasting, myotonia, cataract,
and often cardiac conduction abnormalities; adults
may become physically disabled and may have a
shortened life span.

\item[{Congenital DM1}] hypotonia and severe generalized weakness at
birth, often with respiratory insufficiency and
early death; intellectual disability is common.
\end{description}
\end{itemize}

\item Diagnostic Testing
\label{sec:orgf56e196}
\begin{itemize}
\item caused by expansion of a CTG trinucleotide repeat in the noncoding region of DMPK.
\item molecular genetic testing of DMPK.
\item CTG repeat length exceeding 34 repeats is abnormal.
\item Molecular genetic testing detects pathogenic variants in nearly 100\%
of affected individuals.

\begin{description}
\item[{Normal alleles}] 5-34 CTG repeats
\item[{Mutable normal (premutation) alleles}] 35-49 CTG repeats
\item[{Full-penetrance alleles}] \(\ge\) 50 CTG repeats
\end{description}
\end{itemize}

\item Genetic Counseling
\label{sec:org61cc694}
\begin{itemize}
\item AD, DMPK
\item Offspring of an affected individual have a 50\% chance of inheriting the expanded allele.
\item Pathogenic alleles may expand in length during gametogenesis
\begin{itemize}
\item \(\to\) transmission of longer trinucleotide repeat alleles
\item \(\to\) earlier onset and more severe disease the parent
\end{itemize}
\item Prenatal testing is possible for pregnancies at increased risk when
the diagnosis of DM1 has been confirmed by molecular genetic testing
in an affected family member.
\end{itemize}
\end{enumerate}
\subsection{Angelman syndrome}
\label{sec:org33903e7}
\begin{enumerate}
\item Clinical Characteristics
\label{sec:org77236d4}
\begin{itemize}
\item Severe developmental delay or intellectual disability, severe speech
impairment, gait ataxia and/or tremulousness of the limbs
\item unique behavior with an inappropriate happy demeanor that includes
frequent laughing, smiling, and excitability.
\item Microcephaly and seizures are also common.
\item Developmental delays are first noted at around age six months
\item clinical features of AS do not become manifest until after age one year
\begin{itemize}
\item can take several years before the correct clinical diagnosis is obvious.
\end{itemize}
\end{itemize}

\item Diagnostic Testing
\label{sec:orgc50bee8}

\begin{figure}[htbp]
\centering
\includegraphics[width=0.6\textwidth]{./figures/aspws.jpg}
\caption{\label{fig:org392c620}
AS and PWS}
\end{figure}

\begin{itemize}
\item molecular genetic testing deficient expression or function of the
maternally inherited UBE3A allele.
\item parent-specific DNA methylation imprints in the 15q11.2-q13 chromosome region detects approximately 80\%
\begin{itemize}
\item including deletion, uniparental disomy (UPD), imprinting defect (ID)
\end{itemize}
\item \textless{} 1\% have a cytogenetically visible chromosome rearrangement (i.e., translocation or inversion).
\item UBE3A sequence analysis detects pathogenic variants \textasciitilde{}11\% of individuals.
\item molecular genetic testing (methylation analysis and UBE3A sequence
analysis) \textasciitilde{}90\% of individuals.
\item Remaining 10\% with classic phenotypic features of AS have the
disorder as a result of an as-yet unidentified genetic mechanism
\begin{itemize}
\item not amenable to diagnostic testing
\end{itemize}
\end{itemize}
\item Genetic Counseling
\label{sec:org65e555a}
\begin{itemize}
\item caused by disruption of maternally imprinted UBE3A located within
the 15q11.2-q13 Angelman syndrome/Prader-Willi syndrome region.
\item The risk to sibs of a proband depends on the genetic mechanism
leading to the loss of UBE3A function
\begin{itemize}
\item typically less than 1\% risk for probands with a deletion or UPD
\item as high as 50\% for probands with an ID or a pathogenic variant of UBE3A.
\end{itemize}
\item Members of the mother's extended family are also at increased risk
when an ID or a UBE3A pathogenic variant is present.
\item Cytogepnetically visible chromosome rearrangements may be inherited,usually \emph{de novo}.
\item Prenatal testing is possible when the underlying genetic mechanism
is a deletion, UPD, an ID, a UBE3A pathogenic variant, or a
chromosome rearrangement.
\end{itemize}
\end{enumerate}
\subsection{Beckwith-Wiedemann syndrome}
\label{sec:orgfbf68d9}
\begin{enumerate}
\item Clinical Characteristics
\label{sec:org656cbbd}
\begin{itemize}
\item growth disorder variably characterized by neonatal hypoglycemia,
macrosomia, macroglossia, hemihyperplasia, omphalocele, embryonal
tumors (e.g., Wilms tumor, hepatoblastoma, neuroblastoma, and
rhabdomyosarcoma), visceromegaly, adrenocortical cytomegaly, renal
abnormalities (e.g., medullary dysplasia, nephrocalcinosis,
medullary sponge kidney, and nephromegaly), and ear creases/pits.

\item a clinical spectrum, may have many of these features or only one or two.

\item Early death may occur from complications of prematurity,
hypoglycemia, cardiomyopathy, macroglossia, or tumors.
\end{itemize}

\item Diagnostic Testing
\label{sec:org1e68b61}

\begin{itemize}
\item Cytogenetically detectable abnormalities involving chromosome 11p15
are found in 1\% or fewer of affected individuals.

\item Molecular genetic testing can identify epigenetic and genomic
alterations of chromosome 11p15 in individuals with BWS:
\begin{itemize}
\item Loss of methylation on the maternal chromosome at imprinting
center 2 (IC2) in 50\% of affected individuals;
\item Paternal uniparental disomy for chromosome 11p15 in 20\%
\item Gain of methylation on the maternal chromosome at imprinting
center 1 (IC1) in 5\%.
\end{itemize}
\item Methylation alterations associated with deletions or duplications in
this region have high heritability.

\item Sequence analysis of CDKN1C identifies a heterozygous maternally
inherited pathogenic variant in approximately 40\% of familial cases
and 5\%-10\% of cases with no family history of BWS.
\end{itemize}

\begin{figure}[htbp]
\centering
\includegraphics[width=0.5\textwidth]{./figures/bws.png}
\caption{\label{fig:orgba08ecf}
BWS Chromosome 11}
\end{figure}

\item Genetic Counseling
\label{sec:org59a5f38}
\begin{itemize}
\item associated with abnormal regulation of gene transcription in two
imprinted domains on chromosome 11p15.5.
\item Most individuals with BWS are reported to have normal chromosome
studies or karyotypes.
\item \textasciitilde{}85\% of individuals with BWS have no family history of BWS
\item \textasciitilde{}15\% have a family history consistent with parent-of-origin
autosomal dominant transmission.
\item Children of subfertile parents conceived by assisted reproductive
technology (ART) may be at increased risk for imprinting disorders,
including BWS.
\item Identification of the underlying genetic mechanism causing BWS
permits better estimation of recurrence risk.
\item Prenatal screening for pregnancies in the general population that
identifies findings suggestive of a diagnosis of BWS may lead to the
consideration of
\begin{itemize}
\item chromosome analysis, chromosomal microarray, and/or molecular genetic testing.
\end{itemize}
\item prenatal testing by chromosome analysis for families with an
inherited chromosome abnormality or by molecular genetic testing for
families in which the molecular mechanism of BWS has been defined
\end{itemize}
\end{enumerate}

\subsection{Prader-Willi syndrome}
\label{sec:org9642b86}
\begin{enumerate}
\item Clinical Characteristics
\label{sec:orgab111eb}
\begin{itemize}
\item severe hypotonia and feeding difficulties in early infancy
\item later infancy or early childhood by excessive eating
\item gradual development of morbid obesity
\item Motor milestones and language development are delayed.
\item All individuals have some degree of cognitive impairment.
\item A distinctive behavioral phenotype (with temper tantrums, stubbornness, manipulative behavior, and obsessive-compulsive characteristics) is common.
\item Hypogonadism is present in both males and females and manifests as genital hypoplasia, incomplete pubertal development, and, in most, infertility.
\item Short stature is common (if not treated with growth hormone);
\item characteristic facial features, strabismus, and scoliosis are often present.
\end{itemize}

\item Diagnostic Testing
\label{sec:orgf019c21}
\begin{itemize}
\item DNA methylation testing to detect abnormal parent-specific
imprinting within the Prader-Willi critical region (PWCR) on
chromosome 15
\item testing determines whether the region is maternally inherited only
\begin{itemize}
\item the paternally contributed region is absent
\item detects more than 99\% of affected individuals
\item DNA methylation-specific testing is important to confirm the
diagnosis of PWS in all individuals,
\end{itemize}
\end{itemize}

\item Genetic Counseling
\label{sec:org00c82ed}
\begin{itemize}
\item PWS is caused by an absence of expression of imprinted genes in the
paternally derived PWS/Angelman syndrome (AS) region (15q11.2-q13)
of chromosome 15:
\begin{itemize}
\item paternal deletion, maternal uniparental disomy 15 and rarely an imprinting defect.
\end{itemize}
\item The risk to the sibs depends on the genetic mechanism.
\begin{itemize}
\item \textless{} 1\% if the affected child has a deletion or uniparental disomy
\item up to 50\% if the affected child has an imprinting defect
\item up to 25\% if a parental chromosome translocation is present
\end{itemize}
\item Prenatal testing is possible for pregnancies at increased risk if
the underlying genetic mechanism is known.
\end{itemize}
\end{enumerate}
\subsection{Russell-Silver syndrome}
\label{sec:orgf22079b}
\begin{enumerate}
\item Clinical Characteristics
\label{sec:org0e600b8}
\begin{itemize}
\item asymmetric gestational growth restriction resulting in affected
individuals being born small for gestational age, with relative
macrocephaly at birth (head circumference \(\le\)1.5 SD above birth
weight and/or length), prominent forehead usually with frontal
bossing, and frequently body asymmetry.
\item This is followed by postnatal growth failure, and in some cases progressive limb length discrepancy and feeding difficulties.
\item Additional clinical features include triangular facies, fifth-finger clinodactyly, and micrognathia with narrow chin.
\item The average adult height in untreated individuals is \textasciitilde{}3.1\textpm{}1.4 SD below the mean.
\end{itemize}
\item Diagnostic Testing
\label{sec:orgf08b6ef}
\begin{itemize}
\item a genetically heterogeneous condition.
\item Genetic testing confirms clinical diagnosis in approximately 60\% of
affected individuals.
\begin{itemize}
\item hypomethylation of the imprinted control region 1 (ICR1) at
11p15.5 causes SRS in 35\%-50\% of individuals
\item maternal uniparental disomy (mUPD7) causes SRS in 7\%-10\% of individuals.
\item a small number of individuals with SRS who have duplications,
deletions or translocations involving the imprinting centers at
11p15.5 or duplications, deletions, or translocations involving
chromosome 7.
\item rarely, affected individuals with pathogenic variants in CDKN1C,
IGF2, PLAG1, and HMGA2 have been described.
\item approximately 40\% of individuals who meet NH-CSS clinical criteria
for SRS have negative molecular and/or cytogenetic testing.
\end{itemize}
\end{itemize}

\begin{figure}[htbp]
\centering
\includegraphics[width=0.5\textwidth]{./figures/rss.png}
\caption{\label{fig:org9dc802a}
11p Duplication in RSS}
\end{figure}

\item Genetic Counseling
\label{sec:org2e439dc}
\begin{itemize}
\item SRS has multiple etiologies and typically has a low recurrence
risk.
\item In most families, a proband with SRS represents a simplex case (a single affected family member) and has
SRS as the result of an apparent \emph{de novo} epigenetic or genetic alteration
\begin{itemize}
\item loss of paternal methylation at the 11p15 ICR1 H19/IGF2 imprinting center 1 or
\item maternal uniparental disomy for chromosome 7.
\end{itemize}
\item SRS may also occur as the result of a genetic alteration associated with up to a 50\% recurrence risk
\begin{itemize}
\item copy number variant on chromosome 7 or 11 or
\item an intragenic pathogenic  variant in CDKN1C, IGF2, PLAG2, or HMGA2
\end{itemize}
\item Accurate assessment of SRS recurrence therefore requires
identification of the causative genetic mechanism in the proband.
\end{itemize}
\end{enumerate}

\section{Neurogenetics}
\label{sec:org8cfa8ce}
\subsection{{\bfseries\sffamily TODO} Rett syndrome}
\label{sec:orgb32ed10}
\subsection{{\bfseries\sffamily TODO} Charcot-Marie-Tooth disease type IA}
\label{sec:org5fa0b1d}
\subsection{{\bfseries\sffamily TODO} Hereditary neuropathy with pressure palsies}
\label{sec:org30380fb}
\subsection{{\bfseries\sffamily TODO} Spinal muscular atrophy}
\label{sec:org8c6bcc9}
\end{document}