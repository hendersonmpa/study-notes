% Created 2019-12-10 Tue 18:06
% Intended LaTeX compiler: pdflatex
\documentclass{scrartcl}
\usepackage[utf8]{inputenc}
\usepackage[T1]{fontenc}
\usepackage{graphicx}
\usepackage{grffile}
\usepackage{longtable}
\usepackage{wrapfig}
\usepackage{rotating}
\usepackage[normalem]{ulem}
\usepackage{amsmath}
\usepackage{textcomp}
\usepackage{amssymb}
\usepackage{capt-of}
\usepackage{hyperref}
\hypersetup{colorlinks,linkcolor=black,urlcolor=blue}
\usepackage{textpos}
\usepackage{textgreek}
\usepackage[version=4]{mhchem}
\usepackage{chemfig}
\usepackage{siunitx}
\usepackage{gensymb}
\usepackage[usenames,dvipsnames]{xcolor}
\usepackage[T1]{fontenc}
\usepackage{lmodern}
\usepackage{verbatim}
\usepackage{tikz}
\usepackage{wasysym}
\usetikzlibrary{shapes.geometric,arrows,decorations.pathmorphing,backgrounds,positioning,fit,petri}
\usepackage{fancyhdr}
\pagestyle{fancy}
\author{Matthew Henderson, PhD, FCACB}
\date{\today}
\title{Disorders of Lipid and Bile Metabolism}
\hypersetup{
 pdfauthor={Matthew Henderson, PhD, FCACB},
 pdftitle={Disorders of Lipid and Bile Metabolism},
 pdfkeywords={},
 pdfsubject={},
 pdfcreator={Emacs 26.1 (Org mode 9.1.9)}, 
 pdflang={English}}
\begin{document}

\maketitle
\tableofcontents


\section{Lipoprotein Metabolism}
\label{sec:org99346df}
\subsection{Introduction}
\label{sec:org89aad81}
\begin{itemize}
\item VLDL are large particles produced in the liver.
\begin{itemize}
\item transport endogenously synthesized TG and cholesterol to the peripheral tissue
\end{itemize}
\item IDL are created with the metabolism of VLDL by lipoprotein lipase
\begin{itemize}
\item may be removed by the liver or converted LDL by hepatic triglyceride lipase
\end{itemize}
\end{itemize}
\begin{enumerate}
\item LDL
\label{sec:org55879fc}
\begin{itemize}
\item LDL contain \textasciitilde{}45\% cholesterol and they are the major carrier of
cholesterol to peripheral tissues
\item LDL particles are heterogeneous:
\begin{itemize}
\item small dense LDL particles have been associated with increased risk
for cardiovascular disease
\item \(\uparrow\) small dense LDL associated with male gender and diabetes in adults
\end{itemize}
\item LDL is recognized by specific LDL receptors
\begin{itemize}
\item highly expressed in the liver
\end{itemize}
\item once bound to receptor are internalized into the cell
\item removes \textasciitilde{}75\% LDL particles
\item remaining LDL are removed by macrophages
\end{itemize}
\item HDL
\label{sec:org594c364}
\begin{itemize}
\item HDL are produced by the liver and the gastrointestinal tract, and
peripheral catabolism of chylomicrons and VLDL particles
\item HDL particles are also heterogeneous:
\begin{itemize}
\item HDL2 is associated with protection from atherosclerosis
\item HDL3 is a smaller particle
\begin{itemize}
\item \(\uparrow\) in alcohol consumption, obesity, diabetes, cigarette
smoking, uraemia and hypertriglyceridemia
\end{itemize}
\item HDL particles are involved in reverse transport of free
cholesterol from peripheral tissues to the liver
\end{itemize}
\end{itemize}
\item Lp(a)
\label{sec:orgb637caa}
\begin{itemize}
\item Lp(a) has also been found to be associated with the atherosclerotic
process
\item Lp(a) is similar to LDL, but with the addition of a large "little a"
protein bound to apoB via a single cysteine-mediated disulphide
bond
\item plasma levels are regulated independently from LDL
\item risk of coronary heart disease is \Uparrowcreased if both LDL and
Lp(a) are elevated
\item the (a) protein is similar to plasminogen, may inhibit the
thrombolytic activity of plasminogen \(\to\) \(\uparrow\) atherosclerosis
\item Inborn errors of lipoprotein metabolism are a group of genetic
disorders exemplified by changes in plasma lipids due to defects in:
\begin{itemize}
\item the protein lipid-carriers (lipoproteins)
\item lipoprotein receptors
\item enzymes responsible for the hydrolysis and clearance of
lipoprotein-lipid complexes
\end{itemize}
\item \textbf{heterozygous familial hypercholesterolaemia is the most common}
\textbf{inherited lipid disorder with a prevalence of 1 in 500}
\end{itemize}

\begin{figure}[htbp]
\centering
\includegraphics[width=1.0\textwidth]{./lipoprotein/figures/lipid_met.png}
\caption{\label{fig:orge80b505}
Lipid and Lipoprotein Metabolism}
\end{figure}
\end{enumerate}

\subsection{LDL Metabolism Disorders}
\label{sec:orgb5c16e0}
\begin{itemize}
\item atherosclerotic process begins in childhood
\item extent and rate of progression has \(\propto\) lipid levels
\item healthy lifestyles and a lower saturated fat intake is the
cornerstone of treatment of lipid disorders in children
\item lipid-lowering therapies are important

\item Five genetic disorders:
\begin{enumerate}
\item familial hypercholesterolaemia (FH) - LDL receptor
\begin{itemize}
\item heterozygous and more severe homozygous forms
\end{itemize}
\item familial ligand defective apo-B (FLDB)
\item autosomal recessive hypercholesterolaemia
\item sitosterolemia
\item proprotein convertase substilisin-like kexin type 9 (PCSK9)
\end{enumerate}
\item \(\Uparrow\) risk of premature atherosclerosis
\item \(\Uparrow\) risk of CAD in adulthood
\end{itemize}

\begin{table}[htbp]
\caption{\label{tab:org2ce3de0}
Selected disorders affecting low density lipoprotein metabolism}
\centering
\begin{tabular}{lllll}
Disorder & Int & Protein & Profile & Freq\\
\hline
AD Familial hypercholesterolemia & AD & LDLR (het) & \(\uparrow\) LDL & 1 in 250-500\\
AR Familial hypercholesterolemia & AR & LDLR (homo/comp het) & \(\uparrow\) LDL & 1 in 10\(^{\text{6}}\)\\
AD Familial hypercholesterolemia & AD & PCSK9 & \(\uparrow\) LDL & \\
Familial ligand defective apo-B, FLBD & AD & Apo-B & \(\uparrow\) LDL & 1 in 500\\
\end{tabular}
\end{table}

\subsection{Triglyceride Metabolism}
\label{sec:org5545a98}
\begin{itemize}
\item may present early in childhood with faltering growth,
hepatosplenomegaly and life threatening pancreatitis
\item TG \textgreater{} 10 mmol/L is rare in children
\begin{itemize}
\item associated with lipoprotein lipase or apoCII defects
\item apoCII activates LPL
\item TG as high as 250 mmol/L possible
\item \(\to\) pancreatitis and eruptive xanthomas
\end{itemize}
\item low fat diet (<10\% fat)
\item Glybera - LPL gene therapy
\item Familial lipoprotein lipase deficiency - Type-I hyperlipidaemia
\begin{itemize}
\item consanguinity 1 in 1 million (homozygous)
\item founder effect in French Canadian population in Quebec
\begin{itemize}
\item carrier frequency 1 in 40
\end{itemize}
\end{itemize}
\end{itemize}
\subsection{HDL Metabolism}
\label{sec:orgbca5ed2}
\begin{itemize}
\item Disorders of HDL are very rare
\item 3 AR inherited disorders described:
\begin{enumerate}
\item Apolipoprotein A-1 deficiency
\item familial hypoalphalipoproteinaemia (Tangier’s disease)
\item lecithin:cholesterol acyltranferase (LCAT) deficiency
\end{enumerate}
\item \(\downarrow\) HDL and Apo A-I
\item \(\uparrow\) cholesterol and triglycerides
\item premature atherosclerosis
\end{itemize}
\subsection{Sterol Storage}
\label{sec:org13c166b}
\begin{itemize}
\item Lysosomal Acid Lipase (LAL) Deficiency is a lysosomal storage
disorder includes:
\begin{itemize}
\item acute infantile onset form - Wolman disease
\begin{itemize}
\item extreme faltering growth, malabsorption, hepatosplenomegaly,
adrenal calcification and death in early infancy
\end{itemize}
\item cholesteryl ester storage disease (CESD) presenting in childhood/adulthood
\begin{itemize}
\item slow progression w hepatosplenomegaly and microvesicular
cirrhosis, premature atherosclerosis and hypercholesterolaemia
(\(\uparrow\) LDL-C, \(\downarrow\) HDL-C)
\end{itemize}
\end{itemize}
\item accumulation of cholesteryl ester in the lysosomes is secondary to a
deficiency of an esterase that is responsible for hydrolysis of
esterified cholesterol in the normal lysosome
\item Sebelipase alfa (recombinant LAL) is licensed for Wolman disease and
CESD
\end{itemize}
\section{Isoprenoid and Cholesterol Synthesis}
\label{sec:org828e977}
\subsection{Introduction}
\label{sec:org322c5c0}
\begin{itemize}
\item isoprenoids function in a variety of cellular processes including:
\begin{itemize}
\item cell growth and differentiation, protein glycosylation, signal
transduction pathways
\end{itemize}
\item isoprenoid synthesis starts from acetyl-CoA
\item in six enzyme reactions is converted into isopentenyl-PP
\item the basic isoprene unit used for the synthesis of subsequent
isoprenoids
\begin{itemize}
\item first committed intermediate in sterol isoprenoid synthesis is
squalene
\item cyclisation \(\to\) lanosterol
\item lanosterol \(\to\) cholesterol may occur via two major routes involving
the same enzymes
\item depending on the timing of reduction of the \(\Delta^{\text{24}}\) double bond either:
\item 7-dehydrocholesterol
\item desmosterol
\end{itemize}
\end{itemize}


\begin{figure}[htbp]
\centering
\includegraphics[width=1.0\textwidth]{./iso_chol/figures/iso_chol_synth.png}
\caption{\label{fig:org88160a6}
Isoprenoid/cholesterol synthesis pathway}
\end{figure}

\subsection{Mevalonate Kinase Deficiency}
\label{sec:org9eacc02}
\begin{enumerate}
\item Clinical Presentation
\label{sec:orgba1a33d}
\begin{itemize}
\item autoinflammatory metabolic disorder
\item onset 1st year of life, often triggered by childhood vaccinations
\item lifelong episodes of fever and inflammation triggered by stress
\item episodes last 3-7 days, recur on average every 4-6 weeks associated with:
\begin{itemize}
\item abdominal pain
\item vomiting and diarrhoea,
\item (cervical) lymphadenopathy, hepatosplenomegaly, arthralgia skin rash
\end{itemize}
\item Two recognized forms in a disease spectrum:
\begin{itemize}
\item classic mevalonic aciduria (MKD-MA) - severe
\item hyper-IgD and periodic fever syndrome (MKD-HIDS) - mild
\end{itemize}

\item MKD-MA patients can present with additional congenital anomalies
such as:
\begin{itemize}
\item mental retardation, ataxia, cerebellar atrophy, hypotonia, severe
FTT, dysmorphic features, high risk of death in early infancy
\end{itemize}
\end{itemize}

\item Metabolic Derangement
\label{sec:orged524ff}
\begin{itemize}
\item mevalonate kinase deficiency (Figure \ref{fig:org88160a6})
\end{itemize}
\ce{mevalonate + ATP ->[MK] 5-phosphomevalonate + ADP}
\begin{itemize}
\item synthesis of all isoprenoids will be affected
\item clinical presentation \(\propto\) residual MK activity
\begin{itemize}
\item activity < LOD in MKD-MA
\item 1-10\% in MKD-HIDS
\end{itemize}
\end{itemize}

\item Genetics
\label{sec:org540159d}
\begin{itemize}
\item AR, MVK
\item most patients are compound heterozygotes
\end{itemize}

\item Diagnostic tests
\label{sec:org0aab106}
\begin{itemize}
\item \(\uparrow\) mevalonic acid in urine organic acids
\begin{itemize}
\item found in MKD-MA not always MKD-HIDS
\end{itemize}
\item best diagnostic tests are:
\begin{itemize}
\item WBC or fibroblasts MK activity
\item molecular analysis of MVK
\end{itemize}
\end{itemize}

\item Treatment and Prognosis
\label{sec:orgf17d2f9}
\begin{itemize}
\item no treatment
\item many MVK-MA die in infancy
\item some MVK-HIDS improve with immunosuppression
\end{itemize}
\end{enumerate}

\subsection{Smith-Lemli-Opitz Syndrome}
\label{sec:org1a6a11f}
\begin{enumerate}
\item Clinical Presentation
\label{sec:org799605e}
\begin{itemize}
\item large and variable spectrum of morphogenic and congenital anomalies
\item clinical and biochemical continuum ranging from hardly recognisable
to very severe (lethal in utero)
\item most affected patients have a characteristic craniofacial appearance
\item cutaneous syndactyly of the second and third toes (>97\% of cases)
\item genital abnormalities may include hypospadias cryptorchidism and
ambiguous male genitalia
\item congenital heart defects, and renal, adrenal, lung and
gastrointestinal anomalies
\end{itemize}

\item Metabolic Derangement
\label{sec:org9037a6b}
\begin{itemize}
\item 7-dehydrocholesterol reductase deficiency
\begin{itemize}
\item catalyses the predominant final step in cholesterol biosynthesis
\item reduce 7-dehydrocholesterol C7-C8 double bond \(\to\) cholesterol
\end{itemize}
\end{itemize}

\item Genetics
\label{sec:orgf8f6005}
\begin{itemize}
\item AR, DHCR7, 1:15,000-60,000
\item \textbf{most frequency cholesterol biosynthesis defect}
\end{itemize}

\item Diagnostic Tests
\label{sec:org37d1cc4}
\begin{itemize}
\item sterol analysis of plasma or tissues of patients by GC-MS,
\begin{itemize}
\item \(\uparrow\) 7-dehydrocholesterol and 8-dehydrocholesterol are diagnostic
\end{itemize}
\item plasma cholesterol is usually low or low normal
\end{itemize}

\item Treatment and Prognosis
\label{sec:orga55f28a}
\begin{itemize}
\item most anomalies occurring in SLOS are due to the unavailability of
sufficient cholesterol during (early) embryonic development
\begin{itemize}
\item \(\therefore\) postnatal therapy not feasible
\end{itemize}
\item sterol supplementation tired w disappointing results
\end{itemize}
\end{enumerate}
\section{Bile Acid Synthesis}
\label{sec:org50fd8de}
\subsection{Introduction}
\label{sec:orgb1e2353}

\begin{itemize}
\item bile acids are biological detergents synthesised from cholesterol
in the liver by modifications of the sterol nucleus and oxidation of
the side chain
\item synthesis can occur by a number of pathways
\begin{itemize}
\item adults starts with conversion of cholesterol to 7\(\alpha\)-hydroxycholesterol
\item infants other pathways are more important
\begin{itemize}
\item one starts with the conversion of cholesterol to 27-hydroxycholesterol
\end{itemize}
\end{itemize}
\item two IEMs affect the modifications of cholesterol nucleus in both pathways:
\begin{enumerate}
\item 3\(\beta\)-hydroxy-\(\Delta^{\text{5}}\)-C\(_{\text{27}}\)-steroid dehydrogenase (3\(\beta\)-dehydrogenase) deficiency
\item \(\Delta\)4-3-oxosteroid 5\(\beta\)-reductase (5\(\beta\)-reductase) deficiency
\end{enumerate}
\item these disorders \(\to\) cholestatic liver disease and malabsorption of
fat and fat-soluble vitamins
\item onset of symptoms \textasciitilde{} 1st year of life
\item untreated, the liver disease can progress to cirrhosis and liver failure
\item treatment with chenodeoxycholic acid and cholic acid
\begin{itemize}
\item dramatic improvement in liver disease and malabsorption
\end{itemize}

\item two disorders affecting oxidation of the cholesterol side chain:
\begin{itemize}
\item sterol 27-hydroxylase deficiency (cerebrotendinous xanthomatosis [CTX])
\item \(\alpha\)-methylacyl-CoA racemase deficiency
\end{itemize}
\item can present with neonatal cholestatic liver disease
\item commonly present later with neurological disease
\item chenodeoxycholic acid has been shown to halt or even reverse
neurological dysfunction in CTX
\item part of a growing list of defects that may present with transient
neonatal cholestatic jaundice followed by a late onset
neurodegenerative disorder
\item likely that the 27-hydroxycholesterol pathway is important in
fuelling bile flow in infancy and production and metabolism of
important oxysterols in the brain later in life
\item other inborn errors of bile acid synthesis include:
\begin{itemize}
\item two bile acid amidation defects
\begin{itemize}
\item cholestatic liver disease and fat-soluble vitamin malabsorption
\end{itemize}
\item cholesterol 7\(\alpha\)-hydroxylase deficiency
\begin{itemize}
\item adults with hyperlipidaemia and gallstones
\end{itemize}
\end{itemize}
\item in disorders of peroxisome biogenesis and peroxisomal
\(\beta\)-oxidation, neurological disease usually predominates
\item a disorder affecting peroxisomal import of CoA esters of DHCA and
THCA produces predominantly liver disease
\item most of the known enzyme deficiencies of bile acid synthesis affect
both the 27-hydroxycholesterol and the 7\(\alpha\)-hydroxycholesterol pathways
\begin{itemize}
\item exceptions are cholesterol 7\(\alpha\)-hydroxylase deficiency and
oxysterol 7\(\alpha\)-hydroxylase deficiency
\end{itemize}
\item because of the broad specificity of many of the enzymes, the major
metabolites are often not those immediately proximal to the block
\item 3\(\beta\)-hydroxy \(\Delta^{\text{5}}\)-C\(_{\text{27}}\)-steroid dehydrogenase deficiency the
major metabolite is not 7\(\alpha\)-hydroxycholesterol
\begin{itemize}
\item \(\to\) unsaturated bile acids with normal bile acid side chain but
persistence of the 3\(\beta\), 7\(\alpha\)-dihydroxy-\(\Delta^{\text{5}}\) structure of
the nucleus
\end{itemize}
\end{itemize}

\begin{figure}[htbp]
\centering
\includegraphics[width=1.0\textwidth]{./bile/figures/bile_synth.png}
\caption{\label{fig:orge2bffc2}
Major reactions the synthesis of bile acids from cholesterol}
\end{figure}


\subsection{3\(\beta\)-dehydrogenase deficiency}
\label{sec:org7bea7c6}
\begin{enumerate}
\item Clinical presentation
\label{sec:org50b3193}
\begin{itemize}
\item present with
\begin{itemize}
\item neonatal conjugated hyperbilirubinaemia
\item rickets
\item hepatomegaly
\item pruritus
\item steatorrhoea
\item failure to thrive
\item fat-soluble vitamin malabsorption
\begin{itemize}
\item \(\downarrow\) 25-OH vitamin D, K and E
\item prolonged prothrombin time
\end{itemize}
\end{itemize}

\item untreated \(\to\) death from complications of cirrhosis before the age
of 5 years
\item patients with milder forms of the disorder may survive, with a
chronic hepatitis or even remain asymptomatic, into their second
decade or beyond.
\end{itemize}
\item Metabolic Derangement
\label{sec:org4436f3a}
\begin{itemize}
\item 3\(\beta\)-dehydrogenase catalyses the second reaction in the major
pathway of synthesis of bile acid (Figure \ref{fig:orge2bffc2},enyzyme 2)
\item 7\(\alpha\)-hydroxycholesterol \(\to\) 7\(\alpha\)-hydroxycholest-4-en-3-one
\item accumulating 7\(\alpha\)-hydroxycholesterol can undergo side-chain
oxidation with or without 12\(\alpha\)-hydroxylation to produce
\begin{itemize}
\item 3\(\beta\),7\(\alpha\)-dihydroxy-5-cholenoic acid
\item 3\(\beta\),7\(\alpha\),12\(\alpha\)-trihydroxy-5-cholenoic acid
\end{itemize}
\item these unsaturated C\(_{\text{24}}\) bile acids are sulphated in the C3 position
\begin{itemize}
\item a proportion is conjugated to glycine
\item can be found in high concentrations in the urine
\end{itemize}
\item probable that the sulphated \(\Delta^{\text{5}}\) bile acids cannot be secreted
into the bile canaliculi and fuel bile flow in the same way as
occurs with the normal bile acids
\begin{itemize}
\item probably inhibit bile acid-dependent bile flow
\end{itemize}
\item two possible ways this \(\to\) hepatocyte damage:
\begin{enumerate}
\item abnormal toxic metabolites
\item failure of bile acid-dependent bile flow
\end{enumerate}
\end{itemize}

\item Genetics
\label{sec:orgecd75df}
\begin{itemize}
\item AR, HSD3B7
\end{itemize}

\item Diagnostic Tests
\label{sec:orged3f670}
\begin{itemize}
\item characteristic plasma or urine bile acids with a
\begin{itemize}
\item \(\Delta^{\text{5}}\) double bond
\item 3\(\beta\)-hydroxyl/sulphate group
\item 7\(\alpha\)-hydroxyl group
\end{itemize}
\item bile acids with a \(\Delta^{\text{5}}\) double bond and a 7-hydroxy group are acid labile.
\item FAB-MS or ESI-MS/MS analysis overcomes this problem
\end{itemize}

\begin{enumerate}
\item Plasma
\label{sec:orgd5741e4}
\begin{itemize}
\item profile of non-sulphated bile acids by FAB-MS, ESI-MS/MS, GC-MS w/o solvolysis:
\begin{itemize}
\item \(\Downarrow\) cholic and chenodeoxycholic acid for an infant with cholestasis
\item \(\uparrow\) 3\(\beta\),7\(\alpha\)-dihydroxy-5-cholestenoic acid
\end{itemize}
\item FAB-MS, ESI-MS/MS, GC-MS w solvolysis:
\begin{itemize}
\item \(\uparrow\) 3\(\beta\),7\(\alpha\)-dihydroxy-5-cholenoic acid (3-sulphate)
\item 3\(\beta\),7\(\alpha\),12\(\alpha\)-trihydroxy-5-cholenoic acid (3-sulphate)
\end{itemize}
\end{itemize}

\item Urine
\label{sec:orge52d0f1}
\begin{itemize}
\item negative ion FAB-MS or ESI-MS shows the characteristic ions of the
diagnostic
\begin{itemize}
\item unsaturated bile acids
\item sulphated \(\Delta^{\text{5}}\) bile acids
\item glycine conjugates of sulphated \(\Delta^{\text{5}}\) bile acids
\end{itemize}
\end{itemize}

\item Fibroblasts
\label{sec:org4124cc3}
\begin{itemize}
\item \(\downarrow\) 3\(\beta\)-Dehydrogenase activity cultured skin fibroblasts using
tritiated 7\(\alpha\)-hydroxycholesterol
\end{itemize}
\end{enumerate}

\item Treatment and Prognosis
\label{sec:orgf1dbf75}
\begin{itemize}
\item emergency treatment of coagulopathy with parenteral vitamin K may be required
\item long term bile acid replacement therapy corrects all the fat-soluble
vitamin deficiencies

\item cholic and chenodeoxycholic acid therapy \(\to\) improvement in symptoms
\end{itemize}
\end{enumerate}

\subsection{5\(\beta\)-reductase deficiency}
\label{sec:orge36784a}
\begin{itemize}
\item 5\(\beta\)-reductase deficiency (Figure \ref{fig:orge2bffc2}, enzyme 3)
\item excrete 3-oxo-\(\Delta^{\text{4}}\) bile acids as the major urinary bile acids
\begin{itemize}
\item \(\uparrow\) 7\(\alpha\)-hydroxy-3-oxo-4-cholenoic acid glycine conjugate
\item \(\uparrow\) 7\(\alpha\),12\(\alpha\)-dihydroxy-4-cholenoic acid glycine conjugate
\end{itemize}
\item 8 patients
\item treat w chenodeoxycholic acid plus cholic acid
\end{itemize}
\subsection{Cerebrotendinous xanthomatosis}
\label{sec:orgdb5e742}
\begin{enumerate}
\item Clinical Presentation
\label{sec:org14efa9e}
\begin{itemize}
\item average age of diagnosis is 35 years w a diagnostic delay of 16 years
\begin{itemize}
\item described as a pediatric disease diagnosed in adulthood
\end{itemize}
\item signs and symptoms include:
\begin{itemize}
\item adult-onset progressive neurological dysfunction
\item non-neurologic manifestations
\begin{itemize}
\item tendon xanthomas
\item premature atherosclerosis
\item osteoporosis
\item respiratory insufficiency
\end{itemize}
\end{itemize}
\end{itemize}

\item Metabolic Derangement
\label{sec:org3622b83}
\begin{itemize}
\item sterol 27-hydroxylase deficiency (Figure \ref{fig:orge2bffc2}, enzyme 4)
\item mitochondrial catalyses first step inside-chain oxidation.
\begin{itemize}
\item required to convert a C27 sterol into a C24 bile acid
\end{itemize}
\item 5\(\beta\)-cholestane-3\(\alpha\),7\(\alpha\),12\(\alpha\)-triol cannot be hydroxylated in the C\(_{\text{27}}\)
position and accumulates in the liver
\begin{itemize}
\item products of secondary reactions also accumulate
\item converted to cholestanol
\end{itemize}
\item reduced rate of bile-acid synthesis
\begin{itemize}
\item \(\therefore\) the normal feedback inhibition of cholesterol
7\(\alpha\)-hydroxylase by bile acids is disrupted (Figure \ref{fig:orge2bffc2}, enzyme 1)
\end{itemize}
\item symptoms partly due to accumulation of cholestanol and cholesterol
\item lack of 3\(\beta\),7\(\alpha\)-dihydroxy-5-cholestenoic acid may contribute to motor
neuron damage
\end{itemize}

\item Genetics
\label{sec:org4cbd674}
\begin{itemize}
\item AR, CYP27A1
\end{itemize}

\item Diagnostic Tests
\label{sec:org014cd17}
\begin{itemize}
\item molecular
\end{itemize}
\begin{enumerate}
\item Plasma
\label{sec:orgb911f3e}
\begin{itemize}
\item \(\uparrow\) cholestanol by GC or HPLC
\item \(\uparrow\) cholestanol/cholesterol ratio
\item \(\downarrow\) 7-hydroxycholesterol
\end{itemize}
\item Urine
\label{sec:org952af9a}
\begin{itemize}
\item major cholanoids are cholestanepentol glucuronides by FAB-MS or ESI-MS/MS
\end{itemize}
\end{enumerate}

\item Treatment and Prognosis
\label{sec:org5b231c4}
\begin{itemize}
\item Chenodeoxycholic acid
\end{itemize}
\end{enumerate}
\subsection{\(\alpha\)-Methylacyl-CoA racemase deficiency}
\label{sec:org6a8d0ea}

\begin{enumerate}
\item Clinical Presentation
\label{sec:orgddf803a}
\begin{itemize}
\item neurological problems start from childhood to late adult life and
include:
\begin{itemize}
\item mental delay, cognitive decline
\item acute encephalopathy
\item tremor, ataxia
\item pigmentary retinopathy
\item hemiparesis, spastic paraparesis, peripheral neuropathy
\item depression, headache
\end{itemize}
\end{itemize}

\item Metabolic Derangement
\label{sec:org2e6dc52}
\begin{itemize}
\item \(\alpha\)-methylacyl-CoA racemase deficiency (Figure \ref{fig:orge2bffc2}, enzyme 5)
\item side-chain oxidation of cholesterol produces:
\begin{itemize}
\item 25R isomer of 3\(\alpha\),7\(\alpha\),12\(\alpha\)-trihydroxycholestanoyl-CoA [(25R)-THC-CoA]
\end{itemize}
\item \(\alpha\)-oxidation of dietary phytanic acid produces (some):
\begin{itemize}
\item (2R)-pristanoyl-CoA
\end{itemize}
\item these need to be converted to the S-isomers by AMACR before they can
undergo peroxisomal \(\beta\)-oxidation
\end{itemize}

\item Genetics
\label{sec:orgc0c6e61}
\begin{itemize}
\item AR, AMACR
\end{itemize}

\item Diagnostic Tests
\label{sec:org2d02332}
\begin{itemize}
\item \(\uparrow\) plasma DHCA and THCA by GC-MS
\item \(\uparrow\) pristanic acid
\item \(\uparrow\)/n plasma phytanic acid
\item normal VLFCA
\end{itemize}

\item Treatment and Prognosis
\label{sec:org25a748a}
\begin{itemize}
\item vitamin K
\item cholic acid
\item phytanic acid
\end{itemize}
\end{enumerate}
\section{Triglyceride and Phospholid Metabolism}
\label{sec:org0ad3c43}
\subsection{Introduction}
\label{sec:org84b9234}
\begin{itemize}
\item acylglycerols and phospholipids (PL) play a myriad of organ-specific
roles in cell structure, biochemistry and signalling
\item inborn errors of glycerolipid metabolism cause a correspondingly
vast array of clinical phenotypes
\item molecular analysis is currently the principal diagnostic
technique
\item NGS \(\to\) discovery of several genetic defects of the biosynthesis and
remodelling of triglycerides (TGs), other acyl-glycerols and of PLs
\item few disorders of glycerolipid metabolism have distinct metabolite
patterns by conventional techniques
\item some of these conditions produce striking clinical syndromes
identifiable in infancy
\item others are mildly atypical forms of common adult conditions like:
\begin{itemize}
\item metabolic syndrome
\item type II diabetes
\end{itemize}
\item inborn errors of intracellular TG metabolism often affect adipose
tissue, but non-adipose signs can dominate their clinical
presentation
\item inborn errors of PL metabolism often affect the central and
peripheral nervous systems, but also muscle, eye, skin, bone,
cartilage, liver, kidney and immune system
\end{itemize}

\begin{figure}[htbp]
\centering
\includegraphics[width=1.0\textwidth]{./tg_pl/figures/tg.png}
\caption{\label{fig:orgf90a4d9}
The common pathway, triglyceride synthesis and lipolysis}
\end{figure}


\begin{itemize}
\item PLs form the bilayer of cell membranes
\begin{itemize}
\item a phosphate group that constitutes a hydrophilic head, located on the surface of the bilayer
\item a fatty acid chain that forms a hydrophobic tail, located inside
\end{itemize}
\item TGs are extremely hydrophobic, and form droplets
that are surrounded by a PL monolayer
\item the first steps of TG and of PL synthesis are the same
\item \emph{de novo} synthesis begins from glycerol-3-phosphate (G3P), which is
formed from glycerol or from dihydroxyacetone phosphate (DHAP)
\begin{itemize}
\item DHAP is an intermediate of both glycolysis and of gluconeogenesis
\end{itemize}
\item G3P is esterified twice:
\begin{enumerate}
\item glycerolphosphate acyltransferases (GPATs) to form lysophosphatidic acid (LPA)
\item acylglycerol acyltransferases (AGPATs) to form phosphatidic acid (PA)
\end{enumerate}
\item PA can either be converted into
\begin{itemize}
\item an sn-1,2-diacylglycerol (DG) by phosphatidic acid phosphatases or
\item CDP-diacylglycerol by phosphatidic acid cytidyltransferase (CDP-DAG synthase)
\end{itemize}
\item DG is an essential intermediate for synthesis of TGs and three major PLs:
\begin{enumerate}
\item phosphatidyl choline
\item phosphatidyl ethanolamine
\item phosphatidyl serine
\end{enumerate}
\end{itemize}

\subsection{TG biosynthesis and lipolysis}
\label{sec:orgfeb6b7c}
\begin{itemize}
\item phosphatidic acid phosphatases (PAPs, LPIN1-3) catalyse phosphate
removal, the first dedicated step of TG synthesis, creating a sn-1,2-diacylglycerol
\item one of two diacylglycerol acyltransferases (DGATs) can catalyze TG synthesis
\item TGs accumulate between the leaflets of the ER lipid bilayer
\item budding of lipid droplets (LDs) requires seipin
\item adipocyte lipolysis is activated during fasting to export fatty
acids (FAs) and glycerol for fuel
\item The main TG lipase in adipocytes, adipocyte triglyceride lipase
(ATGL) is on the LD surface
\item DGs are hydrolyzed by hormone-sensitive lipase (HSL), producing
mainly sn-1-MGs
\item Monoacylglyceride hydrolase(s) liberate glycerol and a FA
\item Perilipins are found on the surface of all LDs
\item Perilipin 1 suppresses lipolysis in the fed state; beta adrenergic
stimulation and low circulating insulin levels favor PLIN1 and HSL
phosphorylation by protein kinase A (PKA), activating lipolysis
\end{itemize}

\subsection{Phospholipid biosynthesis and remodelling}
\label{sec:orgcafbe7c}
\begin{figure}[htbp]
\centering
\includegraphics[width=1.0\textwidth]{./tg_pl/figures/pl.png}
\caption{\label{fig:orgb75d0b7}
Phospholipid biosynthesis (top of the figure) and remodelling (bottom of the figure)}
\end{figure}


\begin{enumerate}
\item Inborn Errors of Phospholipid Biosynthesis
\label{sec:orgfdfc599}
\begin{itemize}
\item Seven inherited disorders of de novo PL synthesis have been described:
\begin{itemize}
\item three involve proteins located in the endoplasmic reticulum
(Figure \ref{fig:orgb75d0b7} steps 1,2 and 3)
\item four are proteins of the mitochondrial membrane (Figure
\ref{fig:orgb75d0b7} steps 4, 5 and 6)
\end{itemize}
\item distinction between \emph{de novo} synthesis and re-modelling defects is
unclear in some cases
\end{itemize}

\item Inborn Errors related to Phospholipid Remodeling
\label{sec:org51af65f}
\begin{itemize}
\item PLs are a source of bioactive lipids released from
membranes by a large family of enzymes, called phospholipases
\item Diacylglycerol can be formed from membrane phospholipids by
phospholipase C
\begin{itemize}
\item converted by diacylglycerol lipase into 2-arachidonoylglycerol
\item hydrolyzed into arachidonic acid by alpha beta hydrolase 12 (ABHD12)
\end{itemize}
\item PLs also hydrolyzed at the sn-2 position by phospholipases A2, such
as PLA2G6
\item at the sn-1 position by phospholipases A1, such as DDHD1 and DDHD2,
releasing free fatty acid and lysophopholipid
(LysoPC)
\item lysophospholipids are hydrolyzed by lysophos pholipases, such as
PNPLA6 (NTE) into phosphoglycerol and fatty acid
\item archidonic acid released by phospholipases is a precursor of eicosanoids such as
hydroxy eicosatetraenoic acid (19-HETE and 20-HETE)
\item phosphatidylinositol (P-Ins) is a membrane phospholipid composed of
diacylglycerol and a D-myo-inositol head group.
\item inositol ring can be phosphorylated and dephosphorylated by a number
of kinases and phosphatases to yield seven phosphoinositide
derivatives (PI-3P, PI-4P, PI-5P etc\ldots{})
\item arachidonate-rich phosphoinositides are also believed to be another
source of PLA2-mediated arachidonic acid release for the synthesis
of prostaglandins and leukotrienes
\end{itemize}
\end{enumerate}
\end{document}