% Created 2019-11-27 Wed 10:23
% Intended LaTeX compiler: pdflatex
\documentclass{scrartcl}
\usepackage[utf8]{inputenc}
\usepackage[T1]{fontenc}
\usepackage{graphicx}
\usepackage{grffile}
\usepackage{longtable}
\usepackage{wrapfig}
\usepackage{rotating}
\usepackage[normalem]{ulem}
\usepackage{amsmath}
\usepackage{textcomp}
\usepackage{amssymb}
\usepackage{capt-of}
\usepackage{hyperref}
\hypersetup{colorlinks,linkcolor=black,urlcolor=blue}
\usepackage{textpos}
\usepackage{textgreek}
\usepackage[version=4]{mhchem}
\usepackage{chemfig}
\usepackage{siunitx}
\usepackage{gensymb}
\usepackage[usenames,dvipsnames]{xcolor}
\usepackage[T1]{fontenc}
\usepackage{lmodern}
\usepackage{verbatim}
\usepackage{tikz}
\usepackage{wasysym}
\usetikzlibrary{shapes.geometric,arrows,decorations.pathmorphing,backgrounds,positioning,fit,petri}
\usepackage{fancyhdr}
\pagestyle{fancy}
\author{Matthew Henderson, PhD, FCACB}
\date{\today}
\title{Disorders of Lipid and Bile Metabolism}
\hypersetup{
 pdfauthor={Matthew Henderson, PhD, FCACB},
 pdftitle={Disorders of Lipid and Bile Metabolism},
 pdfkeywords={},
 pdfsubject={},
 pdfcreator={Emacs 26.1 (Org mode 9.1.9)}, 
 pdflang={English}}
\begin{document}

\maketitle
\tableofcontents


\section{Lipoprotein Metabolism}
\label{sec:org7834530}
\subsection{Introduction}
\label{sec:orgd860997}
\begin{itemize}
\item VLDL are large particles produced in the liver.
\begin{itemize}
\item transport endogenously synthesized TG and cholesterol to the peripheral tissue.
\end{itemize}
\item IDL are created with the metabolism of VLDL by lipoprotein lipase.
\begin{itemize}
\item may be removed by the liver or converted LDL by hepatic triglyceride lipase.
\end{itemize}
\item LDL contain \textasciitilde{}45\% cholesterol and they are the major carrier of
cholesterol to peripheral tissues.
\item LDL particles are heterogeneous:
\begin{itemize}
\item small dense LDL particles have been associated with increased risk
for cardiovascular disease.
\item \(\uparrow\) small dense LDL associated with male gender and diabetes in adults
\end{itemize}
\item LDL is recognized by specific LDL receptors
\begin{itemize}
\item highly expressed in the liver.
\end{itemize}
\item once bound to receptor are internalized into the cell
\item removes \textasciitilde{}75\% LDL particles
\item remaining LDL are removed by macrophages
\item HDL are produced by the liver and the gastrointestinal tract, and
peripheral catabolism of chylomicrons and VLDL particles.
\item HDL particles are also heterogeneous:
\begin{itemize}
\item HDL2 is associated with protection from atherosclerosis
\item HDL3 is a smaller particle
\begin{itemize}
\item \(\uparrow\) in alcohol consumption, obesity, diabetes, cigarette
smoking, uraemia and hypertriglyceridemia
\end{itemize}
\item HDL particles are involved in reverse transport of free
cholesterol from peripheral tissues to the liver
\end{itemize}
\item Lp(a) has also been found to be associated with the atherosclerotic
process.
\item Lp(a) is similar to LDL, but with the addition of a large "little a"
protein bound to apoB via a single cysteine-mediated disulphide
bond.
\item plasma levels are regulated independently from LDL
\item risk of coronary heart disease is \Uparrowcreased if both LDL and
Lp(a) are elevated
\item the (a) protein is similar to plasminogen, may inhibit the
thrombolytic activity of plasminogen \(\to\) \(\uparrow\) atherosclerosis

\item Inborn errors of lipoprotein metabolism are a group of genetic
disorders exemplified by changes in plasma lipids due to defects in:
\begin{itemize}
\item the protein lipid-carriers (lipoproteins)
\item lipoprotein receptors
\item enzymes responsible for the hydrolysis and clearance of
lipoprotein-lipid complexes
\end{itemize}
\item \textbf{heterozygous familial hypercholesterolaemia is the most common}
\textbf{inherited lipid disorder with a prevalence of 1 in 500}
\end{itemize}

\begin{figure}[htbp]
\centering
\includegraphics[width=0.9\textwidth]{./lipoprotein/figures/lipid_met.png}
\caption{\label{fig:orgd077669}
Lipid and Lipoprotein Metabolism}
\end{figure}


\begin{enumerate}
\item Disorders
\label{sec:org7221d77}
\end{enumerate}

\subsection{LDL Metabolism}
\label{sec:orgc2d9aaf}
\subsection{Triglyceride Metabolism}
\label{sec:org59305c3}
\subsection{HDL Metabolism}
\label{sec:orgb3b48b6}
\subsection{Sterol Storage}
\label{sec:org2e50bfd}
\section{Isoprenoid and Cholesterol Synthesis}
\label{sec:org243dbb4}

\section{Bile Acid Synthesis}
\label{sec:org6d33341}

\section{Triglyceride and Phospholid Metabolism}
\label{sec:org2d71043}
\end{document}