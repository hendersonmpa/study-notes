% Created 2020-06-27 Sat 11:51
% Intended LaTeX compiler: pdflatex
\documentclass[12pt]{scrartcl}
\usepackage[utf8]{inputenc}
\usepackage[T1]{fontenc}
\usepackage{graphicx}
\usepackage{grffile}
\usepackage{longtable}
\usepackage{wrapfig}
\usepackage{rotating}
\usepackage[normalem]{ulem}
\usepackage{amsmath}
\usepackage{textcomp}
\usepackage{amssymb}
\usepackage{capt-of}
\usepackage{hyperref}
\hypersetup{colorlinks,linkcolor=black,urlcolor=blue}
\usepackage{textpos}
\usepackage{textgreek}
\usepackage[version=4]{mhchem}
\usepackage{chemfig}
\usepackage{siunitx}
\usepackage{gensymb}
\usepackage[usenames,dvipsnames]{xcolor}
\usepackage[T1]{fontenc}
\usepackage{lmodern}
\usepackage{verbatim}
\usepackage{tikz}
\usepackage{wasysym}
\usetikzlibrary{shapes.geometric,arrows,decorations.pathmorphing,backgrounds,positioning,fit,petri}
\usepackage[automark, autooneside=false, headsepline]{scrlayer-scrpage}
\clearpairofpagestyles
\ihead{\leftmark}% section on the inner (oneside: right) side
\ohead{\rightmark}% subsection on the outer (oneside: left) side
\ofoot*{\pagemark}% the pagenumber in the center of the foot, also on plain pages
\pagestyle{scrheadings}
\author{Matthew Henderson, PhD, FCACB}
\date{\today}
\title{Disorders of Lipid and Bile Metabolism}
\hypersetup{
 pdfauthor={Matthew Henderson, PhD, FCACB},
 pdftitle={Disorders of Lipid and Bile Metabolism},
 pdfkeywords={},
 pdfsubject={},
 pdfcreator={Emacs 26.3 (Org mode 9.3.7)}, 
 pdflang={English}}
\begin{document}

\maketitle
\tableofcontents


\setchemfig{atom style={scale=0.75}}

\section{Lipoprotein Metabolism}
\label{sec:orgbd257e5}
\subsection{Introduction}
\label{sec:org5f544dc}
\begin{itemize}
\item VLDL are large particles produced in the liver
\begin{itemize}
\item transport endogenously synthesized TG and cholesterol to peripheral tissue
\end{itemize}
\item IDL result from metabolism of VLDL by lipoprotein lipase
\begin{itemize}
\item may be removed by the liver or converted LDL by hepatic triglyceride lipase
\end{itemize}
\end{itemize}
\begin{enumerate}
\item LDL
\label{sec:org401f8cf}
\begin{itemize}
\item LDL contain \textasciitilde{}45\% cholesterol and are the major carrier of
cholesterol to peripheral tissue
\item LDL particles are heterogeneous
\begin{itemize}
\item small dense LDL particles have been associated with increased risk
for cardiovascular disease
\item \(\uparrow\) small dense LDL associated with male gender and diabetes in adults
\end{itemize}
\item LDL is recognized by specific LDL receptors
\begin{itemize}
\item highly expressed in the liver
\item once bound to receptor are internalized into the cell
\item removes \textasciitilde{}75\% LDL particles
\item remaining LDL are removed by macrophages
\end{itemize}
\end{itemize}
\item HDL
\label{sec:org83054b2}
\begin{itemize}
\item HDL are produced by the liver and the gastrointestinal tract, and
peripheral catabolism of chylomicrons and VLDL particles
\item HDL particles are heterogeneous
\begin{itemize}
\item HDL2 is associated with protection from atherosclerosis
\item HDL3 is a smaller particle
\begin{itemize}
\item \(\uparrow\) in alcohol consumption, obesity, diabetes, cigarette
smoking, uraemia and hypertriglyceridemia
\end{itemize}
\end{itemize}
\item HDL particles are involved in reverse transport of free
cholesterol from peripheral tissues to the liver
\end{itemize}
\item Lp(a)
\label{sec:orgbab7bd1}
\begin{itemize}
\item Lp(a) is similar to LDL, but with the addition of the "little a"
protein bound to apoB via a single cysteine-mediated disulphide
bond
\begin{itemize}
\item plasma levels are regulated independently from LDL
\item \(\uparrow\) risk of coronary heart disease if both LDL and Lp(a) are
elevated
\item Lp(a) protein is similar to plasminogen, may inhibit the
thrombolytic activity of plasminogen \(\to\) \(\uparrow\) atherosclerosis
\end{itemize}
\item inborn errors of lipoprotein metabolism are a group of genetic
disorders exemplified by changes in plasma lipids due to defects in:
\begin{itemize}
\item protein lipid-carriers (lipoproteins)
\item lipoprotein receptors
\item enzymes responsible for the hydrolysis and clearance of
lipoprotein-lipid complexes
\end{itemize}
\item \textbf{heterozygous familial hypercholesterolaemia is the most common}
\textbf{inherited lipid disorder with a prevalence of 1 in 500}
\end{itemize}

\begin{figure}[htbp]
\centering
\includegraphics[width=1.0\textwidth]{lipoprotein/figures/lipid_met.png}
\caption{\label{fig:orgaad10ca}Lipid and Lipoprotein Metabolism}
\end{figure}
\end{enumerate}

\subsection{LDL Metabolism}
\label{sec:org2eb7f05}
\begin{itemize}
\item five genetic disorders
\begin{enumerate}
\item familial hypercholesterolaemia (FH) - LDL receptor
\begin{itemize}
\item heterozygous and more severe homozygous forms
\end{itemize}
\item familial ligand defective apo-B (FLDB)
\item autosomal recessive hypercholesterolaemia
\item sitosterolemia
\item proprotein convertase substilisin-like kexin type 9 (PCSK9)
\end{enumerate}
\item see table \ref{tab:org5d3c798}
\item in affected individuals 
\begin{itemize}
\item atherosclerotic process begins in childhood
\begin{itemize}
\item \(\Uparrow\) risk of CAD in adulthood
\end{itemize}
\item extent and rate of progression has \(\propto\) lipid levels
\item healthy lifestyles and a lower saturated fat intake is the
cornerstone of treatment of lipid disorders in children
\item lipid-lowering therapies are important
\end{itemize}
\end{itemize}

\begin{table}[htbp]
\caption{\label{tab:org5d3c798}Selected disorders affecting low density lipoprotein metabolism}
\centering
\begin{tabular}{lllll}
Disorder & Int & Protein & Profile & Freq\\
\hline
AD familial hypercholesterolemia & AD & LDLR (het) & \(\uparrow\) LDL & 1 in 250-500\\
AR familial hypercholesterolemia & AR & LDLR (homo/comp het) & \(\uparrow\) LDL & 1 in 10\textsuperscript{6}\\
AD familial hypercholesterolemia & AD & PCSK9 & \(\uparrow\) LDL & \\
familial apo-B-100 deficiency & A semi-D & Apo-B & \(\uparrow\) LDL & 1 in 500\\
\end{tabular}
\end{table}

\begin{enumerate}
\item Apolipoprotein B
\label{sec:orgbd4f1b3}
\begin{itemize}
\item \textbf{apolipoprotein B} deficiency results in
\begin{itemize}
\item \(\downarrow\) LDL and triglycerides
\end{itemize}

\item familial abetalipoproteinemia
\begin{itemize}
\item microsomal triglyceride transfer protein deficiency
\item \(\downarrow\) production of apoB containing lipoproteins
\item fat malabsorption
\end{itemize}

\item familial hypobetalipoproteinemia
\begin{itemize}
\item apoB dysfunction
\item milder form of abetalipoproteinemia
\end{itemize}
\end{itemize}
\end{enumerate}

\subsection{Triglyceride Metabolism}
\label{sec:org4157404}
\begin{itemize}
\item may present early in childhood with faltering growth,
hepatosplenomegaly and life threatening pancreatitis
\item TG \textgreater{} 10 mmol/L is rare in children
\begin{itemize}
\item associated with \textbf{lipoprotein lipase or apoCII} defects
\item apoCII activates LPL
\item TG as high as 250 mmol/L possible
\item \(\to\) pancreatitis and eruptive xanthomas
\end{itemize}
\item low fat diet (<10\% fat)
\item glybera - LPL gene therapy
\end{itemize}

\begin{enumerate}
\item Familial Lipoprotein Lipase Deficiency
\label{sec:orgfdb1757}
\begin{itemize}
\item Type-I Hyperlipidaemia
\item consanguinity 1 in 1 million (homozygous)
\item founder effect in French Canadian population in Quebec
\begin{itemize}
\item carrier frequency 1 in 40
\end{itemize}
\item see above for clinical and biochemical presentation
\end{itemize}
\end{enumerate}
\subsection{HDL Metabolism}
\label{sec:org3cb7b02}
\begin{itemize}
\item disorders of HDL are very rare
\item three AR inherited disorders described
\begin{enumerate}
\item \textbf{apolipoprotein A-1} deficiency
\begin{itemize}
\item \(\downarrow\) HDL early atherosclerosis
\end{itemize}
\item familial hypoalphalipoproteinaemia (Tangier’s disease)
\begin{itemize}
\item defective \textbf{ABCA1 transporter} prevents cholesterol and
phospholipid transport out of cells for pickup by apoA1 in the
bloodstream
\item polyneuropathy, hepatosplenomegaly, CC, CAD
\end{itemize}
\item \textbf{lecithin:cholesterol acyltranferase (LCAT)} deficiency
\begin{itemize}
\item neuropathy, CC
\end{itemize}
\end{enumerate}
\item \(\downarrow\) HDL and apoAI
\item \(\uparrow\) cholesterol and triglycerides
\item premature atherosclerosis
\end{itemize}

\subsection{Sterol Storage}
\label{sec:org929f0cd}
\begin{itemize}
\item \textbf{lysosomal acid lipase (LAL)} deficiency is a lysosomal storage
disorder includes:
\begin{itemize}
\item acute infantile onset form - Wolman disease
\begin{itemize}
\item extreme faltering growth, malabsorption, hepatosplenomegaly,
adrenal calcification and death in early infancy
\end{itemize}
\item cholesteryl ester storage disease (CESD) presenting in childhood/adulthood
\begin{itemize}
\item slow progression w hepatosplenomegaly and microvesicular
cirrhosis, premature atherosclerosis and hypercholesterolaemia
\end{itemize}
\end{itemize}
\item LAL hydrolyses cholesterol esters in lysosomes
\begin{itemize}
\item \(\uparrow\) cholesteryl ester in lysosomes
\item \(\uparrow\) LDL-C
\item \(\downarrow\) HDL-C
\end{itemize}
\item sebelipase alfa (recombinant LAL) is licensed for Wolman disease and
CESD
\end{itemize}
\section{Isoprenoid and Cholesterol Synthesis}
\label{sec:org9a27012}
\subsection{Introduction}
\label{sec:org4bf6210}
\begin{itemize}
\item isoprenoids function in a variety of cellular processes including
\begin{itemize}
\item cell growth and differentiation, protein glycosylation, signal
transduction pathways
\end{itemize}
\item isoprenoid synthesis starts from acetyl-CoA (Figures \ref{fig:org7fad4de})
\begin{itemize}
\item converted in 6 steps to isopentenyl-PP
\item isopentenyl-PP is the basic isoprene unit used for the synthesis
of subsequent isoprenoids
\item first committed intermediate in sterol isoprenoid synthesis is
squalene
\item undergoes cyclisation \(\to\) lanosterol
\item lanosterol \(\to\) cholesterol occurs via two routes (Figures \ref{fig:org301612a})
\begin{itemize}
\item depending on the timing of reduction of the \(\Delta\)\textsuperscript{24} double bond
\item involve the same enzymes
\item penulimate product is either:
\begin{itemize}
\item 7-dehydrocholesterol
\item desmosterol
\end{itemize}
\end{itemize}
\end{itemize}
\end{itemize}


\begin{figure}[htbp]
\centering
\includegraphics[width=1.0\textwidth]{iso_chol/figures/iso_chol_synth.png}
\caption{\label{fig:org301612a}Isoprenoid/Cholesterol Synthesis Pathway}
\end{figure}

\begin{figure}[htbp]
\centering
\includegraphics[width=1.0\textwidth]{iso_chol/figures/Slide18.png}
\caption{\label{fig:org7fad4de}Isoprenoid/Cholesterol Synthesis Pathway}
\end{figure}

\subsection{Mevalonate Kinase Deficiency}
\label{sec:org021d24c}
\begin{enumerate}
\item Clinical Presentation
\label{sec:orga145a5b}
\begin{itemize}
\item auto-inflammatory metabolic disorder
\item onset 1st year of life, often triggered by childhood vaccinations
\item lifelong episodes of fever and inflammation triggered by stress
\item episodes last 3-7 days, recur on average every 4-6 weeks associated with:
\begin{itemize}
\item abdominal pain
\item vomiting and diarrhoea,
\item (cervical) lymphadenopathy, hepatosplenomegaly, arthralgia skin rash
\end{itemize}
\item two recognized forms in a disease spectrum:
\begin{itemize}
\item classic mevalonic aciduria (MKD-MA) - severe
\item hyper-IgD and periodic fever syndrome (MKD-HIDS) - mild
\end{itemize}

\item MKD-MA patients can present with additional congenital anomalies
such as:
\begin{itemize}
\item mental retardation, ataxia, cerebellar atrophy, hypotonia, severe
FTT, dysmorphic features, high risk of death in early infancy
\end{itemize}
\end{itemize}

\item Metabolic Derangement
\label{sec:org3045ab3}
\begin{itemize}
\item \textbf{mevalonate kinase} deficiency (Figure \ref{fig:org7fad4de})
\end{itemize}
\ce{mevalonate + ATP ->[MK] 5-phosphomevalonate + ADP}
\begin{itemize}
\item synthesis of all isoprenoids will be affected
\item clinical presentation \(\propto\) residual MK activity
\begin{itemize}
\item activity < LOD in MKD-MA
\item 1-10\% in MKD-HIDS
\end{itemize}
\end{itemize}

\item Genetics
\label{sec:org75d219f}
\begin{itemize}
\item AR, MVK
\item most patients are compound heterozygotes
\end{itemize}

\item Diagnostic tests
\label{sec:orge89d060}
\begin{itemize}
\item urine organic acids
\begin{itemize}
\item \(\uparrow\) mevalonic acid
\begin{itemize}
\item found in MKD-MA not always MKD-HIDS
\end{itemize}
\item \(\uparrow\) mevalonolactone
\end{itemize}
\item serum
\begin{itemize}
\item \(\uparrow\) CK
\item \(\uparrow\) IgD
\item \(\uparrow\) transaminases
\end{itemize}
\item best diagnostic tests are:
\begin{itemize}
\item WBC or fibroblasts MK activity
\item molecular analysis of MVK
\end{itemize}
\end{itemize}

\item Treatment and Prognosis
\label{sec:org778fd36}
\begin{itemize}
\item no treatment
\item many MVK-MA die in infancy
\item some MVK-HIDS improve with immunosuppression
\end{itemize}
\end{enumerate}

\subsection{Smith-Lemli-Opitz Syndrome}
\label{sec:orga02bb49}
\begin{enumerate}
\item Clinical Presentation
\label{sec:orgc58fcc7}
\begin{itemize}
\item spectrum of morphogenic and congenital anomalies
\item clinical and biochemical continuum ranging from hardly recognizable
to very severe (lethal in utero)
\item most affected patients have a characteristic craniofacial appearance
\item cutaneous syndactyly of the second and third toes (>97\% of cases)
\item genital abnormalities may include hypospadias, cryptorchidism and
ambiguous male genitalia
\item congenital heart defects, renal, adrenal, lung and gastrointestinal
anomalies
\end{itemize}

\item Metabolic Derangement
\label{sec:orgefdf883}
\begin{itemize}
\item \textbf{7-dehydrocholesterol reductase} deficiency (Figure \ref{fig:org7fad4de})
\begin{itemize}
\item catalyses the predominant final step in cholesterol biosynthesis
\begin{itemize}
\item reduction of 7-dehydrocholesterol(DHC) C7-C8 double bond \(\to\) cholesterol
\end{itemize}
\end{itemize}
\end{itemize}

\item Genetics
\label{sec:org97786dc}
\begin{itemize}
\item AR DHCR7 1:15,000-60,000
\item \textbf{most frequent cholesterol biosynthesis defect}
\end{itemize}

\item Diagnostic Tests
\label{sec:org45d3be8}
\begin{itemize}
\item sterol analysis of plasma or tissues of patients by GC-MS
\begin{itemize}
\item \(\uparrow\) 7-DHC and 8-DHC are diagnostic
\end{itemize}
\item \(\downarrow\)-N plasma cholesterol
\end{itemize}

\item Treatment and Prognosis
\label{sec:org562db6b}
\begin{itemize}
\item most anomalies occurring in SLOS are due to the unavailability of
sufficient cholesterol during early embryonic development
\begin{itemize}
\item \(\therefore\) postnatal therapy not feasible
\end{itemize}
\item sterol supplementation tried w disappointing results
\end{itemize}
\end{enumerate}
\section{Bile Acid Synthesis}
\label{sec:orgcc9f691}
\subsection{Introduction}
\label{sec:org7607cde}
\begin{itemize}
\item bile acids are biological detergents synthesized from cholesterol
in the liver by modifications of the sterol nucleus and oxidation of
the side chain
\item the first reaction (7\(\alpha\)-hydroxylation) is rate limiting
\begin{itemize}
\item some steps involve peroxisomal \(\beta\)-oxidation
\end{itemize}
\item the two main bile acids cholic and chenodeoxycholic
\begin{itemize}
\item are activated to CoA esters
\item conjugated with either glycine or taurine to form bile salts
\item excreted in bile
\end{itemize}
\item bile acids are essential for lipid resorption in the gut
\begin{itemize}
\item regulate hepatic cholesterol synthesis via inhibition of HMG-CoA reductase
\end{itemize}
\item synthesis can occur by a number of pathways (Figure \ref{fig:org27c9053})
\begin{itemize}
\item adults starts with conversion of cholesterol to 7\(\alpha\)-hydroxycholesterol
\item in infants other pathways are more important
\begin{itemize}
\item e.g. conversion of cholesterol to 27-hydroxycholesterol
\end{itemize}
\end{itemize}
\item due to the broad specificity of many of the enzymes, the major
metabolites are often not those immediately proximal to the block
\end{itemize}

\begin{figure}[htbp]
\centering
\includegraphics[width=1.0\textwidth]{bile/figures/bile_synth.png}
\caption{\label{fig:org27c9053}Major reactions the synthesis of bile acids from cholesterol}
\end{figure}

\subsection{3\(\beta\)-Dehydrogenase Deficiency}
\label{sec:orga6e4f02}
\begin{enumerate}
\item Clinical Presentation
\label{sec:orgbae5e8e}
\begin{itemize}
\item presents in childhood with
\begin{itemize}
\item neonatal conjugated hyperbilirubinaemia
\item rickets
\item hepatomegaly
\item pruritus
\item steatorrhoea
\item failure to thrive
\item fat-soluble vitamin malabsorption
\begin{itemize}
\item \(\downarrow\) 25-OH vitamin D, K and E
\item prolonged prothrombin time
\end{itemize}
\end{itemize}

\item untreated \(\to\) death from complications of cirrhosis before the age
of 5 years
\item patients with milder forms of the disorder may survive, with a
chronic hepatitis or even remain asymptomatic, into their second
decade or beyond.
\end{itemize}
\item Metabolic Derangement
\label{sec:org3ae7ba0}
\begin{itemize}
\item 3\(\beta\)-dehydrogenase catalyses the second reaction in the major
pathway of synthesis of bile acid (Figure \ref{fig:org27c9053}, enzyme 2)
\item 7\(\alpha\)-hydroxycholesterol \(\to\) 7\(\alpha\)-hydroxycholest-4-en-3-one
\item accumulating 7\(\alpha\)-hydroxycholesterol can undergo side-chain
oxidation with or without 12\(\alpha\)-hydroxylation to produce
\begin{itemize}
\item 3\(\beta\),7\(\alpha\)-dihydroxy-5-cholenoic acid
\item 3\(\beta\),7\(\alpha\),12\(\alpha\)-trihydroxy-5-cholenoic acid
\end{itemize}
\item these unsaturated C\textsubscript{24} bile acids are sulphated in the C3 position
\begin{itemize}
\item a proportion is conjugated to glycine
\item high concentrations in the urine
\end{itemize}
\item sulphated \(\Delta\)\textsuperscript{5} bile acids cannot be secreted into the bile
canaliculi and fuel bile flow in the same way as occurs with the
normal bile acids
\begin{itemize}
\item inhibits bile acid-dependent bile flow \(\to\) hepatocyte damage:
\begin{enumerate}
\item abnormal toxic metabolites
\item failure of bile acid-dependent bile flow
\end{enumerate}
\end{itemize}
\end{itemize}

\item Genetics
\label{sec:org915312f}
\begin{itemize}
\item AR HSD3B7
\end{itemize}

\item Diagnostic Tests
\label{sec:orgd52f5d6}
\begin{itemize}
\item characteristic plasma or urine bile acids with
\begin{itemize}
\item \(\Delta\)\textsuperscript{5} double bond
\item 3\(\beta\)-hydroxyl/sulphate group
\item 7\(\alpha\)-hydroxyl group
\end{itemize}
\item bile acids with a \(\Delta\)\textsuperscript{5} double bond and a 7-hydroxy group are acid labile
\item FAB-MS or ESI-MS/MS analysis overcomes this problem
\end{itemize}

\begin{enumerate}
\item Plasma
\label{sec:org815eb23}
\begin{itemize}
\item profile of non-sulphated bile acids by FAB-MS, ESI-MS/MS, GC-MS w/o solvolysis:
\begin{itemize}
\item \(\Downarrow\) cholic and chenodeoxycholic acid for an infant with cholestasis
\item \(\uparrow\) 3\(\beta\),7\(\alpha\)-dihydroxy-5-cholestenoic acid
\end{itemize}
\item FAB-MS, ESI-MS/MS, GC-MS w solvolysis:
\begin{itemize}
\item \(\uparrow\) 3\(\beta\),7\(\alpha\)-dihydroxy-5-cholenoic acid (3-sulphate)
\item 3\(\beta\),7\(\alpha\),12\(\alpha\)-trihydroxy-5-cholenoic acid (3-sulphate)
\end{itemize}
\end{itemize}

\item Urine
\label{sec:org6f5a6d1}
\begin{itemize}
\item negative ion FAB-MS or ESI-MS shows diagnostic
\begin{itemize}
\item unsaturated bile acids
\item sulphated \(\Delta\)\textsuperscript{5} bile acids
\item glycine conjugates of sulphated \(\Delta\)\textsuperscript{5} bile acids
\end{itemize}
\end{itemize}

\item Fibroblasts
\label{sec:org7720cc5}
\begin{itemize}
\item \(\downarrow\) 3\(\beta\)-Dehydrogenase activity cultured skin fibroblasts using
tritiated 7\(\alpha\)-hydroxycholesterol
\end{itemize}
\end{enumerate}

\item Treatment and Prognosis
\label{sec:org431dc5e}
\begin{itemize}
\item emergency treatment of coagulopathy with parenteral vitamin K may be required
\item long term bile acid replacement therapy corrects all the fat-soluble
vitamin deficiencies

\item cholic and chenodeoxycholic acid therapy \(\to\) improvement in symptoms
\end{itemize}
\end{enumerate}

\subsection{5\(\beta\)-Reductase Deficiency}
\label{sec:org4bb8bb5}
\begin{itemize}
\item 5\(\beta\)-reductase deficiency (Figure \ref{fig:org27c9053}, enzyme 3)
\item excrete 3-oxo-\(\Delta\)\textsuperscript{4} bile acids as the major urinary bile acids
\begin{itemize}
\item \(\uparrow\) 7\(\alpha\)-hydroxy-3-oxo-4-cholenoic acid glycine conjugate
\item \(\uparrow\) 7\(\alpha\),12\(\alpha\)-dihydroxy-4-cholenoic acid glycine conjugate
\end{itemize}
\item 8 patients
\item treat w chenodeoxycholic acid plus cholic acid
\end{itemize}
\subsection{Cerebrotendinous Xanthomatosis}
\label{sec:orgb4e4411}
\begin{enumerate}
\item Clinical Presentation
\label{sec:orgd4d52c3}
\begin{itemize}
\item average age of diagnosis is 35 years w a diagnostic delay of 16 years
\begin{itemize}
\item described as a pediatric disease diagnosed in adulthood
\end{itemize}
\item signs and symptoms include:
\begin{itemize}
\item adult-onset progressive neurological dysfunction
\item non-neurologic manifestations
\begin{itemize}
\item tendon xanthomas
\item premature atherosclerosis
\item osteoporosis
\item respiratory insufficiency
\end{itemize}
\end{itemize}
\end{itemize}

\item Metabolic Derangement
\label{sec:orgf592dbf}
\begin{itemize}
\item sterol 27-hydroxylase deficiency (Figure \ref{fig:org27c9053}, enzyme 4)
\item mitochondrial enzyme catalyses first step inside-chain oxidation
\begin{itemize}
\item required to convert a C27 sterol into a C24 bile acid
\end{itemize}
\item 5\(\beta\)-cholestane-3\(\alpha\),7\(\alpha\),12\(\alpha\)-triol cannot be hydroxylated in the C\textsubscript{27}
position and accumulates in the liver
\begin{itemize}
\item products of secondary reactions also accumulate
\item converted to cholestanol
\end{itemize}
\item reduced rate of bile-acid synthesis
\begin{itemize}
\item \(\therefore\) the normal feedback inhibition of cholesterol
7\(\alpha\)-hydroxylase by bile acids is disrupted (Figure \ref{fig:org27c9053}, enzyme 1)
\end{itemize}
\item symptoms partly due to accumulation of cholestanol and cholesterol
\item lack of 3\(\beta\),7\(\alpha\)-dihydroxy-5-cholestenoic acid may contribute to motor
neuron damage
\end{itemize}

\item Genetics
\label{sec:orgfdb7d43}
\begin{itemize}
\item AR CYP27A1
\end{itemize}
\item Diagnostic Tests
\label{sec:orgd1aabf3}
\begin{itemize}
\item molecular
\end{itemize}
\begin{enumerate}
\item Plasma
\label{sec:org927a905}
\begin{itemize}
\item \(\uparrow\) cholestanol by GC or HPLC
\item \(\uparrow\) cholestanol/cholesterol ratio
\item \(\downarrow\) 7-hydroxycholesterol
\end{itemize}
\item Urine
\label{sec:org9033dbb}
\begin{itemize}
\item major cholanoids are cholestanepentol glucuronides by FAB-MS or ESI-MS/MS
\end{itemize}
\end{enumerate}

\item Treatment and Prognosis
\label{sec:org7da4cd0}
\begin{itemize}
\item chenodeoxycholic acid
\item statins
\end{itemize}
\end{enumerate}
\subsection{\(\alpha\)-Methylacyl-CoA Racemase Deficiency}
\label{sec:org30edf5e}
\begin{enumerate}
\item Clinical Presentation
\label{sec:orgd16e977}
\begin{itemize}
\item neurological problems start from childhood to late adult life and
include:
\begin{itemize}
\item mental delay, cognitive decline
\item acute encephalopathy
\item tremor, ataxia
\item pigmentary retinopathy
\item hemiparesis, spastic paraparesis, peripheral neuropathy
\item depression, headache
\end{itemize}
\end{itemize}

\item Metabolic Derangement
\label{sec:org6ce6a18}
\begin{itemize}
\item \(\alpha\)-methylacyl-CoA racemase deficiency (Figure \ref{fig:org27c9053}, enzyme 5)
\item side-chain oxidation of cholesterol produces:
\begin{itemize}
\item 25R isomer of 3\(\alpha\),7\(\alpha\),12\(\alpha\)-trihydroxycholestanoyl-CoA [(25R)-THC-CoA]
\end{itemize}
\item \(\alpha\)-oxidation of dietary phytanic acid produces (some):
\begin{itemize}
\item (2R)-pristanoyl-CoA
\end{itemize}
\item these must be converted to their S-isomers by AMACR before they can
undergo peroxisomal \(\beta\)-oxidation
\end{itemize}

\item Genetics
\label{sec:org436bb65}
\begin{itemize}
\item AR AMACR
\end{itemize}
\item Diagnostic Tests
\label{sec:orga4596cb}
\begin{itemize}
\item \(\uparrow\) plasma DHCA and THCA by GC-MS
\item \(\uparrow\) pristanic acid
\item \(\uparrow\)/n plasma phytanic acid
\item normal VLFCA
\end{itemize}

\item Treatment and Prognosis
\label{sec:org78b3dd4}
\begin{itemize}
\item vitamin K
\item cholic acid
\item phytanic acid
\end{itemize}
\end{enumerate}
\section{Triglyceride and Phospholid Metabolism}
\label{sec:org6bf6bed}
\subsection{Introduction}
\label{sec:org3eb1319}
\begin{itemize}
\item acylglycerols and phospholipids (PL) play many organ-specific roles
in cell structure, biochemistry and signalling
\item PLs form the bilayer of cell membranes
\begin{itemize}
\item a phosphate group that constitutes a hydrophilic head, located on the surface of the bilayer
\item a fatty acid chain that forms a hydrophobic tail, located inside
\end{itemize}
\item TGs are extremely hydrophobic, and form droplets
that are surrounded by a PL monolayer
\item the first steps of TG and of PL synthesis are the same (Figure \ref{fig:org7089c0b})
\item \emph{de novo} synthesis begins from glycerol-3-phosphate (G3P), which is
formed from glycerol or from dihydroxyacetone phosphate (DHAP)
\begin{itemize}
\item DHAP is an intermediate of both glycolysis and of gluconeogenesis
\end{itemize}
\item G3P is esterified by:
\begin{enumerate}
\item glycerolphosphate acyltransferases (GPATs) to form lysophosphatidic acid (LPA)
\item acylglycerol acyltransferases (AGPATs) to form phosphatidic acid (PA)
\end{enumerate}
\item PA can either be converted into
\begin{itemize}
\item diacylglycerol (DG) by phosphatidic acid phosphatases or
\item CDP-diacylglycerol by phosphatidic acid cytidyltransferase (CDP-DAG synthase)
\end{itemize}
\item DG is an essential intermediate for synthesis of TGs and three major PLs:
\begin{enumerate}
\item phosphatidyl choline
\item phosphatidyl ethanolamine
\item phosphatidyl serine
\end{enumerate}
\end{itemize}


\begin{itemize}
\item inborn errors of glycerolipid metabolism cause a vast array of
clinical phenotypes
\item molecular analysis is currently the principal diagnostic
technique
\item NGS \(\to\) discovery of several genetic defects of the biosynthesis and
remodelling of triglycerides (TGs), other acyl-glycerols and of PLs
\item few disorders of glycerolipid metabolism have distinct metabolite
patterns by conventional techniques
\item some of these conditions produce striking clinical syndromes
identifiable in infancy
\item others are mildly atypical forms of common adult conditions like:
\begin{itemize}
\item metabolic syndrome
\item type II diabetes
\end{itemize}
\end{itemize}

\begin{figure}[htbp]
\centering
\includegraphics[width=1.0\textwidth]{tg_pl/figures/tg.png}
\caption{\label{fig:org7089c0b}The common pathway, triglyceride synthesis and lipolysis}
\end{figure}
\subsection{Common Pathway}
\label{sec:orgf3d3b07}
\begin{enumerate}
\item Lipin-1 Deficiency
\label{sec:org85c925f}
\begin{itemize}
\item recurrent childhood rhabdomyolysis with myoglobinuria
\item accounts for \textasciitilde{}10\% of patients with severe recurrent childhood
rhabdomyolysis
\begin{itemize}
\item occurring particularly between 2–6 years of age
\item CK \textgreater{} 10,000 after excercise
\end{itemize}
\item \textbf{phosphatidic acid phosphatases} (PAPs, LPIN1-3) deficiency
\begin{itemize}
\item catalyse phosphate removal, the first dedicated step of TG
synthesis, creating a diacylglycerol
\end{itemize}
\end{itemize}
\end{enumerate}

\subsection{TG Biosynthesis and Lipolysis}
\label{sec:orgda17b4c}
\begin{itemize}
\item extremely rare
\item inborn errors of intracellular TG metabolism often affect adipose
tissue, but non-adipose signs can dominate their clinical
presentation
\end{itemize}

\subsection{Phospholipid Biosynthesis and Remodelling}
\label{sec:org6d21e55}
\begin{itemize}
\item PLs are a source of bioactive lipids released from
membranes by a large family of enzymes, called phospholipases
\item inborn errors of PL metabolism often affect the central and
peripheral nervous systems, but also muscle, eye, skin, bone,
cartilage, liver, kidney and immune system 

\begin{figure}[htbp]
\centering
\includegraphics[width=1.0\textwidth]{tg_pl/figures/pl.png}
\caption{\label{fig:org710bc89}Phospholipid biosynthesis (top of the figure) and remodelling (bottom of the figure)}
\end{figure}
\end{itemize}

\begin{enumerate}
\item Inborn Errors of Phospholipid Biosynthesis
\label{sec:org0f89eee}
\begin{itemize}
\item seven inherited disorders of \emph{de novo} PL synthesis have been described:
\begin{itemize}
\item three involve proteins located in the endoplasmic reticulum
(Figure \ref{fig:org710bc89} steps 1,2 and 3)
\item four are proteins of the mitochondrial membrane (Figure
\ref{fig:org710bc89} steps 4, 5 and 6)
\end{itemize}
\item distinction between \emph{de novo} synthesis and re-modelling defects is
unclear in some cases
\end{itemize}

\item Inborn Errors related to Phospholipid Remodeling
\label{sec:org4ce07c2}
\begin{itemize}
\item six inborn errors of metabolism linked to hydrolases and lipases
involved in the remodelling of membrane phos pholipids have been
described
\item most present with neurodegenerative features including spastic
paraplegia, peripheral neuropathy, neuroendocrine and ophthalmologic
findings
\end{itemize}
\end{enumerate}
\end{document}