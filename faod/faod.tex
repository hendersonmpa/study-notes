% Created 2019-11-09 Sat 17:57
% Intended LaTeX compiler: pdflatex
\documentclass{scrartcl}
\usepackage[utf8]{inputenc}
\usepackage[T1]{fontenc}
\usepackage{graphicx}
\usepackage{grffile}
\usepackage{longtable}
\usepackage{wrapfig}
\usepackage{rotating}
\usepackage[normalem]{ulem}
\usepackage{amsmath}
\usepackage{textcomp}
\usepackage{amssymb}
\usepackage{capt-of}
\usepackage{hyperref}
\hypersetup{colorlinks,linkcolor=black,urlcolor=blue}
\usepackage{textpos}
\usepackage{textgreek}
\usepackage[version=4]{mhchem}
\usepackage{chemfig}
\usepackage{siunitx}
\usepackage{gensymb}
\usepackage[usenames,dvipsnames]{xcolor}
\usepackage[T1]{fontenc}
\usepackage{lmodern}
\usepackage{verbatim}
\usepackage{tikz}
\usepackage{wasysym}
\usetikzlibrary{shapes.geometric,arrows,decorations.pathmorphing,backgrounds,positioning,fit,petri}
\usepackage{fancyhdr}
\pagestyle{fancy}
\author{Matthew Henderson, PhD, FCACB}
\date{\today}
\title{Fatty Acid Oxidation Disorders}
\hypersetup{
 pdfauthor={Matthew Henderson, PhD, FCACB},
 pdftitle={Fatty Acid Oxidation Disorders},
 pdfkeywords={},
 pdfsubject={},
 pdfcreator={Emacs 26.1 (Org mode 9.1.9)}, 
 pdflang={English}}
\begin{document}

\maketitle
\setcounter{tocdepth}{2}
\tableofcontents


\section{Fatty Acid Oxidation}
\label{sec:org64ff894}
\subsection{Introduction}
\label{sec:orga5c83d9}
\subsubsection{Fatty Acids}
\label{sec:org91c797c}
\begin{itemize}
\item usually straight aliphatic chains with a methyl group at one end
(\(\omega\)-carbon) and a carboxyl group and the other end.
\end{itemize}

\definesubmol{x}{-[1,.6]-[7,.6]}
\definesubmol{a}{-[1,.6]\beta{}-[7,.6]\alpha{}}
\definesubmol{y}{!x!x!x!x!x!x!x!x}
\definesubmol{b}{!x!x!x!x!x!x!x!a}
%\chemfig{H{_3}C!y-[1]C(=[1]O)-[7]O{^-}}
\chemname{\chemfig{\omega{}!b-[1]C(=[1]O)-[7]O{^-}}}{\small stearic acid 18:0}

\subsubsection{Fatty Acid Nomenclature}
\label{sec:orgc20f8ce}
\begin{itemize}
\item Non-systematic historical names most commonly used.
\begin{description}
\item[{Palmitic acid}] discovered in palm oil
\item[{Stearic acid}] from the Greek word "stear", which means tallow.
\item[{Oleic acid}] oleic means related to, or derived from, olive oil
\end{description}
\item The position of a double bond is designated by the number of the carbon in the double bond that is closest to the carboxyl group
\end{itemize}


\definesubmol{x}{-[1,.6]-[7,.6]}
\definesubmol{y}{-[7,.6]-[1,.6]}
\definesubmol{d}{=[0,.6](-[7,0.25,,,draw=none]\scriptstyle\color{red}9)-[1,.6]}
\definesubmol{e}{!x!x!x!x!d!y!y!y}
\chemname{\chemfig{\omega{}(-[3,0.25,,,draw=none]\scriptstyle\color{red}18)!e(-[2,0.25,,,draw=none]\scriptstyle\color{red}2)-[7,.6]COOH}}{\small Oleic acid 18:1,\Delta{}$^9$}

\begin{itemize}
\item 18 carbons
\item 1 double bond
\item \(\Delta^{\text{9}}\), double bond between 9th and 10th carbon.
\item Also 18:1(9)
\item Distance from \(\omega\) methyl group, \(\omega\)-9
\end{itemize}

\subsubsection{Fatty Acid Chain Length}
\label{sec:org283fbd4}

\begin{description}
\item[{Very long-chain}] > C20
\item[{Long-chain}] C12-C20
\item[{Medium-chain}] C6-C12
\item[{Short-chain}] C4
\end{description}

\subsubsection{Fatty Acids as an Energy Source}
\label{sec:org004ca62}

\begin{itemize}
\item Long chain fatty acids released from adipose tissue triacylglycerol
stores during periods of increased fuel demand or fasting.
\item \(\downarrow\) insulin, \(\uparrow\) glucagon \(\to\) \(\uparrow\) lipolysis
\begin{itemize}
\item dietary lipids
\item triacylglycerols synthesis in the liver
\item palmitate (C16:0), oleate (C18:1, \(\Delta^{\text{9}}\)) and stearate (C18:0)
\end{itemize}
\item FA transported to tissue bound to albumin
\item Energy derived from oxidation of FA to acetyl-CoA in \(\beta\)-oxidation.
\item The acetyl-CoA is oxidized in the TCA cycle or converted to ketone bodies in the liver.
\end{itemize}

\subsection{\(\beta\)-oxidation}
\label{sec:org2bbe4ec}
\subsubsection{Overview of Mitochondrial Long-Chain Fatty Acid Metabolism}
\label{sec:org31d0ff9}

\begin{figure}[htbp]
\centering
\includegraphics[width=0.5\textwidth]{./fao/figures/23_1.png}
\caption{\label{fig:orga1d6e6b}
\(\beta\)-oxidation}
\end{figure}

\begin{itemize}
\item Steps 
\begin{enumerate}
\item Membrane transport
\begin{itemize}
\item FaBP
\end{itemize}
\item Activation
\begin{itemize}
\item CoA
\item Carnitine
\item CoA
\end{itemize}
\item \(\beta\)-oxidation spiral
\begin{itemize}
\item Acyl-CoA \(\to\) Ketone bodies or TCA
\end{itemize}
\end{enumerate}
\end{itemize}

\subsubsection{Activation of Fatty Acids}
\label{sec:orgebc66c4}
\begin{figure}[htbp]
\centering
\includegraphics[width=0.5\textwidth]{./fao/figures/23_2.png}
\caption{\label{fig:orgcc91d51}
FA activation}
\end{figure}

\subsubsection{Transport of Long-Chain Fatty Acids into Mitochondria}
\label{sec:org1a18011}
\begin{figure}[htbp]
\centering
\includegraphics[width=0.5\textwidth]{./fao/figures/23_5.png}
\caption{\label{fig:orgd7afa47}
Transport of Long-Chain Fatty Acids into Mitochondria}
\end{figure}

\begin{itemize}
\item chain length specificity
\item Organ and organelle specificity
\begin{itemize}
\item Acyl-CoA synthetases
\item Acyltransferases
\item Acyl-CoA dehydrogenases
\end{itemize}
\end{itemize}


\subsubsection{Chain Length Specificity}
\label{sec:org825a447}

\begin{table}[htbp]
\caption{\label{tab:org721855c}
Acyl-CoA Synthetases}
\centering
\begin{tabular}{lrl}
Enzyme & Length & location\\
\hline
V.L. chain & 14-26 & pex\\
L. chain & 12-20 & ER, mito, pex\\
M. chain & 6-12 & mito - kidney, liver\\
acetyl & 2-4 & cyto, ?mito?\\
\end{tabular}
\end{table}


\begin{table}[htbp]
\caption{\label{tab:org39ae3dd}
Acyl-CoA Dehydrogenases}
\centering
\begin{tabular}{lrl}
Enzyme & Length & location\\
\hline
VLCAD & 14-20 & IMM\\
LCAD & 12-18 & MM\\
MCAD & 4-12 & MM\\
SCAD & 2-4 & MM\\
\end{tabular}
\end{table}


\begin{table}[htbp]
\caption{\label{tab:org3c6b5fd}
Other}
\centering
\begin{tabular}{lrl}
Enzyme & Length & comment\\
\hline
Enoyl-CoA hydratase,SC & >4 & \(\downarrow\) activity w \(\uparrow\) length\\
Hydroxyacyl-CoA dehydrogenase, SC & 4-16 & \(\downarrow\) activity w \(\uparrow\) length\\
Acetoacetyl-CoA thiolase & 4 & Acetoacetyl-CoA specific\\
Trifunctional protein & 12-16 & \(\uparrow\) activity w \(\uparrow\) length\\
\end{tabular}
\end{table}


\subsubsection{\(\beta\)-oxidation of Long-Chain Fatty Acids}
\label{sec:org87d57d3}
\begin{figure}[htbp]
\centering
\includegraphics[width=0.5\textwidth]{./fao/figures/23_7.png}
\caption{\label{fig:org3217120}
\(\beta\)-oxidation of Long-Chain Fatty Acids}
\end{figure}

\subsubsection{Oxidation of Unsaturated Fatty Acids}
\label{sec:orgf8abf65}

\begin{figure}[htbp]
\centering
\includegraphics[width=0.3\textwidth]{./fao/figures/23_9.png}
\caption{\label{fig:org3ce8180}
Oxidation of Unsaturated Fatty Acids}
\end{figure}

\begin{itemize}
\item isomerase and reductase change location of the double bonds
\begin{itemize}
\item correct configuration for B-oxidation
\end{itemize}
\end{itemize}
\subsubsection{Odd-Chain Length Fatty Acids}
\label{sec:orgf6d2b3f}
\begin{figure}[htbp]
\centering
\includegraphics[width=0.3\textwidth]{./fao/figures/23_10.png}
\caption{\label{fig:org23b48e3}
Odd-Chain Length Fatty Acids}
\end{figure}

\subsubsection{Oxidation of Medium-Chain Length Fatty Acids}
\label{sec:orgecc3255}

\begin{itemize}
\item \(\uparrow\) solubility
\item not stored in adipose triacylglycerol
\item gut \(\to\) portal vein \(\to\) liver
\item \(\to\) mito matrix via the monocarboxylate transporter
\item activated in the mito matrix
\item \(\beta\)-oxidation

\item There is a general consensus that short-chain and medium-chain fatty
acids (C4 to C12) diffuse freely across plasma and mitochondrial
membranes
\item Butyrate is taken up by enterocytes, presumably by means of the
monocarboxylate transporter 1 (MCT-1) and the sodium-coupled
monocarboxylate transporter 1 (SMCT-1)
\end{itemize}

\subsubsection{Monocarboxylate Transporter 1}
\label{sec:orgd108f7e}
\begin{itemize}
\item Out of 14 known mammalian MCTs, six isoforms have been functionally
characterized to transport monocarboxylates and short chain fatty
acids (MCT1-4), thyroid hormones (MCT8-10) and aromatic amino
acids (MCT10)

\item MCT1 mediates the movement of lactate and pyruvate across cell
membranes.
\begin{itemize}
\item erythrocytes, muscle, intestine, liver and kidney
\end{itemize}
\end{itemize}

\begin{table}[htbp]
\caption[Monocarboxylate Transporter 1]{\label{tab:org54a8840}
MCT1 deficiency (SLC16A1)}
\centering
\begin{tabular}{ll}
Phenotype & Inheritance\\
\hline
Erythrocyte lactate transporter defect & AD\\
Hyperinsulinemic hypoglycemia, familial, 7\footnotemark & AD\\
Monocarboxylate transporter 1 deficiency & AR, AD\\
 & \\
\end{tabular}
\end{table}\footnotetext[1]{\label{org041e091}promoter-activating mutations in patients with hyperinsulinemic
hypoglycemia induce SLC16A1 expression in beta cells, where this
gene is not usually transcribed, permitting pyruvate uptake and
pyruvate-stimulated insulin release despite ensuing hypoglycemia}

\subsubsection{Regulation of \(\beta\)-oxidation}
\label{sec:org5306b90}
\begin{figure}[htbp]
\centering
\includegraphics[width=0.5\textwidth]{./fao/figures/23_12.png}
\caption{\label{fig:org8d36388}
Regulation of \(\beta\)-oxidation}
\end{figure}

\begin{enumerate}
\item Lipolysis or gut
\item Regulation of CPT1 activity
\item Re-oxidation of NAD\(^{\text{+}}\) and FAD\(^{\text{2+}}\)
\end{enumerate}

\subsection{Alternative Routes of Fatty Acid Oxidation}
\label{sec:orgc9cf0b0}

\subsubsection{Peroxisomal Oxidation of Fatty Acids}
\label{sec:orgfe6336c}

\begin{figure}[htbp]
\centering
\includegraphics[width=0.6\textwidth]{./fao/figures/23_14.png}
\caption{\label{fig:orge2a3102}
Peroxisomal Oxidation of Fatty Acids}
\end{figure}

\begin{itemize}
\item very long chain FA C24-26 mandatory
\item long chain optional
\item carnitine not required for entry into peroxisomes
\end{itemize}

\subsubsection{First Step of Oxidation of Fatty Acids}
\label{sec:org5f1c0fe}
\begin{figure}[htbp]
\centering
\includegraphics[width=0.3\textwidth]{./fao/figures/23_13.png}
\caption{\label{fig:org8acd8b6}
First Step of Oxidation of Fatty Acids}
\end{figure}

\subsubsection{Long-Chain Branched-Chain Fatty Acids}
\label{sec:orga7667ab}

\begin{figure}[htbp]
\centering
\includegraphics[width=0.6\textwidth]{./fao/figures/ff22.png}
\caption{\label{fig:org9d9a84f}
Long-Chain Branched-Chain Fatty Acids}
\end{figure}

\begin{itemize}
\item \(\alpha\)-oxidation of phytanic acid takes place in peroxisomes.
\item Pristanic acid can then undergo \(\beta\)-oxidation.
\begin{itemize}
\item Propionyl-CoA is released when the \(\alpha\) carbon is substituted
\end{itemize}
\end{itemize}

\subsubsection{\(\omega\)-Oxidation of Fatty Acids}
\label{sec:orge16485b}

\begin{figure}[htbp]
\centering
\includegraphics[width=0.3\textwidth]{./fao/figures/23_16.png}
\caption{\label{fig:orgba20944}
\(\omega\)-Oxidation of Fatty Acids}
\end{figure}

\begin{itemize}
\item occurs in the ER
\item the \(\omega\)-carbon is oxidized to an alcohol
\item dehydrogenated to a carboxylic acid \(\to\) dicarboxylic acid
\begin{itemize}
\item excreted in urine as medium chain dicarboxylic acids
\end{itemize}
\item xenobiotic compounds w FA like structure
\end{itemize}
\section{Mitochondrial FAODs}
\label{sec:org4c48368}
\subsection{Introduction}
\label{sec:org8b5c293}
\begin{figure}[htbp]
\centering
\includegraphics[width=0.9\textwidth]{./mito_faod/figures/b_oxidation.png}
\caption{\label{fig:orgec08761}
\(\beta\)-oxidation}
\end{figure}

\subsection{Carnitine Cycle Defects}
\label{sec:org7c985e4}
\subsubsection{Carnitine Transporter Deficiency}
\label{sec:orga3cc592}
\begin{itemize}
\item Organic cation/carnitine transporter(OCTN2) responsible for
carnitine uptake. 
\begin{itemize}
\item Analysis of carnitine transport in different tissues suggests the
presence of heterogeneous transporters.
\item Liver and brain have a low-affinity (Km 2–10 uM), high-capacity transporter
\item Fibroblast, muscle, and heart cells have a high-affinity (Km 5–10 uM), low-capacity system.
\end{itemize}
\item Defects \(\to\) primary carnitine deficiency with \(\uparrow\) renal loss of carnitine
\begin{itemize}
\item low plasma concentrations
\item low intracellular concentrations \(\to\) impair fatty acid oxidation
\end{itemize}
\begin{itemize}
\item Precipitated by infection, fasting, pregnancy or antibiotics containing pivalate
\begin{itemize}
\item pivalate is excreted bound to carnitine, \(\downarrow\) carnitine concentration
\item isobaric with C5-carnitine
\end{itemize}
\item Some present in infancy with hypoglycaemia, liver dysfunction and hyperammonaemia
\item Other children develop heart failure due to cardiomyopathy,
thickened ventricular walls and reduced contractility.
\begin{itemize}
\item often accompanied by skeletal muscle weakness.
\end{itemize}
\item Adults may suffer fatigue or arrhythmias
\item Screening has shown that many subjects with low plasma carnitine remain asymptomatic
\begin{itemize}
\item Faroe Islands prevalence is 1:300.
\end{itemize}
\end{itemize}
\end{itemize}

\subsubsection{Carnitine Palmitoyltransferase I Deficiency}
\label{sec:org9830485}
\begin{itemize}
\item \textbf{CPT Ia} liver and kidney
\item \textbf{CPT Ib}  muscle and heart
\item \textbf{CPT Ic}  brain

\item Only CPT Ia deficiency has been identified.
\item Usually present by the age of 2 years with hypoketotic hypoglycaemia,
\begin{itemize}
\item induced by fasting or illness.
\end{itemize}
\item Accompanied by hepatomegaly, liver dysfunction and occasionally cholestasis
\begin{itemize}
\item may also be transient lipaemia and renal tubular acidosis.
\end{itemize}
\item CPT I deficiency is extremely common in the Inuit population of Canada and Greenland.
\begin{itemize}
\item c.1436C>T, P479L
\end{itemize}
\item A few of these patients present with hypoglycaemia as neonates or young children
\begin{itemize}
\item most remain asymptomatic.
\end{itemize}
\end{itemize}

\subsubsection{Carnitine Acylcarnitine Translocase Deficiency}
\label{sec:org57c01e6}

\begin{itemize}
\item This rare disorder usually presents in the neonatal period, with
death by 3 months of age
\begin{itemize}
\item severe hypoglycaemia and hyperammonaemia, cardiomyopathy,
atrioventricular block and ventricular arrhythmias.
\end{itemize}
\item A few more mildly affected patients present later with hypoglycaemic
encephalopathy
\begin{itemize}
\item precipitated by fasting or infections.
\end{itemize}
\end{itemize}

\subsubsection{Carnitine Palmitoyltransferase II Deficiency}
\label{sec:orgcb0f682}
\begin{enumerate}
\item Neonatal
\label{sec:org1c51a0a}
\begin{itemize}
\item Severe neonatal onset CPT II deficiency is usually lethal.
\item Patients become comatose within a few days of birth
\begin{itemize}
\item hypoglycaemia and hyperammonaemia.
\item may have cardiomyopathy, arrhythmias and congenital malformations,
principally renal cysts and neuronal migration defects.
\end{itemize}
\item There is also an intermediate form of CPT II deficiency that causes
episodes of hypoglycaemia and liver dysfunction, sometimes
accompanied by cardiomyopathy and arrhythmias
\end{itemize}

\item Childhood
\label{sec:orgc264a1b}
\begin{itemize}
\item Episodes may be brought on by infections or exercise
\item In moderate or severe episodes there is myoglobinuria, \(\uparrow\) CK
\begin{itemize}
\item may lead to acute renal failure
\item CK often normalises between episodes but may remain moderately
elevated
\end{itemize}
\end{itemize}

\item Adolescence,  young adult
\label{sec:org0ef0e21}
\begin{itemize}
\item Most common form is a partial deficiency that presents with
episodes of rhabdomyolysis.
\begin{itemize}
\item usually precipitated by prolonged exercise
\item particularly in the cold or after fasting
\end{itemize}
\end{itemize}
\end{enumerate}

\subsection{\(\beta\)-Oxidation Defects}
\label{sec:org8507bc6}
\subsubsection{Very-Long-Chain Acyl-CoA Dehydrogenase Deficiency}
\label{sec:orgfb88308}
\begin{enumerate}
\item Early infancy
\label{sec:org8c80eb5}
\begin{itemize}
\item Severely affected patients present in early infancy with
cardiomyopathy, in addition to the problems seen in milder patients.
\end{itemize}

\item Childhood
\label{sec:org0d82291}
\begin{itemize}
\item patients present in childhood with hypoglycaemia but suffer exercise
or illness induced rhabdomyolysis or chronic weakness at a later age.
\end{itemize}

\item Adolescence, Adult
\label{sec:org628085f}
\begin{itemize}
\item Mildly affected patients present as adolescents or adults with
exercise-induced rhabdomyolysis.
\end{itemize}

\item Screening
\label{sec:orgfdb5f29}
\begin{itemize}
\item Second most common FAOD in Europe and the USA
\item prevalence between 1:50,000 and 1:100,000
\item Much higher than detected clinically
\item likely that many patients diagnosed by screening would remain
asymptomatic without intervention
\end{itemize}
\end{enumerate}

\subsubsection{Mitochondrial Trifunctional Protein}
\label{sec:orgd9f30cd}
\begin{itemize}
\item MTP a hetero-octamer composed of four \(\alpha\)-subunits and four \(\beta\)-subunits;
\item \(\alpha\)-subunit has long-chain enoyl-CoA hydratase (LCEH) and LCHAD activities
\item \(\beta\)-subunit has long-chain ketoacyl-CoA thiolase (LCKAT) activity.
\item Patients may have isolated LCHAD deficiency or a generalised deficiency of all three enzyme activities.

\item Mothers who are heterozygous for LCHAD or MTP deficiency have a high
risk of illness during pregnancies when carrying an affected fetus
\item The main problems are HELLP syndrome (Haemolysis, Elevated Liver
enzymes and Low Platelets) and acute fatty liver of pregnancy
(AFLP).
\end{itemize}

\subsubsection{Long-Chain 3-Hydroxyacyl-CoA Dehydrogenase}
\label{sec:org81145c2}
\begin{itemize}
\item Isolated LCHAD deficiency usually presents acutely before 6 months of age
\begin{itemize}
\item hypoglycaemia, liver dysfunction, lactic acidosis
\item Many have cardiomyopathy, some have hypoparathyroidism or ARDS
\end{itemize}
\item Other patients present with chronic symptoms
\begin{itemize}
\item failure to thrive, hypotonia, occasionally cholestasis or cirrhosis.
\end{itemize}
\item Subsequently, episodes of rhabdomyolysis are common.
\item Many patients develop retinopathy, may start as early as 2 years of age.
\item Granular pigmentation followed by chorioretinal atrophy w deteriorating central vision.
\item Some patients develop cataracts
\end{itemize}

\subsubsection{Mitochondrial Trifunctional Protein Deficiency}
\label{sec:org91903f3}
\begin{itemize}
\item Presentation of generalised MTP deficiency is heterogeneous
\item Patients with severe deficiency present as neonates
\begin{itemize}
\item cardiomyopathy, respiratory distress, hypoglycaemia and liver dysfunction
\item most die within a few months, regardless of treatment.
\end{itemize}
\item Other patients resemble those with isolated LCHAD deficiency.
\item A milder neuromyopathic phenotype:
\begin{itemize}
\item exercise induced rhabdomyolysis and progressive peripheral
neuropathy
\item can present at any age from infancy to adulthood.
\end{itemize}
\end{itemize}

\subsubsection{Long-Chain Acyl-CoA Dehydrogenase Deficiency}
\label{sec:orgc1c30b9}
\begin{itemize}
\item No human disease-causing mutations have been identified
\item role  in  human  metabolism  is unclear.
\item \emph{In vitro}, the substrate specificity of LCAD overlaps with that of
VLCAD and ACAD9.
\item enzymes have strong activity toward long-chain acyl-CoAs (C14-20)
\item surfactant deficiency and altered lung mechanics in LCAD deficient
mice.
\begin{itemize}
\item postulated that LCAD deficiency in humans may manifest primarily
as a lung disease
\end{itemize}
\end{itemize}
\subsubsection{Medium-Chain Acyl-CoA Dehydrogenase Deficiency}
\label{sec:org1187e17}
\begin{itemize}
\item most common FAOD with an incidence of approximately 1:10,000-20,000
in Europe,USA and Australia.
\item Before NBS, presented 4 months to 4 years
\begin{itemize}
\item acute hypoglycaemic encephalopathy and liver dysfunction, not always
\item some deteriorated rapidly and died.
\end{itemize}
\item Precipitated by prolonged fasting or infection with vomiting
\item Some babies still present within 72 hours of birth before
newborn screening results are available
\begin{itemize}
\item hypoglycaemia and/or arrhythmias
\item breast-fed babies are at higher risk, due to the small supply of
breast milk at this stage.
\end{itemize}
\item MCAD deficiency only presents clinically if exposed to an
appropriate environmental stress.
\item Prior to NBS \textasciitilde{} 30-50\% remained asymptomatic
\item NBS and preventative measures, hypoglycaemia is rare.
\begin{itemize}
\item Patients do not develop cardiomyopathy or myopathy and few present
initially as adults.
\end{itemize}
\item Healty MCAD deficient children > 1 year can fast for 12-14 hours without problems.
\item >14 hours \(\to\) non-ketotic (inappropriately low) hypoglycaemia.
\item Shorter fasts may cause problems in infancy
\item Encephalopathy may occur without hypoglycaemia
\begin{itemize}
\item accumulation of FFA acids and carnitine/CoA esters.
\end{itemize}
\end{itemize}

\subsubsection{Short-Chain Acyl-CoA Dehydrogenase Deficiency}
\label{sec:org89835da}
\begin{itemize}
\item non-disease?
\begin{itemize}
\item previous association with symptoms due to ascertainment bias?
\end{itemize}
\end{itemize}

\subsubsection{3-Hydroxyacyl-CoA Dehydrogenase Deficiency}
\label{sec:orgfae207c}
\begin{itemize}
\item This defect (HADH), previously called SCHAD deficiency, causes
hyperinsulinaemic hypoglycaemia
\item Role in modulation of ATP production inhibition of GDH
\end{itemize}

\subsubsection{Acyl-CoA dehydrogenase 9}
\label{sec:orgcec7976}
\begin{itemize}
\item A complex I assembly factor with a moonlighting function in fatty
acid oxidation deficiencies.
\item ACAD9 is most homologous to VLCAD
\item recombinant ACAD9 displays activity towards long-chain acyl-CoAs,
very similar to VLCAD.
\item Responsible for production of C14:1-carnitine and C12-carnitine in
VLCAD deficiency.
\begin{itemize}
\item VLCAD\(^{\text{-/-}}\) cell lines accumulate C14:1
\item VLCAD\(^{\text{-/-}}\)/ACAD9\(^{\text{-/-}}\) cell lines accumulate C18:1
\end{itemize}
\item Patients with ACAD9 defects present in infancy or childhood with
myopathy or hypertrophic cardiomyopathy and lactic acidaemia;
\item some also have neurological problems.
\item The myopathic patients often respond to treatment with riboflavin
\begin{itemize}
\item FAD is the enzyme-bound prosthetic group of all acyl-CoA
dehydrogenases
\end{itemize}
\end{itemize}

\subsection{Electron Transfer Defects}
\label{sec:org6ceb62e}
\subsubsection{Multiple acyl-CoA dehydrogenase deficiency}
\label{sec:org6582ccc}
\begin{itemize}
\item AKA glutaric aciduria type II.
\item Electron transfer flavoprotein (ETF) and ETF ubiquinone
oxidoreductase (ETFQO) carry electrons to the respiratory chain from
multiple FAD-linked dehydrogenases.
\item Includes enzymes of amino acid, choline metabolism and acyl-CoA
dehydrogenases of \(\beta\)-oxidation
\item Defects in ETF or ETFQO \(\to\)
\item GAII less often, a result of defects of riboflavin transport or
metabolism
\item ETF and ETFQO deficiencies \(\to\) wide range of clinical severity.
\item Severely affected patients present in the first few days of life
\begin{itemize}
\item hypoglycaemia, hyperammonaemia and acidosis
\item hypotonia and hepatomegaly.
\end{itemize}
\item There is usually an odour of sweaty feet similar to that in isovaleric acidaemia.
\item Some patients have congenital anomalies
\begin{itemize}
\item Large cystic kidneys, hypospadias and neuronal migration defects and facial dysmorphism
\begin{itemize}
\item low set ears, high forehead and midfacial hypoplasia.
\end{itemize}
\end{itemize}
\item The malformations resemble those seen in CPT II deficiency but the pathogenesis is unknown.
\item Most patients with neonatal presentation die within a week of birth
\item Others develop cardiomyopathy and die within a few months.
\item Less severe cases can present at any age from infancy to adulthood
\begin{itemize}
\item with hypoglycaemia, liver dysfunction and weakness
\item usually precipitated by an infection
\end{itemize}
\item Cardiomyopathy is common in infants.
\item Rarer problems include stridor and leukodystrophy.
\item Mildly affected children may have recurrent bouts of vomiting.
\item Muscle weakness is the commonest presentation in adolescents and adults.
\begin{itemize}
\item Predominantly affects proximal muscles and may lead to scoliosis,
hypoventilation or an inability to lift the chin off the chest.
\end{itemize}
\item Weakness can worsen rapidly during infection or pregnancy, myoglobinuria is rare.
\begin{itemize}
\item Many milder defects respond to riboflavin
\end{itemize}
\end{itemize}

\subsection{Metabolic Derangement}
\label{sec:orgcf7cf75}
\begin{itemize}
\item Fasting hypoglycaemia is the classic metabolic disturbance in FAODs
\begin{itemize}
\item primarily due to increased peripheral glucose consumption
\item hepatic glucose output is also reduced under some conditions.
\end{itemize}
\item The hypoglycaemia is hypoketotic.
\begin{itemize}
\item Ketone bodies can be synthesised
\begin{itemize}
\item medium-or short-chain FAODs or if there is high residual enzyme activity
\item plasma concentrations are lower than expected for hypoglycaemia or the plasma free fatty acid concentrations.
\end{itemize}
\end{itemize}
\item Hyperammonaemia occurs in some severe defects
\begin{itemize}
\item with normal or low glutamine concentrations
\item decreased acetyl-CoA production reducing the synthesis of N-acetylglutamate
\end{itemize}
\item Lactic acidaemia is seen in long-chain FAODs(LCHAD and MTP deficiencies)
\begin{itemize}
\item inhibitory effects of metabolites on pyruvate metabolism.
\end{itemize}
\item Moderate hyperuricaemia - frequent finding during acute attacks.
\item Secondary hyperprolinaemia occurs in some babies with MAD deficiency.
\item Accumulating long-chain acylcarnitines may be responsible for
arrhythmias and may interfere with surfactant metabolism.
\item In LCHAD and MTP deficiencies, long chain hydroxy-acylcarnitine
concentrations correlate with the severity of retinopathy and may
cause both this and the peripheral neuropathy
\end{itemize}

\subsection{Summary}
\label{sec:orgff12f7c}
\subsubsection{Common manifestations in FAODs}
\label{sec:orgf58b159}
\begin{figure}[htbp]
\centering
\includegraphics[width=0.9\textwidth]{./mito_faod/figures/Ch101f016.png}
\caption{\label{fig:orgc7e32e6}
Common manifestations in FAODs [fn:1]}
\end{figure}

\begin{itemize}
\item Green squares indicate that the feature is frequently seen in the disorder
\item Yellow squares represent an intermediate rate of occurrence
\item Red squares denote that it is uncommon
\end{itemize}

\section{Routine Testing for Mitochondrial FAODs}
\label{sec:org2134b3b}
\subsection{Routine Biochemistry}
\label{sec:org0b9f7f4}
\begin{itemize}
\item Glucose
\item Ketones
\item Lactate
\item Ammonia
\item Uric Acid
\item Creatine Kinase (CK)
\item Acute Fatty Liver of Pregnancy (AFLP) \& Hemolysis, Elevated Liver enzymes Low Platlets (HELLP)
\begin{itemize}
\item Mothers who are heterozygous for LCHAD or MTP deficiency when
carrying an affected fetus
\end{itemize}
\end{itemize}
\subsubsection{Fasting Hypoglycaemia  in FAODs}
\label{sec:orgfabfeb8}
\begin{itemize}
\item Fasting hypoglycaemia is the classic metabolic disturbance in FAODs
\begin{itemize}
\item primarily due to increased peripheral glucose consumption
\end{itemize}
\begin{itemize}
\item The hypoglycaemia is hypoketotic.
\begin{itemize}
\item Ketone bodies can be synthesised (medium-or short-chain FAODs or
if there is high residual enzyme activity)
\item plasma concentrations are lower than expected for hypoglycaemia or
the plasma free fatty acid concentrations.
\end{itemize}
\end{itemize}
\end{itemize}

\subsubsection{Randle Cycle \footnote{Hue, L., Taegtmeyer, H., Randle, P., Garland, P., \& Hales,
N. (2009). The Randle cycle revisited. Am J Physiol Endocrinol Metab,
297, 578–591.

\begin{figure}[htbp]
\centering
\includegraphics[width=0.5\textwidth]{./faod_routine_testing/figures/randle.png}
\caption{\label{fig:org47c4324}
Randle Cycle}
\end{figure}}}
\label{sec:org32ace4f}
\begin{itemize}
\item Fasted state:
\begin{itemize}
\item Fatty Acids inhibit glucose oxidation at pyruvate dehydrogenase
(PDH)
\item Long-chain acyl-CoA derivatives directly inhibit glucokinase
\end{itemize}
\item Fed state:
\begin{itemize}
\item Insulin inhibits lipolysis
\item Malonyl-CoA inhibits CPT1
\begin{itemize}
\item \(\uparrow\) esterification of FA
\end{itemize}
\end{itemize}
\end{itemize}

\subsubsection{Inhibition of Glucose Utilization by Fatty Acids}
\label{sec:org739ec30}
\begin{figure}[htbp]
\centering
\includegraphics[width=0.5\textwidth]{./faod_routine_testing/figures/glucose_oxidation_inhibition.png}
\caption{\label{fig:org2c430ab}
Inhibition of Glucose Utilization by FA}
\end{figure}

\begin{itemize}
\item The extent of inhibition is graded and most severe at the level of
pyruvate dehydrogenase (PDH) and less severe at the level of
6-phosphofructo-1-kinase (PFK) and glucose uptake.
\item PDH inhibition is caused by acetyl-CoA and NADH accumulation
resulting from fatty acid oxidation, whereas PFK inhibition results
from citrate accumulation in the cytosol. The mechanism of
inhibition of glucose uptake is not clear.
\item These effects reroute glucose toward glycogen synthesis and pyruvate
to anaplerosis and/or gluconeogenesis.
\end{itemize}

\subsubsection{Inhibition of Fatty Acid Utilization by Glucose}
\label{sec:org30a2f25}

\begin{figure}[htbp]
\centering
\includegraphics[width=0.5\textwidth]{./faod_routine_testing/figures/FAO_inhibition.png}
\caption{\label{fig:orgd2a8876}
Inhibition of FA Utilization by Glucose}
\end{figure}

\begin{itemize}
\item This mechanism is mediated by malonyl-CoA, the concentration of
which depends on ACC activity and which inhibits the entry of
long-chain fatty acyl (LCFAcyl-CoA) moieties into mitochondria.
\item This effect reroutes fatty acids toward esterification.
\item In extrahepatic tissues, the effect of glucose is stimulated by
insulin.
\end{itemize}

\subsubsection{Lactic Acidemia in FAODs}
\label{sec:org8912a79}
\begin{itemize}
\item Lactic acidaemia is seen in long-chain FAODs (VLCAD, LCHAD and MTP deficiencies)
\item Long-chain acyl-CoA esters have been found to inhibit a large
variety of enzymes including the mitochondrial ATP/ADP carrier.
\begin{itemize}
\item The mitochondrial ATP/ADP carrier catalyses the exchange of intra- and
extra-mitochondrial adenine nucleotides.
\end{itemize}
\item \emph{In vivo} the ATP/ADP carrier catalyses the transport of ATP
synthesized in the mitochondrion by the F\(_{\text{1}}\) F\(_{\text{0}}\)-ATPase reaction to the
extra-mitochondrial space in exchange for cytosolic ADP
\begin{itemize}
\item \(\uparrow\) intra-mitochondrial ATP/ADP ratio and \(\downarrow\) oxidation of
NADH \(\to\) NAD\(^{\text{+}}\)
\item \(\uparrow\) ATP/ADP and NADH/NAD\(^{\text{+}}\) ratios will activate PDK
\item PDK inactivates PDH by phosphorylation.
\end{itemize}
\end{itemize}

\subsubsection{Lactic Acidemia in FAODs}
\label{sec:orge355104}
\begin{figure}[htbp]
\centering
\includegraphics[width=0.9\textwidth]{./faod_routine_testing/figures/NAD.png}
\caption{\label{fig:org63724b9}
Lactic Acidemia in FAODs}
\end{figure}

\begin{itemize}
\item Overview of the major metabolic pathways involved in cytosolic and
mitochondrial oxidation and reduction of NAD
\end{itemize}

\subsubsection{Hyperammonaemia in FAODs}
\label{sec:orgca6d16e}
\begin{itemize}
\item Hyperammonaemia occurs in some severe defects,
\begin{itemize}
\item with normal or low glutamine concentrations;
\item decreased acetyl-CoA production reducing the synthesis of N-acetylglutamate
\end{itemize}
\item Influence of metabolic disorders on function of urea cycle leading
to secondary hyperammonemia
\end{itemize}

\begin{figure}[htbp]
\centering
\includegraphics[width=0.9\textwidth]{./faod_routine_testing/figures/2nd_ammonemia.png}
\caption{\label{fig:orgd0f70c3}
Hyperammonaemia in FAODs}
\end{figure}



\subsubsection{Moderate hyperuricaemia and elevated CK}
\label{sec:org57cd2da}
\begin{itemize}
\item Seen in acute attacks 
\begin{itemize}
\item The association suggests that the mechanism is a breakdown of
cells, particularly muscle.
\item ATP depletion \(\to\) \(\downarrow\) Na/K-ATPase
\end{itemize}
\item Release of CK
\item The uric acid excess is the product of nucleic acid and nucleotide catabolism.
\end{itemize}

\subsubsection{AFLP/HELLP}
\label{sec:org00c4034}
\begin{itemize}
\item Mothers who are heterozygous for LCHAD or MTP deficiency when
carrying an affected fetus
\item A woman whose affected fetus has the Glu474Gln mutation on one or
both alleles of the \(\alpha\)-subunit of the trifunctional protein is
likely to have acute fatty liver of pregnancy or the HELLP syndrome 
\begin{itemize}
\item Long-chain 3-hydroxyacyl metabolites produced by the fetus or
placenta accumulate in the mother and are highly toxic to the liver
\item Exaggerated by the decreased metabolic utilization of fatty acids
during pregnancy.
\end{itemize}
\end{itemize}
\end{document}