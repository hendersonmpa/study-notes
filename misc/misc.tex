% Created 2020-08-02 Sun 19:26
% Intended LaTeX compiler: pdflatex
\documentclass[12pt]{scrartcl}
\usepackage[utf8]{inputenc}
\usepackage[T1]{fontenc}
\usepackage{graphicx}
\usepackage{grffile}
\usepackage{longtable}
\usepackage{wrapfig}
\usepackage{rotating}
\usepackage[normalem]{ulem}
\usepackage{amsmath}
\usepackage{textcomp}
\usepackage{amssymb}
\usepackage{capt-of}
\usepackage{hyperref}
\hypersetup{colorlinks,linkcolor=black,urlcolor=blue}
\usepackage{textpos}
\usepackage{textgreek}
\usepackage[version=4]{mhchem}
\usepackage{chemfig}
\usepackage{siunitx}
\usepackage{gensymb}
\usepackage[usenames,dvipsnames]{xcolor}
\usepackage{lmodern}
\usepackage{verbatim}
\usepackage{tikz}
\usepackage{wasysym}
\usetikzlibrary{shapes.geometric,arrows,decorations.pathmorphing,backgrounds,positioning,fit,petri}
\usepackage[automark, autooneside=false, headsepline]{scrlayer-scrpage}
\clearpairofpagestyles
\ihead{\leftmark}% section on the inner (oneside: right) side
\ohead{\rightmark}% subsection on the outer (oneside: left) side
\addtokomafont{pagehead}{\upshape}% header upshape instead of italic
\ofoot*{\pagemark}% the pagenumber in the center of the foot, also on plain pages
\pagestyle{scrheadings}
\author{Matthew Henderson, PhD, FCACB}
\date{\today}
\title{Miscellaneous}
\hypersetup{
 pdfauthor={Matthew Henderson, PhD, FCACB},
 pdftitle={Miscellaneous},
 pdfkeywords={},
 pdfsubject={},
 pdfcreator={Emacs 26.3 (Org mode 9.3.7)}, 
 pdflang={English}}
\begin{document}

\maketitle
\setcounter{tocdepth}{2}
\tableofcontents


\setchemfig{atom style={scale=0.75}}

\section{Neurotransmission}
\label{sec:org0129113}
\subsection{Introduction}
\label{sec:org5e61628}
\begin{itemize}
\item classical neurotransmitter systems involve:
\begin{itemize}
\item inhibitory
\begin{description}
\item[{aminoacidergic}] \(\gamma\)-aminobutyric acid (GABA) and glycine
\end{description}
\item excitatory
\begin{description}
\item[{aminoacidergic}] aspartate and glutamate
\item[{cholinergic}] acetylcholine
\item[{monoaminergic}] adrenaline, noradrenaline, dopamine, and serotonin
\item[{purinergic}] adenosine and adenosine mono, di, and triphosphate
\end{description}
\end{itemize}

\item GABA is formed from glutamic acid by glutamic acid decarboxylase
(Figure \ref{fig:org3fe932d})
\item catabolized into succinic acid through the sequential action of two
mitochondrial enzymes:
\begin{itemize}
\item GABA transaminase and succinic semialdehyde dehydrogenase
\end{itemize}
\item glutamic acid decarboxylase and GABA transaminase require PLP as a coenzyme
\begin{itemize}
\item PLP also participates in the synthesis of dopamine and serotonin
\item many other pathways including the glycine cleavage system
\end{itemize}
\item \textbf{GABA is a major inhibitory neurotransmitter} present in high
concentration in the CNS
\begin{itemize}
\item predominantly in the gray matter
\end{itemize}
\item GABA modulates brain activity by binding to
sodium-independent high-affinity  mostly GABA A receptors
\item \textbf{glycine is one of the major inhibitory neurotransmitters} in the CNS
\item inhibitory glycine receptors are mostly found in the brain stem
and spinal cord
\item \textbf{glutamate is the major excitatory neurotransmitter} in the brain
\begin{itemize}
\item requires rapid uptake to replenish intracellular
neuronal pools following extracellular release
\end{itemize}
\item inborn errors of neurotransmission include neurotransmitter
metabolism, receptor and transporter defects:
\begin{itemize}
\item three defects of GABA catabolism have been reported (section \ref{sec:org5b57d30})
\item three receptor defects (section \ref{sec:orgfb8d14c})
\item three transportopathies are reported (section \ref{sec:orgfb8d14c})
\item six disorders of monoamine metabolism (section \ref{sec:org84a2137})
\item see Amino Acids for disorders of glycine metabolism
\end{itemize}
\end{itemize}
\subsection{GABA catabolism}
\label{sec:org5b57d30}
\begin{itemize}
\item three defects of \(\gamma\)-aminobutyric acid (GABA) catabolism have been reported:
\begin{enumerate}
\item GABA transaminase deficiency (very rare, severe, and untreatable)
\item succinic semialdehyde dehydrogenase (SSADH) deficiency
\item homocarnosinosis
\end{enumerate}
\end{itemize}

\begin{figure}[htbp]
\centering
\includegraphics[width=0.9\textwidth]{neuro/figures/gaba.png}
\caption{\label{fig:org3fe932d}Brain Metabolism of GABA: 1 glutamic acid decarboxylase; 2 GABA transaminase; 3 succinic semialdehyde dehydrogenase}
\end{figure}

\begin{figure}[htbp]
\centering
\includegraphics[width=0.9\textwidth]{neuro/figures/Slide23.png}
\caption{\label{fig:orga7c9819}\(\gamma\)-Aminobutryic Acid Metabolism}
\end{figure}

\subsubsection{Succinic Semialdehyde Dehydrogenase Deficiency}
\label{sec:orgf39be11}
\begin{itemize}
\item first reported as \(\gamma\)-hydroxybutyric aciduria (4-hydroxybutyric
aciduria)
\item most prevalent of the disorders of GABA metabolism
\end{itemize}

\begin{enumerate}
\item Clinical Presentation
\label{sec:orgd5da949}
\begin{itemize}
\item non-specific clinical manifestations including:
\begin{itemize}
\item developmental delay and early-onset hypotonia
\item later-onset expressive language impairment
\item obsessive-compulsive disorder
\item hyporeflexia, non-progressive ataxia, and epilepsy
\end{itemize}
\end{itemize}

\item Metabolic Derangement
\label{sec:org5a63a77}
\begin{itemize}
\item key feature is an accumulation of \(\gamma\)-hydroxybutyrate in urine,
plasma, and CSF
\item \(\gamma\)-hydroxybutyrate and GABA (\(\uparrow\) 3x in CSF) are
neuropharmacologically active compounds
\item additional biochemical abnormalities relate to GABA, succinic semialdehyde, and GHB
\begin{itemize}
\item \(\uparrow\) CSF homocarnosine
\item \(\uparrow\) CSF guanidinobutyrate
\item \(\uparrow\) urine D-2-hydroxyglutarate
\item \(\uparrow\) urine succinic semialdehyde
\item \(\uparrow\) urine 4,5-dihydroxyhexanoic acid
\end{itemize}
\item many of these intermediates likely derive from succinic semialdehyde
\item \(\downarrow\) glutathione, the major intracellular antioxidant
\end{itemize}

\item Genetics
\label{sec:org0d7d846}
\begin{itemize}
\item AR ALDH5A1
\end{itemize}

\item Diagnostic Tests
\label{sec:org4463d77}
\begin{itemize}
\item \(\uparrow\) urine \(\gamma\)-hydroxybutyric acid (GHB)
\item \(\uparrow\) urine 4,5-dihydroxyhexanoic acid
\begin{itemize}
\item both the free and lactone form (threo-, erythro-) in the organic acid profile
is highly suggestive
\end{itemize}
\item molecular diagnosis via sequencing of ALDH5A1
\item GHB is also employed clinically (Xyrem R) for cataplexy and
illicitly for induction of euphoria
\end{itemize}

\item Treatment and Prognosis
\label{sec:org1f009ac}
\begin{itemize}
\item sudden death can occur in patients often in association with
epilepsy (sudden unexplained death in epilepsy or SUDEP)
\item therapeutic intervention has historically employed vigabatrin
\begin{itemize}
\item an irreversible inhibitor of GABA-transaminase
\item beneficial in some
\item remains to be determined whether enhancing GABA levels in SSADHD
(which are already elevated) is prudent
\end{itemize}
\end{itemize}
\end{enumerate}

\subsection{Receptors and Transporters of Neurotransmitters}
\label{sec:orgfb8d14c}
\begin{itemize}
\item receptors defects:
\begin{itemize}
\item dominantly inherited defect of the \(\alpha\)1-subunit of the glycine
receptor causes  causes hyperekplexia (excessive startle responses)
\begin{itemize}
\item treatable with clonazepam
\end{itemize}
\item mutations in GABA A receptor are a cause of dominantly inherited
epilepsy
\item mutations in glutamate receptors associate with neurodevelopmental
and psychiatric disorders
\end{itemize}
\item three transportopathies are reported:
\begin{itemize}
\item mitochondrial glutamate transporter defect
\begin{itemize}
\item a cause of severe epileptic encephalopathy
\end{itemize}
\item diseases that produce early parkinsonism-dystonia:
\begin{itemize}
\item dopamine transporter defect
\item vesicular monoamine transporter type 2 defect
\end{itemize}
\end{itemize}
\end{itemize}
\subsection{Monoamine Metabolism}
\label{sec:org84a2137}
\begin{itemize}
\item the monoamines: adrenaline, noradrenaline, dopamine, and serotonin
are metabolites of the amino acids tyrosine and tryptophan
\item dopaminergic modulation of ion fluxes regulates emotion, activity,
behaviour, nerve conduction, and the release of a number of hormones
via G-protein-coupled cell-surface dopamine
receptors
\item serotoninergic neurotransmission modulates body temperature, blood
pressure, endocrine secretion, appetite, sexual behaviour, movement,
emesis, and pain
\item first step in formation is catalysed by amino acid specific
hydroxylases which require tetrahydrobiopterin (BH\textsubscript{4}) as a cofactor
(Figure \ref{fig:org4f261b8})
\item synthesis of BH\textsubscript{4} from GTP is initiated by the rate-limiting GTP
cyclohydrolase (GTPCH)
\begin{itemize}
\item forms dihydroneopterin triphosphate
\item see Phenylalanine
\end{itemize}
\item BH\textsubscript{4} is also a cofactor of phenylalanine hydroxylase
\item B\textsubscript{6}-dependent aromatic L-amino acid decarboxylase (AADC) converts:
\begin{itemize}
\item L-dopa \(\to\) dopamine, the precursor of the catecholamines,
adrenaline and noradrenaline
\item 5-hydroxytryptophan (5-HTP) to serotonin (AKA 5-hydroxytryptamine)
\end{itemize}
\item monoamino oxidase A (MAO-A) is involved in the catabolism of
\begin{itemize}
\item adrenaline to vanillylmandelic acid (VMA)
\item noradrenaline to 3-methoxy-4-hydroxyphenylethyleneglycol (MHPG)
\item dopamine to homovanillic acid (HVA) via 3-methoxytyramine (3-MT)
\item serotonin to 5-hydroxyindoleacetic acid (5-HIAA)
\end{itemize}
\end{itemize}


\begin{figure}[htbp]
\centering
\includegraphics[width=0.9\textwidth]{neuro/figures/Slide21.png}
\caption{\label{fig:orgf898b60}Tetrahydrobiopterin Metabolism}
\end{figure}

\begin{figure}[htbp]
\centering
\includegraphics[width=0.9\textwidth]{neuro/figures/Slide22.png}
\caption{\label{fig:org4f261b8}Biogenic Amines}
\end{figure}

\begin{figure}[htbp]
\centering
\includegraphics[width=1.0\textwidth]{neuro/figures/monoamines.png}
\caption{\label{fig:orga3e8847}Metabolism of Adrenaline, Noradrenaline, Dopamine and Serotonin}
\end{figure}


\subsubsection{Tyrosine Hydroxylase (TH) Deficiency}
\label{sec:orgcd04e18}
\begin{itemize}
\item impairs synthesis of L-dopa and causes a neurological disease with
prominent extrapyramidal signs
\item variable response to L-dopa
\item \(\downarrow\) CSF HVA
\item normal CSF 5-HIAA
\item \(\downarrow\) CSF HVA/5-HIAA
\begin{itemize}
\item most sensitive marker
\end{itemize}
\end{itemize}
\subsubsection{Dopamine \(\beta\)-Hydroxylase Deficiency}
\label{sec:orgb6c1fe5}
\begin{itemize}
\item severe orthostatic hypotension with sympathetic failure
\item \(\Downarrow\) plasma noradrenaline and adrenaline
\item \(\uparrow\) plasma dopamine
\end{itemize}
\subsubsection{Aromatic-Amino Acid Decarboxylase (AADC) Deficiency}
\label{sec:orgb1a5f74}
\begin{itemize}
\item located upstream of the neurotransmitter amines
\begin{itemize}
\item \(\downarrow\) synthesis of seratonin and dopamine
\item treatment can be challenging
\end{itemize}
\item \(\downarrow\) CSF HVA and 5-HIAA
\item \(\uparrow\) CSF 3-O-methyl-l-dopa and 5-HTP
\item normal pterin concentrations
\item \(\uparrow\) urine vanillyllactic acid in organic acids
\end{itemize}
\subsubsection{Monoamine-Oxidase A (MAO-A) Deficiency}
\label{sec:orge99e20f}
\begin{itemize}
\item located downstream, mainly causes behavioural disturbances
\item no effective treatment is known
\item \(\uparrow\) urine serotonin, normetanephrine, metanephrine and 3-MT
\end{itemize}
\subsubsection{GTPCH and SR Deficiency}
\label{sec:orgcc9701c}
\begin{itemize}
\item dopamine responsive dystonia (Figure \ref{fig:orgf898b60} and Table \ref{tab:org74bfd4c})
\item GTPCH catalyses the first step in BH\textsubscript{4} synthesis
\ce{GTP ->[GTPCH] NH2TP}
\item sepiaterin reductase (SR) catalyses the final steps in BH\textsubscript{4} synthesis
\ce{PTP ->[SR] BH4}
\item pterin disorders upstream of L-dopa and 5-hydroxytryptophan
(5-HTP) with normal baseline phenylalaninemia and effective
treatment
\begin{itemize}
\item GTPCH labs
\begin{itemize}
\item \(\downarrow\) CSF biopterin
\item \(\downarrow\) CSF neopterin
\item normal or \(\downarrow\) CSF HVA
\item \(\downarrow\) CSF 5-HIAA
\item normal or \(\uparrow\) plasma Phe
\end{itemize}
\item SR labs
\begin{itemize}
\item \(\uparrow\) CSF biopterin
\item \(\uparrow\) CSF sepiapterin
\item normal CSF neopterin
\item \(\Downarrow\) CSF HVA
\item \(\Downarrow\) CSF 5-HIAA
\item normal plasma Phe
\end{itemize}
\end{itemize}
\item treatment
\begin{itemize}
\item L-dopa
\item 5-OH tryptophan
\item BH\textsubscript{4}
\item \(\downarrow\) Phe
\end{itemize}
\end{itemize}

\begin{table}[htbp]
\caption{\label{tab:org74bfd4c}Results in Biopterin Disorders}
\centering
\begin{tabular}{lrlllll}
Deficiency & Phe & biopterin\footnotemark & neopterin\textsuperscript{\ref{orgc23c73e}} & primapterin\textsuperscript{\ref{orgc23c73e}} & CSF 5-HIAA HVA & DHPR activity\\
\hline
PAH & \textgreater{}120 & \(\uparrow\) & \(\uparrow\) & - & N & N\\
GTPCH & 50-1200 & \(\Downarrow\) & \(\Downarrow\) & - & \(\downarrow\) & N\\
PTPS & 240-2500 & \(\Downarrow\) & \(\Uparrow\) & - & \(\downarrow\) & N\\
DHPR & 180-2500 & \(\Downarrow\) & N or \(\uparrow\) & - & \(\downarrow\) & \(\downarrow\)\\
PCD & 180-1200 & \(\downarrow\) & \(\uparrow\) & \(\Uparrow\) &  & N\\
SR & N & N & N & N & \(\Downarrow\) & N\\
\end{tabular}
\end{table}\footnotetext[1]{\label{orgc23c73e}blood or urine}
\section{Peptides}
\label{sec:orgae6330d}
\subsection{TMA and Choline}
\label{sec:org3a59806}
\subsubsection{Trimethylamine Metabolism}
\label{sec:org35873cf}
\begin{itemize}
\item trimethylamine (TMA) is a volatile tertiary amine which smells of rotting fish
\item bacterial metabolite which is produced by anaerobes in the colon
from choline (present in lecithin), carnitine, betaine, and from
trimethylamine-N-oxide (TMAO) in marine fish and shellfish
\item TMA is absorbed from the intestine and oxidised in the liver by
flavin-containing monooxygenase 3 (FMO3)
\begin{itemize}
\item product is TMA-N-oxide (TMAO) which is non-odorous and is
excreted in urine
\end{itemize}
\item \textgreater{} 90\% of the absorbed TMA is oxidised
\end{itemize}

\begin{figure}[htbp]
\centering
\includegraphics[width=0.9\textwidth]{peptides/figures/tma.png}
\caption{\label{fig:org022ab32}Trimethlyamine Metabolism}
\end{figure}

\subsubsection{Catabolism of Choline}
\label{sec:orgeeb2b5d}
\begin{itemize}
\item choline is an essential nutrient required for synthesis of:
\begin{itemize}
\item choline phospholipids
\item acetyl choline
\item SAM
\end{itemize}
\item catabolism occurs within mitochondria
\begin{itemize}
\item involves the sequential removal of two methyl groups by:
\begin{itemize}
\item dimethylglycine dehydrogenase (DMGDH)
\item sarcosine dehydrogenase (SDH)
\end{itemize}
\item flavin enzymes with covalently linked FAD which use folate as co-factor
\end{itemize}
\item methyl groups from dimethylglycine and sarcosine are transferred to
tetrahydrofolate (THF)
\begin{itemize}
\item forming 5,10-methylene THF
\end{itemize}
\end{itemize}

\begin{figure}[htbp]
\centering
\includegraphics[width=0.9\textwidth]{peptides/figures/choline.png}
\caption{\label{fig:org0b61018}Choline Catabolism}
\end{figure}

\subsubsection{Trimethylaminuria (Fish Malodour Syndrome)}
\label{sec:orge4547e4}
\begin{enumerate}
\item Clinical Presentation
\label{sec:orgd47ee3b}
\begin{itemize}
\item TMA is excreted in the breath, sweat, urine and vaginal secretions
\item individuals with trimethylaminuria (TMAU) excrete excessive amounts
of TMA and when this happens have an unpleasant, pervasive body
odour of decaying fish
\begin{description}
\item[{Severe TMAU}] null mutation, onset in infancy
\item[{Mild TMAU}] polymorphisms, adult onset
\item[{Transient Childhood}] choline supplementation, mild variants, 2\degree{} to illness
\end{description}
\end{itemize}

\item Metabolic Derangement
\label{sec:org9efaa84}
\begin{itemize}
\item FMO3 deficiency
\item \(\uparrow\) urine TMA results when the oxidative capacity of FMO3 is
overwhelmed by TMA substrate
\end{itemize}

\item Genetics
\label{sec:org70eede4}
\begin{itemize}
\item AR FMO3 \textasciitilde{}1\% carrier freq
\end{itemize}

\item Diagnostic Tests
\label{sec:org2f4f10f}
\begin{itemize}
\item common causes of abnormal body odour and UTIs are excluded
\item random urine sample is collected if the odour is obvious and not
menstruating
\item otherwise urine is collected after a high choline meal (eggs, baked
beans, marine fish)
\item urine is acidified quickly to pH 2.0 and frozen until analysis
\begin{itemize}
\item NMR spectroscopy or mass spectrometry
\item TMA, TMAO and total TMA
\item \(\uparrow\) TMA
\item \(\downarrow\) TMAO:TMA
\end{itemize}
\end{itemize}

\item Treatment
\label{sec:orgd4c7a3b}
\begin{itemize}
\item \(\downarrow\) choline diet
\item \(\uparrow\) folate
\end{itemize}
\end{enumerate}

\subsection{Glutathione}
\label{sec:orgd98d7cd}
\begin{itemize}
\item glutathione is a tripeptide consisting of glutamate, cysteine and
glycine
\item one of the most important antioxidants
\begin{itemize}
\item drug metabolism, free-radical scavenging, biosynthesis of DNA and
proteins as well as amino acid transport
\end{itemize}
\item synthesized and metabolized in the \(\gamma\)-glutamyl cycle (Figure
\ref{fig:orgbf2836c})
\item six enzymes involved in synthesis and turnover
\item synthesised from glutamate by sequential actions of
\(\gamma\)-glutamylcysteine synthetase and glutathione
synthetase
\item degradation involves four enzymes:
\begin{itemize}
\item \(\gamma\)-glutamyl transpeptidase initiates the breakdown by
catalysing the transfer of its \(\gamma\)-glutamyl-group to
acceptors
\item \(\gamma\)-glutamyl residues are substrates of the
\(\gamma\)-glutamyl-cyclotransferase which converts them to
5-oxoproline and the corresponding amino acids
\end{itemize}
\item conversion of 5-oxoproline to glutamate is catalysed by
5-oxoprolinase
\item a dipeptidase splits cysteinylglycine, which is formed in the
transpeptidation reaction, into glycine and cysteine
\item biosynthesis of glutathione is feedback regulated
\begin{itemize}
\item glutathione inhibits of \(\gamma\)-glutamylcysteine synthetase
\end{itemize}
\item genetic defects have been described in five of the six enzymes of
the \(\gamma\)-glutamyl cycle
\end{itemize}

\begin{figure}[htbp]
\centering
\includegraphics[width=0.9\textwidth]{peptides/figures/gsh.png}
\caption{\label{fig:orgbf2836c}The \(\gamma\)-Glutamyl Cycle}
\end{figure}


\begin{figure}[htbp]
\centering
\includegraphics[width=0.9\textwidth]{peptides/figures/Slide08.png}
\caption{\label{fig:org89618a6}The \(\gamma\)-Glutamyl Cycle}
\end{figure}


\subsubsection{Glutathione Synthetase Deficiency}
\label{sec:orgb665cb9}
\begin{enumerate}
\item Clinical Presentation
\label{sec:org750bfbb}
\begin{itemize}
\item classified as mild, moderate or severe
\begin{description}
\item[{mild}] mild hemolytic anemia
\item[{moderate}] present during the neonatal period, with severe and
chronic metabolic acidosis, hemolytic anemia, jaundice
and 5-oxoprolinuria
\item[{severe}] above plus progressive CNS symptoms
\end{description}
\end{itemize}

\item Metabolic Derangement
\label{sec:org93ba240}
\begin{itemize}
\item \textbf{glutathione synthetase (GS)} catalyses the last step of glutathione synthesis
\item deficiency \(\to\) \(\downarrow\) cellular glutathione and \(\uparrow\) \(\gamma\)-glutamylcysteine
\begin{itemize}
\item due to \(\downarrow\) feedback inhibition of \(\gamma\)-glutamylcysteine synthetase
\end{itemize}
\item \(\gamma\)-glutamylcysteine \(\to\) 5-oxoproline by \(\gamma\)-glutamyl
cyclotransferase
\item \(\Uparrow\) 5-oxoproline exceeds the capacity of 5-oxoprolinase
\item \(\uparrow\) 5-oxoproline \(\to\) metabolic acidosis and 5-oxoprolinuria
\end{itemize}

\item Genetics
\label{sec:org38201ef}
\begin{itemize}
\item AR GSS
\end{itemize}

\item Diagnostic Tests
\label{sec:orgda7605b}
\begin{itemize}
\item \(\uparrow\) urine 5-oxoproline
\item \(\downarrow\) RBC glutathione
\item \(\downarrow\) RBC or fibroblast GS activity
\begin{itemize}
\item 1-30\% of normal
\end{itemize}
\item mutation analysis
\end{itemize}

\item Treatment
\label{sec:orgb9da54e}
\begin{itemize}
\item management of GS deficient patients is aimed at correction of
acidosis, prevention of hemolytic crises and support of endogenous
defence against reactive oxygen species
\item bicarbonate in acute acidosis
\item blood transfusion for hemolysis
\item vitamins E and C
\end{itemize}
\end{enumerate}
\section{Metal}
\label{sec:org5512672}
\subsection{Copper}
\label{sec:org77c170f}
\begin{itemize}
\item copper is essential for cellular metabolism and toxic at [\(\Uparrow\)]
\item \textgreater{} 90\% of circulating copper is bound to ceruloplasmin
\item absorbed each day from the intestine
\item removed from portal circulation by hepatocytes
\item excretion of copper by liver into bile is only method of removal
\item copper transporter CTR1 responsible for copper uptake in enterocytes and hepatocytes
\item intracellular transport done by two related ATPases
\begin{description}
\item[{Menkes}] absorption : enterocytes : ATP7A
\item[{Wilson}] excretion : hepatocytes : ATP7B
\end{description}
\end{itemize}

\begin{figure}[htbp]
\centering
\includegraphics[width=1\textwidth]{metal/figures/copper.PNG}
\caption[copper]{\label{fig:org13a0e16}Copper Metabolism}
\end{figure}
\subsubsection{Menkes Disease}
\label{sec:org406c47e}
\begin{enumerate}
\item Clinical Presentation
\label{sec:org005ef82}
\begin{itemize}
\item male infants 2-3 months
\item neurodegeneration manifests as:
\begin{itemize}
\item seizures, hypotonia, loss of milestones
\end{itemize}
\item non-specific signs at birth:
\begin{itemize}
\item prematurity, large cephalhaematomas, skin laxity, hypothermia
\item hair breaks easily, sandpaper feel
\end{itemize}
\end{itemize}

\item Metabolic Derangement
\label{sec:org84a25bc}
\begin{itemize}
\item defect in \textbf{ATP7A}
\item normal copper uptake, can not be exported from enterocytes into circulation
\item insufficient copper for incorporation into \textasciitilde{}20 cuproenzymes including:
\begin{description}
\item[{lysloxidase}] collagen cross-linking
\item[{tyrosinase}] melanin formation
\item[{dopamine \(\beta\)-hydroxylase}] catacholamine biosynthesis
\item[{peptidyl glycine monooxygenase}] neuropeptide precursors
\item[{cytochrome c-oxidase}] ETC
\end{description}
\end{itemize}

\item Genetics
\label{sec:org38fc321}
\begin{itemize}
\item \textbf{XL ATB7A} 1:250,000
\begin{itemize}
\item 1/3 \emph{de novo} mutations
\end{itemize}
\item expressed in all tissues except liver
\end{itemize}

\item Diagnostic Tests
\label{sec:org0b27381}
\begin{itemize}
\item \(\downarrow\) serum copper
\item \(\downarrow\) serum ceruloplasmin 
\begin{itemize}
\item not specific in 0-3 months of life
\end{itemize}
\item \(\uparrow\) plasma dopamine/norepinephrine
\item \(\uparrow\) copper retention in cultured fibroblasts
\end{itemize}

\item Treatment
\label{sec:orgf594837}
\begin{itemize}
\item often fatal < 3 years
\begin{itemize}
\item infection or vascular complications
\end{itemize}
\item parenteral treatment should bypass ATP7A
\begin{itemize}
\item disappointing results
\item near normal intellectual and motor development only possible with
residual ATP7A activity
\end{itemize}
\end{itemize}
\end{enumerate}
\subsubsection{Wilson Disease}
\label{sec:org9505bbc}
\begin{enumerate}
\item Clinical Presentation
\label{sec:org3539a04}
\begin{itemize}
\item hepatic symptoms 8-20 years
\item neurological symptoms 2nd-3rd decade
\item suspect in patients with:
\begin{itemize}
\item liver disease w no obvious cause
\item movement disorder
\end{itemize}
\item occasionally isolated:
\begin{itemize}
\item \(\uparrow\) transaminases
\item Kayser-Fleischer rings
\item hemolysis
\end{itemize}
\item diagnosis often made in siblings of patient
\end{itemize}

\item Metabolic Derangement
\label{sec:org47c3906}
\begin{itemize}
\item \textbf{ATP7B} has 6 copper binding domains
\item expressed predominantly in liver and kidney
\item defect in trans-Golgi protein ATP7B
\begin{itemize}
\item required for excretion of copper and incorporation into ceruloplasmin
\end{itemize}
\item \(\downarrow\) t\textsubscript{1/2} of ceruloplasmin with out bound copper
\item rare patients with excretion defect and normal ceruloplasmin binding
\item \(\downarrow\) excretion of copper into bile
\item accumulation of copper in liver
\begin{itemize}
\item secondary accumulation in brain, kidney and eyes
\end{itemize}
\end{itemize}

\item Genetics
\label{sec:org2a99e22}
\begin{itemize}
\item AR ATP7B
\item 1:30,000 may be higher
\item \(\sim\) 1/90 carriers
\end{itemize}

\item Diagnostic tests
\label{sec:org10c4f6e}
\begin{itemize}
\item \(\downarrow\) serum ceruloplasmin
\item \(\downarrow\) serum copper
\item \(\uparrow\) urine copper
\item \(\uparrow\) liver copper
\item \(\uparrow\) free copper
\begin{itemize}
\item 1 mg ceruloplasmin contains 3.4 ug copper
\end{itemize}
\item results should be taken together, there is a scoring system \footnote{Clinical Practice Guidelines: Wilson's Disease, J Hepatol 56:671-685}
\item genetic analysis in family
\item possible candidate for NBS
\end{itemize}

\item Treatment
\label{sec:org47844a5}
\begin{itemize}
\item excellent prognosis if treated before severe damage
\item penicillamine chelates copper and is excreted in urine
\item oral zinc induces metallothionein synthesis
\begin{itemize}
\item metallothionein binds copper preferentially to zinc
\item fecal excretion
\end{itemize}
\item trien (triethylenetetramine) is a chelator
\begin{itemize}
\item used when penicillamine is not tolerated
\end{itemize}
\item combination therapy should be staggered - don't chelate treatment
\end{itemize}
\end{enumerate}

\subsection{Iron}
\label{sec:org6586232}
\begin{itemize}
\item iron is essential for the synthesis of haem and other
metalloproteins
\begin{itemize}
\item iron sulfur cluster has a crucial role in mitochondrial metabolism (Figure \ref{fig:org51afa83})
\end{itemize}
\item \textgreater{} 20 mg of iron per day is required, only 1–2 mg from intestinal absorption,
\begin{itemize}
\item the remainder is re-used
\end{itemize}
\item not actively secreted from the body
\item \(\uparrow\) [iron] \(\to\) \(\uparrow\) [circulating free iron]
\begin{itemize}
\item primarily taken up by the liver, the pancreas and the heart
\item syndromes manifest with cirrhosis, diabetes and cardiomyopathy
\end{itemize}
\item absorption of iron occurs primarily in the duodenum via the
divalent-metal transporter (DMT1)
\item major recycling route for iron is removal from erythrocytic heme by
hemeoxygenase
\begin{itemize}
\item in macrophages and enterocytes
\end{itemize}
\item circulating free iron binds to to apo-transferrin forming holo-transferrin
\item transferrin can only bind iron in the ferric state
\item ceruloplasmin catalyses the oxidization of Fe\textsuperscript{2+} into Fe\textsuperscript{3+}
\item transferrin receptor mediates the uptake of transferrin
\item hepcidin inhibits iron absorption by binding to ferroportin
\item iron is released from intracellular transferrin by a specific isoform of DMT1
\item iron can be stored bound to ferritin until needed in several cell
types, including macrophages
\end{itemize}

\begin{figure}[htbp]
\centering
\includegraphics[width=0.9\textwidth]{metal/figures/iron_met.png}
\caption[iron]{\label{fig:orgc025ea3}Iron Metabolism}
\end{figure}

\begin{figure}[htbp]
\centering
\includegraphics[width=0.9\textwidth]{metal/figures/fes.png}
\caption[fes]{\label{fig:org51afa83}Iron Sulfur Cluster Proteins}
\end{figure}

\subsubsection{Classic Hereditary Haemochromatosis}
\label{sec:orgc194558}
\begin{enumerate}
\item Clinical Presentation
\label{sec:org77ece2e}
\begin{itemize}
\item also called Type 1 or HFE related haemochromatosis
\item slow but progressive accumulation of iron in various organs
\item clinically apparent by the fourth or fifth decade of life
\item initial symptoms are nonspecific and include:
\begin{itemize}
\item fatigue, weakness, abdominal pain, weight loss and arthralgia
\end{itemize}
\item increased awareness, and improved diagnostic testing
\begin{itemize}
\item classic symptoms of full-blown haemochromatosis are rarely seen
\begin{itemize}
\item liver fibrosis and cirrhosis, diabetes, cardiomyopathy,
hypogonadotrophic hypogonadism, arthropathy and skin
pigmentation
\end{itemize}
\end{itemize}
\end{itemize}

\item Metabolic Derangement
\label{sec:org543f291}
\begin{itemize}
\item caused by a disturbance in iron homeostasis associated with hepcidin
deficiency and systemic accumulation of iron
\item exact role of HFE is unclear
\begin{itemize}
\item sensing iron levels and thus indirectly for regulating hepcidin
synthesis
\end{itemize}
\end{itemize}

\item Genetics
\label{sec:org95da791}
\begin{itemize}
\item AR HFE
\item \textasciitilde{} 0.5\% of the Northern European population are homozygous for the
C282Y mutation in HFE
\begin{itemize}
\item only 5\% of male and <1\% of female C282Y homozygotes eventually
develop liver fibrosis or cirrhosis
\end{itemize}
\end{itemize}

\item Diagnostic Tests
\label{sec:orgb2fdd49}
\begin{itemize}
\item \(\Uparrow\) transferrin saturation initially
\item followed by \(\Uparrow\) serum ferritin
\begin{itemize}
\item reflects increasing iron overload
\end{itemize}
\item genetic testing of HFE should be performed when:
\begin{itemize}
\item transferrin saturation is above 45\%
\item serum ferritin is elevated:
\begin{itemize}
\item >200 ng/ml in adult females
\item >300 ng/ml in adult males
\end{itemize}
\end{itemize}
\end{itemize}

\item Treatment and Prognosis
\label{sec:org1ea6cd4}
\begin{itemize}
\item \textgreater{} half male and female C282Y homozygotes have normal serum
ferritin levels and may never require therapy
\item many have moderately elevated serum ferritin levels
\begin{itemize}
\item unclear whether all should have regular
phlebotomies to reduce systemic iron load
\end{itemize}
\item serum ferritin levels exceeding 1000 ng/ml a phlebotomy regimen is clearly
necessary.
\begin{itemize}
\item in adults initially 500 ml blood is removed weekly or bi-weekly
\item phlebotomy frequency is usually reduced to once every 3-6 months
when serum ferritin levels are \textless{} 50 ng/ml
\end{itemize}
\end{itemize}
\end{enumerate}

\subsubsection{Systemic Iron Overload Syndromes}
\label{sec:org59ba295}
\begin{enumerate}
\item Juvenile Hereditary Haemochromatosis (Type 2)
\label{sec:org1645477}
\begin{itemize}
\item most severe type of hereditary haemochromatosis
\begin{itemize}
\item probably because hepcidin deficiency is more pronounced
\end{itemize}
\item patients present in the 2nd and 3rd decade
\begin{itemize}
\item mostly w hypogonadotropic hypogonadism and cardiomyopathy due to
iron overload
\end{itemize}
\item type 2A is caused by mutations in the HJV gene encoding for hemojuvelin
\begin{itemize}
\item necessary for proper hepcidin synthesis
\end{itemize}
\item type 2B from mutations in the HAMP gene encoding hepcidin
\item \(\uparrow\) serum ferritin and transferrin saturation as in
classic HFE-related haemochromatosis
\item diagnosis is made by mutation analysis
\item phlebotomy is the treatment of choice and may prevent organ damage
if initiated early
\end{itemize}

\item TFR2-Related Hereditary Haemochromatosis (Type 3)
\label{sec:orgc07560d}
\begin{itemize}
\item transferrin receptor 2 is important for sensing the intracellular
iron status (e.g erythroid cells)
\item iron overload phenotype which resembles classic HFE-related
haemochromatosis
\begin{itemize}
\item patients are generally younger
\end{itemize}
\item \(\downarrow\) hepcidin
\item \(\uparrow\) transferrin saturation
\item \(\uparrow\) ferritin
\item \(\uparrow\) liver iron content
\item diagnosis made by mutation analysis
\item phlebotomy is the treatment of choice
\end{itemize}

\item Ferroportin Related Hereditary Haemochromatosis (Type 4)
\label{sec:org3ac9510}
\begin{itemize}
\item differs in several ways from the other three subtypes of haemochromatosis
\item AD inheritance and caused by mutations in SLC40A1 encoding ferroportin
\item expressed at the enterocyte and plasma membrane of macrophages
\item loss of function mutations impair the export of iron from macrophages causing an iron
deficiency in erythrocytic precursors
\item patients present with a combination of mild microcytic anaemia with
low transferrin saturation
\begin{itemize}
\item iron overload predominantly in macrophages
\item tolerance to phlebotomy is limited by the concurrent anaemia
\end{itemize}
\item gain of function mutations cause resistance to feedback inhibition
by hepcidin
\begin{itemize}
\item these patients present with a more classic hepatic iron overload
haemochromatosis phenotype
\end{itemize}
\end{itemize}

\item Neonatal Haemochromatosis
\label{sec:orga362d7b}
\begin{itemize}
\item once thought to be an AR inherited disorder now recognized as
acquired
\begin{itemize}
\item any disease state that chronically prevents the synthesis or
activity of hepcidin will lead to haemochromatosis.
\end{itemize}
\item patients present in the first few weeks of life with severe liver
failure
\item caused by a maternal allo-immune reaction to the infant liver
\begin{itemize}
\item starts \emph{in utero}
\end{itemize}
\item liver injury leads to a decrease in hepcidin
\begin{itemize}
\item manifests as severe siderosis of both liver and extrahepatic organs
\end{itemize}
\item diagnosis is made in any child with neonatal liver failure in
combination with high serum ferritin and extrahepatic siderosis
\begin{itemize}
\item shown by MRI and/or oral mucosal biopsy
\item iron deposits in minor salivary glands in patients with NH
\end{itemize}
\item therapy is by exchange transfusion in combination with IVIGs to
remove/bind maternally derived IgG
\item risk of recurrence in a subsequent pregnancy from a mother who has
given birth to an affected child is as high as 90\%
\begin{itemize}
\item recurrence risk reduced by IVIGs during pregnancy
\end{itemize}
\end{itemize}
\end{enumerate}

\subsubsection{Iron Deficiency and Distribution Disorders}
\label{sec:orgaae423a}
\begin{enumerate}
\item Iron-Refractory Iron Deficiency Anaemia (IRIDA)
\label{sec:orga6e6916}
\begin{itemize}
\item caused by a deficiency of \textbf{matriptase-2}
\item AR TMPRSS6
\item normal cleavage of haemojuvelin is interrupted \(\to\) \(\uparrow\) hepcidin
\item presents in youth
\begin{itemize}
\item iron deficiency
\item \(\downarrow\) transferrin saturation (<10\%)
\item microcytic anaemia
\end{itemize}
\item oral iron supplementation is not effective
\begin{itemize}
\item high hepcidin levels will prevent iron release from the
enterocytes
\item requires intravenous iron therapy
\end{itemize}
\end{itemize}

\item Atransferrinaemia
\label{sec:org988c129}
\begin{itemize}
\item rare
\item AR TF
\item presents with moderate to severe hypochromic microcytic anaemia and
growth retardation along with signs of haemochromatosis
\item \(\downarrow\) serum transferrin
\item \(\uparrow\) serum ferritin
\item treated with plasma infusions to increase the transferrin pool
\end{itemize}
\end{enumerate}
\section{Purine and Pyrimidine Metabolism}
\label{sec:orgd399b08}
\begin{figure}[htbp]
\centering
\includegraphics[width=1\textwidth]{pp/figures/Slide17.png}
\caption{\label{fig:org29bc090}Purine and Pyrimidine Metabolism}
\end{figure}

\subsection{Purine Metabolism}
\label{sec:org7c5481d}
\subsubsection{Introduction}
\label{sec:org1590613}
\begin{itemize}
\item purine nucleotides are involved in energy transfer, metabolic
regulation, and synthesis of DNA and RNA
\item purine metabolism can be divided into three pathways:
\begin{itemize}
\item biosynthetic pathway often termed \emph{de novo}
\begin{itemize}
\item starts with the formation of phosphoribosyl pyrophosphate (PRPP)
and leads to the synthesis of IMP
\item IMP conversion to AMP and GMP
\begin{itemize}
\item \(\to\) di- and triphosphates and deoxyribonucleotides
\end{itemize}
\end{itemize}
\item catabolic pathway
\begin{itemize}
\item starts from GMP, IMP and AMP
\item produces uric acid, a poorly soluble compound, which tends to
crystallize once its plasma \(\ge\) 0.4 mmol/L
\end{itemize}
\item salvage pathway
\begin{itemize}
\item utilizes the purine bases guanine, hypoxanthine and adenine
\begin{itemize}
\item provided by food or the catabolic pathway
\end{itemize}
\item converts them into GMP, IMP and AMP
\item also salvage of purine nucleosides, adenosine and guanosine, and
their deoxy counterparts, catalyzed by kinases
\item also converts several pharmacological anticancer and antiviral
nucleoside analogs into their active forms
\end{itemize}
\end{itemize}

\item inborn errors of purine metabolism comprise defects or
superactivities of:

\begin{itemize}
\item purine nucleotide synthesis and interconversion
\begin{itemize}
\item phosphoribosyl pyrophosphate synthetase (PRS) superactivity and deficiency
\item adenylosuccinase (ADSL) deficiency
\item AICA-ribosiduria caused by ATIC deficiency
\end{itemize}
\item purine catabolism
\begin{itemize}
\item muscle AMP deaminase (AMPD, also termed myoadenylate deaminase)
\item adenylate kinase (AK)
\item adenosine deaminase (ADA)
\item purine nucleoside phosphorylase (PNP)
\item xanthine oxidase (XO, also termed xanthine dehydrogenase)
\end{itemize}
\item purine salvage
\begin{itemize}
\item hypoxanthine-guanine phosphoribosyltransferase (HPRT)
\item adenine phosphoribosyltransferase (APRT)
\item adenosine kinase (ADK)
\end{itemize}
\item deoxyguanosine kinase (DGUOK) deficiency causes mitochondrial DNA depletion
\item thiopurine S-methyltransferase (TPMT) deficiency results in less
efficient methylation
\begin{itemize}
\item \(\therefore\) enhanced toxicity of pharmacologic thiopurine analogs
\end{itemize}
\item inosine triphosphate pyrophosphatase (IT-Pase) deficiency
increases the toxicity of thiopurines
\end{itemize}
\item with the exception of the deficiencies of muscle AMPD and TPMT, all
these inborn errors are very rare
\end{itemize}

\begin{figure}[htbp]
\centering
\includegraphics[width=0.9\textwidth]{pp/figures/purine_met.png}
\caption{\label{fig:org433c067}Purine Metabolism:1 PRPP synthetase; 2 adenylosuccinase; 3 AICAR transformylase; 4 IMP cyclohydrolase; 5 adenylosuccinate synthetase; 6 AMP deaminase; 7 5‘-nucleotidase; 8 adenosine deaminase; 9 purine nucleoside phosphorylase; 10 hypoxanthine-guanine phosphoribosyltransferase; 11 adenine phosphoribosyltransferase; 12 adenosine kinase; 13 guanosine kinase; 14 xanthine oxidase}
\end{figure}

\subsubsection{Phosphoribosyl Pyrophosphate Synthetase Superactivity}
\label{sec:org93d69ef}
\begin{enumerate}
\item Clinical Presentation
\label{sec:org0a9b214}
\begin{itemize}
\item gouty arthritis and/or uric acid lithiasis in young males
\end{itemize}

\item Metabolic Derangement
\label{sec:org1e072c3}
\begin{itemize}
\item PRPP is the first intermediate in \emph{de novo} synthesis of purine nucleotides
\end{itemize}
\ce{ribose-5-phosphate + ATP ->[PRPS] PRPP}
\begin{itemize}
\item PRPP synthetase is highly regulated
\item two proposed mechanisms to explain PRPS1 superactivity
\begin{enumerate}
\item gain-of-function point mutations in the PRPS1 gene resulting in an
altered regulatory site
\item \(\uparrow\) expression of PRPS1 with normal kinetic properties
\end{enumerate}
\item PRPP amidotransferase is the first and rate-limiting enzyme of the
\emph{de novo} pathway
\begin{itemize}
\item \(\uparrow\) purine synthesis to \(\uparrow\) uric acid
\end{itemize}
\end{itemize}

\item Genetics
\label{sec:orgebb28e2}
\begin{itemize}
\item \textbf{XL PRPS1}
\end{itemize}

\item Diagnostic Tests
\label{sec:orge4792b8}
\begin{itemize}
\item extensive kinetic studies of the enzyme in erythrocytes and cultured
fibroblasts is performed in only a few laboratories in the world
\item molecular testing of PRPS1 gene
\end{itemize}

\item Treatment
\label{sec:orge14f111}
\begin{itemize}
\item treated with allopurinol, which inhibits xanthine oxidase, the last
enzyme of purine catabolism
\item results in a decrease of the production of uric acid and in its
replacement by
\begin{itemize}
\item hypoxanthine which is about 10-fold more soluble
\item xanthine which is slightly more soluble
\end{itemize}
\end{itemize}
\end{enumerate}

\subsubsection{Muscle Adenosine Monophosphate Deaminase 1 Deficiency}
\label{sec:org4b941f4}
\begin{itemize}
\item AKA myoadenylate deaminase deficiency
\end{itemize}
\begin{enumerate}
\item Clinical Presentation
\label{sec:org5fbc101}
\begin{itemize}
\item present in 1-2\% of the Caucasian population
\item majority asymptomatic
\item muscular weakness, fatigue, cramps or myalgias following moderate to
vigorous exercise
\item patients may display a gradual progression of their symptoms
\begin{itemize}
\item dressing and walking a few steps lead to fatigue and myalgias
\end{itemize}
\item sometimes accompanied by an increase in serum creatine kinase, myoglobinuria and minor electromyographic
abnormalities
\end{itemize}

\item Metabolic Derangement
\label{sec:org288c559}
\begin{itemize}
\item AMPD, adenylosuccinate synthetase and adenylosuccinase form the
purine nucleotide cycle
\end{itemize}
\ce{AMP + H2O ->[AMPD] IMP + NH3}

\item Genetics
\label{sec:orgd2092da}
\begin{itemize}
\item AR AMPD1
\end{itemize}

\item Diagnostic Tests
\label{sec:org1a18821}
\begin{itemize}
\item NIET
\begin{itemize}
\item several-fold elevation of venous plasma ammonia, seen in normal
subjects, is absent in AMPD deficiency
\end{itemize}
\end{itemize}

\item Treatment
\label{sec:org1801b88}
\begin{itemize}
\item ribose reported to improve muscular strength and endurance
\end{itemize}
\end{enumerate}
\subsubsection{Adenosine Deaminase 1 Deficiency}
\label{sec:orgae67836}
\begin{itemize}
\item two isoforms of adenosine deaminase (ADA)
\begin{itemize}
\item ADA1 is found in most cells, particularly lymphocytes and macrophages
\item ADA2 is predominant in plasma
\end{itemize}
\end{itemize}
\begin{enumerate}
\item Clinical Presentation
\label{sec:orgfa81a6a}
\begin{itemize}
\item clinical spectrum is very broad
\begin{itemize}
\item from a profound impairment of both humoral and cellular immunity
in infants, known SCID
\item to delayed and less severe later onset in older children or
adults
\item even benign partial ADA1 deficiency in adults
\end{itemize}
\item \textasciitilde{} 80\% of patients present within the first weeks or months after
birth
\item recurrent opportunistic infections caused by a variety of organisms,
which rapidly become life-threatening
\item infections are mainly localized in the skin, the respiratory, and the
gastrointestinal tract
\item \textgreater{} 6 months develop hypoplasia or apparent absence of lymphoid
tissue (tonsils, lymph nodes, thymus shadow on x-ray)
\item non-immunological symptoms are also found
\begin{itemize}
\item 50\% have bone abnormalities
\item cognitive, behavioural, and neurological abnormalities can present
\begin{itemize}
\item lower IQ, hyperactivity, attention deficits, spasticity, head
lag, nystagmus, inability to focus, and high frequency
sensorineural deafness
\end{itemize}
\end{itemize}
\item disease is progressive since residual B and T-cell function which
may be found at birth disappear later on
\end{itemize}
\item Metabolic Derangement
\label{sec:org364ff27}
\begin{itemize}
\item accumulation in body fluids of adenosine and deoxyadenosine
\begin{itemize}
\item normally \textasciitilde{}undetectable
\end{itemize}
\end{itemize}
\ce{adenosine + H2O ->[ADA1] inosine + NH3}
\begin{itemize}
\item \(\to\) premature death of lymphoid progenitor cells
\item \(\therefore\) impair generation of T, B, and NK lymphocytes
\item ADA deficiency affects to varying extent bone, brain, lung and liver
\end{itemize}

\item Genetics
\label{sec:org1c723e6}
\begin{itemize}
\item AR ADA1
\begin{itemize}
\item \textasciitilde{} 40\% of SCID
\end{itemize}
\end{itemize}

\item Diagnostic Tests
\label{sec:orgc10f3e6}
\begin{itemize}
\item SCID can be confirmed by relatively simple laboratory tests:
\begin{itemize}
\item lymphopenia involving B, T and natural killer (NK) cells
\item hypogammaglobulinemia
\item IgM deficiency may be detected early
\item IgG deficiency becomes manifest only after the age of 3 months
when the maternal supply has been exhausted
\end{itemize}
\item enzymatic diagnosis is mostly confirmed on red blood cells
\item severity of disease correlates with the loss of ADA1 activity:
\begin{itemize}
\item 0-1\% activity in children with neonatal onset
\item 1-5\% activity in individuals with later onset
\end{itemize}
\end{itemize}

\item Treatment
\label{sec:org6cc8430}
\begin{itemize}
\item HSCT
\item ERT with PEG-ADA1
\item gene therapy
\end{itemize}
\end{enumerate}

\subsubsection{Purine Nucleoside Phosphorylase Deficiency}
\label{sec:orgc6a8b2b}
\begin{enumerate}
\item Clinical Presentation
\label{sec:orgeb442c5}
\begin{itemize}
\item recurrent infections are usually of later onset
\item starting from the end of the first year to up to 5-6 years of age
\item initially less severe than in ADA1 deficiency
\item 2/3 have neurologic symptoms
\begin{itemize}
\item spastic tetra- or diplegia, ataxia and tremor, and mild to severe
mental retardation
\end{itemize}
\item 1/3 have autoimmune disorders
\begin{itemize}
\item hemolytic anemia, idiopathic thrombocytopenic purpura and
autoimmune neutropenia
\end{itemize}
\end{itemize}

\item Metabolic Derangement
\label{sec:org2dc80e3}
\begin{itemize}
\item accumulation of four PDP purine nucleoside substrates:
\begin{itemize}
\item guanosine, deoxyguanosine, inosine, deoxyinosine
\end{itemize}
\end{itemize}
\ce{purine nucleoside + phosphate ->[PNP] purine + \alpha-D-ribose 1-phosphate}
\begin{itemize}
\item \(\downarrow\) formation of uric acid
\item T-cells accumulate dGTP \(\to\) impaired immunity
\begin{itemize}
\item dGTP is formed from deoxyguanosine and inhibits ribonucleotide
reductase, and hence cell division.
\end{itemize}
\item ubiquitous expression of PNP explains the presence of nonimmunologic
symptoms in its deficiency
\end{itemize}

\item Genetics
\label{sec:org92bd639}
\begin{itemize}
\item AR PNP
\end{itemize}

\item Diagnostic Tests
\label{sec:org05b8c7a}
\begin{itemize}
\item \(\downarrow\) plasma uric acid
\item \(\uparrow\)  plasma guanosine, deoxyguanosine, inosine, deoxyinosine
\item \(\downarrow\) urine uric acid
\begin{itemize}
\item other causes of hypouricemia such as xanthine oxidase deficiency,
and drug administration (acetylsalicylic acid, thiazide diuretics)
should be ruled out
\end{itemize}
\item RBC enzyme activity
\end{itemize}

\item Treatment
\label{sec:org6fef480}
\begin{itemize}
\item bone marrow transplantation
\item repeated transfusions of irradiated erythrocytes
\end{itemize}
\end{enumerate}

\subsubsection{Xanthinuria}
\label{sec:orge2582c4}
\begin{itemize}
\item AKA Xanthine Oxidase Deficiency, Xanthine Dehydratase Deficiency
\end{itemize}
\begin{enumerate}
\item Clinical Picture
\label{sec:org4c341e7}
\begin{itemize}
\item three types of deficiencies of \textbf{xanthine oxidase} all cause
xanthinuria
\begin{enumerate}
\item type I classical xanthinuria
\begin{itemize}
\item isolated XO deficiency
\end{itemize}
\item type II classical xanthinuria
\begin{itemize}
\item XO and aldehyde oxidase (AO) deficiency
\end{itemize}
\item molybdenum cofactor deficiency
\begin{itemize}
\item combined deficiency of XO, AO and sulfite oxidase
\end{itemize}
\end{enumerate}
\item type I and type II xanthinuria can be completely asymptomatic
\item \(\sim\) 1/3 of cases \(\to\) kidney stones
\item myopathy w pain stiffness
\end{itemize}

\item Metabolic Derangement
\label{sec:orge114e7a}
\begin{itemize}
\item deficiency of XO results in the near total replacement of uric acid
in plasma and urine by hypoxanthine and xanthine as the end products
of purine catabolism
\end{itemize}

\ce{hypoxanthine + H2O + O2 ->[XDH] xanthine + H2O2}

\ce{xanthine + H2O + O2 ->[XDH] uric acid + H2O2}

\begin{itemize}
\item plasma hypoxanthine is not or minimally elevated
\begin{itemize}
\item due to reutilization by hypoxanthine-guanine phospho-ribosyltransferase
\end{itemize}
\item plasma xanthine \(\uparrow\) 10x
\item deficiency of AO \(\to\) inability to metabolize synthetic purine
analogues (e.g. allopurinol)
\item combined XO, AO, and SO deficiency is caused molybdenum cofactor
(MoCo) deficiency (see Sulfur Amino Acids)
\end{itemize}

\item Genetics
\label{sec:org947edab}
\begin{itemize}
\item AR XDH
\end{itemize}

\item Diagnostic Tests
\label{sec:org63e321c}
\begin{itemize}
\item \(\downarrow\) plasma uric acid
\item \(\downarrow\) urine uric acid
\item \(\Uparrow\) plasma xanthine
\end{itemize}

\item Treatment
\label{sec:org06f9a62}
\begin{itemize}
\item type I and II XO deficiency are mostly benign
\begin{itemize}
\item \(\downarrow\) purine diet w \(\uparrow\) fluid intake to prevent renal stones
\end{itemize}
\item prognosis of combined XO, AO and SO deficiency improved by daily
infusion of cyclic pyranopterin monophosphate (cPMP)
\end{itemize}
\end{enumerate}

\subsubsection{Hypoxanthine-Guanine Phosphoribosyltransferase}
\label{sec:org97184c7}
\begin{enumerate}
\item Clinical Presentation
\label{sec:org694b24e}
\begin{itemize}
\item clinical spectrum is very wide and determined by residual enzyme activity
\item \textbf{Lesch-Nyhan syndrome} = complete or near-complete deficiency of HPRT
\item affected children generally appear normal during the first months of
life
\item \(\sim\) 3-6 months a neurological syndrome evolves
\begin{itemize}
\item classified as severe action dystonia, superimposed on a baseline hypotonia
\end{itemize}
\item patients develop a striking neuro-psychological profile comprising:
\begin{itemize}
\item compulsive self-destructive behaviour involving biting of their
fingers and lips
\item physical and verbal aggression
\end{itemize}
\item speech is hampered by athetoid (writhing) dysarthria
\item most patients have IQ’s around 60-70, some display normal intelligence
\item form uric acid stones
\item if untreated, the uric acid nephrolithiasis progresses to
obstructive uropathy and renal failure during the first decade of
life
\end{itemize}

\item Metabolic Derangement
\label{sec:orgef39622}
\begin{itemize}
\item hypoxanthine-guanine phosphoribosyltransferase deficiency
\end{itemize}

\ce{hypoxanthine + PRPP ->[HPRT] IMP + P2O2^4-}

\ce{guanine + PRPP ->[HPRT] GMP + P2O2^4-}

\ce{xanthine + PRPP ->[HPRT] XMP + P2O2^4-}

\begin{itemize}
\item transfers the 5-phosphoribosyl group from 5-phosphoribosyl
1-pyrophosphate (PRPP) to the purine
\item HGPRT plays a central role in the generation of purine nucleotides
through the purine salvage pathway
\item \(\uparrow\) PRPP \(\to\) \(\Uparrow\) production of uric acid
\begin{itemize}
\item due to \(\uparrow\) \emph{de novo} purine synthesis
\end{itemize}
\end{itemize}

\item Genetic
\label{sec:orge65f7d0}
\begin{itemize}
\item \textbf{XL HPRT}
\end{itemize}

\item Diagnostic Tests
\label{sec:orgcb8f381}
\begin{itemize}
\item \(\Uparrow\) urine and plasma uric acid
\begin{itemize}
\item uric acid/creatinine
\end{itemize}
\item \(\uparrow\) hypoxanthine and xanthine
\item RBC HPRT activity is nearly undetectable
\end{itemize}

\item Treatment and Prognosis
\label{sec:org8eeb976}
\begin{itemize}
\item allopurinol prevents urate nephropathy
\begin{itemize}
\item even when given from birth or in combination with adenine has no
effect on the neurological symptoms
\end{itemize}
\end{itemize}
\end{enumerate}

\subsection{Pyrimidine Metabolism}
\label{sec:org4030f73}
\subsubsection{Introduction}
\label{sec:orgbbcd8db}
\begin{itemize}
\item metabolism of the pyrimidine nucleotides can be divided into three
pathways:
\begin{enumerate}
\item biosynthetic \emph{de novo} pathway:
\begin{itemize}
\item starts with the formation of carbamoylphosphate by cytosolic
carbamoylphosphate synthetase (CPS II)
\item followed by the synthesis of UMP, CMP and TMP
\end{itemize}
\item catabolic pathway:
\begin{itemize}
\item starts from CMP, UMP and TMP
\item yields \(\beta\)-alanine and \(\beta\)-aminoisobutyrate
\item converted into intermediates of TCA cycle
\end{itemize}
\item salvage pathway:
\begin{itemize}
\item composed of kinases
\item converts pyrimidine nucleosides, cytidine, uridine, and
thymidine \(\to\) CMP, UMP, and TMP
\item also converts several pharmacological anticancer and antiviral
nucleoside analogs into their active forms
\end{itemize}
\end{enumerate}

\item inborn errors of pyrimidine metabolism comprise defects of:
\begin{itemize}
\item pyrimidine synthesis:
\begin{itemize}
\item CAD (carbamoylphosphate synthetase II, aspartate transcarbamylase, dihydroorotase) Deficiency
\item UMP synthase deficiency
\item Miller syndrome
\end{itemize}
\item pyrimidine catabolism:
\begin{itemize}
\item deficiencies of dihydropyrimidine
dehydrogenase (DPD) dihydropyrimidinase (DHP)
\item ureidopropionase, thymidine phosphorylase
\item pyrimidine 5’-nucleotidase and cytidine deaminase
\item super-activity of cytosolic 5’-nucleotidase
\end{itemize}
\item pyrimidine salvage:
\begin{itemize}
\item thymidine kinase 2 deficiency
\end{itemize}
\end{itemize}
\end{itemize}

\begin{figure}[htbp]
\centering
\includegraphics[width=0.9\textwidth]{pp/figures/pyrimidine_met.png}
\caption{\label{fig:orge926c03}Pyrimidine Metabolism: 1, carbamoylphosphate synthetase II ; 2, aspartate transcarbamylase ; 3, dihydroorotase (1 to 3 form CAD); 4, dihydroorotate dehydrogenase ; 5, orotate phos- phoribosyltransferase ; 6, orotidine decarboxylase (5 and 6 form UMP synthase); 7, pyrimidine (cytosolic) 5’-nucleotidase; 8, cytidine kinase; 9, uridine kinase ; 10, thymidine kinase ; 11, thymidine phosphorylase ; 12, dihydropyrimidine dehydrogenase ; 13, dihydropyrim- idinase ; 14, ureidopropionase ; 15, cytidine deaminase.}
\end{figure}

\subsubsection{UMP Synthase Deficiency}
\label{sec:org02963f9}
\begin{itemize}
\item AKA Hereditary Orotic Aciduria
\end{itemize}
\begin{enumerate}
\item Clinical Presentation
\label{sec:org85865a1}
\begin{itemize}
\item megaloblastic anaemia a few weeks or months after birth
\begin{itemize}
\item usually the first manifestation
\end{itemize}
\item peripheral blood smears often show anisocytosis, poikilocytosis, and
moderate hypochromia
\item bone marrow examination reveals erythroid hyperplasia and numerous
megaloblastic erythroid precursors
\item characteristically, the anemia does not respond to iron, folic acid
or vitamin B\textsubscript{12}
\item untreated disorder leads to FTT and retardation of growth and
psychomotor development
\end{itemize}

\item Metabolic Derangement
\label{sec:org4eafe72}
\begin{itemize}
\item UMP synthase is a bifunctional enzyme of the \emph{de novo} synthesis of
pyrimidines
\item first domain orotate phosphoribosyltransferase (OPRT) converts
orotic acid into OMP
\item second domain orotidine-5’-monophosphate decarboxylase (ODC)
decarboxylates OMP into UMP
\end{itemize}
\ce{orotic ->[UMPS] OMP ->[UMPS] UMP}
\begin{itemize}
\item deficiency \(\to\) massive overproduction of orotic acid
\begin{itemize}
\item due to \(\downarrow\) feedback inhibition exerted by the pyrimidine
nucleotides on the first enzyme of their \emph{de novo} synthesis CPS2
and deficiency of pyrimidine nucleotides
\end{itemize}
\item \(\downarrow\) pyrimidine nucleotides \(\to\) \(\downarrow\) cell division \(\to\)
megaloblastic anemia
\end{itemize}

\item Genetics
\label{sec:org791f29f}
\begin{itemize}
\item AR UMPS
\end{itemize}

\item Diagnostic Tests
\label{sec:org841114b}
\begin{itemize}
\item \(\Uparrow\) urine orotic acid 200-1000X
\end{itemize}

\item Treatment
\label{sec:org7c835b2}
\begin{itemize}
\item enzyme defect can be by-passed by the administration of uridine
\begin{itemize}
\item converted into UMP by uridine kinase
\end{itemize}
\end{itemize}
\end{enumerate}

\subsubsection{Dihydropyrimidine Dehydrogenase Deficiency}
\label{sec:org285b2b2}
\begin{enumerate}
\item Clinical Presentation
\label{sec:org8b76ff2}
\begin{itemize}
\item two forms:
\begin{enumerate}
\item infantile, severe
\begin{itemize}
\item epilepsy, motor and mental retardation
\item hypertonia, hyperreflexia, growth delay, microcephaly, autistic features
\end{itemize}
\item adult, partial
\begin{itemize}
\item found in adults who receive pyrimidine analog, 5-fluorouracil
\begin{itemize}
\item 5-fluorouracil used to treat cancers including breast, ovary, colon
\end{itemize}
\item severe toxicity, manifested by profound neutropenia, stomatitis,
diarrhea and neurologic symptoms, including ataxia, paralysis
and stupor
\end{itemize}
\end{enumerate}
\end{itemize}

\item Metabolic Derangement
\label{sec:orgf65085c}
\begin{itemize}
\item DPD catalyzes the catabolism of uracil and thymine \(\to\) dihydrouracil
and dihydrothymine
\begin{itemize}
\item accumulation of uracil and thymine
\end{itemize}
\end{itemize}

\item Genetics
\label{sec:org9471a14}
\begin{itemize}
\item AR DPYD
\end{itemize}

\item Diagnostic Tests
\label{sec:org5e04b4d}
\begin{itemize}
\item \(\Uparrow\) urine uracil
\item \(\Uparrow\) urine thyamine
\item enzyme activity in fibroblasts, liver and blood cells, with the
exception of erythrocytes
\end{itemize}

\item Treatment
\label{sec:orgfd134ab}
\begin{itemize}
\item infantile: none
\item adult: avoid 5-fluorouracil
\end{itemize}
\end{enumerate}
\section{Porphyrias}
\label{sec:org2b2dad3}
\subsection{Introduction}
\label{sec:org9adeb12}
\begin{itemize}
\item porphyria is a group of disorders caused by abnormalities in the
chemical steps that lead to heme production
\item heme is most abundant in the blood, bone marrow, and liver
\item heme is a component of several iron-containing proteins called
hemoproteins, including hemoglobin and the cytochrome P450 family of
enzymes
\item porphyrins are heterocyclic macrocycles composed of four modified
pyrrole subunits
\item interconnected at their \(\alpha\)-carbon atoms via methine bridges
(=CH−)
\item typically have very intense absorption bands in the visible region
and may be deeply colored
\item biosynthesis:
\begin{itemize}
\item organelle: mitochondria \(\to\) cytoplasm \(\to\) mitochondria
\begin{itemize}
\item starts with succinyl-CoA and glycine in mitochondria
\end{itemize}
\item tissue: 70-80\% in bone marrow
\begin{itemize}
\item 15\% in other tissue ie. liver \(\to\) Cyto P450, cytochromes
\end{itemize}
\end{itemize}
\end{itemize}

\definesubmol{P}{-[::-60]-[::60](=[::60]O)-[::-60]OH}
\definesubmol{M}{CH_3}
\definesubmol{V}{=[::-60]CH_2}
\chemname{\chemfig[]{?[a]=[::+72]*5(-N?[b]=(-=[::-72]*5(-N?[c]
    (-[::-33,1.5,,,draw=none]{\color{red}Fe}?[b]?[c]?[d]?[e])-(=-[::-36]*5(=N?[d]-(=-[::-72]*5(-N?[e]-?[a]
    =(-!{M})-(-!{P})=))
    -(-!{P})=(-!{M})-))
    -(-!{V})=(-!{M})-))
    -(-!{V})=(-!{M})-)}}{Heme}

\begin{figure}[htbp]
\centering
\includegraphics[width=0.9\textwidth]{porphyrins/figures/heme_synth.png}
\caption{\label{fig:org04f9a05}Heme Synthesis}
\end{figure}

\begin{figure}[htbp]
\centering
\includegraphics[width=0.9\textwidth]{porphyrins/figures/Slide19.png}
\caption{\label{fig:orgd0e243e}Heme Synthesis}
\end{figure}

\begin{table}[htbp]
\caption{\label{tab:orgbd6df95}Route of excretion dictated by solubility for porphyrin precursors}
\centering
\begin{tabular}{ll}
Porphyrin & Route\\
\hline
ALA & urine\\
PBG & urine\\
Uro & urine\\
CI & >fecal\\
CIII & >urine\\
Proto & fecal\\
\end{tabular}
\end{table}

\begin{table}[htbp]
\caption{\label{tab:orga5bb2eb}Porphyrin Function}
\centering
\begin{tabular}{ll}
Protein & Functions\\
\hline
hemoglobin & oxygen transport\\
myoglobin & storage of oxygen in muscle\\
cytochrome c & electron transport\\
cytochrome P450 & drug metabolism\\
catalase & \ce{H2O2} breakdown\\
tryptophan pyrolase & tryptophan oxidation\\
\end{tabular}
\end{table}

\begin{table}[htbp]
\caption{\label{tab:org5332e69}Main types of human porphyrias}
\centering
\begin{tabular}{llll}
Enzyme & Substrate & Disorder & Clinical\footnotemark\\
\hline
 & glycine + succinyl CoA &  & \\
ALAS & \(\downarrow\) & XLSA & A\\
 & \(\sigma\)-ALA &  & \\
ALAD & \(\downarrow\) & ADP & N\\
 & PBG &  & \\
HMBS & \(\downarrow\) & AIP & N\\
 & hydroxymethylbilane &  & \\
UROS & \(\downarrow\) & CEP & C\\
 & uroporphyrinogen-III &  & \\
UROD & \(\downarrow\) & PCT & C\\
 & coproporphyrinogen-III &  & \\
CPOX & \(\downarrow\) & HCP & N,C\\
 & protoporphyrinogen-IX &  & \\
PPOX & \(\downarrow\) & VP & N,C\\
 & protoporphyrin-IX &  & \\
FECH & \(\downarrow\) & EPP & C\\
\end{tabular}
\end{table}\footnotetext[3]{\label{orgb427184}A = anemia, N = neuroviseral, C = cutaneous}

\subsubsection{Clinical Classification of the Porphyrias}
\label{sec:orgbc341ca}
\begin{itemize}
\item Acute \textbf{Neurovisceral attacks}
\begin{itemize}
\item AIP, ADP (rare)
\end{itemize}

\item Neuroviseral and/or Cutaneous
\begin{itemize}
\item HCP, VP
\end{itemize}

\item Cutaneous \textbf{photo-sensitivity, bullae, skin fragility}
\begin{itemize}
\item PCT, CEP, HEP, EPP
\end{itemize}
\end{itemize}

\begin{table}[htbp]
\caption{\label{tab:orgf6e0aef}Main types of human porphyrias: Classification by onset}
\centering
\begin{tabular}{llllll}
Disorder & Enzyme & Prevalence & NV & Lesions & Site\\
\hline
Acute &  &  &  &  & \\
\hline
ADP & ALAD & - & - & - & \\
AIP & HMBS & 1-2:100,000 & + & - & hepatic\\
HCP & CPO & 1-2:10\textsuperscript{6} & + & fragile,bullae & hepatic\\
VP & PPOX & 1:2:50,000 & + & fragile,bullae & hepatic\\
\hline
Non-acute &  &  &  &  & \\
\hline
CEP & UROS & 1:10\textsuperscript{6} & - & fragile,bullae & erythropoietic\\
PCT & UROD & 1:25,000 & - & fragile,bullae & hepatic\\
EPP & FECH & 1:140,000 & - & photosensitive,bullae & erythropoietic\\
\end{tabular}
\end{table}

\subsection{Acute Porphyrias}
\label{sec:org74abd04}
\subsubsection{Clinical Features}
\label{sec:org4378472}
\begin{itemize}
\item life threatening neuroviseral attack occur in AIP,VP and HCP
are clinically identical
\item low clinical penetrance is a promenent feature of all AD acute porphyrias
\item 25\% of patients with overt acute porphyria have no family history
\begin{itemize}
\item sporadic presentation reflects high prevalence and low penetrance
\item acute porphyria caused by \emph{de novo} mutation is uncommon
\end{itemize}
\item allelic heterogenetity
\end{itemize}

\begin{table}[htbp]
\caption{\label{tab:org4f37a27}Clinical features of acute neuroviseral attacks}
\centering
\begin{tabular}{lr}
Symptom/Sign & Percent\\
\hline
Abdominal pain & 97\\
Nonabdominal pain & 25\\
Vomiting & 85\\
Constipation & 46\\
Psychologic symptoms & 8\\
Convulsions & 5\\
Muscle weakness & 8\\
Sensory loss & 2\\
Hypertension (Diastolic >85 mmHg & 64\\
Tachycardia (>80/min) & 65\\
Hyponatremia & 37\\
\end{tabular}
\end{table}

\begin{itemize}
\item persistent psychiatric illness is not a feature of acute porphyrias.
\begin{itemize}
\item disappears with remission
\end{itemize}
\end{itemize}
\begin{enumerate}
\item Precipitating factors
\label{sec:org5622002}
\begin{enumerate}
\item drugs
\item alcohol, especially binge drinking
\item the menstrual cycle
\item calorie restriction
\item infection
\item stress
\end{enumerate}
\item Drugs
\label{sec:org477cc99}
\begin{itemize}
\item barbiturates, sulfonamides, progestogens, anticonvulsants
\item \url{http://www.drugs-porphyria.org}
\end{itemize}
\item Long term complications
\label{sec:org597ddc9}
\begin{itemize}
\item chronic renal failure
\item hypertension
\item primary hepatocellular carcinoma
\end{itemize}
\end{enumerate}

\subsubsection{Metabolic Derangement}
\label{sec:org441360c}
\begin{enumerate}
\item ALA Dehydratase Porphyria
\label{sec:org691e056}
\begin{itemize}
\item \textbf{aminolevulinic acid dehydratase (ALAD)} 
\begin{itemize}
\item aka: porphobilinogen synthase
\item requires zinc, inhibited by lead
\item \textasciitilde{}five cases reported
\begin{description}
\item[{Urine ALA}] \(\Uparrow\) \(\Uparrow\) \(\Uparrow\)
\item[{Urine PBG}] Not elevated
\end{description}
\end{itemize}
\end{itemize}

\item Acute Intermittent Porphyria
\label{sec:org1212ceb}
\begin{itemize}
\item \textbf{hydroxymethylbilane synthase (HMBS)}
\item aka: PBG deaminase
\item four PBGs are combined through deamination
\item susceptible to allosteric inhibition by CIII and protoporphyrinogen
\item HMB is unstable \(\to\) \(\uparrow\) URO I
\begin{description}
\item[{Urine PBG}] \(\Uparrow\) \(\Uparrow\) \(\Uparrow\)
\item[{Urine ALA}] \(\Uparrow\) \(\Uparrow\) \(\Uparrow\)
\end{description}
\item rule out VP and HCP
\item \(\uparrow\) urine uroporphyrin arises from non-enzymatic
condensation of micro-molar concentrations of PBG
\end{itemize}

\item Hereditary Coproporphyria
\label{sec:org1fbd3fb}
\begin{itemize}
\item \textbf{coproporphyrinogen oxidase (CPOX)}
\item mitochondrial inter-membrane space
\item inhibited by metals
\item specific for CIII
\begin{description}
\item[{Urine PBG}] \(\Uparrow\) \(\Uparrow\) \(\Uparrow\)
\item[{Fecal copro III}] \(\Uparrow\) \(\Uparrow\) \(\Uparrow\)
\end{description}
\end{itemize}

\item Varigate Porphyria
\label{sec:org65cb797}
\begin{itemize}
\item \textbf{Protoporphyrinogen Oxidase (PPOX)}
\item inner mitochondrial membrane
\item[{Urine PBG}] \(\Uparrow\) \(\Uparrow\) \(\Uparrow\)
\begin{description}
\item[{Fecal copro-III}] \(\uparrow\) \(\uparrow\)
\item[{Plasma fluorescence scan}] \(\Uparrow\) \(\Uparrow\) \(\Uparrow\)
\end{description}
\end{itemize}
\end{enumerate}

\subsubsection{Genetics}
\label{sec:org5a416ee}
\begin{description}
\item[{ALAD}] AR ALAD
\item[{AIP}] AD HMBS
\item[{VP}] AD PPOX with reduced penetrance
\item[{HCP}] AD CPOX
\end{description}
\subsubsection{Diagnostic Tests}
\label{sec:org71897fd}
\begin{enumerate}
\item Acute Attack
\label{sec:orgfd2a0a3}
\begin{itemize}
\item during an acute attack \textbf{a normal PBG essentially excludes all acute
neuro-visceral porphyrias}
\begin{itemize}
\item except ADP
\end{itemize}
\item patients with cutaneous symptoms (VP, HCP) should also have excessive
production of porphyrins
\item when suspicion of an acute porphyria remains high while crisis is
resolving
\begin{itemize}
\item analysis of fecal and plasma porphyrins and urinary ALA is
advisable even if PBG is normal
\end{itemize}
\item elevated PBG and ALA doesn't mean symptoms are caused by AIP
\item genetic and/or enzyme studies are rarely helpful for diagnosis
\end{itemize}

\item Non-AIP Acute Porphyrias
\label{sec:orgee9eed5}
\begin{itemize}
\item VP and HCP may not have skin lesions \(\to\) fecal porphyrins
\begin{itemize}
\item if normal, w \(\uparrow\) PBG, VP \& HCP are excluded \(\to\) AIP
\item if \(\uparrow\) total fecal porphyrins \(\to\) fractionate by HPLC
\begin{description}
\item[{HCP}] coproporphyrin-III \(\Uparrow\) \(\Uparrow\) \(\Uparrow\)
\item[{VP}] protoporphyrin-IX \(\Uparrow\) \(\Uparrow\) \(\Uparrow\)
\end{description}
\item can also be due to diet or GI bleed
\item follow-up with plasma porphyrin emission scan
\end{itemize}
\end{itemize}
\end{enumerate}
\subsubsection{Treatment}
\label{sec:org1b37a84}
\begin{itemize}
\item acute
\begin{itemize}
\item IV glucose, hematin
\end{itemize}
\item avoid triggers
\begin{itemize}
\item adverse drugs
\item hypoglycemia
\item smoking, drinking
\item progesterone
\end{itemize}
\end{itemize}

\subsection{Non-Acute Porphyrias}
\label{sec:orgb1de202}
\subsubsection{X-Linked Sideroblastic Anemia}
\label{sec:orgd07cbd5}
\begin{itemize}
\item \textbf{\(\Delta\)-aminolevulinate synthase 2 (ALAS2)}
\item X-linked ALAS2
\item mitochondrial
\item rate limiting step under normal conditions
\item microcytic, hypochromic red cells
\item abnormal accumulation of iron in red blood cells \(\to\) ring
sideroblasts
\end{itemize}
\subsubsection{Porphyria Cutanea Tarda}
\label{sec:orgb92b6e8}
\begin{enumerate}
\item Clinical Presentation
\label{sec:org2181982}
\begin{itemize}
\item both sexes
\item most common porphyria
\item onset during 5th and sixth decade
\item lesions on sun-exposed skin: back of hands, forearm, face
\item fragile skin
\item subepidermal bullae, milia, hypertrichosis of the face, patchy pigmentation
\item skin lesions with liver damage associated with:
\begin{itemize}
\item alcohol abuse
\item estrogens
\item infection with heptotropic viruses, HCV
\item hemochromatosis, iron overload
\end{itemize}
\end{itemize}
\item Metabolic Derangement
\label{sec:org27799ab}
\begin{itemize}
\item \textbf{uroporphyrinogen decarboxylase (UROD)}
\item last cytoplasmic enzyme, \(\downarrow\) polar
\item \(\downarrow\) activity of UROD in liver \(\to\) \(\uparrow\) URO
\item 50\% \(\downarrow\) in UROD activity does not \(\to\) overt PCT
\begin{itemize}
\item further inactivation in the liver is required
\end{itemize}
\item 80\% of patients have sporadic (type I)
\begin{itemize}
\item enzyme defect is restricted to the liver
\item typically no family history
\end{itemize}
\item \(\uparrow\) LFTs in 50\%
\item hepta, hexa and pentacarboxylate formed at the same active site
\item \(\downarrow\) UROD \(\to\) increase in intermediates and uroporphyrins
\end{itemize}
\item Genetics
\label{sec:org12c171b}
\begin{itemize}
\item famillial PCT
\begin{itemize}
\item AD UROD w reduced penetrance
\item mutation in one UROD gene \(\to\) 1/2 normal activity
\end{itemize}
\item acquired PCT 
\begin{itemize}
\item exposure to polyhalogenated aromatic hydrocarbons
\end{itemize}
\end{itemize}
\item Diagnostic Tests
\label{sec:orgf00a206}
\begin{itemize}
\item urine porphyrin quantitation
\begin{itemize}
\item \(\Uparrow\) \(\Uparrow\) \(\Uparrow\) Uro I \& III
\end{itemize}
\end{itemize}
\item Treatment
\label{sec:orgd810fb7}
\begin{itemize}
\item \(\downarrow\) exposure to light
\item iron depletion
\item chloroquine
\end{itemize}
\end{enumerate}
\subsubsection{Congential Erythropoietic Porphyria}
\label{sec:orga59c06b}
\begin{enumerate}
\item Clinical Presentation
\label{sec:org65cd321}
\begin{itemize}
\item varying severity
\begin{itemize}
\item hydrops fetalis
\item onset in infancy of severe skin lesions, transfusion dependent
hemolytic anemia
\item mid-life onset of mild skin lesions resembling PCT
\end{itemize}
\item most present in early infancy
\begin{itemize}
\item blisters on skin after UV exposure
\item red-brown staining of diapers by urinary porphyrins
\end{itemize}
\item ongoing destruction of ears, nose and eyelids, alopecia
\item red brown teeth
\item skin changes usually accompanied by hemolytic anemia and splenomegaly
\end{itemize}
\item Metabolic Derangement
\label{sec:orge07a65a}
\begin{itemize}
\item \textbf{uroporphyrinogen III synthase (UROS)}
\item \(\downarrow\) UROS \(\to\) \(\uparrow\) UI
\begin{itemize}
\item HMB condensed \(\to\) Uro I or III
\item HMB \(\rightarrow\) Uro I: spontaneous
\item HMB \(\rightarrow\) Uro III: UROS
\end{itemize}
\end{itemize}
\item Genetics
\label{sec:orgc500b2c}
\begin{itemize}
\item least common, most severe of the cutaneous porphyrias, < 1:million in UK
\item AR UROS
\begin{itemize}
\item usually heteroallelic
\end{itemize}
\item X-linked GATA1 very rare
\begin{itemize}
\item a transcription factor
\end{itemize}
\end{itemize}
\item Diagnostic tests
\label{sec:org3541f10}
\begin{itemize}
\item urine porphyrin quantitation
\begin{itemize}
\item \(\Uparrow\) \(\Uparrow\) \(\Uparrow\) Uro I
\item \(\Uparrow\) \(\Uparrow\) \(\Uparrow\)  Copro I
\end{itemize}
\item fecal porphyrin quantitation
\begin{itemize}
\item \(\Uparrow\) \(\Uparrow\) \(\Uparrow\) Copro I
\end{itemize}
\end{itemize}

\item Treatment
\label{sec:org5835705}
\begin{itemize}
\item \(\downarrow\) UV exposure
\item curative treatment - allogenic bone marrow transplantation
\item investigating gene therapy
\end{itemize}
\end{enumerate}

\subsubsection{Erythropoietic Protoporphyria}
\label{sec:org21e330f}
\begin{enumerate}
\item Clinical Features
\label{sec:org56b032f}
\begin{itemize}
\item life long acute photosensitivity due to \(\uparrow\) protoporphyrin-IX
in the skin
\item onset birth \(\to\) age 6, median = 1 year
\item both sexes
\item onset after 40 very rare
\begin{itemize}
\item most cases associated with myelodysplasia, caused by aquired
somatic mutation of FECH.
\end{itemize}
\item absence of fragile skin, subepidermal bullae, and hypertrichosis
distinguishes it from all other cutaneous porphyrias
\item most severe complication is progressive hepatic failure
\item cholelithiasis - gallstones promoted by \(\uparrow\) protoporphyri in bile
\item mild microcytic anemia
\end{itemize}

\item Metabolic Derangement
\label{sec:orgf5e49d4}
\begin{itemize}
\item \textbf{ferrochelatase (FECH)}
\item threshold FECH activity = 35\% \(\to\) \(\uparrow\) protoporphyrin
\begin{itemize}
\item inserts ferrous iron into protoporphyrin to form heme
\item aka: heme synthase
\item in iron deficient states forms \(\to\) zinc protoporphyrin
\end{itemize}
\item \(\uparrow\) protoporphyrin mainly in bone marrow
\end{itemize}

\item Genetics
\label{sec:orgae7a55f}
\begin{itemize}
\item AR FECH
\item X-linked dominant protoporphyria (XLDPP)
\begin{itemize}
\item 2\% of EPP cases
\item due to gain of function ALAS2 mutation
\end{itemize}
\end{itemize}

\item Diagnostic Tests
\label{sec:org5c63526}
\begin{itemize}
\item \(\Uparrow\) \(\Uparrow\) \(\Uparrow\) RBC free protoporphyrin
\end{itemize}

\item Treatment
\label{sec:org926b05b}
\begin{itemize}
\item avoid sunlight
\item annual LFTs
\item orthotopic liver transplantation
\item bone marrow transplantation
\end{itemize}
\end{enumerate}

\subsection{Secondary Causes}
\label{sec:org3e830d2}
\begin{itemize}
\item more common cause of abnormal porphyrin metabolism than porphyria
\end{itemize}
\subsubsection{Lead and Other Heavy Metals}
\label{sec:orgd220a7d}
\begin{itemize}
\item lead exposure \(\uparrow\) urinary ALA and coproporphyrin III excretion
and accumulation of ZN-protoporphyrin in erythrocytes
\begin{itemize}
\item inhibition of ALAD, CPOX
\item Pb causes mito deficiency in Fe \(\to\) Zn replaces Fe as
substrate for FECH
\item \(\uparrow\) ALA excretion secondary to inhibition of ALAD
\begin{itemize}
\item caused by lead displacing zinc at catalytic site
\item ALAD2 isoform more susceptible than ALAD1
\end{itemize}
\end{itemize}
\end{itemize}
\subsubsection{Secondary Coproporphyrinuria: Hepatobiliary and other Disorders}
\label{sec:orgf285aa9}
\begin{itemize}
\item most common cause of abnormal porphyrin excretion
\begin{description}
\item[{alcohol intake}] CIII
\item[{impaired biliary excretion of CI}] \(\to\) urine
\begin{itemize}
\item cholestatic jaundice, hepatitis, and cirrhosis
\item reversal of normal ratio: CI dominates
\item drugs
\item severe infection
\end{itemize}
\item[{Dubin-Johnson}] \(\uparrow\) CI, \(\downarrow\) CII
\item[{Rotor}] \(\uparrow\) CI, normal CIII
\item[{Gilbert}] \(\uparrow\) CI, \(\uparrow\) CIII
\end{description}
\end{itemize}

\subsubsection{Increased Fecal Porphyrin Concentration}
\label{sec:org4830695}
\begin{itemize}
\item protoporphyrin and other dicarboxylic porphyrins derived from
bacterial metabolism
\item additional protoporphyrin and other dicarboxylic porphyrins formed
from heme containing proteins from diet or gastrointestinal
hemorrhage
\item even minor hemorrhage (ie FOBT negative) \(\uparrow\) dicarboxylic porphyrins
\item confusion with EPP may occur when associated iron deficiency
\(\uparrow\) erythrocyte total porphyrin, and skin lesions for other reasons
\item Confusion with VP when coexisting liver disease causes
coproporphyrinuria
\item porphyria is excluded when no porphyrin fluorescence is detectable
on fluorescence emission spectroscopy of plasma and fecal
coproporphyrin excretion is normal
\item consumption of brewers yeast caused profile indistinguishable from
VP
\end{itemize}

\subsubsection{Increase Plasma Porphyrin Concentration: Renal and other Disorders}
\label{sec:org08fc3b8}
\begin{itemize}
\item plasma porphyrin concentration increased due to \(\downarrow\) renal or
hepatobiliary excretion is impaired
\item ESRF marked increase in concentration, poor clearance by dialysis
\begin{itemize}
\item similar to PCT, but not as high
\item PCT uncommon complicaiton of ESRF
\item distinguish with fecal porphyrin analysis
\end{itemize}
\end{itemize}
\subsubsection{Hematologic Disorders}
\label{sec:org770b221}
\begin{itemize}
\item iron deficiency anemia, Zn acts as alternate substrate for FECH
\begin{itemize}
\item results in \(\uparrow\) erythrocyte ZPP
\item also sideroblastic megaloblastic, and hemolytic anemias
\end{itemize}
\end{itemize}
\subsubsection{Hereditary Tyrosinemia Type I}
\label{sec:org624d112}
\begin{itemize}
\item \(\uparrow\) succinylacetone resembles ALA, inhibits ALAD
\begin{itemize}
\item \(\uparrow\) ALA accumulates in urine
\end{itemize}
\end{itemize}
\section{Oxalate}
\label{sec:org0d8a5d6}
\subsection{Primary Hyperoxalouria}
\label{sec:org001ce23}
\begin{itemize}
\item the three known types of primary hyperoxaluria (PH) are
\begin{itemize}
\item PH1 (due to mutation of AGXT)
\item PH2 (mutation of GRHPR)
\item PH3 (mutation of HOGA1)
\end{itemize}
\item each gene encodes an enzyme for different metabolic pathways
relevant for the metabolism of glyoxylate
\item of the primary hyperoxalurias, approximately
\begin{itemize}
\item 70\% PH1
\item 10\% PH2
\item 10\% PH3
\item 10\% do not have an identified genetic cause
\end{itemize}
\item the clinical manifestations of the three known types of
PH overlap considerably
\end{itemize}
\subsubsection{Clinical Presentation}
\label{sec:org069879d}
\begin{itemize}
\item risk for recurrent
\begin{itemize}
\item nephrolithiasis - deposition of calcium oxalate in the renal pelvis and urinary tract
\item nephrocalcinosis - deposition of calcium oxalate in the renal parenchyma
\item end-stage renal disease (ESRD)
\end{itemize}
\item age at onset of symptoms ranges from infancy to the sixth decade
\item systemic calcium oxalate deposition (systemic oxalosis) not observed in PH3
\end{itemize}
\subsubsection{Metabolic Derangement}
\label{sec:org668e908}
\begin{enumerate}
\item PH1
\label{sec:orgd55eb2e}
\begin{itemize}
\item deficiency of \textbf{liver peroxisomal enzyme alanine:glyoxylate-aminotransferase (AGT)}
\begin{itemize}
\item conversion of glyoxylate to glycine
\end{itemize}
\item when AGT activity is absent, glyoxylate is converted to oxalate,
which forms insoluble calcium oxalate crystals that accumulate in
the kidney and other organs
\end{itemize}
\item PH2
\label{sec:orgf8d0370}
\begin{itemize}
\item deficiency of  \textbf{glyoxylate reductase/hydroxypyruvate reductase (GR/HPR)}
\begin{itemize}
\item reduction of glyoxylate to glycolate, using the cofactor NADH or NADPH
\end{itemize}
\end{itemize}
\item PH3
\label{sec:orgb9f4f20}
\begin{itemize}
\item deficiency of \textbf{4-hydroxy-2-oxoglutarate aldolase}
\begin{itemize}
\item conversion of 4-hydroxy-2-oxoglutarate to pyruvate and glyoxylate
\end{itemize}
\end{itemize}
\end{enumerate}
\subsubsection{Genetics}
\label{sec:org5a85fab}
\begin{description}
\item[{PH1}] AR AGXT
\item[{PH2}] AR GRHPR
\item[{PH3}] AR HOGA1
\end{description}
\subsubsection{Diagnostic Tests}
\label{sec:orgfd42704}
\begin{itemize}
\item differences in urinary metabolites other than oxalate can provide
clues regarding the most likely type of PH

\begin{itemize}
\item \(\uparrow\) urinary glycolate in PH1 or occasionally in PH3
\item \(\uparrow\) urine glycerate in most individuals with PH2
\item \(\uparrow\) 4-hydroxy-2-oxoglutarate (HOG) in PH3
\begin{itemize}
\item undetectable in PH1 and PH2
\end{itemize}
\end{itemize}
\end{itemize}

\begin{figure}[htbp]
\centering
\includegraphics[width=0.9\textwidth]{oxalate/figures/phdiag.png}
\caption{\label{fig:org4fa78f5}Diagnosis of Primary Hyperoxaluria}
\end{figure}

\subsubsection{Treatment}
\label{sec:org93663f7}
\begin{itemize}
\item \(\uparrow\) fluid intake
\item alkalinization of urine with potassium citrate
\item limit oxalate intake may be helpful
\begin{itemize}
\item oxalate excess is due to endogenous formation
\end{itemize}
\item prevent kidney injury
\end{itemize}
\end{document}