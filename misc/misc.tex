% Created 2019-12-12 Thu 17:15
% Intended LaTeX compiler: pdflatex
\documentclass{scrartcl}
\usepackage[utf8]{inputenc}
\usepackage[T1]{fontenc}
\usepackage{graphicx}
\usepackage{grffile}
\usepackage{longtable}
\usepackage{wrapfig}
\usepackage{rotating}
\usepackage[normalem]{ulem}
\usepackage{amsmath}
\usepackage{textcomp}
\usepackage{amssymb}
\usepackage{capt-of}
\usepackage{hyperref}
\hypersetup{colorlinks,linkcolor=black,urlcolor=blue}
\usepackage{textpos}
\usepackage{textgreek}
\usepackage[version=4]{mhchem}
\usepackage{chemfig}
\usepackage{siunitx}
\usepackage{gensymb}
\usepackage[usenames,dvipsnames]{xcolor}
\usepackage[T1]{fontenc}
\usepackage{lmodern}
\usepackage{verbatim}
\usepackage{tikz}
\usepackage{wasysym}
\usetikzlibrary{shapes.geometric,arrows,decorations.pathmorphing,backgrounds,positioning,fit,petri}
\usepackage{fancyhdr}
\pagestyle{fancy}
\author{Matthew Henderson, PhD, FCACB}
\date{\today}
\title{Miscellaneous IEM}
\hypersetup{
 pdfauthor={Matthew Henderson, PhD, FCACB},
 pdftitle={Miscellaneous IEM},
 pdfkeywords={},
 pdfsubject={},
 pdfcreator={Emacs 26.1 (Org mode 9.1.9)}, 
 pdflang={English}}
\begin{document}

\maketitle
\setcounter{tocdepth}{2}
\tableofcontents


\section{Purine and Pyrimidine Metabolism}
\label{sec:orgdf1a14a}
\subsection{Purine Metabolism}
\label{sec:org679ed72}
\subsubsection{Introduction}
\label{sec:org62ade3f}
\begin{itemize}
\item Purine nucleotides are involved in energy transfer, metabolic
regulation, and synthesis of DNA and RNA
\item Purine metabolism can be divided into three pathways:
\begin{itemize}
\item biosynthetic pathway, often termed \emph{de novo}
\begin{itemize}
\item starts with the formation of phosphoribosyl pyrophosphate (PRPP)
and leads to the synthesis of IMP
\item IMP interconversions to AMP and GMP
\item go on to di- and triphosphates, to and deoxyribonucleotides
\end{itemize}
\item catabolic pathway
\begin{itemize}
\item starts from GMP, IMP and AMP
\item produces uric acid , a poorly soluble compound, which tends to
crystallize once its plasma \(\ge\) 0.4 mmol/L
\end{itemize}
\item salvage pathway
\begin{itemize}
\item utilizes the purine bases guanine, hypoxanthine and adenine
\begin{itemize}
\item provided by food or the catabolic pathway
\end{itemize}
\item converts them into GMP, IMP and AMP
\item salvage of purine nucleosides, adenosine and guanosine, and
their deoxy counterparts, catalyzed by kinases, also occurs.
\item also converts several pharmacological anticancer and antiviral
nucleoside analogs into their active forms
\end{itemize}
\end{itemize}

\item inborn errors of purine metabolism comprise defects or
superactivities of:

\begin{itemize}
\item purine nucleotide synthesis and interconversions:
\begin{itemize}
\item phosphoribosyl pyrophosphate synthetase (PRS) superactivity and deficiency
\item adenylosuccinase (ADSL) deficiency
\item AICA-ribosiduria caused by ATIC deficiency
\end{itemize}
\item purine catabolism:
\begin{itemize}
\item muscle AMP deaminase (AMPD, also termed myoadenylate deaminase)
\item adenylate kinase (AK)
\item adenosine deaminase (ADA)
\item purine nucleoside phosphorylase (PNP)
\item xanthine oxidase (XO)
\end{itemize}
\item purine salvage:
\begin{itemize}
\item hypoxanthine-guanine phosphoribosyltransferase (HPRT)
\item adenine phosphoribosyltransferase (APRT)
\item adenosine kinase (ADK)
\end{itemize}
\item deficiency of deoxyguanosine kinase (DGUOK) causes mitochondrial DNA depletion
\item deficiency of thiopurine S-methyltransferase (TPMT) results in
less efficient methylation and hence in enhanced toxicity of
pharmacologic thiopurine ana logs
\item deficiency of inosine triphosphate pyrophosphatase (IT-Pase) also
increases the toxicity of thiopurines
\end{itemize}
\item With the exception of the deficiencies of muscle AMPD and TPMT, all
these inborn errors are very rare
\end{itemize}

\begin{figure}[htbp]
\centering
\includegraphics[width=0.9\textwidth]{./pp/figures/purine_met.png}
\caption{\label{fig:org5442a57}
Purine Metabolism}
\end{figure}

\subsubsection{Phosphoribosyl Pyrophosphate Synthetase Superactivity}
\label{sec:orga8a5b62}
\begin{enumerate}
\item Clinical Presentation
\label{sec:org8fafca8}
\begin{itemize}
\item gouty arthritis and/or uric acid lithiasis in young males
\end{itemize}

\item Metabolic Derangement
\label{sec:orgd977eee}
\ce{ribose-5-phosphate + ATP ->[PRPS] PRPP}

\begin{itemize}
\item PRPP is the first intermediate in \emph{de novo} synthesis of purine nucleotides
\item PRPP synthetase is highly regulated
\item Two proposed mechanisms to explain PRS-I superactivity:
\begin{enumerate}
\item gain-of-function point mutations in the PRPS1 gene resulting in an
altered regulatory site
\item \(\uparrow\) expression of PRPS1 with normal kinetic properties
\end{enumerate}
\item PRPP amidotransferase, the first and rate-limiting enzyme
of the de novo pathway, is physiologically not saturated by PRPP,
the synthesis of purine nucleotides increases, and hence the
production of uric acid
\end{itemize}

\item Genetics
\label{sec:org0d739bd}
\begin{itemize}
\item X-linked
\end{itemize}

\item Diagnostic Tests
\label{sec:orgde1a8c0}
\begin{itemize}
\item extensive kinetic studies of the enzyme in erythrocytes and cultured
fibroblasts is performed in only a few laboratories in the world
\item molecular testing of PRPS1 gene
\end{itemize}

\item Treatment
\label{sec:org75acb4b}
\begin{itemize}
\item treated with allopurinol, which inhibits xanthine oxidase, the last
enzyme of purine catabolism
\item results in a decrease of the production of uric acid and in its
replacement by
\begin{itemize}
\item hypoxanthine which is about 10-fold more soluble
\item xanthine which is slightly more soluble
\end{itemize}
\end{itemize}
\end{enumerate}

\subsubsection{Adenosine Deaminase 1 Deficiency}
\label{sec:orga6fb962}
\begin{itemize}
\item two isoforms of adenosine deaminase (ADA)
\begin{itemize}
\item ADA1 is found in most cells, particularly lymphocytes and macrophages
\item ADA2 is predominant in plasma
\end{itemize}
\end{itemize}
\begin{enumerate}
\item Clinical Presentation
\label{sec:org89e4f19}
\begin{itemize}
\item clinical spectrum is very broad
\begin{itemize}
\item from a profound impairment of both humoral and cellular immunity
in infants, known as severe combined immunodeficienc (SCID)
\item to delayed and less severe later onset in older children or
adults
\item even benign partial ADA1 deficiency in adults
\end{itemize}
\item \textasciitilde{} 80\% of patients display, within the first weeks or months after
birth
\item multiple, recurrent opportunistic infections caused by a variety of
organisms, which rapidly become life-threatening.
\item infections are mainly localized in the skin, the respiratory,and the
gastrointestinal tract
\item in affected children over 6 months of age hypoplasia or apparent
absence of lymphoid tissue (tonsils, lymph nodes, thymus shadow on x-ray)
\item nonimmunological symptoms are also found
\begin{itemize}
\item 50\% have bone abnormalities
\item cognitive, behavioural, and neurological abnormalities can present
\begin{itemize}
\item lower IQ, hyperactivity, attention deficits, spasticity, head
lag, nystagmus, inability to focus, and high frequency
sensorineural deafness
\end{itemize}
\end{itemize}
\end{itemize}

\item Metabolic Derangement
\label{sec:orge38d7f5}
\begin{itemize}
\item accumulation in body fluids of adenosine and deoxyadenosine
\begin{itemize}
\item normally nearly undetectable
\end{itemize}
\item these compounds induce the premature death of lymphoid progenitor
cells, and thereby profoundly impair the generation of T, B, and NK
lymphocytes.
\item ADA deficiency affects to varying extent bone, brain, lung and liver
\end{itemize}

\item Genetics
\label{sec:orge25c055}
\begin{itemize}
\item AR, ADA1
\begin{itemize}
\item \textasciitilde{} 40\% of SCID
\end{itemize}
\end{itemize}

\item Diagnostic Tests
\label{sec:orgc407cc0}
\begin{itemize}
\item SCID can be confirmed by relatively simple laboratory tests:
\begin{itemize}
\item lymphopenia involving B, T and natural killer (NK) cells
\item hypogammaglobulinemia
\item IgM deficiency may be detected early
\item IgG deficiency becomes manifest only after the age of 3 months
when the maternal supply has been exhausted
\end{itemize}
\item the disease is progressive, since residual B- and T-cell function
which may be found at birth, disappears later on.

\item The enzymatic diagnosis is mostly confirmed on red blood cells
\item severity of disease correlates with the loss of ADA1 activity:
\begin{itemize}
\item 0-1\% activity in children with neonatal onset
\item 1-5\% activity in individuals with later onset
\end{itemize}
\end{itemize}

\item Treatment
\label{sec:org02da75d}
\begin{itemize}
\item HSCT
\item ERT - PEG-ADA1
\item gene therapy
\end{itemize}
\end{enumerate}

\subsubsection{Purine Nucleoside Phosphorylase Deficiency}
\label{sec:org8f7808d}
\begin{enumerate}
\item Clinical Presentation
\label{sec:orge22d01c}
\begin{itemize}
\item recurrent infections are usually of later onset
\item starting from the end of the first year to up to 5–6 years of age
\item initially less severe than in ADA1 deficiency
\item 2/3 have neurologic symptoms
\begin{itemize}
\item spastic tetra- or diplegia, ataxia and tremor, and mild to severe
mental retardation
\end{itemize}
\item 1/3 have autoimmune disorders
\begin{itemize}
\item hemolytic anemia, idiopathic thrombocytopenic purpura and
autoimmune neutropenia
\end{itemize}
\end{itemize}

\item Metabolic Derangement
\label{sec:org2907828}
\begin{itemize}
\item accumulation of four PDP substrates:
\begin{itemize}
\item guanosine, deoxyguanosine, inosine, deoxyinosine
\end{itemize}
\item \(\downarrow\) formation of uric acid
\item T-cells accumulate dGTP \(\to\) impaired immunity
\begin{itemize}
\item dGTP is formed from deoxyguanosine and inhibits ribonucleotide
reductase, and hence cell division.
\end{itemize}
\item ubiquitous expression of PNP explains the presence of nonimmunologic
symptoms in its deficiency.
\end{itemize}

\item Genetics
\label{sec:org4eac4a6}
\begin{itemize}
\item AR
\end{itemize}

\item Diagnostic Tests
\label{sec:org8f00933}
\begin{itemize}
\item \(\downarrow\) plasma uric acid
\item \(\downarrow\) urine uric acid

\item other causes of hypouricemia such as xanthine oxidase deficiency,
and drug administration (acetylsalicylic acid, thiazide diuretics),
should be ruled out.
\item enzymatic diagnosis of PNP deficiency is usually performed on red
blood cells
\end{itemize}

\item Treatment
\label{sec:org427735c}
\begin{itemize}
\item bone marrow transplantation
\item repeated transfusions of normal, irradiated erythrocytes
\end{itemize}
\end{enumerate}

\subsubsection{Xanthine Oxidase Deficiency}
\label{sec:orgc869ee1}
\begin{enumerate}
\item Clinical Picture
\label{sec:org9b22507}
\begin{itemize}
\item three types of deficiencies of xanthine oxidase
\item all cause xanthinuria
\begin{enumerate}
\item type I classi cal xanthinuria
\begin{itemize}
\item caused by isolated XO deficiency
\end{itemize}
\item type II classical xanthinuria
\begin{itemize}
\item deficiency of both XO and aldehyde oxidase (AO)
\end{itemize}
\item combined deficiency of XO, AO and sulfite oxidase
\end{enumerate}
\item type I and type II xanthinuria can be completely asymptomatic
\item about 1/3 of cases kidney stones are formed
\item myopathy w pain, stiffness
\end{itemize}

\item Metabolic Derangement
\label{sec:org803efb1}
\begin{itemize}
\item deficiency of XO results in the near total replacement of uric acid,
in plasma and urine, by hypoxanthine and xanthine as the end
products of purine catabolism
\item plasma hypoxanthine is not or minimally elevated
\begin{itemize}
\item due to reutilization by hypoxanthine-guanine phospho-ribosyltransferase
\end{itemize}
\item plasma xanthine \(\uparrow\) 10x

\item deficiency of AO \(\to\) inability to metabolize synthetic purine
analogues - allopurinol
\item The combined deficiency of XO, AO, and SO is caused by failure to
synthesize a molybdenum cofactor (MoCo), common to the three
oxidases
\end{itemize}

\item Diagnostic Tests
\label{sec:org82fb12f}
\begin{itemize}
\item \(\downarrow\) plasma uric acid
\item \(\downarrow\) urine uric acid
\item \(\Uparrow\) plasma xanthine
\end{itemize}

\item Treatment
\label{sec:org783062b}
\begin{itemize}
\item Type I and II XO deficiency are mostly benign
\begin{itemize}
\item \(\downarrow\) purine diet w \(\uparrow\) fluid intake to prevent renal stones.
\end{itemize}
\item The prognosis of combined XO,AO and SO deficiency improved by daily
infusion of cyclic pyranopterin monophosphate (cPMP)
\end{itemize}
\end{enumerate}

\subsubsection{Hypoxanthine-Guanine Phosphoribosyltransferase}
\label{sec:orgf226fbe}
\begin{enumerate}
\item Clinical Presentation
\label{sec:org9ed1426}
\begin{itemize}
\item clinical spectrum of this disorder is very wide and determined by
the residual activity of the enzyme
\item Lesch-Nyhan syndrome = complete or near-complete deficiency of HPRT
\item affected children generally appear normal during the first months of
life.
\item at 3 to 6 months of age, a neurological syndrome evolves
\begin{itemize}
\item classified as severe action dystonia, superimposed on a baseline hypotonia.
\end{itemize}
\item Patients develop a striking neuro-psychological profile comprising:
\begin{itemize}
\item compulsive self-destructive behaviour involving biting of their
fingers and lips
\item physical and verbal aggression
\end{itemize}
\item Speech is hampered by athetoid dysarthria
\item most patients have IQ’s around 60–70, some display normal intelligence.
\item form uric acid stones.
\item if untreated, the uric acid nephrolithiasis progresses to
obstructive uropathy and renal failure during the first decade of
life
\end{itemize}

\item Metabolic Derangement
\label{sec:org0a2642a}
\begin{itemize}
\item \(\Uparrow\) production of uric acid due to \(\uparrow\) \emph{de novo} synthesis
\begin{itemize}
\item caused by \(\uparrow\) PRPP, which is not recycled by HPRT
\end{itemize}
\end{itemize}

\item Genetic
\label{sec:org40c228b}
\begin{itemize}
\item XLR, HPRT
\end{itemize}

\item Diagnostic Tests
\label{sec:org2469770}
\begin{itemize}
\item \(\Uparrow\) urine and plasma uric acid
\begin{itemize}
\item uric acid/creatinine
\end{itemize}
\item patients with the Lesch-Nyhan syndrome display nearly undetectable
HPRT activity in red blood cells
\end{itemize}

\item Treatment and Prognosis
\label{sec:org1a6e0e3}
\begin{itemize}
\item allopurinol prevents urate nephropathy.
\begin{itemize}
\item even when given from birth or in combination with adenine has no
effect on the neurological symptoms
\end{itemize}
\end{itemize}
\end{enumerate}

\subsection{Pyrimidine Metabolism}
\label{sec:orgb545799}
\subsubsection{Introduction}
\label{sec:org2a72237}
\begin{itemize}
\item metabolism of the pyrimidine nucleotides can be divided into three
pathways:
\begin{enumerate}
\item biosynthetic \emph{de novo} pathway:
\begin{itemize}
\item starts with the formation of carbamoylphosphate by cytosolic
carbamoylphosphate synthetase (CPS II)
\item followed by the synthesis of UMP, CMP and TMP
\end{itemize}
\item catabolic pathway:
\begin{itemize}
\item starts from CMP, UMP and TMP
\item yields \(\beta\)-alanine and \(\beta\)-aminoisobutyrate
\item converted into intermediates of TCA cycle
\end{itemize}
\item salvage pathway:
\begin{itemize}
\item composed of kinases
\item converts pyrimidine nucleosides, cytidine, uridine, and
thymidine \(\to\) CMP, UMP, and TMP
\item also converts several pharmacological anticancer and antiviral
nucleoside analogs into their active forms
\end{itemize}
\end{enumerate}

\item inborn errors of pyrimidine metabolism comprise defects of:
\begin{itemize}
\item pyrimidine synthesis:
\begin{itemize}
\item CAD, UMP synthase deficiency and Miller syndrome
\end{itemize}
\item pyrimidine catabolism:
\begin{itemize}
\item deficiencies of dihydropyrimidine
dehydrogenase (DPD) dihydropyrimidinase (DHP)
\item ureidopropionase, thymidine phosphorylase
\item pyrimidine 5’-nucleotidase and cytidine deaminase
\item super-activity of cytosolic 5’-nucleotidase
\end{itemize}
\item pyrimidine salvage:
\begin{itemize}
\item thymidine kinase 2 deficiency
\end{itemize}
\end{itemize}
\end{itemize}

\subsubsection{UMP Synthase Deficiency}
\label{sec:orgcf6b37e}
\begin{itemize}
\item AKA: Hereditary Orotic Aciduria
\end{itemize}
\begin{enumerate}
\item Clinical Presentation
\label{sec:orgd7c03c6}
\begin{itemize}
\item megaloblastic anaemia a few weeks or months after birth
\begin{itemize}
\item usually the first manifestation
\end{itemize}
\item peripheral blood smears often show anisocytosis, poikilocytosis, and
moderate hypochromia
\item bone marrow examination reveals erythroid hyperplasia and numerous
megaloblastic erythroid precursors
\item characteristically, the anemia does not respond to iron, folic acid
or vitamin B 12
\item unrecognized, the disorder leads to FTT and to retardation of growth
and psychomotor development
\end{itemize}

\begin{figure}[htbp]
\centering
\includegraphics[width=0.9\textwidth]{./pp/figures/pyrimidine_met.png}
\caption{\label{fig:org9fbfec8}
Pyrimidine Metabolism}
\end{figure}

\item Metabolic Derangement
\label{sec:org9fef4a5}

\begin{itemize}
\item UMP synthase is a bifunctional enzyme of the \emph{de novo} synthesis of
pyrimidines
\item first reaction, orotate phosphoribosyltransferase (OPRT),converts
orotic acid into OMP
\item second, orotidine-5’-monophosphate decarboxylase (ODC),
decarboxylates OMP into UMP
\item deficiency \(\to\) massive overproduction of orotic acid
\begin{itemize}
\item due to \(\downarrow\) feedback inhibition exerted by the pyrimidine
nucleotides on the first enzyme of their \emph{de novo synthesis} CPS2
and deficiency of pyrimidine nucleotides
\end{itemize}
\item \(\downarrow\) pyrimidine nucleotides \(\to\) \(\downarrow\) cell division \(\to\) megaloblastic anemia
\end{itemize}

\item Genetics
\label{sec:org5f35280}
\begin{itemize}
\item AR, UMPS
\end{itemize}

\item Diagnostic Tests
\label{sec:orgd579398}
\begin{itemize}
\item \(\Uparrow\) urine orotic acid, 200-1000X
\end{itemize}

\item Treatment
\label{sec:org1db70a6}
\begin{itemize}
\item enzyme defect can be by-passed by the administration of uridine
\begin{itemize}
\item converted into UMP by uridine kinase
\end{itemize}
\end{itemize}
\end{enumerate}

\subsubsection{Dihydropyrimidine Dehydrogenase Deficiency}
\label{sec:org1c3b87a}
\begin{enumerate}
\item Clinical Presentation
\label{sec:org240ffb1}
\begin{itemize}
\item two forms:
\begin{enumerate}
\item infantile, severe
\begin{itemize}
\item epilepsy, motor and mental retardation
\item hypertonia, hyperreflexia, growth delay, microcephaly, autistic features
\end{itemize}
\item adult, partial
\begin{itemize}
\item found in adults who receive pyrimidine analog, 5-fluorouracil
\begin{itemize}
\item 5-fluorouracil used to treat cancers including breast, ovary colon
\end{itemize}
\item evere toxicity, manifested by profound neutropenia, stomatitis,
diarrhea and neurologic symptoms, including ataxia, paralysis
and stupor
\end{itemize}
\end{enumerate}
\end{itemize}

\item Metabolic Derangement
\label{sec:org5dc05d6}
\begin{itemize}
\item DPD catalyzes the catabolism of uracil and thymine \(\to\) dihydrouracil
and dihydrothymine
\item accumulation of uracil and thymine
\end{itemize}

\item Genetics
\label{sec:org7246eb2}
\begin{itemize}
\item AR, DPD for infantile form
\item adult form found in certain heterozygotes
\begin{itemize}
\item IVS14+1G>A
\end{itemize}
\end{itemize}

\item Diagnostic Tests
\label{sec:org3f1be64}
\begin{itemize}
\item \(\Uparrow\) urine uracil
\item \(\Uparrow\) urine thyamine
\item Enzyme activity in fibroblasts, liver and blood cells, with the
exception of erythrocytes
\end{itemize}

\item Treatment
\label{sec:orgcb5fa83}
\begin{itemize}
\item Infantile: None
\item Adult: avoid 5-fluorouracil
\end{itemize}
\end{enumerate}

\section{Porphyrias}
\label{sec:orgd174a04}
\subsection{Introduction}
\label{sec:org24bac13}
\begin{itemize}
\item Porphyria is a group of disorders caused by abnormalities in the
chemical steps that lead to heme production
\item Heme is a vital molecule for all of the body's organs, although it is
most abundant in the blood, bone marrow, and liver
\item Heme is a component of several iron-containing proteins called
hemoproteins, including hemoglobin and the cytochrome P450 family of
enzymes
\end{itemize}

\definesubmol{P}{-[::-60]-[::60](=[::60]O)-[::-60]OH}
\definesubmol{M}{CH_3}
\definesubmol{V}{=[::-60]CH_2}
\chemname{\chemfig[]{?[a]=[::+72]*5(-N?[b]=(-=[::-72]*5(-N?[c]
    (-[::-33,1.5,,,draw=none]{\color{red}Fe}?[b]?[c]?[d]?[e])-(=-[::-36]*5(=N?[d]-(=-[::-72]*5(-N?[e]-?[a]
    =(-!{M})-(-!{P})=))
    -(-!{P})=(-!{M})-))
    -(-!{V})=(-!{M})-))
    -(-!{V})=(-!{M})-)}}{Heme}

\begin{table}[htbp]
\caption{\label{tab:org5c7eeab}
Porphyrin names and the corresponding number of carboxyl groups}
\centering
\begin{tabular}{lr}
Porphyrin & Number of carboxyl groups\\
\hline
Uroporphyrin (octacarboxyl porphyrin) & 8\\
Heptacarboxyl porphyrin & 7\\
Hexacarboxyl porphyrin & 6\\
Pentacarboxyl porphyrin & 5\\
Coproporphyrin (tetracarboxyl porphyrin) & 4\\
Harderoporphyrin (tricarboxyl porphyrin) & 3\\
Protoporphyrin (dicarboxyl porphyrin) & 2\\
\end{tabular}
\end{table}


\begin{table}[htbp]
\caption{\label{tab:orgcfbbfe2}
Route of excretion dictated by solubility for porphyrin precursors}
\centering
\begin{tabular}{ll}
Porphyrin & Route\\
\hline
ALA & Urine\\
PBG & Urine\\
Uro & Urine\\
CI & >Fecal\\
CIII & >Urine\\
Proto & Fecal\\
\end{tabular}
\end{table}

\begin{itemize}
\item Porphyrins are heterocyclic macrocycles composed of four modified
pyrrole subunits
\item interconnected at their \(\alpha\)-carbon atoms via methine bridges
(=CH−)
\item Porphyrins are aromatic
\item Thus porphyrin macrocycles are highly conjugated systems
\item As a result they typically have very intense absorption bands in the
visible region and may be deeply colored
\item the name "porphyrin" comes from a Greek word for purple

\item Form complexes with metal ions
\begin{description}
\item[{Iron}] Heme
\item[{Magnesium}] Chlorophyll
\end{description}
\item Binds in hemoglobin
\item The highly reactive binds to the CYP450 heme iron
\item The CYP450 enzyme oxidizes substrates by transfer of the oxygen atom
\item Porphyrias can therefore lead to a wide variety of symptoms and
clinical manifestations
\end{itemize}

\begin{table}[htbp]
\caption{\label{tab:orgce8dff6}
Porphyrin Function}
\centering
\begin{tabular}{ll}
Protein & Functions\\
\hline
Hemoglobin & Oxygen transport\\
Myoglobin & Storage of oxygen in muscle\\
Cytochrome c & Electron transport\\
Cytochrome P450 & Drug Metabolism\\
Catalase & \ce{H2O2} breakdown\\
Tryptophan Pyrolase & Oxidation of tryptophan\\
\end{tabular}
\end{table}


\begin{table}[htbp]
\caption{\label{tab:orgf677366}
Main types of human porphyrias}
\centering
\begin{tabular}{llll}
Enzyme & Substrate & Disorder & Clinical\\
\hline
 & Glycine + Succinyl CoA &  & \\
ALAS & \(\downarrow\) &  & \\
 & \(\sigma\)-ALA &  & \\
ALAD & \(\downarrow\) & ADP & N\\
 & PBG &  & \\
HMBS & \(\downarrow\) & AIP & N\\
 & Hydroxymethylbilane &  & \\
UROS & \(\downarrow\) &  & C\\
 & Uroporphyrinogen-III &  & \\
UROD & \(\downarrow\) & PCT & C\\
 & Coproporphyrinogen-III &  & \\
CPO & \(\downarrow\) & HCP & N,C\\
 & Protoporphyrinogen-IX &  & \\
PPOX & \(\downarrow\) & VP & N,C\\
 & Protoporphyrin-IX &  & \\
FECH & \(\downarrow\) & EPP & C\\
\end{tabular}
\end{table}

\subsubsection{Nomenclature}
\label{sec:org10ddc4c}
\begin{itemize}
\item Porphyrinogens
\begin{itemize}
\item Reduced tetrapyrrole structures
\item Non-resonating
\item Non-photosensitizing
\item Only these can function as substrates for enzymes in heme synthesis
\begin{itemize}
\item exception protoporphyrin IX
\end{itemize}
\end{itemize}

\item Porphyrins
\begin{itemize}
\item Oxidized tetrapyrrole structures
\item Resonating
\item Photo-sensitizing
\item These absorb solar energy and cause cutaneous injury
\end{itemize}
\end{itemize}

\subsubsection{Clinical Classification of the Porphyrias}
\label{sec:orgd9d3dd6}
\begin{itemize}
\item Acute \textbf{Neurovisceral attacks}
\begin{itemize}
\item AIP
\item ADP: rare
\end{itemize}

\item Neuroviseral and/or Cutaneous
\begin{itemize}
\item HCP
\item VP
\end{itemize}

\item Cutaneous \textbf{photo-sensitivity, bullae, skin fragility}
\begin{itemize}
\item PCT
\item CEP
\item HEP
\item EPP
\end{itemize}
\end{itemize}
\begin{center}
\begin{tabular}{llllll}
Disorder & Enzyme & Prevalence & NV & Lesions & Site\\
\hline
Acute &  &  &  &  & \\
\hline
ADP & ALAD & - & - & - & \\
AIP & HMBS & 1-2:100,000 & + & - & hepatic\\
HCP & CPO & 1-2:10\(^{\text{6}}\) & + & fragile,bullae & hepatic\\
VP & PPOX & 1:2:50,000 & + & fragile,bullae & hepatic\\
\hline
Non-acute &  &  &  &  & \\
\hline
CEP & UROS & 1:10\(^{\text{6}}\) & - & fragile,bullae & erythropoietic\\
PCT & UROD & 1:25,000 & - & fragile,bullae & hepatic\\
EPP & FECH & 1:140,000 & - & photosensitiv,bullae & erythropoietic\\
\end{tabular}
\end{center}

\subsection{Biosynthesis and Biochemistry}
\label{sec:orga59cc25}
\subsubsection{Location of Heme Biosynthesis}
\label{sec:orgd7d4293}
\begin{itemize}
\item organelle: mitochondria \(\to\) cytoplasm \(\to\) mitochondria
\begin{itemize}
\item Starts with succinyl-CoA and glycine in mitochondria
\end{itemize}
\item Tissue: 70-80\% in bone marrow
\item 15\% in other tissue ie. liver \(\to\) Cyto P450, cytochromes
\end{itemize}

\begin{figure}[htbp]
\centering
\includegraphics[width=0.9\textwidth]{./porphyrins/figures/heme_synth.png}
\caption{\label{fig:orgdfdf3fd}
Heme Synthesis}
\end{figure}

\subsubsection{Reactions and Enzymes}
\label{sec:org02c64b9}
\begin{enumerate}
\item ALAS: 5-Aminolevulinate Synthase
\label{sec:orgc48f589}
\begin{enumerate}
\item X-linked sideroblastic anemia
\label{sec:orge7eb199}
\begin{itemize}
\item mitochondrial
\item rate limiting step under normal conditions
\item microcytic, hypochromic red cells
\item abnormal accumulation of iron in red blood cells \(\to\) ring
sideroblasts
\end{itemize}
\end{enumerate}

\item ALAD: Aminolevulinic Acid Dehydratase
\label{sec:orgdf24244}
\begin{enumerate}
\item ADP (ALA Dehydratase Porphyria)
\label{sec:orgc50e7c9}
\begin{itemize}
\item aka: porphobilinogen synthase
\item requires zinc, inhibited by lead
\item \textasciitilde{}five cases reported
\item[{Urine ALA:}] \(\uparrow\) \(\uparrow\) \(\uparrow\)
\item[{Urine PBG:}] Not elevated
\end{itemize}
\end{enumerate}

\item HMBS: Hydroxymethylbilane Synthase
\label{sec:org3067b62}
\begin{enumerate}
\item AIP (Acute Intermittent Porphyria)
\label{sec:orgac05012}
\begin{itemize}
\item aka: PBG deaminase
\item Four PBGs are combined through deamination
\item susceptible to allosteric inhibition by CIII and protoporphyrinogen
\item HMB is unstable \(\to\) \(\uparrow\) URO I
\item[{Urine PBG:}] \(\uparrow\) \(\uparrow\) \(\uparrow\)
\item[{Urine ALA:}] \(\uparrow\) \(\uparrow\) \(\uparrow\)
\item[{Rule Out:}] VP and HCP
\item \(\uparrow\) urine uroporphyrin arises from non-enzymatic
condensation of micro-molar concentrations of PBG.
\end{itemize}
\end{enumerate}
\item Non-AIP Acute Porphyrias
\label{sec:org8a71920}
\begin{itemize}
\item VP and HCP may not have skin lesions \(\to\) \textbf{fecal porphyrins}
\begin{itemize}
\item If normal, w \(\uparrow\) PBG, VP \& HCP are excluded \(\to\) \textbf{AIP}
\item If total \(\uparrow\) fecal porphyrins \(\to\) fractionate by HPLC
\end{itemize}

\item[{HCP}] Coproporphyrin-III \(\uparrow\) \(\uparrow\) \(\uparrow\)

\item[{VP}] Protoporphyrin-IX \(\uparrow\) \(\uparrow\) \(\uparrow\)
\begin{itemize}
\item \emph{Can also be due to diet or GI bleed}
\item Follow-up with plasma porphyrin emission scan
\end{itemize}
\end{itemize}

\item UROS: Uroporphyrinogen III synthase
\label{sec:org444c2db}
\begin{enumerate}
\item CEP (Congential Erythropoietic Porphyria)
\label{sec:orgb191bd9}
\begin{itemize}
\item HMB condensed \(\to\) Uro I or III
\item HMB \(\rightarrow\) Uro I: spontaneous
\item HMB \(\rightarrow\) Uro III: UROS
\item[{Urine Uro I:}] \(\uparrow\) \(\uparrow\) \(\uparrow\)
\item[{Urine Copro I:}] \(\uparrow\) \(\uparrow\) \(\uparrow\)
\item[{Fecal Copro I:}] \(\uparrow\) \(\uparrow\) \(\uparrow\)
\end{itemize}
\end{enumerate}

\item UROD: Uroporphyrinogen Decarboxylase
\label{sec:org75f61b4}
\begin{enumerate}
\item PCT (Porphyria Cutanea Tarda)
\label{sec:orga0cc180}
\begin{itemize}
\item last cytoplasmic enzyme, \(\downarrow\) polar
\item hepta, hexa and pentacarboxylate formed at the same active site
\begin{itemize}
\item \(\downarrow\) UROD \(\to\) increase in intermediates and uroporphyrins
\end{itemize}
\item[{Urine Uro I \& III:}] \(\uparrow\) \(\uparrow\) \(\uparrow\)
\end{itemize}
\end{enumerate}
\item CPOX: Coproporphyrinogen Oxidase
\label{sec:org9e7c994}
\begin{enumerate}
\item HCP (Hereditary Coproporphyria)
\label{sec:org7d9fa09}
\begin{itemize}
\item mitochondrial intermembrane space
\item inhibited by metals
\item specific for CIII
\item[{Urine PBG}] \(\uparrow\) \(\uparrow\) \(\uparrow\)
\item[{Fecal copro III}] \(\uparrow\) \(\uparrow\) \(\uparrow\)
\end{itemize}
\end{enumerate}

\item PPOX: Protoporphyrinogen Oxidase
\label{sec:org299a6fc}
\begin{enumerate}
\item VP (Varigate Porphyria)
\label{sec:orga9c86e7}
\begin{itemize}
\item inner mitochondrial membrane
\item[{Urine PBG:}] \(\uparrow\) \(\uparrow\) \(\uparrow\)
\item[{Fecal proto-IX}] \(\uparrow\) \(\uparrow\) \(\uparrow\)
\item[{Fecal copro-III}] \(\uparrow\) \(\uparrow\)
\item[{Plasma fluorescence scan}] \(\uparrow\) \(\uparrow\) \(\uparrow\)
\end{itemize}
\end{enumerate}

\item FECH: Ferrochelatase
\label{sec:org2c2d3a5}
\begin{enumerate}
\item EPP (Erythropoietic Protoporphyria)
\label{sec:org017af97}
\begin{itemize}
\item inserts ferrous iron into protoporphyrin to form heme
\item aka: heme synthase
\item in iron deficient states forms \(\to\) zinc protoporphyrin
\item[{RBC free protoporphyrin}] \(\uparrow\) \(\uparrow\) \(\uparrow\)
\end{itemize}
\end{enumerate}
\end{enumerate}

\subsubsection{Acute Porphyrias (Table \ref{tbl:onset})}
\label{sec:org0c78976}
\begin{itemize}
\item \textbf{ADP, AIP, VP, HCP}
\item Low clinical penetrance is a promenent feature of all AD acute porphyrias
\item 25\% of patients with overt acute porphyria have no family history
\begin{itemize}
\item sporadic presentation reflects high prevalence and low penetrance
\item acute porphyria caused by de novo mutation is uncommon
\end{itemize}
\item Allelic heterogenetity
\end{itemize}
\begin{enumerate}
\item Clinical Features
\label{sec:orgcedb5d6}
\begin{itemize}
\item Life threatening neuroviseral attack occur in AIP,VP and HCP
are clinically identical
\end{itemize}
\begin{table}[htbp]
\caption{\label{tab:org573892e}
Clinical features of acute neuroviseral attacks}
\centering
\begin{tabular}{lr}
Symptom/Sign & Percent\\
\hline
Abdominal pain & 97\\
Nonabdominal pain & 25\\
Vomiting & 85\\
Constipation & 46\\
Psychologic symptoms & 8\\
Convulsions & 5\\
Muscle weakness & 8\\
Sensory loss & 2\\
Hypertension (Diastolic >85 mmHg & 64\\
Tachycardia (>80/min) & 65\\
Hyponatremia & 37\\
\end{tabular}
\end{table}

\begin{itemize}
\item persistent psychiatric illness is not a feature of acute porphyrias.
\begin{itemize}
\item disappears with remission
\end{itemize}
\end{itemize}
\begin{enumerate}
\item Precipitating factors
\label{sec:orgbe4b3a7}
\begin{enumerate}
\item drugs
\item alcohol, especially binge drinking
\item the menstrual cycle
\item calorie restriction
\item infection
\item stress
\end{enumerate}
\item Drugs
\label{sec:orgc8c339a}
\begin{itemize}
\item barbiturates, sulfonamides, progestogens, anticonvulsants
\item \url{http://www.drugs-porphyria.org}
\end{itemize}
\item Long term complications
\label{sec:orgc42889d}
\begin{itemize}
\item chronic renal failure
\item hypertension
\item primary hepatocellular carcinoma
\end{itemize}
\end{enumerate}
\end{enumerate}

\subsubsection{Non-acute Porphyrias (Table \ref{tbl:onset})}
\label{sec:org1d8f068}
\begin{itemize}
\item \textbf{PCT, CEP, EPP}
\end{itemize}
\begin{enumerate}
\item PCT
\label{sec:org23d458c}
\begin{itemize}
\item \textbf{UROD}
\item most common, 2-5/million in UK
\item both sexes
\item onset during 5th and sixth decade
\end{itemize}
\begin{enumerate}
\item Clincal features
\label{sec:org51ce30e}
\begin{itemize}
\item lesions on sun-exposed skin
\begin{itemize}
\item back of hands
\item forearm
\item face
\end{itemize}
\item fragile skin
\item subepidermal bullae, milia, hypertrichosis of the face, patchy pigmentation
\item \(\uparrow\) LFTs in 50\%
\item Skin lesions with liver damage associated with:
\begin{itemize}
\item alcohol abuse
\item estrogens
\item infection with heptotropic viruses, HCV
\item hemochromatosis, iron overload
\end{itemize}
\end{itemize}
\item Pathogenesis and Molecular Genetics
\label{sec:orgecb5f86}
\begin{itemize}
\item \(\downarrow\) activity of UROD in liver \(\to\) \(\uparrow\) URO
\item 50\% \(\downarrow\) in UROD activity does not \(\to\) overt PCT
\begin{itemize}
\item further inactivation in the liver is required
\end{itemize}
\item 80\% of patients have sporadic (type I)
\begin{itemize}
\item enzyme defect is restricted to the liver
\item typically no family history
\end{itemize}
\item Famillial (type II)
\begin{itemize}
\item mutation in one UROD gene \(\to\) 1/2 normal activity
\end{itemize}
\item Exposure to polyhalogenated aromatic hydrocarbons
\end{itemize}
\item Treatment
\label{sec:orgffbebce}
\begin{itemize}
\item \(\downarrow\) exposure to light
\item iron depletion
\item chloroquine
\end{itemize}
\end{enumerate}
\item CEP
\label{sec:orgc321b20}
\begin{itemize}
\item \textbf{UROS}
\item least common, most severe of the cutaneous porphyrias.
\begin{itemize}
\item < 1:million in UK
\end{itemize}
\end{itemize}
\begin{enumerate}
\item Clinical Features
\label{sec:org503b281}
\begin{itemize}
\item Varying severity
\begin{itemize}
\item hydrops fetalis
\item onset in infancy of severe skin lesions, transfusion dependent
hemolytic anemia
\item mid-life onset of mild skin lesions resembling PCT
\end{itemize}
\item Most present in early infancy
\begin{itemize}
\item blisters on skin after UV exposure
\item reb-brown staining of diapers by urinary porphyrins
\end{itemize}
\item Ongoing destruction of ears, nose and eyelids, alopecia
\item red brown teeth
\item Skin changes usually accompanied by hemolytic anemia and splenomegaly
\end{itemize}
\item Pathogenesis and Molecular Genetics
\label{sec:org5ce5b05}
\begin{itemize}
\item Autosomal recessive, mutations in UROS or rarely GATA1
\item \(\downarrow\) UROS \(\to\) \(\uparrow\) UI
\item usually heteroallelic
\end{itemize}
\item Treatment
\label{sec:orgef186c7}
\begin{itemize}
\item \(\downarrow\) UV exposure
\item curative treatment - allogenic bone marrow transplantation
\item investigating gene therapy
\end{itemize}
\end{enumerate}

\item EPP
\label{sec:org285376a}
\begin{itemize}
\item \textbf{FECH}
\item X-linked dominant protoporphyria (XLDPP)
\begin{itemize}
\item 2\% of EPP cases
\item due to gain of function \textbf{ALAS2} mutation
\end{itemize}
\item life long acute photosensitivity due to \(\uparrow\) protoporphyrin-IX
in the skin
\item Absence of fragile skin, subepidermal bullae, and hypertrichosis
distinguishes it from all other cutaneous porphyrias.
\end{itemize}

\begin{enumerate}
\item Clinical Features
\label{sec:org1980960}
\begin{itemize}
\item acute photosensitivity
\item onset birth \(\to\) age 6, median = 1 year
\item both sexes
\item onset after 40 very rare
\begin{itemize}
\item most cases associated with myelodysplasia, caused by aquired
somatic mutation of FECH.
\end{itemize}
\item Most severe complication is progressive hepatic failure
\item Cholelithiasis - gallstones promoted by \(\uparrow\) protoporphyri in bile
\item mild microcytic anemia
\end{itemize}

\item Molecular Pathology and Genetics
\label{sec:org6830645}
\begin{itemize}
\item \(\uparrow\) protoporphyrin mainly in bone marrow
\item FECH mutation is autosomal recessive
\item compound heterozygotes
\item Threshold FECH activity = 35\% \(\to\) \(\uparrow\) protoporphyrin.
\item see XLDPP above
\end{itemize}
\item Treatment
\label{sec:org8e82a0e}
\begin{itemize}
\item avoid sunlight
\item annual LFTs
\item orthotopic liver transplantation
\item bone marrow transplantation
\end{itemize}
\end{enumerate}
\end{enumerate}

\subsubsection{Abnormalities of Porphyrin Metabolism not caused by Porphyria}
\label{sec:org12c4c2f}
\begin{itemize}
\item More common cause of abnormal porphyrin metabolism than porphyria.
\end{itemize}
\begin{enumerate}
\item Lead and Other Heavy Metals
\label{sec:org8bcf278}
\begin{itemize}
\item Lead exposure \(\uparrow\) urinary ALA and coproporphyrin III excretion
and accumulation of ZN-protoporphyrin in erythrocytes
\begin{itemize}
\item inhibition of ALAD, CPOX
\item Pb causes mito deficiency in Fe \(\to\) Zn replaces Fe as
substrate for FECH
\item \(\uparrow\) ALA excretion secondary to inhibition of ALAD
\begin{itemize}
\item caused by lead displacing zinc at catalytic site
\item ALAD2 isoform more susceptible than ALAD1
\end{itemize}
\end{itemize}
\end{itemize}
\item Secondary Coproporphyrinuria: Hepatobiliary and other Disorders
\label{sec:org1196dfa}
\begin{itemize}
\item Most common cause of abnormal porphyrin excretion
\begin{description}
\item[{alcohol intake}] CIII
\item[{impaired biliary excretion of CI}] \(\to\) urine
\begin{itemize}
\item cholestatic jaundice, hepatitis, and cirrhosis
\item reversal of normal ratio: CI dominates
\item drugs
\item severe infection
\end{itemize}
\item[{Dubin-Johnson}] \(\uparrow\) CI, \(\downarrow\) CII
\item[{Rotor}] \(\uparrow\) CI, normal CIII
\item[{Gilbert}] \(\uparrow\) CI, \(\uparrow\) CIII
\end{description}
\end{itemize}

\item Increased Fecal Porphyrin Concentration
\label{sec:org9ff7af9}
\begin{itemize}
\item Protoporphyrin and other dicarboxylic porphyrins derived from
bacterial metabolism.
\item Additional protoporphyrin and other dicarboxylic porphyrins formed
from heme containing proteins from diet or gastrointestinal
hemorrhage.
\item Even minor hemorrhage (ie FOBT negative) \(\uparrow\) dicarboxylic porphyrins.
\item Confusion with EPP may occur when associated iron deficiency
\(\uparrow\) erythrocyte total porphyrin, and skin lesions for other reasons.
\item Confusion with VP when coexisting liver disease causes
Coproporphyrinuria.
\item Porphyria is excluded when no porphyrin fluorescence is detectable
on fluorescenceq emission spectroscopy of plasma and fecal
coproporphyrin excretion is normal.
\item Consumption of Brewers yeast caused profile indistinguishable from
VP.
\end{itemize}

\item Increase Plasma Porphyrin Concentration: Renal and other Disorders
\label{sec:org33da045}
\begin{itemize}
\item Plasma Porphyrin concentration increased due to \(\downarrow\) renal or
hepatobiliary excretion is impaired.
\item ESRF marked increase in concentration, poor clearance by dialysis
\begin{itemize}
\item Similar to PCT, but not as high
\item PCT uncommon complicaiton of ESRF
\item Distinguish with fecal porphyrin analysis
\end{itemize}
\end{itemize}
\item Hematologic Disorders
\label{sec:org037c171}
\begin{itemize}
\item iron deficiency anemia, Zn acts as alternate substrate for FECH
\begin{itemize}
\item results in \(\uparrow\) erythrocyte ZPP
\item also sideroblastic megaloblastic, and hemolytic anemias
\end{itemize}
\end{itemize}
\item Hereditary Tyrosinemia Type I
\label{sec:orgfba4aa4}
\end{enumerate}

\subsection{Diagnostic Tests}
\label{sec:orgac00301}
\subsubsection{Laboratory Tests in an Acute Attack}
\label{sec:org55f100f}
\begin{itemize}
\item Acute attacks should have excess urinary excretion of or both.
\item Patients with cutaneous symptoms (VP,HCP) should also have excessive
production of porphyrins
\item \textbf{Genetic and/or enzyme studies are rarely helpful for diagnosis}
\item Urine should be \textbf{markedly elevated}
\begin{itemize}
\item Use a fresh RANDOM specimen, protect from light
\end{itemize}
\item During an acute attack, \textbf{a normal PBG essentially excludes all acute
neuro-visceral porphyrias} (except ADP).
\item When suspicion of an acute porphyria remains high while crisis is
resolving
\begin{itemize}
\item analysis of fecal and plasma porphyrins and urinary ALA is
advisable even if PBG is normal
\end{itemize}
\item \textbf{NB: Elevated PBG and ALA doesn't mean symptoms are caused by AIP}
\item Watson-Schwartz Test for PBG
\begin{itemize}
\item Urine is passed sequentially through an anion exchange column which
retains PBG and a cation exchange column which retains ALA.
\item Ehrlich's reagent is used to detect indoles and pyrroles.
\end{itemize}
\end{itemize}

\begin{figure}[htbp]
\centering
\includegraphics[width=0.9\textwidth]{./porphyrins/figures/urine.pdf}
\caption{\label{fig:org61811f8}
Urine Porphyrins}
\end{figure}

\begin{figure}[htbp]
\centering
\includegraphics[width=0.9\textwidth]{./porphyrins/figures/fecal.pdf}
\caption{\label{fig:orgcd350b3}
Fecal Porphyrins}
\end{figure}
\end{document}