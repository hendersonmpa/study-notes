% Created 2020-01-09 Thu 14:40
% Intended LaTeX compiler: pdflatex
\documentclass{scrartcl}
\usepackage[utf8]{inputenc}
\usepackage[T1]{fontenc}
\usepackage{graphicx}
\usepackage{grffile}
\usepackage{longtable}
\usepackage{wrapfig}
\usepackage{rotating}
\usepackage[normalem]{ulem}
\usepackage{amsmath}
\usepackage{textcomp}
\usepackage{amssymb}
\usepackage{capt-of}
\usepackage{hyperref}
\hypersetup{colorlinks,linkcolor=black,urlcolor=blue}
\usepackage{textpos}
\usepackage{textgreek}
\usepackage[version=4]{mhchem}
\usepackage{chemfig}
\usepackage{siunitx}
\usepackage{gensymb}
\usepackage[usenames,dvipsnames]{xcolor}
\usepackage[T1]{fontenc}
\usepackage{lmodern}
\usepackage{verbatim}
\usepackage{tikz}
\usepackage{wasysym}
\usetikzlibrary{shapes.geometric,arrows,decorations.pathmorphing,backgrounds,positioning,fit,petri}
\usepackage{fancyhdr}
\pagestyle{fancy}
\author{Matthew Henderson, PhD, FCACB}
\date{\today}
\title{Amino Acids}
\hypersetup{
 pdfauthor={Matthew Henderson, PhD, FCACB},
 pdftitle={Amino Acids},
 pdfkeywords={},
 pdfsubject={},
 pdfcreator={Emacs 26.1 (Org mode 9.1.9)}, 
 pdflang={English}}
\begin{document}

\maketitle
\setcounter{tocdepth}{2}
\tableofcontents


\section{Phenylalanine}
\label{sec:org0e6bbc9}
\subsection{Introduction}
\label{sec:org3f0d34e}
\begin{itemize}
\item Phe is an essential aromatic amino acid
\item mainly metabolised in the liver by the PHE hydroxylase (PAH) system
\item first step in the irreversible catabolism of Phe is hydroxylation to
Tyr by PAH
\item PAH requires the active pterin, tetrahydrobiopterin (BH\(_{\text{4}}\))
\item BH\(_{\text{4}}\) is formed in three steps from GTP
\begin{itemize}
\item also an obligate co-factor for tyrosine hydroxylase and tryptophan hydroxylase
\begin{itemize}
\item \(\therefore\) required for production of dopamine,catecholamines,
melanin and serotonin
\end{itemize}
\item alkylglycerol monooxygenase (AGMO) and nitric oxide synthase
\end{itemize}

\item defects in PAH and those involving production or recycling of BH\(_{\text{4}}\)
are associated with hyperphenylalaninaemia (HPA)
\item defects in PAH or production or recycling of BH\(_{\text{4}}\) may result in:
\begin{itemize}
\item hyperphenylalaninaemia (HPA)
\item deficiency of TYR, L-dopa, dopamine, melanin, catecholamines and 5-hydroxytryptophan (5HT)
\end{itemize}
\item when hydroxylation to Tyr is impeded, Phe may be transaminated to
phenylpyruvic acid and further reduced and decarboxylated
\end{itemize}


\begin{table}[htbp]
\label{tab:orgc2d926f}
\centering
\begin{tabular}{lll}
Class & Phe (\textmu{}mol/l) & PAH activity\\
\hline
Classic & \(\ge\) 1200 & \textless{}1\%\\
HPA & 600-1200 & 1-5\%\\
MHP & 120-600 & \textgreater{}5\%\\
BH\(_{\text{4}}\) responsive & \(\ge\) 30\% decrease & \\
\end{tabular}
\end{table}

\begin{figure}[htbp]
\centering
\includegraphics[width=0.9\textwidth]{./phe/figures/pah.png}
\caption{\label{fig:orgbe65f43}
Phenylalanine Hydroxylation}
\end{figure}

\subsection{Phenylalanine Hydroxylase Deficiency}
\label{sec:orgc25e571}
\subsubsection{Clinical Presentation}
\label{sec:org8c6612f}
\begin{itemize}
\item natural history of PKU is for affected individuals to suffer
progressive, irreversible neurological impairment during infancy and
childhood
\item untreated patients develop mental, behavioural, neurological and
physical impairments
\item most common outcome is a moderate to profound intellectual
developmental disorder (IQ \(\le\) 50) often associated with:
\begin{itemize}
\item mousey odour \(\to\) excretion of phenylacetic acid
\item reduced hair, skin and iris pigmentation \(\to\) reduced melanin synthesis
\item reduced growth and microcephaly
\item neurological impairments:
\begin{itemize}
\item 25\% epilepsy, 30\% tremor, 5\% spasticity of the limbs, 80\% EEG abnormalities
\end{itemize}
\end{itemize}
\item clinical phenotype correlates with Phe blood levels
\begin{itemize}
\item reflects the degree of PAH deficiency
\end{itemize}
\end{itemize}
\subsubsection{Metabolic Derangement}
\label{sec:org4e38aca}
\begin{itemize}
\item pathogenesis of brain damage in PKU is not fully understood
\begin{itemize}
\item it is causally related to the \(\uparrow\) blood Phe
\end{itemize}
\item Tyr becomes a semi-essential amino acid
\begin{itemize}
\item \(\downarrow\) blood levels \(\to\) impaired synthesis of biogenic amines including:
\begin{itemize}
\item melanin , dopamine and norepinephrine
\end{itemize}
\end{itemize}
\item \(\uparrow\) blood PHE levels result in an imbalance of other large
neutral amino acids (LNAA) within the brain
\begin{itemize}
\item \(\to\) decreased brain concentrations of Tyr and serotonin
\item ratio of Phe levels in blood/brain is about 4:1
\end{itemize}
\end{itemize}

\subsubsection{Genetics}
\label{sec:orgfe9182c}
\begin{itemize}
\item AR, PAH
\item R408W mutation \(\sim\) 30\% of alleles in Europeans with PKU
\item genotypes correlate well with biochemical phenotypes, pre-treatment
Phe levels and Phe tolerance
\end{itemize}

\subsubsection{Diagnostic Tests}
\label{sec:orgdd84b96}
\begin{itemize}
\item NBS for Phe and Phe/Tyr
\item exclude cofactor defects via pterins in blood or urine and
dihydropteridine reductase (DHPR) in blood
\item HPA may be found in preterm and sick babies particularly
\begin{itemize}
\item after parenteral feeding with amino acids
\item liver disease (where blood levels of methionine, TYR,
leucine/isoleucine are usually also raised)
\item treatment with inhibitors of dihydrofolate reductase (e.g. methotrexate, trimethoprim)
\begin{itemize}
\item folate promotes recycling of BH\(_{\text{4}}\)
\end{itemize}
\end{itemize}
\end{itemize}

\subsubsection{Treatment}
\label{sec:orgd75fcfa}
\begin{enumerate}
\item Diet
\label{sec:org2a95d21}
\begin{itemize}
\item reduce the blood Phe concentration sufficiently to prevent the
neuropathological effects but also to fulfil age-dependent
requirements for protein synthesis
\item Blood Phe is primarily a function of residual PAH activity and Phe
intake
\end{itemize}
\item BH\(_{\text{4}}\)
\label{sec:org99d0df0}
\begin{itemize}
\item pharmacological doses of BH\(_{\text{4}}\) can reduce blood phenylalanine levels
in some patients with PKU
\item sapropterin dihydrochloride (Kuvan), a synthetic formulation of the
active 6R-isomer of BH\(_{\text{4}}\) is approved in Europe and the USA for the
treatment of responsive patients with HPA and PKU, of all ages
\begin{itemize}
\item reduction of \(\ge\)30\% in blood Phe level after a single dose
\end{itemize}
\end{itemize}
\item Alternative/Experimental
\label{sec:org5370daa}
\begin{itemize}
\item liver transplantation
\item phenylalanine ammonia lyase (PAL)
\begin{itemize}
\item converts Phe \(\to\) harmless transcinnamic acid
\end{itemize}
\item large neutral amino acids
\begin{itemize}
\item tyr, trp,leu, ile, val compete with phe for the same transport at
the blood brain barrier
\end{itemize}
\end{itemize}
\end{enumerate}

\subsection{Maternal PKU}
\label{sec:org18a3150}
\subsubsection{Clinical Presentation}
\label{sec:orgdcd7b5f}
\begin{itemize}
\item first description of maternal PKU syndrome (MPKUS) recognised the
teratogenic effects of high maternal PHE levels
\item offspring of women with untreated classical PKU suffer developmental
delay, microcephaly, cardiac defects, low birth weight and
dysmorphic features
\item pathogenesis is poorly understood
\end{itemize}
\subsubsection{Metabolic Derangement}
\label{sec:org20ef9d0}
\begin{itemize}
\item maternal Phe \textless{} 360 umol/l \(\to\) no deleterious effect on the foetus
\item maternal Phe \textgreater{} 360 μmol/l, developmental indices decreased by
about three points for every 60 umol/l rise in average maternal Phe
\item \(\uparrow\) CHD \(\ge\) 900 umol/l
\end{itemize}
\subsubsection{Prevention}
\label{sec:org129544b}
\begin{itemize}
\item plan pregnancy
\item start diet before conception
\item monitoring expecting mothers 2x weekly
\end{itemize}

\subsection{HPA and Disorders of Biopterin Metabolism}
\label{sec:org7a1f52d}
\begin{itemize}
\item Disorders of BH\(_{\text{4}}\) associated with HPA and biogenic amine deficiency
include deficiencies of:
\begin{itemize}
\item GTP cyclohydrolase I (GTPCH)
\item 6-pyruvoyl-tetrahydropterin synthase (PTPS)
\item dihydropteridine reductase (DHPR)
\item pterin-4a-carbinolamine dehydratase (PCD)
\end{itemize}
\item Dopa-responsive dystonia (DRD) due to a dominant form of GTPCH
deficiency, and sepiapterin reductase (SR) deficiency (see Disorders of Monoamine Metabolism)
\end{itemize}
\subsubsection{Clinical Presentation}
\label{sec:org5d13bcb}
\begin{itemize}
\item can present in any of three ways:
\begin{enumerate}
\item Asymptomatic, but with raised PHE found following NBS; as part of
the standard screening protocol the infant is then investigated
further for biopterin defects
\item Symptomatic, with neurological deterioration in infancy despite a
low-PHE diet. This will occur where no further investigations are
routinely undertaken after a finding of HPA in NBS which is
wrongly assumed to be PAH deficiency
\item Symptomatic, with neurological deterioration in infancy on a
normal diet. This will occur either where there has been no NBS
for HPA or if the PHE level is not sufficiently raised to have
resulted in a positive screen or to require dietary treatment
\end{enumerate}
\end{itemize}
\subsubsection{Metabolic Derangement}
\label{sec:org11cc015}
\begin{itemize}
\item associated with decreased activity of PAH, tyrosine hydroxylase,
tryptophan hydroxylase and nitric oxide synthase (Figure \ref{fig:orgbe65f43})
\item degree of HPA is highly variable
\begin{itemize}
\item blood PHE concentrations ranging from normal to \textgreater{}2000 umol/l
\item CNS amine deficiency is most often profound and responsible for
the clinical symptoms
\item Decreased concentration of HVA in CSF is a measure of reduced
dopamine turnover
\item 5-HIAA deficiency is a measure of reduced serotonin metabolism
\end{itemize}
\end{itemize}

\subsubsection{Genetics}
\label{sec:org09d79d8}
\begin{itemize}
\item AR: GTPCH, PTPS, DHPR, PCD
\item biopterin disorders account for 1-3\% of infants found to have a
raised PHE on newborn screening
\end{itemize}

\subsubsection{Diagnostic Tests}
\label{sec:org732f3af}
\begin{itemize}
\item Urine or blood pterin analysis and blood DHPR assay
\item BH\(_{\text{4}}\) loading test
\item CSF neurotransmitters
\end{itemize}

\begin{table}[htbp]
\caption{\label{tab:org91db278}
Results in Biopterin Disorders}
\centering
\begin{tabular}{lrlllll}
Deficiency & Phe & biopterin\footnotemark & neopterin\textsuperscript{\ref{org142f2c1}} & primapterin\textsuperscript{\ref{org142f2c1}} & CSF 5HIAA HVA & DHPR activity\\
\hline
PAH & \textgreater{}120 & \(\uparrow\) & \(\uparrow\) & - & N & N\\
GTPCH & 50-1200 & \(\Downarrow\) & \(\Downarrow\) & - & \(\downarrow\) & N\\
PTPS & 240-2500 & \(\Downarrow\) & \(\Uparrow\) & - & \(\downarrow\) & N\\
DHPR & 180-2500 & \(\Downarrow\) & N or \(\uparrow\) & - & \(\downarrow\) & \(\downarrow\)\\
PCD & 180-1200 & \(\downarrow\) & \(\uparrow\) & \(\Uparrow\) &  & N\\
\end{tabular}
\end{table}\footnotetext[1]{\label{org142f2c1}blood or urine}

\subsubsection{Treatment}
\label{sec:org7db9d68}
\begin{itemize}
\item BH\(_{\text{4}}\)
\item CNS amine replacement
\end{itemize}

\section{Tyrosine}
\label{sec:org63dcc9e}
\subsection{Introduction}
\label{sec:org4622752}

\begin{itemize}
\item Tyrosine is a non-essential amino acid derived from diet and hydroxylation of phenylalanine
\item a precursor of DOPA, thyroxine and melanin
\item glucogenic and ketogenic \(\to\) catabolism in liver cytosol \(\to\) fumarate and acetoacetate
\item Tyr \(\to\) 4-hydroxyphenylpyruvate by cytosolic tyrosine aminotransferase
\begin{itemize}
\item also in liver and other tissues by mitochondrial aspartate aminotransferase
\begin{itemize}
\item minor role under normal conditions
\end{itemize}
\end{itemize}
\item penultimate intermediates maleylacetoacetate and fumarylacetoacetate
are reduced to succinylacetoacetate
\begin{itemize}
\item decarboxylation to succinylacetone
\item succinylacetone is the most potent known inhibitor of the heme biosynthetic enzyme ALAD
\end{itemize}
\end{itemize}

\begin{figure}[htbp]
\centering
\includegraphics[width=0.9\textwidth]{./tyr/figures/tyr.png}
\caption{\label{fig:org7f93529}
Tyrosine Catabolism:1 Tyrosine aminotransferase; 2 4-hydroxyphenylpyruvate dioxygenase; 3 homogentisate dioxygenase; 4 fumarylacetoacetase; 5 AST; 6 ALAD}
\end{figure}


\begin{itemize}
\item Five inherited disorders of tyrosine metabolism are known:
\begin{itemize}
\item Tyrosinemia type I is characterised by progressive
liver disease and renal tubular dysfunction with rickets
\item Tyrosinemia type II presents with keratitis and
blistering lesions of the palms and soles and neurological
complications
\item Tyrosinemia type III may be asymptomatic or associated with
mental retardation
\item Hawkinsinuria may be asymptomatic or present with failure to
thrive and metabolic acidosis in infancy
\item Alkaptonuria presents as adult with symptoms of osteoarthritis
\end{itemize}
\item Other inborn errors of tyrosine metabolism discussed in Monoamine Metabolism
\end{itemize}

\subsubsection{Transient Tyrosinemia}
\label{sec:orga18f43c}
\begin{itemize}
\item one of the most common amino acid disorders
\begin{itemize}
\item clinically asymptomatic
\end{itemize}
\item believed to be caused by late fetal maturation of
4-hydroxyphenylpyruvate dioxygenase
\item more common in premature infants than in full-term newborns
\item protein intake is an important aetiological factor
\begin{itemize}
\item incidence has fallen dramatically in the last 4 decades, with the
reduction in the protein content of newborn formula
\end{itemize}
\end{itemize}


\subsection{Tyrosinemia Type I}
\label{sec:org7beac70}
\begin{itemize}
\item Hepatorenal Tyrosinemia
\end{itemize}
\subsubsection{Clinical presentation}
\label{sec:org2e1b6a0}
\begin{itemize}
\item very variable, presents at any time from the neonatal period to adulthood
\begin{description}
\item[{acute}] before 6 months of age with acute liver failure
\item[{subacute}] between 6 months and 1 year of age with liver disease,
failure to thrive, coagulopathy, hepatosplenomegaly,
rickets and hypotonia
\item[{chronic}] after the 1st year with chronic liver disease, renal
disease, rickets, cardiomyopathy \textpm{} porphyria-like
syndrome
\end{description}

\item liver is the major affected organ
\begin{itemize}
\item morbidity and mortality
\end{itemize}
\item renal disease is detected in most patients
\begin{itemize}
\item proximal tubular disease
\end{itemize}
\item acute neurological crisis can occur
\begin{itemize}
\item painful paresthesias and autonomic signs that may progress to
paralysis
\end{itemize}
\end{itemize}

\subsubsection{Metabolic derangement}
\label{sec:orgd554dea}
\begin{itemize}
\item deficiency of the enzyme fumarylacetoacetate hydrolase (FAH)
\end{itemize}
\ce{fumarylacetoacetate ->[FAH] fumarate + acetoacetate}
\begin{itemize}
\item mainly expressed in the liver and kidney
\item compounds immediately upstream from the FAH reaction,
maleylacetoacetate (MAA) and fumarylacetoacetate (FAA), and their
derivatives, succinylacetone (SA) and succinylacetoacetate (SAA)
accumulate and have important pathogenic effects
\item effects of FAA and MAA occur only in the cells of the organs in which they are produced
\begin{itemize}
\item these compounds are not found in body fluids of patients
\end{itemize}
\item SA and SAA are readily detectable in plasma and urine and have
widespread effects
\item FAA, MAA and SA disrupt sulfhydryl metabolism by forming glutathione
adducts \(\to\) free rad ical damage
\item Disruption of sulfhydryl metabolism is also believed to cause
secondary deficiency of two other hepatic enzymes,
4-hydroxyphenylpyruvate dioxygenase and methionine
adenosyltransferase, resulting in hypertyrosinemia and
hypermethioninemia
\item SA is a potent inhibitor of ALAD
\end{itemize}

\subsubsection{Genetics}
\label{sec:org864ff05}
\begin{itemize}
\item AR, FAH
\item most common mutation is c.1062+5G>A
\begin{itemize}
\item is found in about 25\% of the alleles worldwide
\item predominant mutation in the French-Canadian population, in which
it accounts for >90\% of alleles
\end{itemize}
\end{itemize}

\subsubsection{Diagnostic Tests}
\label{sec:org175ac08}
\begin{itemize}
\item \(\uparrow\) SA in urine, plasma or DBS is pathognomonic
\item \(\uparrow\) tyrosine
\item \(\uparrow\) phenylalanine
\item \(\uparrow\) methionine
\item \(\uparrow\) urine ALA
\item symptomatic patients, biochemical tests of liver function are
usually abnormal
\begin{itemize}
\item coagulopathy and/or hypoalbuminaemia
\end{itemize}
\item acutely ill patients
\begin{itemize}
\item \(\Uparrow\) \(\alpha\)-fetoprotein
\item Fanconi-type tubulopathy is often present with:
\begin{itemize}
\item aminoaciduria, phosphaturia and glycosuria
\item radiological evidence of rickets may be present
\end{itemize}
\end{itemize}
\end{itemize}

\subsubsection{Treatment}
\label{sec:org0cb7008}
\begin{itemize}
\item Nitisinone (aka: NTBC) is the recommended therapy, in combination
with a tyrosine and phenylalanine restricted diet
\begin{itemize}
\item inhibits 4-hydroxyphenylpyruvate dioxygenase turning Type I into Type III
\end{itemize}
\item nitisinone block tyrosine degradation at an early step
\begin{itemize}
\item \(\downarrow\) FAA, MAA and SA
\item \(\uparrow\) tyrosine and 4-hydroxyphenylpyruvate
\end{itemize}
\item liver transplantation \(\to\) functional cure
\begin{itemize}
\item normal diet
\item mortality and life long immunosuppressive therapy
\end{itemize}
\end{itemize}


\subsection{Tyrosinemia Type II}
\label{sec:org5a6a526}
\begin{itemize}
\item Oculocutaneous Tyrosinemia
\end{itemize}
\subsubsection{Clinical presentation}
\label{sec:org9583d31}
\begin{itemize}
\item any combination of: 
\begin{itemize}
\item ocular lesions
\item skin lesions
\item neurological complications
\end{itemize}
\item usually presents in infancy but can be any age
\end{itemize}

\subsubsection{Metabolic derangement}
\label{sec:orgd12be84}
\begin{itemize}
\item hepatic cytosolic tyrosine aminotransferase
\end{itemize}
\ce{tyrosine ->[TAT] 4-hydroxyphenylpyruvate}
\begin{itemize}
\item \(\uparrow\) tyrosine in CSF and serum
\end{itemize}
\begin{itemize}
\item \(\uparrow\) phenolic acids 4-hydroxyphenylpyruvate,
4-hydroxyphenyllactate and 4-hydroxyphenylacetate
\end{itemize}

\subsubsection{Genetics}
\label{sec:org4fafbc5}
\begin{itemize}
\item AR, TAT
\end{itemize}

\subsubsection{Diagnostic tests}
\label{sec:orgbbbe194}
\begin{itemize}
\item \(\Uparrow\) plasma tyrosine > 1200 umol/L
\begin{itemize}
\item if lower consider Type III
\end{itemize}
\item Urine organic acids
\begin{itemize}
\item \(\Uparrow\) urine 4-hydroxyphenylpyruvate
\item \(\Uparrow\) 4-hydroxyphenyllactate
\item \(\Uparrow\) 4-hydroxyphenylacetate
\item \(\uparrow\) N-acetyltyrosine
\item \(\uparrow\) 4-tyramine
\end{itemize}
\end{itemize}

\subsubsection{Treatment}
\label{sec:org35c6c99}
\begin{itemize}
\item tyrosine and phenylalanine restricted diet
\end{itemize}

\subsection{Tyrosinemia Type III}
\label{sec:org7b769ac}
\subsubsection{Clinical presentation}
\label{sec:orgea6cb58}
\begin{itemize}
\item very rare, 13 cases described
\item most common long-term complication is intellectual impairment
\end{itemize}
\subsubsection{Metabolic derangement}
\label{sec:org8c4e566}
\begin{itemize}
\item 4-hydroxyphenylpyruvate dioxygenase
\end{itemize}
\ce{4-hydroxyphenylpyruvate ->[HPD] homogentisate}
\begin{itemize}
\item \(\uparrow\) plasma tyrosine
\item \(\uparrow\) urine 4-hydroxyphenylpyruvate, 4-hydroxyphenyllactate and 4-hydroxyphenylacetate
\end{itemize}
\subsubsection{Genetics}
\label{sec:orge0fdcc5}
\begin{itemize}
\item AR, HPD
\end{itemize}
\subsubsection{Diagnostic tests}
\label{sec:org5d99878}
\begin{itemize}
\item \(\uparrow\) plasma tyrosine 300-1300 umol/L
\item \(\uparrow\) urine 4-hydroxyphenylpyruvate, 4-hydroxyphenyllactate and 4-hydroxyphenylacetate
\end{itemize}


\subsection{Alkaptonuria}
\label{sec:org2baf3a6}
\subsubsection{Clinical presentation}
\label{sec:orged21cc3}
\begin{itemize}
\item clinical symptoms first appear in adulthood
\begin{itemize}
\item some cases diagnosed in infancy due to darkening of urine when
exposed to air
\end{itemize}
\item most prominent symptoms relate to joint and connective tissue involvement;
\item significant cardiac disease and urolithiasis may be detected in later years
\end{itemize}
\subsubsection{Metabolic derangement}
\label{sec:orga8e605e}
\begin{itemize}
\item first identified IEM in 1902 by Garrod
\item homogentisate dioxygenase expressed mainly in the liver and the
kidneys
\end{itemize}
\ce{homogentisate ->[HGD] maleylacetoacetate}
\begin{itemize}
\item accumulation of homogentisate and its oxidised derivative
benzoquinone acetic acid (the toxic metabolite) in various tissues
\end{itemize}
\subsubsection{Genetics}
\label{sec:org20a0dca}
\begin{itemize}
\item AR, HGD
\item 1:250000-1:1000000
\end{itemize}
\subsubsection{Diagnostic tests}
\label{sec:orgd5c64c6}
\begin{itemize}
\item Alkalinisation of the urine \(\to\) immediate dark brown colouration
\item \(\uparrow\) urine homogentisate \(\to\) positive test for reducing substances
\item \(\uparrow\) UOA homogentisic acid
\end{itemize}
\subsubsection{Treatment}
\label{sec:orgaaaf848}
\begin{itemize}
\item vitamin C
\item nitisinone with \(\downarrow\) phe and tyr diet
\begin{itemize}
\item 3-year clinical trial of nitisinone \(\to\) 95\% \(\downarrow\) urine and plasma homogentisic acid
\item no demonstrable effects on clinical symptoms
\end{itemize}
\end{itemize}

\subsection{Hawkinsinuria}
\label{sec:orgebc61f1}
\subsubsection{Clinical presentation}
\label{sec:org3a72300}
\begin{itemize}
\item only been described in a few families
\item FTT and metabolic acidosis in infancy
\item early weaning from breastfeeding seems to precipitate the disease
\begin{itemize}
\item may be asymptomatic in breastfed infants
\end{itemize}
\end{itemize}

\subsubsection{Metabolic derangement}
\label{sec:org8b660cb}
\begin{itemize}
\item abnormal metabolites produced in hawkinsinuria
\begin{itemize}
\item (hawkinsin (2-cysteinyl-1,4-dihydroxycyclohexenylacetate)
\item 4-hydroxycycloxylacetate)
\end{itemize}
\item thought to derive from in-complete conversion of
4-hydroxyphenylpyruvate to homogentisate caused by a defect in
4-hydroxyphenylpyruvate dioxygenase
\end{itemize}
\ce{4-hydroxyphenylpyruvate ->[HPD] homogentisate}
\begin{itemize}
\item hawkinsin is product of a reaction of an epoxide intermediate with
glutathione, which may be depleted
\item metabolic acidosis due to 5-oxoproline accumulation secondary to
glutathione depletion
\end{itemize}

\subsubsection{Genetics}
\label{sec:org096194e}
\begin{itemize}
\item AR, HPD
\item mutations that lead to a retention of partial HPD function,
\begin{itemize}
\item production of hawkinsin and 4-hydroxycyclohexylacetate
\end{itemize}
\end{itemize}
\subsubsection{Diagnostic tests}
\label{sec:orgb291505}
\begin{itemize}
\item UAO hawkinsin or 4-hydroxycyclohexylacetate is diagnostic
\item may be moderate tyrosinaemia, increased urinary
4-hydroxyphenylpyruvate and 4-hydroxyphenyllacate, metabolic
acidosis and 5-oxoprolinuria during infancy
\end{itemize}

\subsubsection{Treatment}
\label{sec:org3403564}
\begin{itemize}
\item return to breastfeeding or low tyr and phe diet
\item asymptomatic after the 1st year of life
\item affected infants are reported to have developed normally
\end{itemize}

\section{Sulfur Amino Acids}
\label{sec:org31dc03c}
\subsection{Introduction}
\label{sec:orgd7916fa}
\begin{itemize}
\item methionine is converted by two methionine adenosyltransferases (MAT
I/III and MATII) to S-adenosylmethionine (SAM)
\item methyl group of SAM is used in numerous methylation reactions,
yielding S-adenosylhomocysteine (SAH)
\item excess SAM is removed from the cycle by glycine N-methyltransferas (GNMT)
\item SAH is cleaved by S-adenosylhomocysteine hydrolase (SAHH) to
homocysteine and adenosine, which is phosphorylated by adenosine
kinase (ADK)
\item homocysteine has two metabolic pathways:
\begin{enumerate}
\item remethylated back to methionine by the remethylation pathway or
using betaine as a methyl-group donor, in patients treated with
this drug
\item irreversibly metabolized to sulfate
by the transsulfuration pathway
\begin{itemize}
\item homocysteine and serine are condensed by cystathionine \(\beta\)-Synthase (CBS) to cystathionine
\item cystathionine is cleaved by cystathionine \(\gamma\)-lyase (CTH) to
form cysteine and \(\alpha\)-ketobutyrate
\end{itemize}
\end{enumerate}
\item CTH can use cysteine and/or homocysteine to synthesize hydrogen
sulfide
\item cysteine can be further converted:
\begin{itemize}
\item in a series of reactions to taurine or
\item via the mitochondrial enzymes, AST and 3-mercaptopyruvate
sulfurtransferase (MPST), to pyruvate and hydrogen sulfide
\end{itemize}
\item mitochondrial oxidation of hydrogen sulfide and of cysteine involves
several steps yielding thiosulfate , sulfite and finally sulfate
\item inorganic sulfur released from cysteine residues \(\to\) mitochondrial
iron-sulfur (FeS) cluster cofactors
\item availability of cysteine in the neonatal period is limited because
its endogenous synthesis from methionine by the transsulfuration
pathway is markedly attenuated
\item activity of the rate limiting enzyme in the pathway, cystathionase ,
is very low at birth and increases slowly during the first few months of life
\begin{itemize}
\item cysteine is considered a conditionally essential amino acid, at
least in preterm infants
\end{itemize}

\item disorders in sulfur amino acid metabolism exhibit:
\begin{itemize}
\item altered methionine, S-adenosylmethionine, sarcosine, S-adenosylhomocysteine,
total homocysteine or cystathionine concentrations in blood
\item adenosine or thiosulfate excretion in urine
\end{itemize}
\end{itemize}


\begin{itemize}
\item CBS deficiency – classical homocystinuria – is the most common
disease in this group
\begin{itemize}
\item severity varies from
\begin{itemize}
\item multisystemic childhood condition with lens dislocation,
osteoporosis, marfanoid features, central nervous system and
vascular complications
\item isolated thromboembolic disease in adults.
\end{itemize}
\end{itemize}
\item CTH deficiency appears to be a biochemical trait with no major
clinical sequelae
\item Disorders of cysteine and hydrogen sulfide oxidation pathway include:
\begin{itemize}
\item ethylmalonic encephalopathy
\item isolated sulfite oxidase deficiency
\item combined sulfite oxidase deficiency
\begin{itemize}
\item due to impaired molybdenum cofactor synthesis
\end{itemize}
\end{itemize}
\item these are severe disorders with early-onset seizures and other
neurological complications
\begin{itemize}
\item other signs include orthostatic acrocyanosis, lens dislocation or
urolithiasis;
\end{itemize}
\item only molybdenum cofactor deficiency type A can be treated
successfully, with a synthetic cofactor
\end{itemize}

\begin{figure}[htbp]
\centering
\includegraphics[width=0.9\textwidth]{./figures/cys.jpg}
\caption{\label{fig:org398f237}
Cysteine vs Cystine}
\end{figure}

\url{http://en.citizendium.org/wiki/File:Cysteine\_vs\_Cystine10.jpg}

\begin{figure}[htbp]
\centering
\includegraphics[width=0.9\textwidth]{./sulfur/figures/sulfuraa.png}
\caption{\label{fig:org500fd6c}
Sulfur amino acid metabolism}
\end{figure}

\begin{figure}[htbp]
\centering
\includegraphics[width=0.9\textwidth]{./figures/sulfaa.png}
\caption{\label{fig:org8a3e882}
Disorders of sulfur amino acid metabolism}
\end{figure}

\subsection{Methionine S-Adenosyltransferase Deficiency}
\label{sec:org983e880}
\begin{itemize}
\item Mudd’s Disease
\end{itemize}
\subsubsection{Clinical presentation}
\label{sec:org58943ca}
\begin{itemize}
\item most patients detected by NBS for CBS deficiency using methionine as a marker
\item neurological abnormalities occur in most patients with plasma methionine \textgreater{} 800 μmol/l
\begin{itemize}
\item rare in subjects with lower levels
\end{itemize}
\end{itemize}
\subsubsection{Metabolic derangement}
\label{sec:orgd5e7bf6}
\begin{itemize}
\item Methionine S-adenosyltransferase converts methionine to S-adenosylmethionine (SAM) using ATP
\item MAT exists in 3 forms
\item MAT I and III are encoded by the same gene
\begin{itemize}
\item tetrameric and dimeric forms, respectively
\item liver specific
\end{itemize}
\item MAT II is encoded by a different gene
\begin{itemize}
\item converts methionine to SAM outside the liver
\item explains why MAT I/III deficiency is relatively benign
\end{itemize}
\end{itemize}
\subsubsection{Genetics}
\label{sec:orgb98ff93}
\begin{itemize}
\item AR, MAT1A
\item some mutation are AD
\end{itemize}
\subsubsection{Diagnostic Tests}
\label{sec:org0288bae}
\begin{itemize}
\item plasma methionine 50 to \textgreater{} 2000 umol/L
\item other causes of hypermethioninemia:
\begin{itemize}
\item liver disease
\item prematurity
\item excessive intake of methionine
\item less often, CBS, S-adenosylhomocysteine hydrolase and ADK deficiencies
\begin{itemize}
\item CBS has \(\Uparrow\) homocysteine
\end{itemize}
\end{itemize}
\end{itemize}
\subsubsection{Treatment}
\label{sec:org562a62a}
\begin{itemize}
\item methionine restricted diet if met \textgreater{} 800 umol/L
\end{itemize}

\subsection{Cystathionine \(\beta\)-Synthase Deficiency}
\label{sec:org1b978c6}
\subsubsection{Clinical presentation}
\label{sec:orgc2b293b}
\begin{itemize}
\item wide spectrum of severity and age at presentation
\item some asymptomatic into adulthood
\item others have severe multisystem disease
\item clinical features predominantly involve four organ systems:
\begin{description}
\item[{eye}] lens dislocation
\item[{skeleton}] excessive growth - Marfanoid but stiff
\item[{brain}] learning disabilities
\item[{vascular}] thromboembolism
\end{description}
\end{itemize}
\subsubsection{Metabolic derangement}
\label{sec:org18fce5d}
\begin{itemize}
\item CBS is a cytosolic tetrameric enzyme
\item expressed predominantly in liver, pancreas, kidney and brain
\item activity can alsobe determined in cultured fibroblasts and in plasma
due to its release from the liver
\item catalytic domain binds heme,the cofactor PLP and substrates
\item regulatory domain binds the allosteric activator SAM
\item pathophysiology is not fully understood
\begin{itemize}
\item \(\uparrow\) SAH impairs methylation reaction
\item \(\uparrow\) homocysteine \(\to\) ER stress, vascular disease
\item enhanced remethylation methionine
\item depletion of cystathionine and cysteine \(\to\) apoptosis, oxidative stress, \(\delta\) protein structure.
\end{itemize}
\end{itemize}

\subsubsection{Genetics}
\label{sec:orgc497a6b}
\begin{itemize}
\item AR, CBS
\end{itemize}

\subsubsection{Diagnostic tests}
\label{sec:orgcb30846}
\begin{itemize}
\item plasma total homocysteine (tHcy)
\begin{itemize}
\item \textgreater{} 100 umol/L
\item plasma should be separated from whole blood within one hour of venepuncture
\end{itemize}
\item measurement of free homocystine is not recommended
\begin{itemize}
\item low sensitivity
\item complicated pre-analytical requirements
\end{itemize}
\item to avoid misdiagnosis in pyridoxine responsive patients
\begin{itemize}
\item pyridoxine supplements including multivitamins should be avoided
for at least 2 weeks prior to testing
\end{itemize}
\item diagnosis very likely if the plasma methionine is high or borderline
high and supported by:
\begin{itemize}
\item \(\downarrow\) to low-normal plasma cystathionine
\item \(\uparrow\) methionine:cystathionine
\end{itemize}
\item can be con firmed by enzyme assay in cultured fibroblasts or plasma,
and/or mutation analysis of the CBS gene
\end{itemize}

\subsubsection{Treatment}
\label{sec:org14a89de}
\begin{itemize}
\item pyridoxine, betaine and a methionine-restricted diet
\end{itemize}

\subsection{Molybdenum Cofactor Deficiency}
\label{sec:orgfff3333}
\subsubsection{Clinical presentation}
\label{sec:org45079dc}
\begin{itemize}
\item usually present soon after birth with poor feeding,hypotonia,
exaggerated startle reactions and intractable seizures, resembling
hypoxic ischaemic encephalopathy
\begin{itemize}
\item \(\to\) multicystic leukoencephalopathy with microcephaly
\end{itemize}
\item dislocation of the ocular lens occurs during infancy and xanthine
renal stones can develop later
\end{itemize}

\subsubsection{Metabolic derangement}
\label{sec:orgfac0e05}
\begin{itemize}
\item molybdenum cofactor (MoCo) synthesis involves three steps:
\begin{description}
\item[{MoCo deficiency type A}] affects the conversion of GTP to cyclic
pyranopterin monophosphate (cPMP)
\item[{MoCo deficiency type B}] cannot convert cPMP to molybdopterin
\item[{MoCo deficiency type C}] affects gephyrin, which catalyses
adenylation of molybdopterin and insertion of molybdenum to
form the cofactor
\end{description}
\item molybdenum cofactor is needed for:
\begin{itemize}
\item sulfite oxidase
\item aldehyde oxidase
\item mitochondrial amidoxime reducing component (mARC)
\item xanthine dehydrogenase
\end{itemize}
\item xanthine dehydrogenase deficiency causes raised xanthine and low
urate concentrations
\item sulfite accumulation is responsible for the neurotoxicity and lens
dislocation
\end{itemize}

\subsubsection{Genetics}
\label{sec:org53bdedb}
\begin{itemize}
\item AR
\item Type A, MOCS1, most common
\item Type B, MOCS2
\item Type C, GPHN, rare
\end{itemize}

\subsubsection{Diagnostic tests}
\label{sec:org67c993a}
\begin{itemize}
\item plasma urate concentration is initially normal but decreases after a
few days and remains low (<0.06 mmol/L)
\item \(\uparrow\) urine xanthine
\item sulfite can be detected in fresh urine using dipsticks but false
positive and negative results occur
\item \(\uparrow\) urine or blood s-sulfocysteine is a more reliable indicator
\item s-sulfocysteine accumulation \(\to\) inhibition of antiquitin
\begin{itemize}
\item secondary elevation of pipecolic acid
\end{itemize}
\item \(\uparrow\) plasma taurine and thiosulfate
\item \(\downarrow\) plasma total cysteine and tHcy
\item diagnosis is confirmed by mutation analysis
\end{itemize}

\subsubsection{Treatment}
\label{sec:org9fc29b1}
\begin{itemize}
\item without treatment, patients have profound handicap and die early
\item successful treatment of Type A with daily intravenous infusions of
cPMP
\item no treatment for Types B \& C
\end{itemize}

\subsection{Isolated Sulfite Oxidase Deficiency}
\label{sec:orgf7fe6d2}
\subsubsection{Clinical presentation}
\label{sec:org658d6c4}
\begin{itemize}
\item resembles MoCo deficiency
\end{itemize}
\subsubsection{Metabolic derangement}
\label{sec:orgb94e119}
\begin{itemize}
\item sulfite derived from cysteine is normally oxidised to form
sulfate
\item in sulfite oxidase deficiency, accumulating sulfite damages the brain
\begin{itemize}
\item partly due to the production of sulfocysteine, which
mediates excitotoxicity
\end{itemize}
\item sulfite probably causes lens dislocation by disrupting cystine
cross-linkages in the suspensory ligament
\end{itemize}

\subsubsection{Genetics}
\label{sec:orgc076fcd}
\begin{itemize}
\item AR, SUOX
\end{itemize}

\subsubsection{Diagnostic tests}
\label{sec:org374b27d}
\begin{itemize}
\item sulfite can be detected in fresh urine using dipsticks
\begin{itemize}
\item not reliable
\end{itemize}
\item \(\uparrow\) urine or blood s-sulfocysteine
\item \(\uparrow\) plasma taurine
\item \(\downarrow\) plasma total cysteine and tHcy
\item normal urate and xanthine
\item diagnosis is confirmed by mutation analysis
\end{itemize}

\subsubsection{Treatment}
\label{sec:orgd0469c9}
\begin{itemize}
\item prognosis for neonatal-onset cases is poor
\item diet low in cysteine and methionine may help patients with a mild
form
\end{itemize}

\subsection{Ethylmalonic Encephalopathy}
\label{sec:org2a44255}
\subsubsection{Clinical presentation}
\label{sec:org5001739}
\begin{itemize}
\item progressive multisystem disease
\item presents in the first months of life with hypotonia, chronic
diarrhoea, orthostatic acrocyanosis, recurrent petechial rash and
bruising (with normal platelets)
\item developmental regression, microcephaly, seizures, episodes of coma,
poor growth and hyperlactataemia
\item most die in early childhood, though some have a milder course
\end{itemize}

\subsubsection{Metabolic derangement}
\label{sec:org107d63e}
\begin{itemize}
\item deficiency of a mitochondrial sulfur dioxygenase necessary for the
detoxification of sulfide
\item hydrogen sulfide (\ce{H2S}) is synthesized endogenously by CBS, CTH
and 3-mercaptosulfurtransferase
\begin{itemize}
\item also formed by bacterial anaerobes in the large intestine
\end{itemize}
\item in EE accumulating \ce{H2S} inhibits cytochrome c oxidase and
short-chain fatty acid oxidation
\begin{itemize}
\item results in ethylmalonic aciduria , and raised C4- and C5-acylcarnitines in blood
\end{itemize}
\item \ce{H2S} also has vasoactive and vasotoxic effects
\begin{itemize}
\item damage to small blood vessels causes bleeding into the
skin
\item production of \ce{H2S} by gut bacteria causes the severe, persistent diarrhea
\end{itemize}
\end{itemize}

\subsubsection{Genetics}
\label{sec:org2a57a65}
\begin{itemize}
\item AR, ETHE1, rare
\end{itemize}

\subsubsection{Diagnostic tests}
\label{sec:org13c7647}
\begin{itemize}
\item \(\uparrow\) urine ethylmalonic acid
\item \(\uparrow\) urine C4- and C5-acylglycines
\item \(\uparrow\) plasma C4- and C5-acylcarnitines
\item \(\Uparrow\) urine thiosulfate is also markedly elevated
\item diagnosis is confirmed by mutation analysis
\end{itemize}

\subsubsection{Treatment}
\label{sec:orgf719853}
\begin{itemize}
\item metronidazole to reduce bacterial \ce{H2S} production
\item N-acetylcysteine a precursor of glutathione, which can accept the sulfur atom of \ce{H2S}
\begin{itemize}
\item leads to some clinical and biochemical improvement the prognosis remains poor
\end{itemize}
\item Liver transplant
\end{itemize}
\end{document}