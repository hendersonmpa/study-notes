% Created 2022-01-06 Thu 13:28
% Intended LaTeX compiler: pdflatex
\documentclass[12pt]{scrartcl}
\usepackage[utf8]{inputenc}
\usepackage[T1]{fontenc}
\usepackage{graphicx}
\usepackage{longtable}
\usepackage{wrapfig}
\usepackage{rotating}
\usepackage[normalem]{ulem}
\usepackage{amsmath}
\usepackage{amssymb}
\usepackage{capt-of}
\usepackage{hyperref}
\hypersetup{colorlinks,linkcolor=black,urlcolor=blue}
\usepackage{textpos}
\usepackage{textgreek}
\usepackage[version=4]{mhchem}
\usepackage{chemfig}
\usepackage{siunitx}
\usepackage{gensymb}
\usepackage[usenames,dvipsnames]{xcolor}
\usepackage{lmodern}
\usepackage{verbatim}
\usepackage{tikz}
\usepackage{wasysym}
\usetikzlibrary{shapes.geometric,arrows,decorations.pathmorphing,backgrounds,positioning,fit,petri}
\usepackage[automark, autooneside=false, headsepline]{scrlayer-scrpage}
\clearpairofpagestyles
\ihead{\leftmark}% section on the inner (oneside: right) side
\ohead{\rightmark}% subsection on the outer (oneside: left) side
\addtokomafont{pagehead}{\upshape}% header upshape instead of italic
\ofoot*{\pagemark}% the pagenumber in the center of the foot, also on plain pages
\pagestyle{scrheadings}
\author{Matthew Henderson, PhD, FCACB}
\date{\today}
\title{Amino Acids}
\hypersetup{
 pdfauthor={Matthew Henderson, PhD, FCACB},
 pdftitle={Amino Acids},
 pdfkeywords={},
 pdfsubject={},
 pdfcreator={Emacs 27.2 (Org mode 9.5)}, 
 pdflang={English}}
\begin{document}

\maketitle
\setcounter{tocdepth}{2}
\tableofcontents


\setchemfig{atom style={scale=0.75}}

\section{Phenylalanine}
\label{sec:org4d702e9}
\subsection{Introduction}
\label{sec:orgb3d04e3}
\begin{itemize}
\item phenylalanine is an essential aromatic amino acid
\end{itemize}

\begin{center}
\chemnameinit{}
\chemname{\chemfig{*6(-=-(-[1]-[7](<[6]NH_2)-[1](=[2]O)-[7]OH)=-=-)}}{\small phenyalanine}
\end{center}

\begin{itemize}
\item mainly metabolised in the liver by the phenylalanine hydroxylase (PAH) system
\item first step in the irreversible catabolism of phenylalanine is hydroxylation to
tyrosine by PAH
\item PAH requires the active pterin, tetrahydrobiopterin (BH\textsubscript{4})
\begin{itemize}
\item BH\textsubscript{4} is formed in three steps from GTP
\item forms a Fe(II)-O-O-BH\textsubscript{4} bridge during the hydroxylation reaction
\item also an obligate co-factor for tyrosine hydroxylase, tryptophan
hydroxylase, alkylglycerol monooxygenase (AGMO) and nitric oxide
synthase
\begin{itemize}
\item \(\therefore\) required for production of dopamine, catecholamines,
melanin and serotonin
\end{itemize}
\end{itemize}

\item defects in PAH or production or recycling of BH\textsubscript{4} may result in:
\begin{itemize}
\item hyperphenylalaninaemia (HPA)
\item deficiency of tyrosine, L-dopa, dopamine, melanin, catecholamines
and 5-hydroxytryptophan (5HT)
\end{itemize}
\item when hydroxylation to tyrosine is impeded excess phenylalanine is converted to phenylketones
\begin{itemize}
\item phenylalanine can be transaminated to phenylpyruvate which is reduced to
phenyllactate or decarboxylated to phenylacetate
\end{itemize}
\end{itemize}

\begin{figure}[htbp]
\centering
\includegraphics[width=0.9\textwidth]{phe/figures/pah.png}
\caption{\label{fig:orgad61702}Phenylalanine Hydroxylation}
\end{figure}

\begin{figure}[htbp]
\centering
\includegraphics[width=0.9\textwidth]{phe/figures/Slide04.png}
\caption{\label{fig:org6f4a2b1}Phenylalanine and Tyrosine Metabolism}
\end{figure}

\begin{figure}[htbp]
\centering
\includegraphics[width=0.9\textwidth]{phe/figures/Slide21.png}
\caption{\label{fig:orga49ea8f}Tetrahydrobiopterin Metabolism}
\end{figure}


\begin{table}[htbp]
\caption{\label{tab:org3838e23}Phenylalaninemia Classification}
\centering
\begin{tabular}{lll}
Class & Phe (\textmu{}mol/l) & PAH activity\\
\hline
Classic & \(\ge\) 1200 & \textless{}1\%\\
HPA & 600-1200 & 1-5\%\\
MHP \footnotemark & 120-600 & \textgreater{}5\%\\
BH\textsubscript{4} responsive & \(\ge\) 30\% decrease & \\
\end{tabular}
\end{table}\footnotetext[1]{\label{orge850da8}mild hyperphenylalaninaemia}

\subsection{Phenylalanine Hydroxylase Deficiency}
\label{sec:orgae79507}
\subsubsection{Clinical Presentation}
\label{sec:org24068a9}
\begin{itemize}
\item natural history of PKU is for affected individuals to suffer
progressive, irreversible neurological impairment during infancy and
childhood
\item untreated patients develop mental, behavioural, neurological and
physical impairments
\item most common outcome is a moderate to profound intellectual
developmental disorder (IQ \(\le\) 50) often associated with:
\begin{itemize}
\item mousey odour \(\to\) excretion of phenylacetic acid
\item reduced hair, skin and iris pigmentation \(\to\) reduced melanin synthesis
\item reduced growth and microcephaly
\item neurological impairments:
\begin{itemize}
\item 25\% epilepsy, 30\% tremor, 5\% spasticity of the limbs, 80\% EEG abnormalities
\end{itemize}
\end{itemize}
\item clinical phenotype correlates with phenylalanine blood levels
\begin{itemize}
\item reflects the degree of PAH deficiency
\end{itemize}
\end{itemize}
\subsubsection{Metabolic Derangement}
\label{sec:org4c02cd2}
\begin{itemize}
\item \textbf{phenylalanine hydroxylase} deficiency
\item pathogenesis of brain damage in PKU is not fully understood
\begin{itemize}
\item causally related to the \(\uparrow\) blood phenylalanine
\end{itemize}
\item tyrosine becomes a semi-essential amino acid
\begin{itemize}
\item \(\downarrow\) blood levels \(\to\) impaired synthesis of biogenic amines including:
\begin{itemize}
\item melanin, dopamine and norepinephrine
\end{itemize}
\end{itemize}
\item \(\uparrow\) blood phenylalanine levels result in an imbalance of other large
neutral amino acids (LNAA) in the brain
\begin{itemize}
\item \(\to\) decreased brain concentrations of tyrosine and serotonin
\item ratio of phenylalanine levels in blood:brain is about 4:1
\end{itemize}
\end{itemize}

\subsubsection{Genetics}
\label{sec:org2851634}
\begin{itemize}
\item AR PAH
\item R408W mutation \(\sim\) 30\% of alleles in Europeans with PKU
\item genotypes correlate well with biochemical phenotypes, pre-treatment
phenylalanine levels and dietary phenylalanine tolerance
\end{itemize}

\subsubsection{Diagnostic Tests}
\label{sec:org8b29784}
\begin{itemize}
\item NBS for phenylalanine and phenyalanine/tyrosine
\item exclude cofactor defects via pterins in blood or urine and
dihydropteridine reductase (DHPR) in blood
\item HPA may be found in preterm and sick babies particularly
\begin{itemize}
\item after parenteral feeding with amino acids
\item liver disease (where blood levels of methionine, tyrosine,
leucine/isoleucine are usually also raised)
\item treatment with inhibitors of dihydrofolate reductase (e.g. methotrexate, trimethoprim)
\begin{itemize}
\item folate promotes recycling of BH\textsubscript{4}
\end{itemize}
\end{itemize}
\item plasma amino acids:
\begin{itemize}
\item \(\uparrow\) phenylalanine
\item \(\downarrow\) or normal tyrosine
\end{itemize}
\item phenylketones in urine organic acids:
\begin{itemize}
\item \(\uparrow\) phenylpyruvate
\item \(\uparrow\) phenyllactate
\item \(\uparrow\) phenylacetate
\end{itemize}
\end{itemize}
\subsubsection{Treatment}
\label{sec:org8590d88}
\begin{enumerate}
\item Diet
\label{sec:orge4b3070}
\begin{itemize}
\item reduce the blood phenylalanine concentration sufficiently to prevent the
neuropathological effects but also to fulfil age-dependent
requirements for protein synthesis
\item blood phenylalanine is primarily a function of residual PAH activity and phenylalanine
intake
\end{itemize}
\item BH\textsubscript{4}
\label{sec:org71a2951}
\begin{itemize}
\item pharmacological doses of BH\textsubscript{4} can reduce blood phenylalanine levels
in some patients with PKU
\item sapropterin dihydrochloride (Kuvan), a synthetic formulation of the
active 6R-isomer of BH\textsubscript{4} is approved in Europe and the USA for the
treatment of responsive\footnote{defined as reduction of \(\ge\)30\% in blood phenylalanine after a
single dose} patients with HPA and PKU, of all
ages
\end{itemize}
\item Alternative/Experimental
\label{sec:org9a1adb6}
\begin{itemize}
\item liver transplantation
\item phenylalanine ammonia lyase (PAL)
\begin{itemize}
\item converts phenylalanine \(\to\) harmless transcinnamic acid
\end{itemize}
\item large neutral amino acids
\begin{itemize}
\item tyr, trp, leu, ile, val compete with phe for the same LNAA transporter at
the blood brain barrier
\end{itemize}
\end{itemize}
\end{enumerate}

\subsection{Maternal PKU}
\label{sec:org2ac3732}
\subsubsection{Clinical Presentation}
\label{sec:org704c6a6}
\begin{itemize}
\item teratogenic effects of high maternal phenylalanine levels
\item offspring of women with untreated classical PKU suffer developmental
delay, microcephaly, cardiac defects, low birth weight and
dysmorphic features
\item pathogenesis is poorly understood
\end{itemize}
\subsubsection{Metabolic Derangement}
\label{sec:org52b935c}
\begin{itemize}
\item maternal phenylalanine \textless{} 360 umol/l \(\to\) no deleterious effect on the foetus
\item maternal phenylalanine \textgreater{} 360 μmol/l \(\to\) developmental indices
decreased by about three points for every 60 umol/l rise in average
maternal phenylalanine
\item \(\uparrow\) CHD \(\ge\) 900 umol/l
\end{itemize}
\subsubsection{Prevention}
\label{sec:org45117ef}
\begin{itemize}
\item plan pregnancy
\item start diet before conception
\item monitoring expecting mothers 2x weekly
\end{itemize}

\subsection{HPA and Disorders of Biopterin Metabolism}
\label{sec:org0736369}
\begin{itemize}
\item disorders of BH\textsubscript{4} metabolism associated with HPA and biogenic amine deficiency (Figure \ref{fig:orga49ea8f})
\begin{itemize}
\item GTP cyclohydrolase I (GTPCH)
\item 6-pyruvoyl-tetrahydropterin synthase (PTPS)
\item dihydropteridine reductase (DHPR)
\item pterin-4a-carbinolamine dehydratase (PCD)
\end{itemize}
\item dopa-responsive dystonia (DRD) due to a dominant form of GTPCH
deficiency, and sepiapterin reductase (SR) deficiency (see Disorders of Monoamine Metabolism)
\end{itemize}
\subsubsection{Clinical Presentation}
\label{sec:orgec75a14}
\begin{itemize}
\item can present in any of three ways:
\begin{enumerate}
\item \textbf{asymptomatic} with raised phenylalanine found following NBS; as part of
the standard screening protocol the infant is then investigated
further for biopterin defects
\item \textbf{symptomatic} with neurological deterioration in infancy despite a
low-phenylalanine diet
\begin{itemize}
\item this will occur where no further investigations are routinely
undertaken after a finding of HPA in NBS which is wrongly
assumed to be PAH deficiency
\end{itemize}
\item \textbf{symptomatic} with neurological deterioration in infancy on a
normal diet
\begin{itemize}
\item this will occur either where there has been no NBS for HPA or
if the phenylalanine level is not sufficiently raised to have
resulted in a positive screen or to require dietary treatment
\end{itemize}
\end{enumerate}
\end{itemize}
\subsubsection{Metabolic Derangement}
\label{sec:org6b89470}
\begin{itemize}
\item associated with decreased activity of PAH, tyrosine hydroxylase,
tryptophan hydroxylase and nitric oxide synthase (Figures \ref{fig:orgad61702}
and \ref{fig:orga49ea8f})
\item degree of HPA is highly variable
\begin{itemize}
\item blood phenylalanine concentrations ranging from normal to \textgreater{}2000
umol/l
\item CNS amine deficiency is most often profound and responsible for
the clinical symptoms
\item decreased concentration of HVA in CSF is a measure of reduced
dopamine turnover
\item 5-HIAA deficiency is a measure of reduced serotonin metabolism
\end{itemize}
\end{itemize}

\subsubsection{Genetics}
\label{sec:org0d02365}
\begin{itemize}
\item AR GTPCH, PTPS, DHPR, PCD
\item biopterin disorders account for 1-3\% of infants found to have a
raised phenylalanine on newborn screening
\end{itemize}

\subsubsection{Diagnostic Tests}
\label{sec:org079a4fc}
\begin{itemize}
\item urine or blood pterin analysis and blood DHPR assay (Table \ref{tab:org5fd6463})
\item BH\textsubscript{4} loading test
\item CSF neurotransmitters
\begin{itemize}
\item L-Dopa
\item 5-OH-tryptophan
\end{itemize}
\end{itemize}

\begin{table}[htbp]
\caption{\label{tab:org5fd6463}Results in Biopterin Disorders}
\centering
\begin{tabular}{lrlllll}
Deficiency & Phe & biopterin\footnotemark & neopterin\textsuperscript{\ref{org51817c0}} & primapterin\textsuperscript{\ref{org51817c0}} & CSF 5-HIAA HVA & DHPR activity\\
\hline
PAH & \textgreater{}120 & \(\uparrow\) & \(\uparrow\) & - & N & N\\
GTPCH & 50-1200 & \(\Downarrow\) & \(\Downarrow\) & - & \(\downarrow\) & N\\
PTPS & 240-2500 & \(\Downarrow\) & \(\Uparrow\) & - & \(\downarrow\) & N\\
DHPR & 180-2500 & \(\Downarrow\) & N or \(\uparrow\) & - & \(\downarrow\) & \(\downarrow\)\\
PCD & 180-1200 & \(\downarrow\) & \(\uparrow\) & \(\Uparrow\) &  & N\\
\end{tabular}
\end{table}\footnotetext[3]{\label{org51817c0}blood or urine}

\subsubsection{Treatment}
\label{sec:org7620ec5}
\begin{itemize}
\item BH\textsubscript{4}
\item CNS amine replacement
\begin{itemize}
\item L-dopa, 5-OH-tryptophan
\end{itemize}
\end{itemize}

\section{Tyrosine}
\label{sec:orgf1b80f7}
\subsection{Introduction}
\label{sec:org68b2113}
\begin{itemize}
\item tyrosine is a non-essential amino acid derived from diet and hydroxylation of phenylalanine
\item a precursor of DOPA, thyroxine and melanin
\end{itemize}

\begin{center}
\chemnameinit{}
\chemname{\chemfig{HO-[0]*6(-=-(-[1]-[7](<[6]NH_2)-[1](=[2]O)-[7]OH)=-=-)}}{\small tyrosine}
\end{center}

\begin{itemize}
\item glucogenic and ketogenic \(\to\) catabolism in liver cytosol \(\to\) fumarate and acetoacetate
\item tyrosine \(\to\) 4-hydroxyphenylpyruvate by cytosolic tyrosine aminotransferase
\begin{itemize}
\item also in liver and other tissues by mitochondrial aspartate aminotransferase
\begin{itemize}
\item minor role under normal conditions
\end{itemize}
\end{itemize}
\item penultimate intermediates maleylacetoacetate and fumarylacetoacetate
are reduced to succinylacetoacetate
\begin{itemize}
\item decarboxylation to succinylacetone
\item succinylacetone is the most potent known inhibitor of the heme biosynthetic enzyme ALAD
\end{itemize}
\end{itemize}

\begin{figure}[htbp]
\centering
\includegraphics[width=0.9\textwidth]{tyr/figures/tyr.png}
\caption{\label{fig:org1321c74}Tyrosine Catabolism:1 Tyrosine aminotransferase; 2 4-hydroxyphenylpyruvate dioxygenase; 3 homogentisate dioxygenase; 4 fumarylacetoacetase; 5 AST; 6 ALAD}
\end{figure}


\begin{figure}[htbp]
\centering
\includegraphics[width=1\textwidth]{tyr/figures/Slide04.png}
\caption{\label{fig:org3096e39}Phenylalanine and Tyrosine Metabolism}
\end{figure}

\begin{itemize}
\item five inherited disorders of tyrosine metabolism are known:
\begin{enumerate}
\item Tyrosinemia Type I is characterised by progressive
liver disease and renal tubular dysfunction with rickets
\item Tyrosinemia Type II presents with keratitis and
blistering lesions of the palms and soles and neurological
complications
\item Tyrosinemia Type III may be asymptomatic or associated with
mental retardation
\item Hawkinsinuria may be asymptomatic or present with failure to
thrive and metabolic acidosis in infancy
\item Alkaptonuria presents as adult with symptoms of osteoarthritis
\end{enumerate}
\end{itemize}

\subsubsection{Transient Tyrosinemia}
\label{sec:org2822150}
\begin{itemize}
\item one of the most common amino acid disorders
\begin{itemize}
\item clinically asymptomatic
\end{itemize}
\item believed to be caused by late fetal maturation of
4-hydroxyphenylpyruvate dioxygenase
\item more common in premature infants than in full-term newborns
\item protein intake is an important aetiological factor
\begin{itemize}
\item incidence has fallen dramatically in the last 4 decades, with the
reduction in the protein content of newborn formula
\end{itemize}
\end{itemize}

\subsection{Tyrosinemia Type I}
\label{sec:org1c5970f}
\begin{itemize}
\item Hepatorenal Tyrosinemia
\end{itemize}
\subsubsection{Clinical Presentation}
\label{sec:orgeaf80ab}
\begin{itemize}
\item very variable, presents at any time from the neonatal period to adulthood
\begin{description}
\item[{acute}] before 6 months of age with acute liver failure
\item[{subacute}] between 6 months and 1 year of age with liver disease,
failure to thrive, coagulopathy, hepatosplenomegaly,
rickets and hypotonia
\item[{chronic}] after the 1st year with chronic liver disease, renal
disease, rickets, cardiomyopathy \textpm{} porphyria-like
syndrome
\end{description}

\item liver is the major affected organ
\begin{itemize}
\item morbidity and mortality
\end{itemize}
\item renal disease is detected in most patients
\begin{itemize}
\item proximal tubular disease
\end{itemize}
\item acute neurological crisis can occur
\begin{itemize}
\item painful paresthesias and autonomic signs that may progress to
paralysis
\item ALAD inhibition by succinylacetone \(\to\) AIP symptoms
\end{itemize}
\end{itemize}

\subsubsection{Metabolic Derangement}
\label{sec:org71b7557}
\begin{itemize}
\item deficiency of the enzyme \textbf{fumarylacetoacetate hydrolase (FAH)}

\ce{fumarylacetoacetate ->[FAH] fumarate + acetoacetate}

\item mainly expressed in the liver and kidney
\item compounds immediately upstream from the FAH reaction,
maleylacetoacetate (MAA) and fumarylacetoacetate (FAA), and their
derivatives, succinylacetone (SA) and succinylacetoacetate (SAA)
accumulate and have important pathogenic effects
\item effects of FAA and MAA occur only in the cells of the organs in which they are produced
\begin{itemize}
\item these compounds are not found in body fluids of patients
\end{itemize}
\item SA and SAA are readily detectable in plasma and urine and have
widespread effects
\item FAA, MAA and SA disrupt sulfhydryl metabolism by forming glutathione
adducts \(\to\) free radical damage
\item disruption of sulfhydryl metabolism is also believed to cause
secondary deficiency of two other hepatic enzymes,
4-hydroxyphenylpyruvate dioxygenase and methionine
adenosyltransferase, resulting in hypertyrosinemia and
hypermethioninemia
\item SA is a potent inhibitor of ALAD
\end{itemize}

\subsubsection{Genetics}
\label{sec:orga9d9cbc}
\begin{itemize}
\item AR FAH
\item most common mutation is c.1062+5G>A
\begin{itemize}
\item is found in about 25\% of the alleles worldwide
\item predominant mutation in the French-Canadian population, in which
it accounts for >90\% of alleles
\end{itemize}
\end{itemize}

\subsubsection{Diagnostic Tests}
\label{sec:orgd5c542e}
\begin{itemize}
\item \(\uparrow\) SA in urine, plasma or DBS is pathognomonic
\item \(\uparrow\) tyrosine
\item \(\uparrow\) phenylalanine
\item \(\uparrow\) methionine
\item \(\uparrow\) urine ALA
\item symptomatic patients, biochemical tests of liver function are
usually abnormal
\begin{itemize}
\item coagulopathy and/or hypoalbuminaemia
\end{itemize}
\item acutely ill patients
\begin{itemize}
\item \(\Uparrow\) \(\alpha\)-fetoprotein
\item Fanconi-type tubulopathy is often present with:
\begin{itemize}
\item aminoaciduria, phosphaturia and glycosuria
\item radiological evidence of rickets may be present
\end{itemize}
\end{itemize}
\end{itemize}

\subsubsection{Treatment}
\label{sec:orgeae0543}
\begin{itemize}
\item nitisinone (aka: NTBC) is the recommended therapy, in combination
with a tyrosine and phenylalanine restricted diet
\begin{itemize}
\item inhibits 4-hydroxyphenylpyruvate dioxygenase turning Type I into Type III
\end{itemize}
\item nitisinone block tyrosine degradation at an early step
\begin{itemize}
\item \(\downarrow\) FAA, MAA and SA
\item \(\uparrow\) tyrosine and 4-hydroxyphenylpyruvate
\end{itemize}
\item liver transplantation \(\to\) functional cure
\begin{itemize}
\item normal diet
\item mortality and life long immunosuppressive therapy
\end{itemize}
\end{itemize}

\subsection{Tyrosinemia Type II}
\label{sec:orgc6d786f}
\begin{itemize}
\item Oculocutaneous Tyrosinemia
\end{itemize}
\subsubsection{Clinical Presentation}
\label{sec:orge243e0e}
\begin{itemize}
\item any combination of: 
\begin{itemize}
\item ocular lesions
\item skin lesions
\item neurological complications
\end{itemize}
\item usually presents in infancy but can be any age
\end{itemize}

\subsubsection{Metabolic Derangement}
\label{sec:org82f68b0}
\begin{itemize}
\item \textbf{hepatic cytosolic tyrosine aminotransferase}

\ce{tyrosine ->[TAT] 4-hydroxyphenylpyruvate}

\begin{itemize}
\item \(\uparrow\) tyrosine in CSF and serum
\end{itemize}
\item \(\uparrow\) phenolic acids 4-hydroxyphenylpyruvate,
4-hydroxyphenyllactate and 4-hydroxyphenylacetate via AST (Figure \ref{fig:org1321c74})
\end{itemize}

\subsubsection{Genetics}
\label{sec:org2e22006}
\begin{itemize}
\item AR TAT
\end{itemize}

\subsubsection{Diagnostic Tests}
\label{sec:org8fa70ca}
\begin{itemize}
\item \(\Uparrow\) plasma tyrosine > 1200 umol/L
\begin{itemize}
\item if lower consider Type III
\end{itemize}
\item urine organic acids
\begin{itemize}
\item \(\Uparrow\) 4-hydroxyphenylpyruvate
\item \(\Uparrow\) 4-hydroxyphenyllactate
\item \(\Uparrow\) 4-hydroxyphenylacetate
\item \(\uparrow\) N-acetyltyrosine
\item \(\uparrow\) 4-tyramine
\end{itemize}
\end{itemize}

\subsubsection{Treatment}
\label{sec:orge9d0f49}
\begin{itemize}
\item tyrosine and phenylalanine restricted diet
\end{itemize}

\subsection{Tyrosinemia Type III}
\label{sec:org9d23ca7}
\subsubsection{Clinical Presentation}
\label{sec:org849172c}
\begin{itemize}
\item very rare, 13 cases described
\item most common long-term complication is intellectual impairment
\end{itemize}
\subsubsection{Metabolic Derangement}
\label{sec:org71edc6f}
\begin{itemize}
\item \textbf{4-hydroxyphenylpyruvate dioxygenase} deficiency
\end{itemize}

\ce{4-hydroxyphenylpyruvate ->[HPD] homogentisate}

\begin{itemize}
\item \(\uparrow\) plasma tyrosine
\item \(\uparrow\) urine 4-hydroxyphenylpyruvate, 4-hydroxyphenyllactate and 4-hydroxyphenylacetate
\end{itemize}
\subsubsection{Genetics}
\label{sec:org32f6f50}
\begin{itemize}
\item AR HPD
\end{itemize}
\subsubsection{Diagnostic Tests}
\label{sec:org87e44cb}
\begin{itemize}
\item \(\uparrow\) plasma tyrosine 300-1300 umol/L
\item urine organic acids
\begin{itemize}
\item \(\uparrow\) 4-hydroxyphenylpyruvate
\item \(\uparrow\) 4-hydroxyphenyllactate
\item \(\uparrow\) 4-hydroxyphenylacetate
\end{itemize}
\end{itemize}

\subsection{Alkaptonuria}
\label{sec:org91ea952}
\subsubsection{Clinical Presentation}
\label{sec:org3db8ed8}
\begin{itemize}
\item clinical symptoms first appear in adulthood
\begin{itemize}
\item some cases diagnosed in infancy due to darkening of urine when
exposed to air
\end{itemize}
\item most prominent symptoms relate to joint and connective tissue involvement
\item significant cardiac disease and urolithiasis may be detected in later years
\end{itemize}
\subsubsection{Metabolic Derangement}
\label{sec:org4f8df13}
\begin{itemize}
\item \textbf{homogentisate dioxygenase} deficiency
\end{itemize}

\ce{homogentisate ->[HGD] maleylacetoacetate}
\begin{itemize}
\item expressed mainly in the liver and the kidneys
\item accumulation of homogentisate and its oxidised derivative
benzoquinone acetic acid (the toxic metabolite) in various tissues
\item first identified IEM in 1902 by Garrod
\end{itemize}

\subsubsection{Genetics}
\label{sec:orgdd191b4}
\begin{itemize}
\item AR HGD
\item 1:250000-1:1000000
\end{itemize}
\subsubsection{Diagnostic Tests}
\label{sec:org4324ae0}
\begin{itemize}
\item alkalinisation of the urine \(\to\) immediate dark brown colour
\item \(\uparrow\) urine homogentisate \(\to\) positive test for reducing substances
\item \(\uparrow\) UOA homogentisic acid
\end{itemize}
\subsubsection{Treatment}
\label{sec:orgc58eff1}
\begin{itemize}
\item vitamin C
\item nitisinone with \(\downarrow\) phenylalanine and tyrosine diet
\begin{itemize}
\item 3-year clinical trial of nitisinone \(\to\) 95\% \(\downarrow\) urine and plasma homogentisic acid
\item no demonstrable effects on clinical symptoms
\end{itemize}
\end{itemize}

\subsection{Hawkinsinuria}
\label{sec:org02543dc}
\subsubsection{Clinical Presentation}
\label{sec:org0346618}
\begin{itemize}
\item only been described in a few families
\item FTT and metabolic acidosis in infancy
\item early weaning from breastfeeding seems to precipitate the disease
\begin{itemize}
\item may be asymptomatic in breastfed infants
\end{itemize}
\end{itemize}

\subsubsection{Metabolic Derangement}
\label{sec:org6d46f53}
\begin{itemize}
\item abnormal metabolites produced in hawkinsinuria
\begin{itemize}
\item hawkinsin (2-cysteinyl-1,4-dihydroxycyclohexenylacetate)
\item 4-hydroxycycloxylacetate
\end{itemize}
\item thought to derive from in-complete conversion of
4-hydroxyphenylpyruvate to homogentisate caused by a defect in
\textbf{4-hydroxyphenylpyruvate dioxygenase}
\end{itemize}

\ce{4-hydroxyphenylpyruvate ->[HPD] homogentisate}

\begin{itemize}
\item same enzyme deficiency in Tyrosinemia Type III
\item hawkinsin is product of a reaction of an epoxide intermediate with
glutathione, which may be depleted
\item metabolic acidosis due to 5-oxoproline accumulation secondary to
glutathione depletion
\end{itemize}

\subsubsection{Genetics}
\label{sec:org72399e4}
\begin{itemize}
\item AD HPD A33T
\item mutations that lead to a retention of partial HPD function
\begin{itemize}
\item production of hawkinsin and 4-hydroxycyclohexylacetate
\end{itemize}
\end{itemize}
\subsubsection{Diagnostic Tests}
\label{sec:orgc17c134}
\begin{itemize}
\item may be moderate tyrosinaemia
\item urine organic acids
\begin{itemize}
\item \(\uparrow\) hawkinsin (4-hydroxycyclohexylacetate) is diagnostic
\item during infancy
\begin{itemize}
\item \(\uparrow\) 4-hydroxyphenylpyruvate
\item \(\uparrow\) 4-hydroxyphenyllactate
\item \(\uparrow\) 5-oxoprolinuria
\item metabolic acidosis
\end{itemize}
\end{itemize}
\end{itemize}

\subsubsection{Treatment}
\label{sec:org79c68b4}
\begin{itemize}
\item return to breastfeeding or low tyrosine and phenylalanine diet
\item asymptomatic after the 1st year of life
\item affected infants are reported to have developed normally
\end{itemize}

\section{BCAA}
\label{sec:orgd36c8e3}
\subsection{Introduction}
\label{sec:org14a1068}
\begin{itemize}
\item the three essential BCAAs: leucine, isoleucine and valine are
initially catabolised by a common pathway
\end{itemize}

\begin{center}
\chemnameinit{}
\chemname{\chemfig{-[1](-[2])-[7](<[6]NH_2)-[1](=[2]O)-[7]OH}}{\small valine}
\hspace{20}
\chemnameinit{}
\chemname{\chemfig{-[7](-[6])-[1]-[7](<[6]NH_2)-[1](=[2]O)-[7]OH}}{\small leucine}
\hspace{20}
\chemnameinit{}
\chemname{\chemfig{-[7]-[1](-[2])-[7](<[6]NH_2)-[1](=[2]O)-[7]OH}}{\small isoleucine}
\end{center}

\begin{itemize}
\item the first reaction, occurs primarily in muscle
\begin{itemize}
\item reversible transamination to 2-oxo-acids \footnote{\(\alpha\)-keto is equivalent to 2-oxo, oxo is preferred}
\item followed by oxidative decarboxylation to coenzyme A (CoA)
derivatives by branched-chain keto-acid dehydrogenase
(BCKDH)
\end{itemize}
\item BCAA degradation pathways then diverge
\item leucine \(\to\) acetoacetate and acetyl-CoA \(\to\) TCA cycle
\item isoleucine \(\to\) acetyl-CoA and propionyl-CoA
\begin{itemize}
\item propionyl-CoA is converted to succinyl-CoA \(\to\) TCA cycle
\end{itemize}
\item valine \(\to\) propionyl-CoA
\item methionine, FFA w odd number of carbons, cholesterol side chains,
gut bacteria \(\to\) propionyl-CoA
\end{itemize}

\begin{figure}[htbp]
\centering
\includegraphics[width=1.1\textwidth]{bcaa/figures/Slide02.png}
\caption{\label{fig:org3515164}BCAA Catabolism}
\end{figure}

\begin{itemize}
\item branched-chain organic acidurias result from an abnormality of
specific enzymes involving the catabolism of BCAAs
\item most common are MSUD, IVA, PA, and MMA
\end{itemize}

\subsection{MSUD, IVA, PA and MMA}
\label{sec:org08d1561}
\subsubsection{Clinical Presentation}
\label{sec:org7bbbeb2}
\begin{itemize}
\item have many clinical and biochemical symptoms in common
\item there are three main clinical presentations:
\begin{enumerate}
\item severe neonatal-onset form of metabolic distress
\item acute and intermittent late-onset form
\item chronic progressive form presenting as hypotonia, failure to
thrive, and developmental delay
\end{enumerate}
\item NBS has identified asymptomatic forms

\item complications include:
\begin{itemize}
\item renal tubular acidosis
\item skin lesions
\item pancreatitis
\item cardiomyopathy
\end{itemize}
\end{itemize}
\begin{enumerate}
\item Severe Neonatal Onset Form
\label{sec:orgeef47f4}
\begin{itemize}
\item toxic encephalopathy with either ketosis or ketoacidosis
\begin{itemize}
\item hours to weeks after birth
\end{itemize}
\item first signs are poor feeding and drowsiness, followed by unexplained
progressive coma
\item at more advanced stage, neurovegetative dysregulation with
respiratory distress, hiccups, apnoeas, bradycardia, and hypothermia
may appear
\end{itemize}

\begin{enumerate}
\item MSUD
\label{sec:org683d5c9}
\begin{itemize}
\item maple syrup or burnt sugar odour
\item no pronounced abnormalities on routine chemistry:
\begin{itemize}
\item no hyperammonaemia
\item not dehydrated
\item normal lactate
\end{itemize}
\end{itemize}
\item IVA, PA, MMA
\label{sec:org9c00dc9}
\begin{itemize}
\item dehydration
\item anion gap metabolic acidosis
\item ketonuria
\item hyperammonaemia in PA and MMA
\end{itemize}
\end{enumerate}

\item Acute Intermittent Form
\label{sec:orgdc8ceff}
\begin{itemize}
\item 25\% present after a symptom-free period
\begin{itemize}
\item commonly longer than 1 year and sometimes until adolescence or adulthood
\end{itemize}
\item neurological presentation: coma, lethargy
\item hepatic: Reye-like syndrome in IVA, PA, MMA
\item hematology and immunology: neutropenia, thrombocytopenia
\end{itemize}

\item Chronic Progressive Form
\label{sec:org360e11d}
\begin{itemize}
\item GI presentation
\item neurological
\end{itemize}
\end{enumerate}

\subsubsection{Metabolic Derangement}
\label{sec:org52b8504}
\begin{enumerate}
\item MSUD
\label{sec:orgc8c67af}
\begin{itemize}
\item \textbf{branched-chain \(\alpha\)-ketoacid dehydrogenase (BCKDH)} deficiency
\begin{itemize}
\item E1 or E2 \(\to\) MSUD
\item E3 \(\to\) dihydrolipamide dehydrogenase deficiency
\end{itemize}
\end{itemize}

\ce{2-oxo-isovalerate ->[BCKD] isobutyryl-CoA}

\ce{2-oxo-3-methylvalerate ->[BCKD] 2-methylbutyryl-CoA}

\ce{2-oxo-3-isocaproate ->[BCKD] isovaleryl-CoA}

\begin{itemize}
\item marked increases in the branched-chain 2-ketoacids in plasma, urine
and CSF
\item due to the reversibility of the initial transamination step BCAAs
also accumulate
\item smaller amounts of the respective 2-hydroxy acids are formed
\item alloisoleucine a diastereomer of isoleucine is found in the blood of
all patients with classic MSUD and in those with variant forms, at
least in those still without dietary treatment
\item leucine and 2-ketoisocaproic acid appear to be the most neurotoxic
\end{itemize}

\item IVA
\label{sec:orgee6b38f}
\begin{itemize}
\item deficiency of \textbf{isovaleryl-CoA dehydrogenase (IVD)}
\end{itemize}

\ce{isovaleryl-CoA ->[IVD] 3-methylcrotonyl-CoA}

\begin{itemize}
\item intramitochondrial flavoenzyme which transfers electrons to ETF
\end{itemize}
\begin{itemize}
\item accumulation of derivatives of isovaleryl-CoA including:
\begin{itemize}
\item free isovaleric increased in both plasma and urine
\item 3-hydroxyisovaleric acid
\item N-isovalerylglycine
\begin{itemize}
\item major derivative of isovaleryl-CoA
\end{itemize}
\item isovaleryl-carnitine (C5)
\end{itemize}
\end{itemize}

\item PA
\label{sec:org54d3e88}
\begin{itemize}
\item deficiency of the mitochondrial enzyme \textbf{propionyl-CoA carboxylase (PCC)}
\begin{itemize}
\item one of the five biotin-dependent enzymes
\end{itemize}
\end{itemize}
\ce{propionyl-CoA ->[PCC] methymalonyl-CoA}
\begin{itemize}
\item \(\uparrow\) free propionic acid in blood and urine
\item \(\uparrow\) derivatives propionyl-carnitine (C3),
3-hydroxypropionate and methylcitrate
\begin{itemize}
\item MCA arises by condensation of propionyl-CoA with oxaloacetate
\item catalysed by citrate synthase
\end{itemize}
\item \textbf{during ketotic episodes, 3-hydroxyisovaleric acid is formed by}
\textbf{condensation of propionyl-CoA with acetyl-CoA, followed by chemical}
\textbf{reduction}
\item \(\uparrow\) organic acids derived from a variety of intermediates of
the isoleucine catabolic pathway, such as:
\begin{itemize}
\item tiglic acid, tiglylglycine, 2-methyl-3-hydroxybutyrate,
3-hydroxybutyrate and propionylglycine can also be found
\end{itemize}
\end{itemize}
\item MMA
\label{sec:orgcda864e}
\begin{itemize}
\item deficiency of \textbf{methylmalonyl-CoA mutase (MCM)}
\begin{itemize}
\item adenosylcobalamin (B\textsubscript{12}) dependent-enzyme
\item disorders that affect AboCbl formation cause variant
forms of MMA
\end{itemize}
\end{itemize}

\ce{methylmalonyl-CoA ->[MCM + AdoCbl] succinyl-CoA}

\begin{itemize}
\item \(\uparrow\) methylmalonyl-CoA results in \(\uparrow\) methylmalonic acid
in urine and blood
\item secondary inhibition of PCC \(\to\) \(\uparrow\) PA and PA derivatives
\end{itemize}

\item Secondary Effects of Elevated Propionyl-CoA
\label{sec:org0996e35}
\begin{itemize}
\item \(\uparrow\) propionylcarnitine \(\to\) carnitine deficiency
\item \(\uparrow\) synthesis of odd numbered LCFAs
\item enzyme inhibition by PA leads to:
\begin{itemize}
\item \(\downarrow\) glucose - \(\uparrow\) insulin
\item \(\uparrow\) lactate and alanine - pyruvate dehydrogenase
\item \(\uparrow\) ammonia - N-acetylglutamate synthase
\item \(\uparrow\) glycine - glycine cleavage system
\begin{itemize}
\item PA was once called ketotic hyperglycinemia
\end{itemize}
\end{itemize}
\end{itemize}
\end{enumerate}

\subsubsection{Genetics}
\label{sec:orgd6ddde0}
\begin{description}
\item[{MSUD}] AR E1\(\alpha\), E1\(\beta\) and E2
\item[{IVA}] AR IVD
\item[{PA}] AR PCCB \& PCCB
\item[{MMA}] AR MUT or cobalamin system (Table \ref{tab:orgbcd2e4c})
\begin{itemize}
\item mut\textsuperscript{-}(\(\downarrow\) activity), mut\textsuperscript{0} (zero activity)
\end{itemize}
\end{description}

\begin{table}[htbp]
\caption{\label{tab:orgbcd2e4c}Isolated Methylmalonic Acidemia Genes}
\centering
\begin{tabular}{ll}
Gene & Protein\\
\hline
MMUT & methylmalonyl-CoA mutase\\
MCEE & methylmalonyl-CoA epimerase\\
MMAA & CblA\\
MMAB & CblB\\
MMADHC\footnotemark & CblC\\
\end{tabular}
\end{table}\footnotetext[5]{\label{org02486ba}deficiency of CblC (MMACHC) causes both MMA and homocysteinemia so not "isolated"}

\subsubsection{Diagnostic Tests}
\label{sec:orgc32bd9c}
\begin{enumerate}
\item MSUD
\label{sec:org499ec28}
\begin{itemize}
\item plasma amino acids:
\begin{itemize}
\item \(\uparrow\) BCAA
\item \(\uparrow\) leucine/alanine
\item \(\uparrow\) alloisoleucine
\end{itemize}
\item urine organic acids:
\begin{itemize}
\item \(\uparrow\) 2-hydroxyisovaleric
\item \(\uparrow\) 2-oxoisocaproic
\end{itemize}
\item \textbf{NB} FIA-MS used in NBS
\begin{itemize}
\item leucine, isoleucine and hydroxyproline are isobaric
\end{itemize}
\end{itemize}

\item IVA
\label{sec:org5d21964}
\begin{itemize}
\item urine organic acids:
\begin{itemize}
\item \(\Uparrow\) 3-hydroxyisovaleric
\item \(\Uparrow\) isovalerylglycine
\end{itemize}
\item \(\uparrow\) plasma isovaleryl-carnitine (C5)
\end{itemize}

\item PA
\label{sec:org44912ac}
\begin{itemize}
\item urine organic acids:
\begin{itemize}
\item \(\Uparrow\) 3-hydroxypropionic
\item \(\uparrow\) methylcitric
\end{itemize}
\item \(\uparrow\) plasma  propionyl-carnitine (C3)
\item \(\uparrow\) plasma glycine and alanine
\end{itemize}

\item MMA
\label{sec:org76e66cb}
\begin{itemize}
\item urine organic acids:
\begin{itemize}
\item \(\uparrow\) methylmalonic
\item \(\uparrow\) 3-hydroxypropionic
\item \(\uparrow\) methylcitric
\end{itemize}
\item \(\uparrow\) plasma propionyl-carnitine (C3) \textpm{} methylmalonyl-carnitine (C4DC)
\item \(\uparrow\) plasma glycine and alanine
\end{itemize}
\end{enumerate}

\subsubsection{Treatment}
\label{sec:org9015b4b}
\begin{itemize}
\item acute treatment of hyperammonemia
\begin{itemize}
\item carbaglu an NAG analog
\end{itemize}
\item MSUD low BCAA diet
\item IVA low protein diet
\begin{itemize}
\item carnitine and glycine \(\to\) acylcarnitine \& acylglycine
\end{itemize}
\item PA \& MMA low protein diet
\begin{itemize}
\item carnitine supplementation
\item MMA test for B\textsubscript{12} response
\end{itemize}
\end{itemize}

\subsection{3-Methylcrotonyl Glycinuria}
\label{sec:org75d700f}
\subsubsection{Clinical Presentation}
\label{sec:org3dc4419}
\begin{itemize}
\item highly variable: neonatal neurological onset with death \(\to\) lack of symptoms
\end{itemize}
\subsubsection{Metabolic Derangement}
\label{sec:orgbc4d098}
\begin{itemize}
\item \textbf{3-methylcrotonyl-CoA carboxylase (MCC)} deficiency
\begin{itemize}
\item involved in leucine catabolism
\item biotin dependant
\end{itemize}
\end{itemize}

\ce{3-methycrotonyl-CoA ->[MCC] 3-methylglutaconyl-CoA}

\begin{itemize}
\item MCC is a heteromeric enzyme consisting of
\(\alpha\)-(biotin-containing) and \(\beta\)-subunits
\item \(\uparrow\) 3-methylcrotonyl-CoA \(\to\) 3-methylcrotonylglycine and
3-methylcrotonic acid
\item 3-hydroxyisovalerate another major metabolite, is derived
through the action of a crotonase on 3-methylcrotonyl-CoA and the
subsequent hydrolysis of the CoA-ester
\end{itemize}
\subsubsection{Genetics}
\label{sec:orga635f85}
\begin{itemize}
\item AR MCCA and MCCB
\end{itemize}
\subsubsection{Diagnostic Tests}
\label{sec:org7d51ac4}
\begin{itemize}
\item \(\Uparrow\) 3-hydroxyisovaleric
\item \(\Uparrow\) 3-methycrotonylglycine
\item \(\Uparrow\) 3-methylcrotonic
\item \(\uparrow\) 3-hydroxyisovaleryl-carnitine (C5OH)
\item without the lactate, methylcitrate, and tiglylglycine found in
multiple carboxlase deficiency
\end{itemize}

\subsubsection{Treatment}
\label{sec:orge2b7c4b}
\begin{itemize}
\item glycine and carnitine supplementation
\end{itemize}

\subsection{3-Methylglutaconic Aciduria}
\label{sec:org88d93f7}
\begin{itemize}
\item defective leucine catabolism
\item primary 3-methylglutaconic aciduria caused by \textbf{3-methylglutaconyl-CoA}
\textbf{hydratase} deficiency (AUH mutations) is very rare
\begin{itemize}
\item \(\uparrow\) urine 3-methylglutaconic and 3-methylglutaric acids
\item \textbf{\(\uparrow\) urine 3-hydroxyisovaleric differentiates from secondary causes}
\end{itemize}
\end{itemize}

\ce{3-methylglutaconyl-CoA ->[MGCH] 3-hydroxy-3-methylglutaryl-CoA}

\begin{itemize}
\item secondary 3-methylglutaconic acidurias are a relatively common finding in a
number of metabolic disorders, particularly mitochondrial disease
\begin{itemize}
\item phospholipid remodelling
\item mitochondrial membrane
\item unknown etiology
\end{itemize}
\end{itemize}

\section{Urea Cycle}
\label{sec:org5e9d3f3}
\subsection{Introduction}
\label{sec:org56f6346}
\begin{itemize}
\item urea cycle is the main route for ammonia (\ce{NH3}) detoxification
\end{itemize}

\begin{center}
\chemnameinit{}
\chemname{\chemfig{H_2N-[::30,,2,](=[::60]O)-PO_4^{-}}}{\small carbamoyl-P}
\hspace{20}
\chemnameinit{}
\chemname{\chemfig{H_2N-[::30,,2,]-[::-60]-[::60]-[::-60](<[::-60]NH_2)-[::60](=[::60]O)-[::-60]OH}}{\small ornithine}
\end{center}

\begin{center}
\chemnameinit{}
\chemname{\chemfig{O=[::-30](-[::-60]OH)-[::60]-[::-60](<[::-60]NH_2)-[::60](=[::60]O)-[::-60]OH}}{\small aspartate}
\end{center}

\begin{center}
\chemnameinit{}
\chemname{\chemfig{H_2N-[::30,,2,](=[::60]O)-[::-60]\chembelow{N}{H}-[::60]-[::-60]-[::60]-[::-60](<[::-60]NH_2)-[::60](=[::60]O)-[::-60]OH}}{\small citrulline}
\end{center}

\begin{center}
\chemnameinit{}
\chemname{\chemfig{H_2N-[::30,,2,](=[::60]NH)-[::-60]\chembelow{N}{H}-[::60]-[::-60]-[::60]-[::-60](<[::-60]NH_2)-[::60](=[::60]O)-[::-60]OH}}{\small arginine}
\hspace{20}
\chemnameinit{}
\chemname{\chemfig{H_2N-[1](=[2]O)-[7]NH_2}}{\small urea}
\begin{center}

\begin{itemize}
\item defects generally cause hyperammonaemia
\item complete cycle is found only in periportal hepatocytes
\item involves two mitochondrial and three cytosolic enzymes as well as
the mitochondrial ornithine/citrulline antiporter and the activating
mitochondrial enzyme N-acetylglutamate synthase, which turns the
cycle on or off
\item required intermediates supplied by:
\begin{itemize}
\item carbonic anhydrase Va (CAVA) \(\to\) bicarbonate
\begin{itemize}
\item required by CPS
\end{itemize}
\item citrin is the aspartate/glutamate mitochondrial antiporter  \(\to\) aspartate
\begin{itemize}
\item required by ASS
\end{itemize}
\item \(\Delta\)1-pyrroline-5-carboxylate synthetase (P5CS) \(\to\) \emph{de novo} ornithine
\begin{itemize}
\item required by OTC
\end{itemize}
\end{itemize}
\end{itemize}

\begin{figure}[htbp]
\centering
\includegraphics[width=1\textwidth]{urea/figures/urea_cycle.png}
\caption{\label{fig:org7b8d79a}Urea Cycle}
\end{figure}


\begin{figure}[htbp]
\centering
\includegraphics[width=1\textwidth]{urea/figures/Slide01.png}
\caption{\label{fig:org7e6b44d}Urea Cycle}
\end{figure}


\begin{figure}[htbp]
\centering
\includegraphics[width=0.9\textwidth]{urea/figures/ammonia_dd.png}
\caption{\label{fig:org40a3b36}Diagnostic Algorithm that can be Applied to any Hyperammonaemic Patient}
\end{figure}

\subsection{Mitochondrial UCDs}
\label{sec:orgf6ae350}
\begin{itemize}
\item \textbf{carbamoyl-phosphate synthase (CPS), ornithine trancarbamylase (OTC) and N-acetylglutamate synthetase (NAGS)} deficiency
\item exclusive role of NAGS is to produce NAG the essential activator of CPS1
\begin{itemize}
\item \(\therefore\) NAGS deficiency is clinically indistinguishable from
CPS1 deficiency
\end{itemize}
\item OTC deficiency is the most frequent urea cycle error (about 60\% of UCD patients)
\item CPS1 and NAGS deficiency are very rare \textless{} 1:1,000,000 live births
\end{itemize}

\subsubsection{Clinical Presentation}
\label{sec:org2d38e3e}
\begin{itemize}
\item acute hyperammonaemia presenting as encephalopathy
\end{itemize}
\begin{enumerate}
\item Newborns
\label{sec:orgd48fff5}
\begin{itemize}
\item healthy at birth but may present by day 2 with a rapidly
progressing encephalopathy
\item presentation similar to bacterial sepsis, which can lead to
significant delay in the start of specific management for
hyperammonaemia
\begin{itemize}
\item vomiting, refusal to feed, somnolence/stupor/coma, muscular
hypotonia, seizures, hyper- or hypoventilation, and hypo- or
hyperthermia
\item respiratory alkalosis is common
\begin{itemize}
\item NH\textsubscript{3} stimulates respiratory center in the brain
\end{itemize}
\end{itemize}
\end{itemize}

\item Children and Adults
\label{sec:org2cb504f}
\begin{itemize}
\item variable
\item present during or shortly after an intercurrent infection or any
other catabolic situation or following a high-protein meal
\end{itemize}

\item Female OTC Carriers
\label{sec:orga2e8c30}
\begin{itemize}
\item variable individual inactivation of the X-chromosome hosting the
mutant OTC gene, they have highly variable residual OTC function
\begin{itemize}
\item some are asymptomatic
\item others report symptoms over many years that are likely explained
by recurrent undiagnosed hyperammonaemia often detected only after
a male offspring is diagnosed
\item others present with frank deficiency
\end{itemize}
\end{itemize}
\end{enumerate}

\subsubsection{Metabolic Derangement}
\label{sec:org537e48d}
\begin{itemize}
\item NAGS, CPS1 and OTC catalyse the initial two steps of the UC
\end{itemize}
\ce{NH3 + HCO3- + 2ATP ->[CPS] CP + Orn ->[OTC] Cit}
\begin{itemize}
\item \(\therefore\) produce the \(\Uparrow\) ammonia of all UCDs
\end{itemize}
\begin{itemize}
\item \(\Downarrow\) citrulline production
\item failure to incorporate ammonia into carbamoyl phosphate (CP)
explains the hyperammonaemia of CPS1 and NAGS deficiency
\begin{itemize}
\item in OTC deficiency it may be due to CPS1 inhibition by CP
that accumulates in the mitochondria
\end{itemize}
\item some CP leaks out of the mitochondria, leading to excessive
pyrimidine biosynthesis and overproduction of orotic acid and
uracil
\begin{itemize}
\item \(\uparrow\) urine excretion of orotic acid and uracil differentiates
conditions in which CP accumulates, such as OTC deficiency, from
those with normal or low CP, such as CPS1 and NAGS deficiencies
\end{itemize}
\item ammonia is neurotoxic
\begin{itemize}
\item NH\textsubscript{3} freely enters brain and is converted to non-permeable
\ce{NH4+}
\end{itemize}
\item \(\uparrow\) ammonia \(\to\) \(\uparrow\) glutamine due to glutamine synthetase
\begin{itemize}
\item increase can occur before and can persist after hyperammonaemia
\item synthesis of non-essential amino acids that require ammonia is increased:
\begin{itemize}
\item glycine, serine, glutamate and alanine
\end{itemize}
\end{itemize}
\end{itemize}

\subsubsection{Genetics}
\label{sec:org21d1a88}
\begin{itemize}
\item AR NAGS \& CPS1
\item \textbf{X-linked OTC}
\end{itemize}

\subsubsection{Diagnostic Tests}
\label{sec:org678d517}
\begin{itemize}
\item hallmark is hyperammonaemia, generally in the absence of
hypoglycaemia, lactic acidosis or ketonuria
\item diagnostic algorithm (Figure \ref{fig:org40a3b36})
\item plasma amino acids
\begin{itemize}
\item \(\uparrow\) glutamine
\item N-\(\downarrow\) citrulline
\item N-\(\downarrow\) arginine
\item \(\uparrow\) lysine
\begin{itemize}
\item \(\because\) \(\downarrow\) \(\alpha\)-ketoglutarate required for metabolism
\end{itemize}
\end{itemize}
\end{itemize}
\begin{enumerate}
\item CPS \& NAGS
\label{sec:org723697d}
\begin{itemize}
\item urine organic acids
\begin{itemize}
\item normal orotic acid
\item normal uracil
\end{itemize}
\end{itemize}
\item OTC
\label{sec:org9dccaac}
\begin{itemize}
\item urine organic acids
\begin{itemize}
\item \(\uparrow\) orotic acid
\item \(\uparrow\) uracil
\end{itemize}
\end{itemize}
\end{enumerate}


\subsubsection{Treatment}
\label{sec:org7002147}
\begin{enumerate}
\item Emergency
\label{sec:org5cf2128}
\begin{itemize}
\item stop natural protein intake
\item stop catabolism with supplementation
\item reduce ammonia with drugs \textpm{} dialysis
\begin{itemize}
\item sodium benzoate and/or sodium phenylbutyrate (Figure \ref{fig:org66f3f9b})
\end{itemize}

\item phenylbutyrate is a prodrug
\begin{itemize}
\item first converted to phenylbutyryl-CoA and then metabolized by mitochondrial
beta-oxidation to the active phenylacetate
\item phenylacetate conjugates with glutamine to phenylacetylglutamine,
which is eliminated with the urine
\item phenylacetylglutamine contains the same amount of nitrogen as
urea, which makes it an alternative to urea for excreting nitrogen
\end{itemize}

\item sodium benzoate combines with glycine to form hippuric acid which is
then excreted
\begin{itemize}
\item this begins with the conversion of benzoate by butyrate-CoA ligase
into an intermediate product, benzoyl-CoA which is then
metabolized by glycine N-acyltransferase into hippuric acid
\end{itemize}
\item citrulline in OTC \& CPS \(\to\) elimination of nitrogen from aspartate
\end{itemize}

\ce{citrulline + aspartate ->[ASS] ASA}

\begin{figure}[htbp]
\centering
\includegraphics[width=0.9\textwidth]{urea/figures/nitrogen_elimination.jpg}
\caption{\label{fig:org66f3f9b}Nitrogen Elimination by Phenylbutyrate and Benzoate}
\end{figure}

\item Prognosis
\label{sec:org209b096}
\begin{itemize}
\item presentation during newborn period have high risk of death
\item severe CPS1 and OTC deficiencies are prone to recurrent
hyperammonaemic crises
\begin{itemize}
\item should undergo liver transplantation as soon as it is possible and
safe
\end{itemize}
\item NAGS deficiency is the only UCD for which drug treatment is almost
curative
\begin{itemize}
\item N-carbamyl-L-glutamate (carbaglu)
\item a synthetic analogue of the physiological activator of CPS1, NAG,
given orally activates CPS1 and thereby urea cycle function
\end{itemize}
\end{itemize}
\end{enumerate}

\subsection{Cytosolic UCDs}
\label{sec:orgc247936}
\begin{itemize}
\item \textbf{arginosuccinic synthase (ASS), arginosuccinic lyase (ASL) and arginase (ARG1)} deficiencies
\item second most frequent among the UCDs
\begin{itemize}
\item ASS and ASL deficiency, each represent \(\sim\) 15\%
\item ARG1 deficiency representing 3\%
\end{itemize}
\end{itemize}

\subsubsection{Clinical Presentation}
\label{sec:org7914649}
\begin{enumerate}
\item Newborns
\label{sec:org974de88}
\begin{itemize}
\item ASS and ASL presentation resembles mitochondrial UCDs
\begin{itemize}
\item hyperammonaemic encephalopathy of similar severity
\item peak plasma ammonia may not be as high
\item onset delayed to day 6-7 of life or even later
\end{itemize}
\item ARG1 deficiency rarely presents in the newborn period
\end{itemize}

\item Children and Adults
\label{sec:orged63be7}
\begin{itemize}
\item ASS and ASL similar to mitochondrial UCDs
\item risk of hyperammonaemic decomposition: ASS \textgreater{} ASL
\item brittle hair due to trichorrhexis nodosa is almost pathognomonic for ASL deficiency
\begin{itemize}
\item results from arginine deficiency and responds to arginine
administration
\end{itemize}
\item ARG1 differs from ASL/ASS
\begin{itemize}
\item developmental delay with neurological and intellectual impairment
\item growth retardation and spastic cerebral palsy
\item seizures
\end{itemize}
\end{itemize}
\end{enumerate}

\subsubsection{Metabolic Derangement}
\label{sec:org98465ba}
\begin{itemize}
\item ASS
\begin{itemize}
\item \ce{citrulline + aspartate ->[ASS] ASA}
\item \(\Uparrow\) citrulline
\item \(\downarrow\) ASA
\item \(\downarrow\) arginine
\end{itemize}
\item ASL
\begin{itemize}
\item \ce{ASA ->[ASL] arginine + fumarate}
\item \(\uparrow\) citrulline
\item \(\Uparrow\) ASA
\item \(\downarrow\) arginine
\end{itemize}
\item ARG1
\begin{itemize}
\item \ce{arginine ->[ARG1] ornithine + urea}
\item \(\uparrow\) citrulline
\item \(\Uparrow\) arginine
\item induction in extrahepatic tissues of ARG2 may explain the modest
increase (about 15-fold) of plasma arginine, and the normal or
near-normal plasma ornithine
\end{itemize}
\end{itemize}


\begin{itemize}
\item citrulline and argininosuccinate include one molecule of ornithine
and one or two atoms of waste nitrogen respectively
\begin{itemize}
\item urinary excretion of these intermediates in ASS and ASL
deficiencies effectively removes waste nitrogen
\item with simultaneous loss of two (ASS) or one (ASL) ornithine
molecules per urea
\begin{itemize}
\item \(\therefore\) ornithine is essential in waste nitrogen excretion in
ASS and ASL deficiencies
\item administration of arginine converted to ornithine upon cleavage by arginase
\end{itemize}
\end{itemize}
\end{itemize}

\subsubsection{Genetics}
\label{sec:org5efdcbc}
\begin{itemize}
\item AR ASS1, ASL, ARG1
\end{itemize}

\subsubsection{Diagnostic Tests}
\label{sec:orge000bd1}
\begin{itemize}
\item plasma amino acids (see above for pattern)
\item urine AA
\begin{itemize}
\item \(\Uparrow\) ASA in ASL
\item \(\Uparrow\) citrulline
\end{itemize}
\item diagnostic algorithm (Figure \ref{fig:org40a3b36})
\end{itemize}

\subsubsection{Treatment}
\label{sec:org7344963}
\begin{itemize}
\item emergency management the same as mitochondrial UCDs
\begin{itemize}
\item give arginine in ASL and ASS
\end{itemize}
\item maintenance treatment for ASS and ASL deficiencies is the same as
CPS1 and OTC deficiencies
\item liver transplant should be considered in ASS, ASL and ARG1
\end{itemize}

\subsection{Mitochondrial Transport}
\label{sec:org6cc602f}
\subsubsection{HHH Syndrome}
\label{sec:org53ccf00}
\begin{itemize}
\item see Ornithine and Proline (Section \ref{sec:orgdbb530f})
\end{itemize}
\subsubsection{Citrin Deficiency}
\label{sec:org9e34d0a}
\begin{enumerate}
\item Clinical Presentation
\label{sec:org4e8861c}
\begin{itemize}
\item citrin is the hepatic mitochondrial aspartate/glutamate antiporter
\begin{itemize}
\item supplies aspartate for the ASS reaction
\end{itemize}
\item two main age dependent clinical presentations:
\begin{itemize}
\item Neonatal Intrahepatic Cholestasis Caused by Citrin Deficiency (NICCD)
\item Citrullinemia Type II (CTLN2)
\begin{itemize}
\item occurs in adolescents and adults
\end{itemize}
\end{itemize}
\item third less common clinical phenotype is Failure To Thrive and
Dyslipidemia Caused by Citrin Deficiency (FTTDCD) may also occur in
childhood
\end{itemize}

\item Metabolic Derangement
\label{sec:orgb476a77}
\begin{itemize}
\item exchange of mitochondrial aspartate for cytosolic glutamate and
the malate/aspartate shuttle are both affected
\end{itemize}

\ce{malate + NAD^+ ->[MDH] oxaloacetate + NADH}

\ce{oxaloacetate + glu + NADH ->[AST] \alpha-ketoglutarate + aspartate}

\ce{citrulline + aspartate ->[ASS] ASA}

\begin{itemize}
\item insufficient supply of aspartate from mitochondria for ASS within
hepatocytes, and the conversion of the fumarate released by ASL, to
form aspartate within the cytosol, is impaired, due to the low
cytosolic NAD\textsuperscript{+} resulting from lack of malate-aspartate shuttle
operation
\begin{itemize}
\item this shuttle transfers reducing equivalents from cytosolic NADH
to the mitochondria, regenerating NAD in the cytosol
\end{itemize}
\item low cytosolic aspartate decreases liver ASS activity, resulting in
citrulline accumulation, and also impairs protein and pyrimidines
synthesis in liver cells
\begin{itemize}
\item both processes are cytosolic and use aspartate, explaining the
hypoalbuminemia and hypoproteinemia of NICCD and the lack of
urinary orotic acid that differentiates citrin deficiency from
ASS deficiency
\end{itemize}
\item high cytosolic NADH/NAD\textsuperscript{+} ratios in the liver explain the
hypoglycaemia and the galactosemia that are frequently observed in
NICCD
\begin{itemize}
\item cytosolic NAD\textsuperscript{+} is needed both for gluconeogenesis from lactate and
for UDP-galactose to UDP-glucose conversion
\end{itemize}
\end{itemize}

\item Genetics
\label{sec:org6a79dd5}
\begin{itemize}
\item AR SLC25A13
\end{itemize}

\item Diagnostic Tests
\label{sec:org8378cb0}
\begin{itemize}
\item newborns with intrahepatic cholestasis the finding of:
\begin{itemize}
\item increased plasma citrulline
\item without significant hyperammonaemia
\item normal or elevated levels of arginine
\item without urinary orotic acid
\item high plasma level of alpha-fetoprotein
\item \textpm{} increased galactose
\end{itemize}
\item strongly suggestive of  NICCD
\end{itemize}

\item Treatment
\label{sec:org759abf7}
\begin{itemize}
\item avoid carbohydrate or glycerol infusions \(\to\) hyperammonaemia
\item maintenance treatment of NICCD involves the use of lactose-free or
MCT-enriched formula
\item when introduced, other foods should be protein rich and fat-rich,
such as eggs or fish
\end{itemize}
\end{enumerate}
\subsection{Ancillary Enzymes}
\label{sec:org86ea0bd}
\subsubsection{P5CS}
\label{sec:org6ff6917}
\begin{itemize}
\item see Ornithine and Proline (Section \ref{sec:orgdbb530f})
\end{itemize}
\subsubsection{CAVA}
\label{sec:org3977a18}
\begin{itemize}
\item Carbonic Anhydrase Va (CAVA) Deficiency
\end{itemize}
\begin{enumerate}
\item Clinical Presentation
\label{sec:org2ca3488}
\begin{itemize}
\item neonatal symptoms identical to those with neonatal onset UCD
\end{itemize}
\item Metabolic Derangement
\label{sec:orgdcfe898}
\begin{itemize}
\item bicarbonate cannot cross the mitochondrial membrane
\item spontaneous conversion of CO\textsubscript{2} to bicarbonate is too slow for the
needs of urea synthesis
\item CAVA accelerates this conversion within liver mitochondria
\begin{itemize}
\item supplying the bicarbonate used within mitochondria by:
\begin{itemize}
\item CPS1
\item pyruvate carboxylase
\item propionyl CoA carboxylase
\item 3-methylcrotonyl CoA carboxylase
\end{itemize}
\end{itemize}
\end{itemize}

\ce{CO2 + H2O ->[CAVA] HCO3- + H^+}

\begin{itemize}
\item \(\therefore\) CAVA deficiency impairs:
\begin{itemize}
\item urea cycle
\item gluconeogenesis
\item BCAA metabolism
\end{itemize}
\end{itemize}

\item Genetics
\label{sec:orgadca299}
\begin{itemize}
\item AR CA5A
\end{itemize}

\item Diagnostic Tests
\label{sec:orga22c976}
\begin{itemize}
\item metabolic acidosis
\item plasma:
\begin{itemize}
\item \(\uparrow\) ammonia
\item \(\downarrow\) citrulline
\item \(\uparrow\) lactate
\item \(\downarrow\) glucose
\item normal acylcarnitines
\end{itemize}
\item urine
\begin{itemize}
\item absence of orotic acid
\item \(\uparrow\) ketone bodies
\item urine organic acids contain carboxylase-related metabolites
(see multiple carboxylase deficiency)
\begin{itemize}
\item pyruvate carboxylase - \(\uparrow\) lactate \& pyruvate
\item propionic acid carboxylase
\item 3-methylcrotonyl-CoA carboxylase
\end{itemize}
\end{itemize}
\end{itemize}

\item Treatment
\label{sec:orgfc64583}
\begin{itemize}
\item emergency management for CAVA deficiency is mainly symptomatic
\begin{itemize}
\item focusing on treating hyperammonaemia as for intramitochondrial UCDs
\end{itemize}
\item good outcome
\end{itemize}
\end{enumerate}

\section{Sulfur Amino Acids}
\label{sec:orgd608a66}
\subsection{Introduction}
\label{sec:org5c9aa54}

\begin{center}
\chemnameinit{}
\chemname{\chemfig{H_3C-[1]S-[7]-[1]-[7](<[6]NH_2)-[1](=[2]O)-[7]OH}}{\small methionine}
\chemnameinit{}
\hspace{20}
\chemname{\chemfig{HS-[7]-[1]-[7](<[6]NH_2)-[1](=[2]O)-[7]OH}}{\small homocysteine}
\chemnameinit{}
\hspace{20}
\chemname{\chemfig{HS-[1]-[7](<[6]NH_2)-[1](=[2]O)-[7]OH}}{\small cysteine}
\end{center}


\begin{figure}[htbp]
\centering
\includegraphics[width=0.4\textwidth]{sulfur/figures/cys.jpg}
\caption{\label{fig:orgf84e5a2}Cysteine and Cystine}
\end{figure}

\begin{itemize}
\item methionine is converted by two methionine adenosyltransferases (MAT
I/III and MATII) to S-adenosylmethionine (SAM)
\item the methyl group of SAM is used in numerous methylation reactions,
yielding S-adenosylhomocysteine (SAH)
\item excess SAM is removed from the cycle by glycine N-methyltransferas (GNMT)
\item SAH is cleaved by S-adenosylhomocysteine hydrolase (SAHH) to
homocysteine and adenosine, which is phosphorylated by adenosine
kinase (ADK)
\item homocysteine has two metabolic pathways (Figure \ref{fig:orgf36fcac}):
\begin{enumerate}
\item remethylated back to methionine by the \textbf{remethylation pathway} or
using betaine as a methyl-group donor, in patients treated with
this drug
\item irreversibly metabolized to sulfate
by the \textbf{transsulfuration pathway}
\begin{itemize}
\item homocysteine and serine are condensed by cystathionine
\(\beta\)-synthase (CBS) to cystathionine
\item cystathionine is cleaved by cystathionine \(\gamma\)-lyase (CTH, AKA cystathionase) to
form cysteine and \(\alpha\)-ketobutyrate
\end{itemize}
\end{enumerate}
\item CTH can use cysteine and/or homocysteine to synthesize hydrogen
sulfide
\item cysteine can be further converted:
\begin{itemize}
\item in a series of reactions to taurine
\item via the mitochondrial enzymes, AST and 3-mercaptopyruvate
sulfurtransferase (MPST), to pyruvate and hydrogen sulfide
\end{itemize}
\item mitochondrial oxidation of hydrogen sulfide (\ce{H2S}) and of cysteine involves
several steps yielding thiosulfate (\ce{S2O3^2-}), sulfite
(\ce{SO3^2-}) and finally sulfate (\ce{SO4-})
\item inorganic sulfur released from cysteine residues \(\to\) mitochondrial
iron-sulfur (FeS) cluster cofactors
\item availability of cysteine in the neonatal period is limited because
its endogenous synthesis from methionine by the transsulfuration
pathway is markedly attenuated
\item activity of the rate limiting enzyme in the pathway cytathione
\(\gamma\)-lyase is very low at birth and increases slowly during the
first few months of life
\begin{itemize}
\item cysteine is considered a conditionally essential amino acid, at
least in preterm infants
\end{itemize}

\item disorders in sulfur amino acid metabolism exhibit:
\begin{itemize}
\item altered methionine, S-adenosylmethionine, sarcosine, S-adenosylhomocysteine,
total homocysteine or cystathionine concentrations in blood
\item adenosine or thiosulfate excretion in urine
\end{itemize}
\end{itemize}


\begin{itemize}
\item CBS deficiency (AKA classical homocystinuria) is the most common
disease in this group
\item CTH deficiency appears to be a biochemical trait with no major
clinical sequelae
\item disorders of cysteine and hydrogen sulfide oxidation pathway include:
\begin{itemize}
\item ethylmalonic encephalopathy
\item isolated sulfite oxidase deficiency
\item combined sulfite oxidase deficiency
\begin{itemize}
\item due to impaired molybdenum cofactor synthesis
\end{itemize}
\end{itemize}
\item these are severe disorders with early-onset seizures and other
neurological complications
\begin{itemize}
\item other signs include orthostatic acrocyanosis, lens dislocation or
urolithiasis
\end{itemize}
\item only molybdenum cofactor deficiency type A can be treated
successfully, with a synthetic cofactor
\item for methionine synthase see Cobalamin and Folate
\end{itemize}

\begin{figure}[htbp]
\centering
\includegraphics[width=0.9\textwidth]{sulfur/figures/sulfuraa.png}
\caption{\label{fig:org874591b}Sulfur Amino Acid Metabolism}
\end{figure}

\begin{figure}[htbp]
\centering
\includegraphics[width=0.9\textwidth]{sulfur/figures/Slide03.png}
\caption{\label{fig:orgf36fcac}Sulfur Amino Acid Metabolism}
\end{figure}

\begin{figure}[htbp]
\centering
\includegraphics[width=0.9\textwidth]{sulfur/figures/sulfaa.png}
\caption{\label{fig:orgac76a1f}Disorders of Sulfur Amino Acid Metabolism}
\end{figure}

\subsection{Methionine S-Adenosyltransferase Deficiency}
\label{sec:org35810ee}
\begin{itemize}
\item Mudd’s Disease
\end{itemize}
\subsubsection{Clinical Presentation}
\label{sec:org7b26026}
\begin{itemize}
\item most patients detected by NBS for CBS deficiency using methionine as a marker
\item neurological abnormalities occur in most patients with plasma methionine \textgreater{} 800 μmol/l
\begin{itemize}
\item rare in subjects with lower levels
\end{itemize}
\end{itemize}
\subsubsection{Metabolic Derangement}
\label{sec:orgdd57d48}
\begin{itemize}
\item methionine S-adenosyltransferase (MAT) converts methionine to S-adenosylmethionine (SAM) using ATP
\item MAT exists in 3 forms
\item MAT I and III tetrameric and dimeric forms encoded MAT1A 
\begin{itemize}
\item liver specific
\end{itemize}
\item MAT II is encoded by a different gene
\begin{itemize}
\item converts methionine to SAM outside the liver
\item explains why MAT I/III deficiency is relatively benign
\end{itemize}
\end{itemize}
\subsubsection{Genetics}
\label{sec:orgfc7fc46}
\begin{itemize}
\item AR MAT1A
\item some mutation are AD
\end{itemize}
\subsubsection{Diagnostic Tests}
\label{sec:org338f32f}
\begin{itemize}
\item plasma methionine 50 to \textgreater{} 2000 umol/L
\item other causes of hypermethioninemia:
\begin{itemize}
\item liver disease
\item prematurity
\item excessive intake of methionine
\item less often, CBS, S-adenosylhomocysteine hydrolase and ADK deficiencies
\begin{itemize}
\item CBS has \(\Uparrow\) homocysteine
\end{itemize}
\end{itemize}
\end{itemize}
\subsubsection{Treatment}
\label{sec:org52389dd}
\begin{itemize}
\item methionine restricted diet if met \textgreater{} 800 umol/L
\end{itemize}

\subsection{Cystathionine \(\beta\)-Synthase Deficiency}
\label{sec:orgaab6a49}
\begin{itemize}
\item AKA Homocystinuria
\end{itemize}
\subsubsection{Clinical Presentation}
\label{sec:org30deb6f}
\begin{itemize}
\item wide spectrum of severity and age at presentation
\item severity varies from
\begin{itemize}
\item multisystemic childhood condition with lens dislocation,
osteoporosis, marfanoid features, central nervous system and
vascular complications
\item isolated thromboembolic disease in adults
\end{itemize}
\item clinical features predominantly involve four organ systems:
\begin{description}
\item[{eye}] lens dislocation
\item[{skeleton}] excessive growth - marfanoid but stiff
\item[{brain}] learning disabilities
\item[{vascular}] thromboembolism
\end{description}
\end{itemize}
\subsubsection{Metabolic Derangement}
\label{sec:orgfc5b087}
\begin{itemize}
\item CBS is a cytosolic tetrameric enzyme
\end{itemize}

\ce{homocysteine ->[CBS] cystathionine}

\begin{itemize}
\item expressed predominantly in liver, pancreas, kidney and brain
\begin{itemize}
\item activity can also be determined in cultured fibroblasts and in plasma
due to its release from the liver
\end{itemize}
\item catalytic domain binds heme, PLP and substrates
\item regulatory domain binds the allosteric activator SAM
\item pathophysiology is not fully understood
\begin{itemize}
\item \(\uparrow\) SAH impairs methylation reaction
\item \(\uparrow\) homocysteine \(\to\) ER stress, vascular disease
\item enhanced remethylation \(\to\) \(\uparrow\) methionine
\item depletion of cystathionine and cysteine \(\to\) apoptosis, oxidative
stress, \(\Delta\) protein structure.
\end{itemize}
\end{itemize}

\subsubsection{Genetics}
\label{sec:org936a8e8}
\begin{itemize}
\item AR CBS
\end{itemize}

\subsubsection{Diagnostic Tests}
\label{sec:org13dd2de}
\begin{itemize}
\item \(\Uparrow\) plasma total homocysteine (tHcy)
\begin{itemize}
\item \textgreater{} 150 umol/L
\end{itemize}
\item \(\uparrow\) plasma methionine
\item measurement of free homocystine is not recommended
\begin{itemize}
\item low sensitivity
\item complicated pre-analytical requirements
\end{itemize}
\item to avoid misdiagnosis in pyridoxine responsive patients
\begin{itemize}
\item pyridoxine supplements including multivitamins should be avoided
for at least 2 weeks prior to testing
\end{itemize}
\item diagnosis very likely if the plasma methionine is high or borderline
high and supported by:
\begin{itemize}
\item \(\downarrow\) to low-normal plasma cystathionine
\item \(\uparrow\) methionine:cystathionine
\end{itemize}
\item can be confirmed by enzyme assay in cultured fibroblasts or plasma,
and/or mutation analysis of the CBS gene
\end{itemize}

\subsubsection{Treatment}
\label{sec:org7f01bcf}
\begin{itemize}
\item pyridoxine, betaine and a methionine-restricted diet
\item oral betaine (trimethylglycine)
\begin{itemize}
\item betaine-homocysteine methyltransferase (BHMT) is betaine dependent
\end{itemize}
\end{itemize}

\ce{trimethylglycine + homocysteine ->[BHMT] dimethylglycine + methionine}

\begin{itemize}
\item in the liver BHMT catalyzes up to 50\% of homocysteine metabolism
\item betaine treatment \(\to\) \(\uparrow\) sarcosine (methlyglycine) in plasma amino acids
\end{itemize}

\subsection{Molybdenum Cofactor Deficiency}
\label{sec:org73f6c54}
\subsubsection{Clinical Presentation}
\label{sec:orge5b014e}
\begin{itemize}
\item usually present soon after birth with poor feeding, hypotonia,
exaggerated startle reactions and intractable seizures, resembling
hypoxic ischaemic encephalopathy
\begin{itemize}
\item \(\to\) multicystic leukoencephalopathy with microcephaly
\end{itemize}
\item dislocation of the ocular lens occurs during infancy and xanthine
renal stones can develop later
\end{itemize}

\subsubsection{Metabolic Derangement}
\label{sec:org485edf2}
\begin{itemize}
\item molybdenum cofactor (MoCo) is a molybdenum containing ptern,
synthesis involves three steps:
\begin{description}
\item[{MoCo deficiency type A}] affects the conversion of GTP to cyclic
pyranopterin monophosphate (cPMP)
\item[{MoCo deficiency type B}] cannot convert cPMP to molybdopterin
\item[{MoCo deficiency type C}] affects gephyrin, which catalyses
adenylation of molybdopterin and insertion of molybdenum to
form the cofactor
\end{description}
\item molybdenum cofactor is needed for:
\begin{itemize}
\item sulfite oxidase, final step in transsulfuration pathway
\item aldehyde oxidase
\item mitochondrial amidoxime reducing component (mARC)
\item xanthine dehydrogenase
\end{itemize}
\item xanthine dehydrogenase deficiency causes raised xanthine and low
urate concentrations
\item sulfite accumulation is responsible for the neurotoxicity and lens
dislocation
\end{itemize}

\subsubsection{Genetics}
\label{sec:org9badc89}
\begin{itemize}
\item AR
\item[{Type A}] MOCS1 most common
\item[{Type B}] MOCS2
\item[{Type C}] GPHN rare
\end{itemize}

\subsubsection{Diagnostic Tests}
\label{sec:org5e01af1}
\begin{itemize}
\item \(\downarrow\) plasma urate concentration
\begin{itemize}
\item initially normal but decreases after a few days and remains low
\end{itemize}
\item \(\uparrow\) urine xanthine
\item sulfite (SO\textsubscript{3}\textsuperscript{2-}) can be detected in fresh urine using dipsticks but false
positive and negative results occur
\item \(\uparrow\) urine or blood S-sulfocysteine is a more reliable indicator
\end{itemize}

\ce{SO3^2- + cystine ->[unknown] S-sulfocysteine}

\begin{itemize}
\item S-sulfocysteine accumulation \(\to\) inhibition of antiquitin
\begin{itemize}
\item secondary elevation of pipecolic acid
\end{itemize}
\end{itemize}
\begin{itemize}
\item \(\uparrow\) plasma taurine and thiosulfate
\item \(\downarrow\) plasma total cysteine and tHcy
\item diagnosis is confirmed by mutation analysis
\end{itemize}

\subsubsection{Treatment}
\label{sec:orgf683b21}
\begin{itemize}
\item without treatment, patients have profound handicap and die early
\item successful treatment of Type A with daily intravenous infusions of
cPMP
\item no treatment for Types B \& C
\end{itemize}

\subsection{Isolated Sulfite Oxidase Deficiency}
\label{sec:org7252554}
\subsubsection{Clinical Presentation}
\label{sec:org414545b}
\begin{itemize}
\item resembles MoCo deficiency
\end{itemize}
\subsubsection{Metabolic Derangement}
\label{sec:orgc678803}
\begin{itemize}
\item in \textbf{sulfite oxidase} deficiency accumulating sulfite damages the
brain
\begin{itemize}
\item partly due to the production of sulfocysteine, which
mediates excitotoxicity
\end{itemize}
\end{itemize}

\ce{SO3^2- + cystine ->[unknown] S-sulfocysteine}

\begin{itemize}
\item sulfite derived from cysteine is normally oxidised to form
sulfate
\item sulfite probably causes lens dislocation by disrupting cystine
cross-linkages
\end{itemize}
\subsubsection{Genetics}
\label{sec:orgb99e469}
\begin{itemize}
\item AR SUOX
\end{itemize}

\subsubsection{Diagnostic Tests}
\label{sec:org8709eee}
\begin{itemize}
\item sulfite can be detected in fresh urine using dipsticks
\begin{itemize}
\item not reliable
\end{itemize}
\item \(\uparrow\) urine or blood S-sulfocysteine
\begin{itemize}
\item S-sulfocysteine accumulation \(\to\) inhibition of antiquitin
\begin{itemize}
\item secondary elevation of pipecolic acid
\end{itemize}
\end{itemize}
\item \(\uparrow\) plasma taurine
\item \(\downarrow\) plasma total cysteine and tHcy
\item normal urate and xanthine
\item diagnosis is confirmed by mutation analysis
\end{itemize}

\subsubsection{Treatment}
\label{sec:orgab4ea45}
\begin{itemize}
\item prognosis for neonatal-onset cases is poor
\item diet low in cysteine and methionine may help patients with a mild
form
\end{itemize}

\subsection{Ethylmalonic Encephalopathy}
\label{sec:org8599845}
\subsubsection{Clinical Presentation}
\label{sec:orga23ca83}
\begin{itemize}
\item progressive multisystem disease
\item presents in the first months of life with hypotonia, chronic
diarrhoea, orthostatic acrocyanosis, recurrent petechial rash and
bruising (with normal platelets)
\item developmental regression, microcephaly, seizures, episodes of coma,
poor growth and hyperlactataemia
\item most die in early childhood, though some have a milder course
\end{itemize}

\subsubsection{Metabolic Derangement}
\label{sec:org0e36d6a}
\begin{itemize}
\item deficiency of a \textbf{mitochondrial persulfide dioxygenase} necessary for
the detoxification of hydrogen sulfide (\ce{H2S})
\item \ce{H2S} is synthesized endogenously by CBS, CTH and
3-mercaptosulfur transferase
\begin{itemize}
\item also formed by bacterial anaerobes in the large intestine
\end{itemize}
\item in EE accumulating \ce{H2S} inhibits cytochrome c oxidase and
short-chain fatty acid oxidation
\begin{itemize}
\item results in ethylmalonic aciduria, and raised butyrl-carnitine (C4)
and isovaleryl-carnintine (C5) in blood
\end{itemize}
\item \ce{H2S} also has vasoactive and vasotoxic effects
\begin{itemize}
\item damage to small blood vessels causes bleeding into the
skin
\item production of \ce{H2S} by gut bacteria causes the severe, persistent diarrhea
\end{itemize}
\end{itemize}

\subsubsection{Genetics}
\label{sec:org61de4c7}
\begin{itemize}
\item AR ETHE1 rare
\end{itemize}
\subsubsection{Diagnostic Tests}
\label{sec:orga12c2d2}
\begin{itemize}
\item \(\uparrow\) lactate
\item \(\uparrow\) urine ethylmalonic and methylsuccinic
\item \(\uparrow\) urine butyrlglycine isovalerylglycine
\item \(\uparrow\) plasma butyrl-carnitine (C4) and isovaleryl-carnintine (C5)
\item \(\Uparrow\) urine thiosulfate is also markedly elevated
\item diagnosis is confirmed by mutation analysis
\end{itemize}

\subsubsection{Treatment}
\label{sec:orgdb80710}
\begin{itemize}
\item metronidazole to reduce bacterial \ce{H2S} production
\item N-acetylcysteine a precursor of glutathione, which can accept the sulfur atom of \ce{H2S}
\begin{itemize}
\item leads to some clinical and biochemical improvement the prognosis remains poor
\end{itemize}
\item liver transplant
\end{itemize}

\section{Ornithine and Proline}
\label{sec:orgdbb530f}
\subsection{Introduction}
\label{sec:org3babc04}
\subsubsection{Ornitine}
\label{sec:org67aeee7}
\begin{itemize}
\item ornithine is an intermediate in metabolic pathways involving the
urea cycle, proline metabolism and the biosynthesis of creatine and
polyamines (e.g. putrescine and spermine)
\end{itemize}

\begin{center}
\chemnameinit{}
\chemname{\chemfig{H_2N-[::30,,2,]-[::-60]-[::60]-[::-60](<[::-60]NH_2)-[::60](=[::60]O)-[::-60]OH}}{\small ornithine}
\end{center}

\begin{itemize}
\item ornithine-\(\delta\)- aminotransferase (OAT) is a pyridoxal
phosphate-requiring, mitochondrial matrix enzyme that plays a
pivotal role in these pathways

\ce{ornithine + \alpha-KG <->[OAT] P5C + glutamate}
\item OAT reaction is freely reversible:
\begin{itemize}
\item \textbf{during the neonatal period the net flux is in the direction of}
\textbf{ornithine and, via the urea cycle, arginine biosynthesis}
\item \textbf{\textgreater{} few months of age the net flux reverses to favour arginine}
\textbf{disposal via the synthesis of \(\Delta\)1-pyrroline-5-carboxylate}
\textbf{(P5C), an intermediate in proline and glutamate synthesis}
\end{itemize}
\item ornithine also plays an essential role as the substrate for urea assembly
\item ornithine produced in the cytoplasm from arginine is transported
into the mitochondrial matrix by an energy-requiring transport
system involving ORNT1
\begin{itemize}
\item an antiporter in the inner mitochondrial membrane, which exchanges
cytosolic ornithine with mitochondrial citrulline
\item defective in HHH syndrome (Section \ref{sec:org53ccf00})
\end{itemize}
\item in the cytoplasm ornithine is decarboxylated to putrescine which is
then converted to spermine
\end{itemize}

\subsubsection{Proline}
\label{sec:org7bfdba4}
\begin{itemize}
\item proline unlike all other amino acids (except hydroxyproline) has
no primary amino group \(\therefore\) an imino acid
\begin{itemize}
\item \(\therefore\) uses a specific set of enzymes for metabolism
\end{itemize}
\end{itemize}

\begin{center}
\chemnameinit{}
\chemname{\chemfig{*5(-\chembelow{N}{H}-(-(=[1]O)-[7]OH)---)}}{\small proline}
\end{center}

\begin{itemize}
\item pyridinoline ring of proline and hydroxyproline contribute to the
structural stability of proteins particularly collagen
\item P5C is the product/precursor of the OAT reaction
\item P5C is precursor/degradation product of proline
\item P5C synthetase is a bifunctional ATP and NADPH-dependent
mitochondrial enzyme that is highly active in the gut
\begin{itemize}
\item also expressed in brain catalyses the reduction of glutamate to
P5C
\end{itemize}
\item P5C/proline cycle transfers reducing/oxidizing potential between
cellular organelles
\end{itemize}


\begin{figure}[htbp]
\centering
\includegraphics[width=1\textwidth]{orn_pro/figures/orn_pro.png}
\caption{\label{fig:orgab7a97b}Ornithine and Proline Metabolism}
\end{figure}

\begin{figure}[htbp]
\centering
\includegraphics[width=1\textwidth]{orn_pro/figures/Slide07.png}
\caption{\label{fig:orgc77bfaa}Ornithine and Proline Metabolism}
\end{figure}

\begin{figure}[htbp]
\centering
\includegraphics[width=1.2\textwidth]{orn_pro/figures/op_diff.png}
\caption{\label{fig:org54435b9}Differential Diagnosis of Ornthine and Proline Disorders}
\end{figure}

\subsection{OAT Deficiency}
\label{sec:orgd7c712d}
\begin{itemize}
\item Gyrate Atrophy of the Choriod Retina
\end{itemize}
\subsubsection{Clinical Presentation}
\label{sec:org71e907f}
\begin{itemize}
\item initial symptoms myopia, night blindness in early-mid childhood
\item fundoscopic appearance of the chorioretinal atrophy in OAT deficiency is highly specific
\begin{itemize}
\item chorioretinal atrophy is progressive \(\to\) blindness by 45-65 years
\end{itemize}
\end{itemize}

\subsubsection{Metabolic Derangement}
\label{sec:orgdd48834}
\begin{itemize}
\item OAT is PLP dependent
\begin{itemize}
\item there is a PLP responsive variant
\end{itemize}
\end{itemize}
\begin{enumerate}
\item Neonatal period
\label{sec:org8c80b4f}
\begin{itemize}
\item present with increased blood ammonia and low levels of plasma
ornithine, citrulline, arginine and orotic aciduria in their first
weeks of life
\item hyperornithinaemia develops later in life
\item net flux in the OAT reaction in the newborn period is in the
direction of ornithine synthesis rather than degradation
\item disruption of the anapleurotic function of the OAT reaction for the
urea cycle can lead to
\begin{itemize}
\item insufficient levels of citrulline and arginine
\item inadequate ureagenesis and consequent hyperammonaemia
\end{itemize}
\end{itemize}
\item Non-neonatal
\label{sec:orgf2368ff}
\begin{itemize}
\item patients develop hyperornithinaemia
\begin{itemize}
\item fasting plasma ornithine in the range of 400-1200 uM
\end{itemize}
\item mechanism of the retinal degeneration is unclear
\item reduced levels of creatine in blood, urine, muscle and brain
\begin{itemize}
\item inhibition of AGAT by \(\uparrow\) ornithine
\end{itemize}
\end{itemize}
\end{enumerate}

\subsubsection{Genetics}
\label{sec:org68fd986}
\begin{itemize}
\item AR OAT
\end{itemize}

\subsubsection{Diagnostic Tests}
\label{sec:org7470d5e}
\begin{itemize}
\item most prominent biochemical abnormality in those ingesting an
unrestricted diet:
\begin{itemize}
\item \(\Uparrow\) 5- to 20-fold plasma ornithine
\end{itemize}
\item \(\uparrow\) urine ornithine
\item when plasma ornithine \textgreater{} 400 uM
\begin{itemize}
\item \(\uparrow\) urine ornithine, lysine, arginine, cystine
\item secondary to competitive inhibition of shared dibasic aa renal transporter
\end{itemize}
\item differentiated from HHH by lack of homocitrulline in urine
\item \textbf{difficult to distinguish neonatal OAT from OTC} in both disorders:
\begin{itemize}
\item \(\downarrow\) plasma ornithine, arginine, and citrulline
\item \(\uparrow\) urine orotic acid
\end{itemize}
\item molecular and enzyme confirmation
\end{itemize}

\subsubsection{Treatment}
\label{sec:org634d6a1}
\begin{itemize}
\item restrict arginine to reduce plasma ornithine levels \textless{} 200 uM
\item \(\uparrow\) dose PLP works in subset of responsive patients
\end{itemize}

\subsection{HHH Syndrome}
\label{sec:org7bdde97}
\begin{itemize}
\item Hyperornithinaemia, Hyperammonaemia and Homocitrullinuria syndrome
\end{itemize}
\begin{center}
\chemnameinit{}
\chemname{\chemfig{H_2N-[::-30,,2,](=[::-60]O)-[::60]\chemabove{N}{H}-[::-60]-[::60]-[::-60]-[::60]-[::-60](<[::-60]NH_2)-[::60](=[::60]O)-[::-60]OH}}{\small homocitrulline}
\end{center}
\subsubsection{Clinical Presentation}
\label{sec:org577c65f}
\begin{itemize}
\item broad spectrum, with some related to episodic hyperammonemia
\item intolerance to protein feeding, vomiting, seizures and developmental
delay from infancy are common
\item neonatal onset of lethargy, hypotonia and seizures, with progression
to coma and death observed in the most severe form
\item liver failure/dysfunction
\item can be chronic and progressive
\begin{itemize}
\item food aversion, central and peripheral neurological dysfunction
\end{itemize}
\end{itemize}

\subsubsection{Metabolic Derangement}
\label{sec:org1cebfb6}
\begin{itemize}
\item HHH syndrome is a disorder of metabolic compartmentation, with
impaired importation of ornithine into the mitochondria
\item results in deficiency of OTC and OAT activity with:
\begin{itemize}
\item \(\Uparrow\) plasma ornithine
\item \(\Uparrow\) plasma ammonia
\end{itemize}
\item \(\downarrow\) intramitochondrial ornithine \(\to\) utilisation of
carbamoyl-phosphate by other pathways with formation of:
\begin{itemize}
\item homocitrulline from lysine
\item orotic acid
\end{itemize}
\end{itemize}

\ce{carbamoyl-P + lysine -> homocitrulline}

\subsubsection{Genetics}
\label{sec:orgcd52440}
\begin{itemize}
\item AR ORNT1 (aka SLC24A15)
\item more frequent in Canada, as a result of a founder mutation in Quebec
\end{itemize}

\subsubsection{Diagnostic Tests}
\label{sec:orgece967a}
\begin{itemize}
\item can be differentiated from other hyperammonaemic syndromes by laboratory findings
\item \(\Uparrow\) plasma ornithine
\item \(\Uparrow\) plasma ammonia
\item \(\uparrow\) urine homocitrulline
\item above triad is pathognomonic
\item plasma ornithine concentration is elevated to 3 to 10 fold
\begin{itemize}
\item lower than in OAT deficiency
\end{itemize}
\item plasma citrulline reduction is less pronounced than in OTC
deficiency
\item when plasma ornithine \textgreater{} 400 uM
\begin{itemize}
\item \(\uparrow\) urine ornithine, lysine, arginine, cystine
\item secondary to competitive inhibition of shared dibasic aa renal
transporter
\end{itemize}
\item orotic aciduria is common in HHH
\end{itemize}

\subsubsection{Treatment}
\label{sec:org0e39762}
\begin{itemize}
\item prevent ammonia toxicity
\item low protein diet
\item citruline and arginine supplementation
\item ammonia scavengers are used:
\begin{itemize}
\item sodium benzoate
\item sodium phenylbutyrate
\end{itemize}
\item prognosis variable, generally good
\end{itemize}

\subsection{P5CS  Deficiency}
\label{sec:org3afcdad}
\begin{itemize}
\item Hypoprolinaemia
\end{itemize}
\subsubsection{Clinical Presentation}
\label{sec:org0fa95a0}
\begin{itemize}
\item central and peripheral neurological
\begin{itemize}
\item progressive ID
\end{itemize}
\item cataracts
\item joint hypermobility
\item see figure \ref{fig:org54435b9} for details
\end{itemize}

\subsubsection{Metabolic Derangement}
\label{sec:org2e57c9c}
\begin{itemize}
\item \textbf{\(\Delta\)1-Pyrroline-5-Carboxylate Synthetase (P5CS)} deficiency
\item P5CS catalyses an essential step in synthesis of proline, ornithine
and arginine from glutamate
\end{itemize}

\ce{glutamate ->[P5CS] P5C}

\begin{itemize}
\item pattern of metabolic abnormalities consistent with impaired proline
and ornithine synthesis
\item hypoornithinaemia, hypocitrullinaemia, hypoargininaemia,
hypoprolinaemia and mild hyperammonaemia
\end{itemize}

\subsubsection{Genetics}
\label{sec:orga9261d9}
\begin{itemize}
\item AR/AD ALDH18A1
\end{itemize}

\subsubsection{Diagnostic Tests}
\label{sec:org8a1f3c9}
\begin{itemize}
\item abnormal metabolite profile is corrected in the fed state
\begin{itemize}
\item \(\therefore\) the metabolic phenotype of P5CS deficiency is easily
missed
\end{itemize}
\item combination of the following should suggest this disorder:
\begin{itemize}
\item \(\downarrow\) fasting: ornithine, citrulline, arginine and proline
\item tendency to paradoxical fasting hyperammonaemia
\item or one of the above together with a clinical phenotype of mental
retardation, connective tissue manifestations and/or cataracts
\end{itemize}
\end{itemize}

\subsubsection{Treatment}
\label{sec:org8bc6087}
\begin{itemize}
\item supplementation of the deficient amino acids seems to be a
reasonable therapeutic approach
\end{itemize}

\subsection{Proline Oxidase Deficiency}
\label{sec:org8e8bdd7}
\begin{itemize}
\item Hyperprolinaemia Type I
\end{itemize}
\subsubsection{Clinical Presentation}
\label{sec:org9264e3f}
\begin{itemize}
\item well tolerated in some individuals
\item may contribute to risk for schizophrenia or other psychiatric,
cognitive or behavioural abnormalities
\end{itemize}

\subsubsection{Metabolic Derangement}
\label{sec:org3baf242}
\begin{itemize}
\item deficiency of \textbf{proline oxidase} a mitochondrial inner-membrane
enzyme
\end{itemize}

\ce{proline ->[POX] P5C}

\subsubsection{Genetics}
\label{sec:org87ddf29}
\begin{itemize}
\item AR PRODH
\item maps to 22q11, in the region deleted in the velocardiofacial
syndrome/DiGeorge syndrome
\end{itemize}

\subsubsection{Diagnostic Tests}
\label{sec:orgb09fdee}
\begin{itemize}
\item \(\Uparrow\) plasma proline
\begin{itemize}
\item usually \(\le\) 2000 uM (normal range 100-450 uM)
\end{itemize}
\item \(\uparrow\) urine and cerebrospinal fluid (CSF) proline
\item hyperprolinaemia (as high as 1000 μM) is also observed as a
secondary phenomenon in hyperlactataemia
\begin{itemize}
\item possibly because proline oxidase is inhibited by lactic acid
\item alanine will also elevated in this situation
\end{itemize}
\end{itemize}
\subsubsection{Treatment}
\label{sec:org4e0d6bd}
\begin{itemize}
\item prognosis excellent
\item no treatment needed
\end{itemize}

\subsection{P5CDH Deficiency}
\label{sec:org517d847}
\begin{itemize}
\item Hyperprolinaemia Type II
\end{itemize}

\subsubsection{Clinical Presentation}
\label{sec:orgb2227c5}
\begin{itemize}
\item relatively benign disorder
\item attenuated phenotype
\item \textasciitilde{} 50\% present with seizures
\end{itemize}
\subsubsection{Metabolic Derangement}
\label{sec:org23835c5}
\begin{itemize}
\item \textbf{pyrroline 5-carboxylate (P5C) dehydrogenase} deficiency
\begin{itemize}
\item mitochondrial inner-membrane enzyme involved in the conversion of
proline into glutamate
\end{itemize}

\ce{P5C ->[P5CDH] glutamate}

\item accumulating P5C is a vitamin B\textsubscript{6} antagonist due to adduct
formation
\item seizures may be due to vitamin B\textsubscript{6} inactivation
\end{itemize}

\subsubsection{Genetics}
\label{sec:org3f2d351}
\begin{itemize}
\item AR ALDH4A1
\end{itemize}

\subsubsection{Diagnostic Tests}
\label{sec:org50b9276}
\begin{itemize}
\item \(\Uparrow\) plasma proline and P5C
\begin{itemize}
\item proline \textgreater{} 2000 uM (normal range 100-450 uM)
\end{itemize}
\item \(\uparrow\) urine and CSF proline and P5C
\end{itemize}
\subsubsection{Treatment}
\label{sec:orgc7b0aa5}
\begin{itemize}
\item seizures are pyridoxine (B\textsubscript{6}) responsive
\end{itemize}

\section{Lysine and Cerebral OA Disorders}
\label{sec:orgec9ac1d}
\subsection{Introduction}
\label{sec:orge2d0cea}
\begin{itemize}
\item lysine, hydroxylysine and tryptophan are thought to be degraded
within mitochondria
\end{itemize}

\begin{center}
\chemnameinit{}
\chemname{\chemfig{H_2N-[::-30,,2,]-[::60]-[::-60]-[::60]-[::-60](<[::-60]NH_2)-[::60](=[::60]O)-[::-60]OH}}{\small lysine}
\hspace{20}
\chemnameinit{}
\chemname{\chemfig{*6(-*5(-\chembelow{N}{H}-=(-[1]-[::-60](<[::-60]NH_2)-[::60](=[::60]O)-[::-60]OH)-)=-=-=)}}{\small tryptophan}
\end{center}

\begin{itemize}
\item initially via separate pathways, which converge into a common 
pathway at:
\begin{itemize}
\item 2-aminoadipic-6-semialdehyde
\begin{itemize}
\item hydroxylysine catabolism and pipecolic acid pathway of lysine
catabolism
\end{itemize}
\item 2-oxoadipic acid
\begin{itemize}
\item tryptophan catabolism
\end{itemize}
\end{itemize}
\item major route of lysine catabolism in most tissues is via the
bifunctional enzyme, 2-aminoadipic-6-semialdehyde synthase (Figure \ref{fig:org89f127a} enzyme 1)
\item small amount of lysine is catabolised via pipecolic acid by the
peroxisomal enzyme, pipecolic acid oxidase (Figure \ref{fig:org89f127a} enzyme 2)
\begin{itemize}
\item this pathway is the major route of lysine catabolism in the
brain
\end{itemize}
\item source of pipecolic acid is not yet fully understood
\begin{itemize}
\item from lysine and 2-aminoadipic-6-semialdehyde
\item microbiome
\end{itemize}
\item 2-oxoadipic acid is primarily converted to glutaryl-CoA by the
ketoglutarate dehydrogenase-like complex (Figure \ref{fig:org89f127a} enzyme 6a)
since its E1 subunit (DHT-KD1) has a higher substrate affinity for
2-oxoadipic acid than the ketoglutarate dehydrogenase complex (KGDH) in
the Krebs cycle (Figure \ref{fig:org89f127a} enzyme 6b)
\begin{itemize}
\item KGDH may serve as an alternative route
\item 2-oxoadipic acid is dehydrogenated and decarboxylated to
crotonyl-CoA by glutaryl-CoA dehydrogenase (Figure \ref{fig:org89f127a} enzyme 7)
\begin{itemize}
\item tranfers electrons to FAD \(\to\) ETC via ETF/ETF-DH
\end{itemize}
\end{itemize}
\item crotonyl-CoA is subsequently converted to 3-hydroxybutyryl-CoA by
short-chain enoyl-CoA hydratase 1 (Figure \ref{fig:org89f127a} crotonase, enzyme 8)
\begin{itemize}
\item this enzyme is multispecific and also acts as a crotonase in the
degradative pathways of valine, isoleucine and short-chain fatty
acids
\end{itemize}
\item 3-hydroxybutyryl-CoA is then converted to acetoacetyl-CoA by
3-hydroxyacyl-CoA dehydrogenase (Figure \ref{fig:org89f127a} enzyme 9)
\item glutaric acid is derived from the intestinal microbiome, spontaneous
disintegration of glutaryl-CoA or other sources
\begin{itemize}
\item normally reactivated by succinyl-CoA-dependent conversion via
succinate-hydroxymethylglutarate CoA transferase to glutaryl-CoA
(Figure \ref{fig:org89f127a} enzyme 10)
\end{itemize}

\item six enzyme deficiencies identified in the degradation of lysine,
only four \(\to\) neurometabolic disorders:
\begin{itemize}
\item antiquitin (Section Vitamin B\textsubscript{6})
\item KGDH - alternative route for lysine catabolism
\item glutaryl-CoA dehydrogenase
\item crotonase
\end{itemize}
\end{itemize}

\subsubsection{Cerebral Organic Acidurias}
\label{sec:orge008619}
\begin{itemize}
\item a group of organic acid disorders present exclusively with
progressive neurological symptoms of ataxia, epilepsy, myoclonus,
extrapyramidal symptoms, metabolic stroke, and macrocephaly
\begin{itemize}
\item glutaric aciduria
\item 2-hydroxyglutaric aciduria
\item \(\gamma\)-hydroxybutyric aciduria (Section Neurotransmitters)
\item N-acetylaspartic aciduria
\end{itemize}
\item in all these disorders the pathological compounds that accumulate
either are odd-chain dicarboxylic acids
\begin{itemize}
\item 2-hydroxyglutarate, 3-hydroxyglutarate, glutarate, N-acetylaspartate
\begin{itemize}
\item sharing the same carbon backbone with the excitatory amino acid
glutamate (2-amino-glutarate)
\end{itemize}
\item or neurotransmitters/negative modulators
\begin{itemize}
\item \(\gamma\)-hydroxybutyrate
\end{itemize}
\end{itemize}
\end{itemize}

\begin{figure}[htbp]
\centering
\includegraphics[width=1.2\textwidth]{lys/figures/cat.png}
\caption{\label{fig:org89f127a}Tryptophan, Hydroxylysine and Lysine Catabolic Pathways}
\end{figure}


\begin{figure}[htbp]
\centering
\includegraphics[width=1\textwidth]{lys/figures/Slide05.png}
\caption{\label{fig:org4b023c2}Tryptophan and Lysine Catabolic Pathways}
\end{figure}

\subsection{Glutaric Aciduria Type I}
\label{sec:org94b9813}
\subsubsection{Clinical Presentation}
\label{sec:orga189dc7}
\begin{itemize}
\item differential diagnosis of any infant who has:
\begin{itemize}
\item macrocephaly combined with progressive atrophic changes on MRI or CT and/or
\item complex extrapyramidal syndrome of predominantly dystonia,
orofacial dyskinesia and dysarthria superimposed on axial hypotonia
\end{itemize}
\item \(\sim\) 9 months the majority of untreated patients suffer an acute
brain injury
\begin{itemize}
\item usually associated with an upper respiratory and/or
gastrointestinal infection
\end{itemize}
\end{itemize}
\subsubsection{Metabolic Derangement}
\label{sec:orge48d014}
\begin{itemize}
\item deficiency of \textbf{glutaryl-CoA dehydrogenase} a mitochondrial
FAD-requiring enzyme
\item catalyses the dehydrogenation of glutaryl-CoA and subsequent
decarboxylation of glutaconyl-CoA to crotonyl-CoA
\end{itemize}
\ce{glutaryl-CoA + FAD ->[GCDH] crotonyl-CoA + FADH2 + CO2}

\begin{itemize}
\item \(\uparrow\) glutaryl-CoA \(\to\) esterified with carnitine to
glutarylcarnitine by carnitine acyltransferase
\begin{itemize}
\item \(\to\) \(\uparrow\) acylcarnitines:free carnitine ratio in plasma and
urine
\item glutarylcarnitine is excreted, contributing to secondary carnitine
deficiency
\item often \(\uparrow\) urinary excretion of dicarboxylic acids,
2-oxoglutarate and succinate indicating disturbed mitochondrial
function
\end{itemize}

\item impaired brain energy metabolism induced by accumulating glutaric
acid, 3-hydroxyglutaric acid and glutaryl-CoA:
\begin{itemize}
\item glutaryl-CoA inhibits \(\alpha\)-ketoglutarate dehydrogenase complex (KGDH)
\item glutaric acid impairs the dicarboxylic acid shuttle between
astrocytes and neurons
\item 3-hydroxyglutaric acid weakly activates glutamatergic neurotransmission
\end{itemize}

\item weak permeability of the blood-brain barrier for dicarboxylic acids
\begin{itemize}
\item traps these metabolites in the brain compartment
\end{itemize}
\end{itemize}

\subsubsection{Genetics}
\label{sec:orgbc4d099}
\begin{itemize}
\item AR GCDH
\item Oji-Cree first nation, incidence of 1 in 300 newborns
\begin{itemize}
\item homozygous for the splice site mutation IVS-1+5 g>t
\end{itemize}
\end{itemize}

\subsubsection{Diagnostic Tests}
\label{sec:orgb9d2129}
\begin{itemize}
\item low excretors: deficiency of glutaryl-CoA dehydrogenase and severe
characteristic neurological disease but with only slight or
inconsistent elevations of glutaric acid or glutaryl-carnitine
\item \(\uparrow\) glutaric acid and 3-OH glutaric acid by UOA
\begin{itemize}
\item 3-OH glutaric acid in urine has a high sensitivity including
\begin{itemize}
\item low-excretor phenotype
\item secondary carnitine depletion
\end{itemize}
\end{itemize}
\item \(\uparrow\) glutaryl-carnitine (C5DC)
\item \(\downarrow\) plasma carnitine
\item \(\Uparrow\) acylcarnitines:free carnitine in urine and plasma
\end{itemize}

\subsubsection{Treatment}
\label{sec:orgd7d96af}
\begin{itemize}
\item early diagnosis key to prevention of acute striatal necrosis and
neurological sequelae
\item emergency treatment during illness
\begin{itemize}
\item \(\uparrow\) CHO feeds, carnitine supplementation
\item lysine free amino acids
\end{itemize}
\item oral supplementation w carnitine and riboflavin
\item lysine free, tryptophan reduced and arginine enriched amino acid
mixtures aims to minimise the risk of malnutrition
\end{itemize}

\subsection{L and/or D 2-Hydroxyglutaric Aciduria}
\label{sec:org72bd271}
\subsubsection{Clinical Presentation}
\label{sec:org4aad8cb}
\begin{description}
\item[{L2}] progressive neurological disease
\begin{itemize}
\item IQ in adolescence \(\sim\) 40-50
\end{itemize}
\item[{D2}] more variable than L2
\item[{L2 \& D2}] neonatal-onset encephalopathy with severe muscular
weakness, intractable seizures, respiratory distress, and lack of
psychomotor development resulting in early death
\end{description}

\subsubsection{Metabolic Derangement}
\label{sec:org4424996}
\begin{itemize}
\item 2-hydroxyglutarate is a "nonsense metabolite" of glutaric acid
\item error in "metabolite repair"
\begin{description}
\item[{L2}] FAD-linked 2-hydroxyglutarate dehydrogenase
\begin{itemize}
\item mitochondrial enzyme converts L-2-hydroxyglutarate to
2-oxoglutarate
\item affected in multiple acyl-CoA dehydrogenase deficiency (MADD/GA2)
\end{itemize}
\item[{D2}] D-2-hydroxyglutarate dehydrogenase
\begin{itemize}
\item enzyme converts D-2-hydroxyglutarate to 2-oxoglutarate
\end{itemize}
\item[{L2 \& D2}] mitochondrial citrate transport protein
\end{description}
\end{itemize}

\subsubsection{Genetics}
\label{sec:org5efcb00}
\begin{description}
\item[{L2}] AR L2HGDH
\item[{D2}] AR D2HGDH but may be genetically heterogeneous
\item[{L2 \& D2}] AR SLC25A1
\end{description}

\subsubsection{Diagnostic Tests}
\label{sec:orge53ee05}
\begin{itemize}
\item L-2- and D-2-hydroxyglutaric acid cannot be differentiated by
conventional GC-MS analysis
\begin{itemize}
\item chromatographic separation of these enantiomers can be performed
using derivatisation with a chiral reagent or a chiral stationary
phase
\begin{description}
\item[{L2}] \(\uparrow\) L-2-hydroxyglutarate in all fluids
\item[{D2}] \(\uparrow\) D-2-hydroxyglutarate in all fluids
\end{description}
\end{itemize}
\end{itemize}

\subsubsection{Treatment}
\label{sec:org0b60b14}
\begin{description}
\item[{L2}] riboflavin reported, poor prognosis
\item[{D2}] none, death in childhood
\end{description}

\subsection{N-Acetylaspartic Aciduria}
\label{sec:org5c3a5a1}
\begin{itemize}
\item Canavan disease
\end{itemize}
\subsubsection{Clinical Presentation}
\label{sec:orgcb9e0c2}
\begin{itemize}
\item 2-4 months w progressive neurological disease
\item macrocephaly by 1 year
\end{itemize}

\subsubsection{Metabolic Derangement}
\label{sec:org9c44516}
\begin{itemize}
\item aspartoacylase deficiency
\end{itemize}

\ce{N-acetyl-L-aspartate + H2O <=>[ASPA] carboxylate + L-aspartate}

\begin{itemize}
\item in the brain, aspartoacylase is located in oligodendrocytes
\begin{itemize}
\item hydrolyses NAA which is formed in neurons from L-aspartate and
L-acetate
\end{itemize}
\item \(\uparrow\) NAA results in
\begin{itemize}
\item \(\downarrow\) brain acetate levels
\item \(\downarrow\) myelin lipid synthesis
\item altered cerebral fluid balance
\end{itemize}
\end{itemize}

\subsubsection{Genetics}
\label{sec:org1e36615}
\begin{itemize}
\item AR ASPA
\end{itemize}

\subsubsection{Diagnostic Tests}
\label{sec:org0ee613f}
\begin{itemize}
\item \(\Uparrow\) N-acetylaspartic on UOA
\begin{itemize}
\item 100x \(\uparrow\) is pathognomonic
\end{itemize}
\item borderline elevation of NAA found in various forms of white matter
disease
\item mutations or enzyme assay to confirm
\end{itemize}

\subsubsection{Treatment}
\label{sec:org318a90f}
\begin{itemize}
\item none
\end{itemize}

\section{Non-Ketotic Hyperglycinemia}
\label{sec:org6d1215f}
\subsection{Introduction}
\label{sec:org9b0a404}
\begin{itemize}
\item Non-Ketotic Hyperglycinemia (NKH) is a genetic disorder characterized
by deficient activity of the \textbf{glycine cleavage system (GCS)}
\end{itemize}

\begin{center}
\chemnameinit{}
\chemname{\chemfig{H_2N-[7]-[1](=[2]O)-[7]OH}}{\small glycine}
\end{center}

\begin{itemize}
\item classic NKH is caused by mutations in genes that encode the protein
components of the GCS
\begin{itemize}
\item based on the clinical severity and developmental outcome classic
NKH is further divided into:
\begin{itemize}
\item severe classic NKH
\item attenuated classic NKH
\end{itemize}
\end{itemize}
\item variant NKH has an overlapping phenotype w classic NKH
\begin{itemize}
\item variant NKH is caused by: 
\begin{enumerate}
\item lipoic acid biosynthesis defects
\begin{itemize}
\item fits into the larger group of the lipoate synthesis
disorders
\item lipoate is a cofactor for:
\begin{itemize}
\item GCS
\item pyruvate dehydrogenase (PDH)
\item \(\alpha\)-ketoglutarate dehydrogenase (KGDH)
\item branched chain ketoacid dehydrogenase (BCKDH)
\end{itemize}
\item lipoate disorders are also called multiple mitochondrial dysfunction
syndromes
\end{itemize}
\item pyridox(am)ine 5-phosphate oxidase (PNPO) deficiency
\begin{itemize}
\item leads to reduced activity of a large number of
pyridoxal-P dependent reactions including the GCS
\end{itemize}
\end{enumerate}
\end{itemize}
\item phenocopies of NKH historically called transient NKH are non-genetic
secondary causes of elevated plasma and CSF glycine levels
\item atypical NKH is a heterogeneous historic term, which should no
longer be used
\end{itemize}

\begin{figure}[htbp]
\centering
\includegraphics[width=0.5\textwidth]{nkh/figures/gce.png}
\caption{\label{fig:org9e32ed0}Glycine Cleavage System}
\end{figure}

\begin{figure}[htbp]
\centering
\includegraphics[width=0.6\textwidth]{nkh/figures/lip.png}
\caption{\label{fig:org41e0a2e}Activation of Lipoate Dependant Enzymes}
\end{figure}

\subsection{Clinical Presentation}
\label{sec:orgd4ba851}
\subsubsection{Severe Classic NKH}
\label{sec:org82e0892}
\begin{itemize}
\item neonatal onset
\item hypotonia, increasing lethargy, coma, apnea, myoclonic jerks,
seizures, and frequent hiccupping, which may have already started
during the prenatal period
\item 15-30\% die in neonatal period
\item developmental age of approximately six weeks
\end{itemize}

\subsubsection{Attenuated Classic NKH}
\label{sec:orgc74e48f}
\begin{itemize}
\item attain developmental milestones with varying degrees of
developmental and intellectual progress
\item may present in the neonatal period, resembling severe classic NKH,
or later in infancy with hypotonia, lethargy, coma, and seizures
\end{itemize}
\subsubsection{Variant NKH}
\label{sec:org7caa90c}
\begin{itemize}
\item lipoate deficiency combines the clinical findings of severe NKH with
mitochondriopathies, such as leukoencephalopathy, optical atrophy,
cardiomyopathy, and sometimes episodes of severe lactic acidosis
\item PNPO deficiency may also resemble classic NKH but with a much more
complex metabolic profile that includes a mildly elevated CSF
glycine
\begin{itemize}
\item symptoms are partly responsive to pyridoxal-P
\end{itemize}
\end{itemize}

\subsection{Metabolic Derangement}
\label{sec:org5c6c0ce}
\begin{itemize}
\item glycine is involved in many biochemical reactions
\item GCS breaks down glycine with tetrahydrofolate into carbon dioxide,
ammonia and methylene-tetrahydrofolate
\item four protein complex located in the mitochondrial matrix
\begin{itemize}
\item H-protein carrying a lipoyl-group as the central core
\end{itemize}
\item octanoate is synthesized by intramitochondrial fatty acid synthesis
on the mitochondrial acylcarrier protein and is transferred to apo-H
protein by lipoyltransferase 2 (LIPT2)
\item lipoate synthase (LIAS) donates the sulfur atoms from its
iron-sulfur cluster to build lipoyl-H
\item lipoyltransferase 1 (LIPT1) transfers the lipoate group from
H-protein to the E2 component of four 2-ketodehydrogenases
\item LIPT1 deficiency causes dysfunction of the E2 subunits of PDH, KGDH,
and BCKDH (Figure \ref{fig:org41e0a2e})
\item LIPT2 and LIAS add deficiency of the GCS
\item some disorders of iron-sulfur cluster biogenesis genes add deficient
activity of complexes I and II of the respiratory chain and
aconitase, which carry iron sulfur clusters
\item in NKH glycine accumulates particularly in the brain
\item glycine is an inhibitory neurotransmitter for the glycinergic
receptors in the brain stem and spinal cord
\begin{itemize}
\item possibly contributing to apnea and hypotonia
\end{itemize}
\item in neuronal stem cells, the glycinergic receptor may be excitatory
rather than inhibitory
\item glycine is an allosteric activator of the excitatory
N-methyl-D-aspartate (NMDA) type glutamate receptor NR1/NR2 type
\begin{itemize}
\item \(\uparrow\) glycine \(\to\) overactivation of this NMDA receptor
\end{itemize}
\end{itemize}
\subsection{Genetics}
\label{sec:org5779b5b}
\begin{itemize}
\item classic NKH is AR
\begin{itemize}
\item GLDC encoding the P-protein \textasciitilde{} 80\%
\item AMT encoding the T-protein
\item no mutations identified in GCSH encoding the H-protein
\end{itemize}
\item variant NKH is AR
\begin{itemize}
\item genes involved in the synthesis of lipoate
\begin{itemize}
\item LIAS, BOLA3, GLRX5, NFU1, ISCA2, LIPT1, LIPT2, and IBA57
\end{itemize}
\item PNPO deficiency
\begin{itemize}
\item PNPO
\end{itemize}
\end{itemize}
\end{itemize}
\subsection{Diagnostic Tests}
\label{sec:org1442d51}
\begin{itemize}
\item \(\uparrow\) glycine are found in plasma, urine, and CSF
\begin{itemize}
\item \(\uparrow\) plasma glycine has low specificity
\item \(\uparrow\) CSF glycine is highly indicative of NKH
\begin{itemize}
\item avoid blood contamination
\end{itemize}
\item \(\uparrow\) CSF glycine:plasma glycine
\end{itemize}
\item valproate inhibits GCS
\item molecular testing of GCS genes
\item transient NKH is a phenocopy
\begin{itemize}
\item elevated CSF glycine levels disappear spontaneously over the next
days to weeks
\end{itemize}
\item variant NKH manifest only mild elevations of glycine in plasma and CSF
\begin{itemize}
\item may also have an increase of
\begin{itemize}
\item plasma alanine, lactate and pyruvate
\item 2-ketoglutarate in urine organic acids
\end{itemize}
\end{itemize}
\item in PNPO deficiency other results point to low pyridoxal phosphate
including:
\begin{itemize}
\item \(\downarrow\) CSF HVA and 5HIAA
\item \(\uparrow\) CSF 3-methoxytyrosine, glycine, threonine, histidine and
taurine
\item \(\downarrow\) plasma arginine
\end{itemize}
\end{itemize}

\subsection{Treatment}
\label{sec:org85180c2}
\begin{itemize}
\item withdrawal of intensive care in the neonatal period is an ethical
consideration given the very poor outcome in severe classic NKH
\item correct distinction between severe and attenuated NKH can aid in
this decision making
\item reduction of glycine plasma levels by benzoate \(\to\) hippurate
\begin{itemize}
\item monitor for carnitine deficiency
\end{itemize}
\item glycine restricted diet
\begin{itemize}
\item dietary glycine has a small contribution to glycine flux
\end{itemize}
\item receptor agonists to block effects of glycine
\end{itemize}

\section{Glutamine, Serine and Asparagine}
\label{sec:orga59bcd0}
\subsection{Glutamine Metabolism}
\label{sec:orgced93fc}
\begin{itemize}
\item glutamine is the most abundant amino acid
\end{itemize}

\begin{center}
\chemnameinit{}
\chemname{\chemfig{H_2N-[::30,,2,](=[::60]O)-[::-60]-[::60]-[::-60](<[::-60]NH_2)-[::60](=[::60]O)-[::-60]OH}}{\small glutamine}
\end{center}
\begin{itemize}
\item synthesized \emph{de novo} from glutamate and ammonia by glutamine synthetase
\begin{itemize}
\item the only way to produce glutamine in the human body
\item there is no net absorption from the intestine
\end{itemize}
\item plays a pivotal role in metabolism as constituent, precursor, and
amino moiety donor for many proteins, purines and pyrimidines,
NAD\textsuperscript{+}, AMP, amino acids and glucose
\end{itemize}

\subsubsection{Disorders}
\label{sec:orgcfa1cb9}
\begin{itemize}
\item one disorder of glutamine synthesis due to glutamine synthetase deficiency
\begin{itemize}
\item reported in only three patients
\item presented as newborns with epileptic encephalopathy
\item the outcome is very poor
\item not known if early start of glutamine therapy improves clinical
prognosis
\end{itemize}
\end{itemize}
\subsection{Serine Metabolism}
\label{sec:org77b254f}
\begin{itemize}
\item synthesized \emph{de novo} from glycolytic intermediate
3-phosphoglycerate
\begin{itemize}
\item can also be synthesized from glycine by serine
hydroxymethyltransferase
\end{itemize}
\end{itemize}

\begin{center}
\chemnameinit{}
\chemname{\chemfig{HO-[::30]-[::-60](<[::-60]NH_2)-[::60](=[::60]O)-[::-60]OH}}{\small serine}
\end{center}
\begin{itemize}
\item precursor of a number of compounds that play an essential role in
neuronal development and function:
\begin{itemize}
\item D-serine
\item glycine
\item cysteine
\item serine phospholipids
\item sphingomyelins
\item cerebrosides
\end{itemize}
\item must be synthesized within the brain because of poor permeability
across the BBB
\begin{itemize}
\item synthesis is confined to astrocytes
\item shuttle to neuronal cells is performed by a dedicated neutral amino
acid transporter ASCT1
\end{itemize}
\item major source of THF and of other one-carbon donors that are required
for the synthesis of purines and thymidine
\end{itemize}

\begin{figure}[htbp]
\centering
\includegraphics[width=0.9\textwidth]{qsn/figures/Slide06.png}
\caption{\label{fig:org62d60dc}Serine and Glycine Metabolism}
\end{figure}

\subsubsection{Disorders}
\label{sec:org3745807}
\begin{itemize}
\item four disorders of serine metabolism are known

\item three are in biosynthesis:
\begin{itemize}
\item 3-phosphoglycerate dehydrogenase deficiency
\begin{itemize}
\item most have severe infantile phenotype called Neu-Laxova syndrome
\begin{itemize}
\item congenital microcephaly, intellectual disability and
intractable seizures
\item perinatal lethality
\end{itemize}
\item treatment is oral L-serine
\item supplemented with glycine if unsatisfactory clinical response
\end{itemize}
\item phosphoserine aminotransferase deficiency
\begin{itemize}
\item Neu-Laxova syndrome or milder phenotype
\item treated with oral serine and glycine
\end{itemize}
\item phosphoserine phosphatase deficiency
\begin{itemize}
\item Neu-Laxova syndrome or milder phenotype
\end{itemize}
\end{itemize}

\item patients with the severe, infantile form, the treatment is oral
L-serine, supplemented with glycine in case of unsatisfactory
clinical response
\end{itemize}

\subsection{Asparagine}
\label{sec:orga9a0c03}
\begin{itemize}
\item synthesized \emph{de novo} by asparagine synthetase (ASNS)
\begin{itemize}
\item catalyzes transfer of ammonia from glutamine to aspartic acid via
a \(\beta\)-aspartyl-AMP intermediate
\end{itemize}
\end{itemize}

\begin{center}
\chemnameinit{}
\chemname{\chemfig{H_2N-[::-30](=[6]O)-[1]-[7](<[6]NH_2)-[1](=[2]O)-[7]OH}}{\small asparagine}
\end{center}
\begin{itemize}
\item must be synthesized within the brain because of poor permeability
across the BBB
\begin{itemize}
\item \(\therefore\) asparagine synthesis is essential for the development
and function of the brain but not of other organs
\end{itemize}
\item ASNS is expressed at low levels in most tissues except the brain
\begin{itemize}
\item there is a brain-specific splice variant
\end{itemize}
\item precursor of the glucogenic amino acid L-aspartate and the
neurotransmitter D-aspartate
\item important role in ammonia detoxification and in asparagine-linked
protein glycosylation
\end{itemize}

\subsubsection{Disorders}
\label{sec:org461135c}
\begin{itemize}
\item one disorder of asparagine synthesis has been reported, due to
asparagine synthetase deficiency
\begin{itemize}
\item severe psychomotor/intellectual disability, progressive
microcephaly, limb hypertonia, hyperreflexia and mostly also
intractable epilepsy
\item death in first year
\item no treatment due to poor BBB permeability for Asp
\end{itemize}
\end{itemize}

\section{Transport}
\label{sec:orgadc5abc}
\subsection{Introduction}
\label{sec:orgdced5fa}
\begin{itemize}
\item epithelial cells in (for example) renal tubules and intestinal
mucosa utilise several different amino acid transport systems which
prefer amino acids with certain physicochemical properties
\item inherited defects in amino acid transport at the cell membrane are
usually expressed as selective renal aminoaciduria
\begin{itemize}
\item the concentration of the affected amino acids is high in the urine
while it is normal or low in plasma
\end{itemize}
\item intestinal absorption of the affected amino acids is also almost
always impaired
\item clinical symptoms result from excess amounts of certain amino acids
in the urine or lack of them in the tissues
\end{itemize}
\subsubsection{Dibasic Amino Acids}
\label{sec:org5e02b11}
\begin{itemize}
\item cystine and structurally related dibasic cationic amino acids
lysine, arginine and ornithine are transported from the intestinal
or renal tubular lumen into epithelial cells by an apical
transporter (system b\textsuperscript{0,+}) in exchange for neutral amino acids
\item dibasic amino acids are then transported from the epithelial cell
into the tissues by a basolateral dibasic amino acid transporter
(system y\textsuperscript{+L}) in exchange for neutral amino acids and sodium
\item both these transporters are heteromers of a heavy subunit and a
light subunit linked by a disulfide bridge
\end{itemize}

\begin{figure}[htbp]
\centering
\includegraphics[width=0.2\textwidth]{transport/figures/cystine.png}
\caption{\label{fig:org8902361}Cystine}
\end{figure}

\subsubsection{Neutral  Amino Acids}
\label{sec:orgcb65dd5}
\begin{itemize}
\item transporter system for neutral amino acids is expressed only at the
luminal border of epithelial cells
\item it transports neutral amino acids into epithelial cells
\item alanine, asparagine, citrulline, glutamine, histidine, isoleucine,
leucine, phenylalanine, serine, threonine, tryptophan, tyrosine
and valine
\end{itemize}
\subsubsection{Dicarboxylic Amino Acids}
\label{sec:orgefc98ed}
\begin{itemize}
\item dicarboxylic amino acids (aspartate, glutamate) have a specific
transporter EAAT3, located at the luminal border
of epithelial cells and also expressed in neurons
\end{itemize}

\begin{figure}[htbp]
\centering
\includegraphics[width=0.9\textwidth]{transport/figures/aatrans.png}
\caption{\label{fig:org5224330}Cationic and Neutral Amino Acid Transport in Epithelial Cells}
\end{figure}

\subsection{Cystinuria}
\label{sec:org89f7a57}
\subsubsection{Clinical Presentation}
\label{sec:org548e13e}
\begin{itemize}
\item life-long risk of urolithiasis
\item 1-2\% of kidney stones in adults
\item 6-8\% of kidney stones in children
\item recurrent urinary tract infections urinary obstruction and
eventually renal failure are possible complications
\end{itemize}

\subsubsection{Metabolic Derangement}
\label{sec:org6be2ae0}
\begin{itemize}
\item defective \textbf{cystine and the dibasic amino acids transporter system}
\textbf{b\textsuperscript{0,+}} from the intestine or renal tubule into epithelial cells
\item \(\downarrow\) absorption of cystine in the intestine and \(\downarrow\)
reabsorption in the kidney
\begin{itemize}
\item normally 99\% of filtered cystine is reabsorbed
\end{itemize}
\item cystine is poorly soluble at neutral or low pH
\end{itemize}

\subsubsection{Genetics}
\label{sec:orgb2f8eea}
\begin{itemize}
\item incidence \(\sim\) 1:7000
\item type A, SLC3A1 (encodes rBAT)
\begin{itemize}
\item AR, \textgreater{} 60\% of cases
\end{itemize}
\item type B, SLC7A9 (encodes b\textsuperscript{0,+}AT)
\begin{itemize}
\item AD, incomplete penetrance
\end{itemize}
\item homozygous contiguous gene deletions on chromosome 2p21 or 2p16
\begin{itemize}
\item cystinuria associated with severe neurological findings (2p16) or
Prader-Willi-like syndrome (2p21)
\end{itemize}
\end{itemize}

\subsubsection{Diagnostic Tests}
\label{sec:org154edfa}
\begin{itemize}
\item cystine stones IR-spectroscopy
\item cyanide nitroprusside test
\begin{itemize}
\item reacts with cystine \(\to\) purple colour
\end{itemize}
\item urine amino acids ("COLA")
\begin{itemize}
\item \(\Uparrow\) cystine
\item \(\uparrow\) arginine
\item \(\uparrow\) ornithine
\item \(\uparrow\) lysine
\end{itemize}
\item plasma amino acids are normal
\end{itemize}
\subsubsection{Treatment}
\label{sec:org19aaa6f}
\begin{itemize}
\item \(\uparrow\) fluid intake
\item alkalinisation of urine
\item cysteine chelation with penicillamine
\end{itemize}

\subsection{Lysinuric Protein Intolerance}
\label{sec:org339f259}
\subsubsection{Clinical Presentation}
\label{sec:org1a1a11a}
\begin{itemize}
\item breast-fed newborns and infants are usually
asymptomatic
\item postprandial episodes of hyperammonaemia usually emerge when formula
with higher protein content or supplementary high-protein foods are
introduced
\item hyperammonaemia may present as refusal to eat, vomiting, stupor and
drowsiness leading to coma, and can be misdiagnosed as food
protein-induced enterocolitis syndrome
\item toddlers and school-age children present most often with growth
failure and hepatosplenomegaly
\item most patients develop a protective aversion to high-protein foods
\begin{itemize}
\item impairs amino acid intake \(\to\) amino acid deficiencies
\end{itemize}
\end{itemize}

\subsubsection{Metabolic Derangement}
\label{sec:org6d9b1e2}
\begin{itemize}
\item defective \textbf{dibasic cationic amino acid transporter y\textsuperscript{+}LAT1} in the
basolateral membrane of epithelial cells in the renal tubules and
small intestine
\begin{itemize}
\item massive amounts of lysine and more moderate amounts of arginine and
ornithine are lost in the urine
\end{itemize}
\item intestinal absorption is limited \(\to\) low plasma concentrations
\item glutamine, glycine and alanine concentrations are often elevated due
to malfunction of the urea cycle
\item \textbf{hyperammonaemia after protein ingestion and diminished protein}
\textbf{tolerance in LPI resemble urea cycle disorders}

\begin{itemize}
\item best explained by functional deficiency of the intermediates
arginine and ornithine in hepatocytes
\end{itemize}
\item carnitine deficiency due to:
\begin{itemize}
\item \(\downarrow\) meat intake
\item \(\downarrow\) endogenous biosynthesis due to lysine deficiency
\end{itemize}
\end{itemize}

\subsubsection{Genetics}
\label{sec:orgf31e9c8}
\begin{itemize}
\item AR SLC7A7
\end{itemize}

\subsubsection{Diagnostic Tests}
\label{sec:org28ac3a2}
\begin{itemize}
\item combination of increased urinary excretion and low plasma
concentrations of the dibasic amino acids, especially lysine
\item urine amino acids ("OLA")
\begin{itemize}
\item \(\uparrow\) lysine
\item \(\uparrow\) arginine
\item \(\uparrow\) ornithine
\item N-\(\uparrow\) citrulline
\end{itemize}
\item plasma amino acids
\begin{itemize}
\item \(\downarrow\) lysine
\item \(\downarrow\) arginine
\item \(\downarrow\) ornithine
\end{itemize}
\item \(\uparrow\) postprandial plasma ammonia
\item postprandial orotic aciduria
\item nonspecific almost constant findings include:
\begin{itemize}
\item \(\uparrow\) serum LDH
\item \(\uparrow\) ferritin
\item \(\uparrow\) triglycerides
\item due to secondary haemophagocytic lymphohistiocytosis (HLH)
\end{itemize}
\end{itemize}

\subsubsection{Treatment}
\label{sec:orga16e1de}
\begin{itemize}
\item principal aims of the treatment are to:
\begin{itemize}
\item prevent hyperammonaemia
\item provide a sufficient supply of protein and essential amino acids
for normal metabolism and growth
\end{itemize}

\item protein tolerance can be improved with supplementary low dose
citrulline
\item citrulline is readily absorbed and partially converted to arginine
and ornithine which improves the function of the urea cycle
\item carnitine supplementation for patients with carnitine deficiency
\end{itemize}

\subsection{Hartnup Disease}
\label{sec:orgeec05c2}
\subsubsection{Clinical Presentation}
\label{sec:orgf1c7d86}
\begin{itemize}
\item classical symptoms of Hartnup disease are pellagra-like dermatitis,
intermittent ataxia and neuropsychiatric abnormalities
\begin{itemize}
\item resembles nutritional niacin deficiency
\end{itemize}
\item can be identified by newborn urine amino acid screening
\begin{itemize}
\item most remain asymptomatic
\end{itemize}
\end{itemize}

\subsubsection{Metabolic Derangement}
\label{sec:orged3c129}
\begin{itemize}
\item defective \textbf{sodium-dependent and chloride-independent neutral amino}
\textbf{acid transporter, B\textsuperscript{0}AT1} facing into the lumen of the renal proximal
tubule and intestinal epithelium
\item impair intestinal uptake and tubular reabsorption of all the neutral
amino acids
\begin{itemize}
\item alanine, serine, threonine,valine, leucine, isoleucine,
phenylalanine, tyrosine, tryptophan, histidine and citrulline
\item and monoamino-dicarboxylic amides asparagine and glutamine
\end{itemize}
\item affected amino acids are readily absorbed in the intestine as short
oligopeptides but not as free amino acids
\begin{itemize}
\item excreted in 5- to 20-fold excess into the urine and feces
\begin{itemize}
\item \(\therefore\) low plasma concentrations
\end{itemize}
\end{itemize}
\item tryptophan deficiency \(\to\) development of clinical symptoms such as
neuropsychiatric signs
\begin{itemize}
\item precursor of the neurotransmitter serotonin
\item reduced availability of nicotinic acid, the precursor of NADPH
\end{itemize}
\end{itemize}

\subsubsection{Genetics}
\label{sec:org88133aa}
\begin{itemize}
\item AR SLC6A19
\end{itemize}

\subsubsection{Diagnostic Tests}
\label{sec:org0afe4a8}
\begin{itemize}
\item \(\uparrow\) urine neutral amino acids
\item normal or low-normal concentrations in plasma
\end{itemize}

\subsubsection{Treatment}
\label{sec:org97bca68}
\begin{itemize}
\item niacin
\item adequate protein for tryptophan requirements
\end{itemize}
\end{document}