% Created 2019-12-10 Tue 18:09
% Intended LaTeX compiler: pdflatex
\documentclass{scrartcl}
\usepackage[utf8]{inputenc}
\usepackage[T1]{fontenc}
\usepackage{graphicx}
\usepackage{grffile}
\usepackage{longtable}
\usepackage{wrapfig}
\usepackage{rotating}
\usepackage[normalem]{ulem}
\usepackage{amsmath}
\usepackage{textcomp}
\usepackage{amssymb}
\usepackage{capt-of}
\usepackage{hyperref}
\hypersetup{colorlinks,linkcolor=black,urlcolor=blue}
\usepackage{textpos}
\usepackage{textgreek}
\usepackage[version=4]{mhchem}
\usepackage{chemfig}
\usepackage{siunitx}
\usepackage{gensymb}
\usepackage[usenames,dvipsnames]{xcolor}
\usepackage[T1]{fontenc}
\usepackage{lmodern}
\usepackage{verbatim}
\usepackage{tikz}
\usepackage{wasysym}
\usetikzlibrary{shapes.geometric,arrows,decorations.pathmorphing,backgrounds,positioning,fit,petri}
\usepackage{fancyhdr}
\pagestyle{fancy}
\author{Matthew Henderson, PhD, FCACB}
\date{\today}
\title{Disorders of Carbohydrate Metabolism}
\hypersetup{
 pdfauthor={Matthew Henderson, PhD, FCACB},
 pdftitle={Disorders of Carbohydrate Metabolism},
 pdfkeywords={},
 pdfsubject={},
 pdfcreator={Emacs 26.1 (Org mode 9.1.9)}, 
 pdflang={English}}
\begin{document}

\maketitle
\setcounter{tocdepth}{2}
\tableofcontents


\section{Carbohydrate Metabolism}
\label{sec:orgd96927d}
\subsection{Introduction}
\label{sec:org32a6f56}
\begin{figure}[htbp]
\centering
\includegraphics[width=0.4\textwidth]{./carbohydrate_metabolism/figures/carb_digest.pdf}
\caption{\label{fig:org46e0143}
Carbohydrate Digestions}
\end{figure}

\begin{figure}[htbp]
\centering
\includegraphics[width=0.6\textwidth]{./carbohydrate_metabolism/figures/glucose_pathways.pdf}
\caption{\label{fig:orgdebb197}
Major Pathways of Glucose Metabolism}
\end{figure}

\begin{figure}[htbp]
\centering
\includegraphics[width=0.45\textwidth]{./carbohydrate_metabolism/figures/glucose_conversion.pdf}
\caption{\label{fig:orgb370317}
Conversion of Glucose}
\end{figure}

\subsection{Regulation}
\label{sec:orgb169d3c}
\subsubsection{Major Hormones of Metabolic Homeostasis}
\label{sec:orgb000d17}
\begin{itemize}
\item \textbf{Insulin} is the main anabolic hormone
\begin{itemize}
\item promotes use of glucose as fuel
\begin{itemize}
\item \(\uparrow\) transport into cells
\end{itemize}
\item storage of glucose as glycogen
\item conversion of glucose \(\to\) TAGs
\item TAG storage in adipose tissue
\item AA uptake and protein synthesis in muscle
\end{itemize}
\item \textbf{Glucagon} is catabolic
\begin{itemize}
\item maintain fuel availability in the absence of dietary glucose
\item stimulates glycogenolysis
\item stimulates gluconeogenesis from lactate, glycerol and AAs
\item mobilizing FAs from adipose TAGs
\item acts on liver and adipose, muscle has no receptor
\end{itemize}
\end{itemize}

\begin{figure}[htbp]
\centering
\includegraphics[width=0.6\textwidth]{./carbohydrate_metabolism/figures/regulation.pdf}
\caption{\label{fig:org1ec6da3}
Glucose Homeostasis}
\end{figure}


\begin{table}[htbp]
\caption[Insuline and Counterregulatory]{\label{tab:org1ca2662}
Insulin and Counterregulatory Hormones}
\centering
\begin{tabular}{lll}
Hormone & Function & Pathway\\
\hline
Insulin & \(\uparrow\) storage & glucose \(\to\) glycogen\\
 & \(\uparrow\) growth & FA synthesis and storage\\
 &  & AA uptake, protein synthesis\\
\hline
Glucagon & mobilizes stores & \(\uparrow\) gluconeogenesis\\
 & maintain blood glucose & \(\uparrow\) glycogenolysis\\
 & during a fast & FA release\\
\hline
Epinephrine & mobilize fuel during & \(\uparrow\) glycogenolysis\\
 & acute stress & FA release\\
\hline
Cortisol & long-term fuel requirements & \(\uparrow\) AA mobilization\\
 &  & from muscle\\
 &  & \(\uparrow\) gluconeogenesis for\\
 &  & glycogen synthesis\\
 &  & \(\uparrow\) FA release\\
\end{tabular}
\end{table}

\begin{figure}[htbp]
\centering
\includegraphics[width=1.0\textwidth]{./carbohydrate_metabolism/figures/counter_hormones.pdf}
\caption{\label{fig:org79e977c}
Low Blood Glucose}
\end{figure}

\subsection{Transport}
\label{sec:orga8dfea9}
\subsubsection{Absorption from the Intestine}
\label{sec:orga68c768}
\begin{itemize}
\item NA-dependent transporter
\begin{itemize}
\item transport glucose \(\uparrow\) the concentration gradient
\item w co-transport of NA \(\downarrow\) the concentration gradient
\end{itemize}
\item Facilitative Glucose Transport
\begin{itemize}
\item transport glucose \(\downarrow\) the concentration gradient
\item GLUT1 to GLUT5
\end{itemize}
\item Galactose and Fructose Transport
\begin{itemize}
\item Gal uses same mechanism as glucose
\item Fructose relies on facilitated diffusion via GLUT5
\end{itemize}
\end{itemize}

\begin{figure}[htbp]
\centering
\includegraphics[width=1.0\textwidth]{./carbohydrate_metabolism/figures/intestine.pdf}
\caption{\label{fig:orgeb29bff}
Absorption from the intestine}
\end{figure}

\begin{table}[htbp]
\caption[GLUTS]{\label{tab:org14b1d9f}
GLUTS}
\centering
\begin{tabular}{lll}
Transporter & Distribution & Comments\\
\hline
GLUT1 & erythrocyte & barrier cells\\
 & brain barrier & \(\uparrow\) affinity transporter\\
 & retina barrier & \\
 & placenta barrier & \\
 & testis barrier & \\
\hline
GLUT2 & Liver & \(\uparrow\) capacity, \(\downarrow\) affinity\\
 & Kidney & may be glucose sensor\\
 & Pancreatic \(\beta\)-cell & in pancreas\\
 & intestine & \\
\hline
GLUT3 & Neurons & \(\uparrow\) affinity  transporter in CNS\\
\hline
GLUT4 & Adipose & insulin sensitive transport\\
 & Skeletal muscle & \(\uparrow\) insulin \(\to\) \(\uparrow\) number\\
 & Heart muscle & \(\uparrow\) affinity\\
\hline
GLUT5 & Intestinal epithelium & fructose transport\\
 & spermatozoa & \\
\end{tabular}
\end{table}

\subsection{Glycogen}
\label{sec:org52d3765}

\begin{itemize}
\item glycogen is the storage form of glucose found in glycogen particles
\item large polymer of branched glucose polysaccharide
\item composed of glucosyl chains linked by \(\alpha\)-1-4-glycosidic bonds
\item \(\alpha\)-1-6-branches every 8 to 10 residues
\begin{itemize}
\item allows parallel processing
\item \(\uparrow\) solubility
\end{itemize}

\item There is regulation of synthesis and degradation
\begin{itemize}
\item Primarily in liver and muscle
\end{itemize}
\end{itemize}

\begin{figure}[htbp]
\centering
\includegraphics[width=0.4\textwidth]{./carbohydrate_metabolism/figures/glycogen_synth.pdf}
\caption{\label{fig:org2b57166}
Glycogen Synthesis}
\end{figure}

\begin{figure}[htbp]
\centering
\includegraphics[width=0.5\textwidth]{./carbohydrate_metabolism/figures/glycogen_degradation.pdf}
\caption{\label{fig:orgbd1ae2e}
Glycogen Degradation}
\end{figure}

\begin{figure}[htbp]
\centering
\includegraphics[width=0.6\textwidth]{./carbohydrate_metabolism/figures/glycogen_synth_deg.pdf}
\caption{\label{fig:org4bcd3b5}
Glycogen Synthesis and Degradation}
\end{figure}

\begin{figure}[htbp]
\centering
\includegraphics[width=0.5\textwidth]{./carbohydrate_metabolism/figures/glycogen_enzyme_reg.pdf}
\caption{\label{fig:org992b506}
Regulation of Glycogen Synthesis and Degradation}
\end{figure}

\begin{table}[htbp]
\caption{\label{tab:orgc99d629}
Liver glycogen regulation}
\centering
\begin{tabular}{lll}
state & regulators & response\\
\hline
Fasting & \(\uparrow\) glucagon & \(\uparrow\) degradation\\
 & \(\downarrow\) insulin & \\
 & \(\uparrow\) cAMP & \\
CHO meal & \(\downarrow\) glucagon & \(\uparrow\) synthesis\\
 & \(\uparrow\) insulin & \\
 & \(\downarrow\) cAMP & \\
exercise \& & \(\uparrow\) epinephrine & \(\uparrow\) degradation\\
stress & \(\uparrow\) cAMP & \\
\end{tabular}
\end{table}

\begin{table}[htbp]
\caption{\label{tab:org0ac51c5}
Muscle glycogen regulation}
\centering
\begin{tabular}{lll}
state & regulators & response\\
\hline
Fasting & \(\downarrow\) insulin & \(\uparrow\) degradation\\
(rest) &  & \(\downarrow\) gluc transport\\
 & \(\uparrow\) cAMP & \\
CHO meal & \(\uparrow\) insulin & \(\uparrow\) synthesis\\
(rest) &  & \(\uparrow\) gluc transport\\
 &  & \\
exercise & \(\uparrow\) epinephrine & \(\uparrow\) glycolysis\\
 & \(\uparrow\) cAMP & \(\downarrow\) synthesis\\
 & \(\downarrow\) AMP & \(\downarrow\) degradation\\
\end{tabular}
\end{table}

\subsection{Sugar Metabolism Pathways}
\label{sec:org94ed647}
\begin{itemize}
\item Fructose
\item Galactose
\item Pentose Phosphate Pathway
\end{itemize}

\begin{figure}[htbp]
\centering
\includegraphics[width=0.7\textwidth]{./carbohydrate_metabolism/figures/fruc_met.pdf}
\caption{\label{fig:orga878a3f}
Fructose Metabolism}
\end{figure}

\begin{figure}[htbp]
\centering
\includegraphics[width=0.5\textwidth]{./carbohydrate_metabolism/figures/gal_met.pdf}
\caption{\label{fig:orgb7a5cb5}
Galactose Metabolism}
\end{figure}

\subsubsection{Pentose Phosphate Pathway}
\label{sec:orgf2a7ed5}
\begin{enumerate}
\item Oxidative Phase
\label{sec:orgfe932a2}
\begin{itemize}
\item glucose 6-P \(\to\) NADPH + ribose 5-P
\item Glucose 6-P dehydrogenase catalyses first step
\item NADPH is for reducing reactions
\begin{itemize}
\item NADPH/NADP\(^{\text{+}}\) \textgreater{}\textgreater{}\textgreater{} NADH/NAD\(^{\text{+}}\)
\item NADH is rapidly converted to NAD\(^{\text{+}}\) in the ETC
\end{itemize}
\end{itemize}
\item Non-oxidative Phase
\label{sec:org2d13111}
\begin{itemize}
\item reversible rxns
\item convert glycolytic intermediates to 5 carbon sugars

\item Ribose-5-P required for purine and pyrimidine synthesis
\item NADPH required for detoxification and synthetic reaction
\begin{itemize}
\item Detoxification
\begin{itemize}
\item Reduction of oxidized glutathione
\item Cytochrome p450 monoxygenases
\end{itemize}
\item Synthetic reactions
\begin{itemize}
\item FA synthesis
\item Cholesterol
\item neurotransmitters
\item deoxynucleotide
\item superoxide
\end{itemize}
\end{itemize}
\end{itemize}
\end{enumerate}

\subsection{Synthesis}
\label{sec:org60c50a8}
\begin{itemize}
\item sugars are activated by addition of nucleotides
\item Uridine diphosphate (UDP)-glucose is a precusor of:
\begin{itemize}
\item glycogen, lactate, UDP-glucuronate
\item CHO chains in proteoglycans glycoproteins and glycolipids
\end{itemize}
\item The polyol pathway is a two-step process that converts glucose to
fructose
\begin{itemize}
\item glucose is reduced to sorbitol, which is oxidized to fructose
\end{itemize}
\end{itemize}

\begin{figure}[htbp]
\centering
\includegraphics[width=0.5\textwidth]{./carbohydrate_metabolism/figures/udp_glu.pdf}
\caption{\label{fig:orged79772}
UDP-glucose metabolism}
\end{figure}

\begin{figure}[htbp]
\centering
\includegraphics[width=0.5\textwidth]{./carbohydrate_metabolism/figures/udp_gln.pdf}
\caption{\label{fig:org629db97}
UDP-glucuronate metabolism}
\end{figure}

\begin{figure}[htbp]
\centering
\includegraphics[width=0.6\textwidth]{./carbohydrate_metabolism/figures/precusors.pdf}
\caption{\label{fig:org92f06d2}
Glucose precusors in gluconeogenesis}
\end{figure}


\begin{figure}[htbp]
\centering
\includegraphics[width=0.7\textwidth]{./carbohydrate_metabolism/figures/fasting.pdf}
\caption{\label{fig:org5d4ac2a}
Tissue interrelationships during fasting}
\end{figure}


\begin{figure}[htbp]
\centering
\includegraphics[width=0.7\textwidth]{./carbohydrate_metabolism/figures/fasting_changes.pdf}
\caption{\label{fig:orgaa7179e}
Changes in metabolic fuels during fasting}
\end{figure}

\section{Galactose Metabolism}
\label{sec:orgd524306}
\subsection{Introduction}
\label{sec:org20f1875}

\begin{itemize}
\item lactose is the principal carbohydrate in milk
\item dairy products are largest exogenous source of galactose
\item endogenous galactose production is also significant
\item lactose is hydrolysed by lactase \(\to\) galactose and glucose
\item galactose is then absorbed using a sodium/glucose-galactose cotransporter (SGLT1)
\item liver is the major organ responsible for galactose metabolism
\begin{itemize}
\item galactose functions:
\begin{itemize}
\item energy source in pre-weaning infants
\item glycosylation
\item glycolipid synthesis
\end{itemize}
\end{itemize}

\item UDPgal is used for synthesis of glycoconjugates, including
glycoproteins, glycolipids and glycosaminoglycans, and their
subsequent degradation
\item All four substrates of GALE: UDPgal, UDPglc, UDPgalNAc, and
UDPglcNAc, are used for glycoconjugate synthesis
\end{itemize}


\begin{figure}[htbp]
\centering
\includegraphics[width=1.0\textwidth]{./galactose/figures/galmet.png}
\caption{\label{fig:org4b5d58b}
Major reactions of galactose metabolism}
\end{figure}


\begin{figure}[htbp]
\centering
\includegraphics[width=1.0\textwidth]{./galactose/figures/hepcarb.png}
\caption{\label{fig:org665ce55}
Hepatic carbohydrate metabolism}
\end{figure}

\begin{figure}[htbp]
\centering
\includegraphics[width=0.4\textwidth]{./galactose/figures/Beta-D-Lactose.png}
\caption[lactose]{\label{fig:orgafada4c}
Lactose is a disaccharide derived from the condensation of galactose and glucose, which form a \(\beta\) 1 \(\to\) 4 glycosidic linkage.}
\end{figure}

\begin{itemize}
\item Three inborn errors of galactose metabolism are known:
\begin{enumerate}
\item classic galactosemia - GALT deficiency
\item GALE deficiency
\item GALK deficiency
\end{enumerate}
\end{itemize}

\subsection{Classical Galactosemia}
\label{sec:org51f7943}
\subsubsection{Clinical Presentation}
\label{sec:org1f7dd55}
\begin{itemize}
\item presents in the neonatal period
\begin{itemize}
\item potentially lethal
\end{itemize}
\item asymptomatic at birth
\item with feeding:
\begin{itemize}
\item poor weight gain, vomiting, diarrhea
\item hepatocellular damage, lethargy, and hypotonia
\end{itemize}
\item May progress to Gram negative sepsis, cataracts
\end{itemize}

\subsubsection{Metabolic Derangement}
\label{sec:org637062f}
\begin{itemize}
\item Caused by deficient activity of galactose-1-phosphate uridylyltransferase
\item ubiquitous enzyme
\item \(\uparrow\)  Galatactose
\item \(\uparrow\) Gal-1-p - pathogenic
\item \(\uparrow\)  galactitol \(\to\) cataracts
\item \(\uparrow\) galactonate
\item \(\downarrow\) UPD-Gal \(\to\) disordered glycosylation, glycolipids
\item \(\downarrow\) UPD-Glc
\end{itemize}

\subsubsection{Genetics}
\label{sec:org2210cd9}
\begin{itemize}
\item AR, GALT, 1:16,000-60,000 live births
\begin{itemize}
\item 336 mutations is the GALT gene described
\end{itemize}
\item c.563A>G, p.Q188R
\begin{itemize}
\item \textasciitilde{}64\% of galactosemic alleles in caucasian pop
\item Irish Travellers 1:430
\item Non-functional variant: Destabilise UMP-GALT
\begin{itemize}
\item no residual RBC GALT activity
\end{itemize}
\end{itemize}
\item c.855G>T, p.K285N
\begin{itemize}
\item slavic origin?
\item no residual RBC GALT activity
\end{itemize}
\end{itemize}
\begin{enumerate}
\item Duarte and Los Angles variants
\label{sec:org84aba4d}
\begin{itemize}
\item Duarte (p.N314D) - 50\% RBC GALT enzyme activity
\item LA (p.N314D) - elevated RBC GALT enzyme activity
\item Same electrophoretic pattern
\item D is in linkage disequilibrium w a 4bp promoter deletion
\item p.D314 is the ancestral allele
\item p.N314 arose early in human evolution
\end{itemize}
\end{enumerate}


\subsubsection{Diagnostic Tests}
\label{sec:org32a6618}
\begin{itemize}
\item reducing substances in urine - not specific or sensitive
\item Gal-1-P, galactose, galactitol in blood or urine
\item RBC Gal-1-P - not specific
\item RBC GALT activity
\begin{itemize}
\item Classic galactosemia - undetectable or 1\% of controls
\end{itemize}
\item NBS not universal
\begin{itemize}
\item must be identified early
\item \textless{} 5 days is ideal
\end{itemize}
\end{itemize}

\subsubsection{Treatment and Prognosis}
\label{sec:org07cbeab}
\begin{itemize}
\item Therapy is life long dietary galactose restriction
\begin{itemize}
\item does not prevent long-term complications
\end{itemize}

\item Endogenous galactose synthesis may be responsible for:
\begin{itemize}
\item Cognitive impairment
\item Ovarian insufficiency
\end{itemize}
\item Dairy restrictions
\begin{itemize}
\item Bone health
\end{itemize}
\end{itemize}

\subsection{Uridine Diphosphate Galactose 4’-Epimerase (GALE) Deficiency}
\label{sec:orgbb139b4}
\subsubsection{Clinical Presentation}
\label{sec:org5e58cf4}
\begin{itemize}
\item GALE deficiency ranges from benign \(\to\) severe:
\begin{description}
\item[{benign}] GALE deficiency restricted to circulating red and white blood cells
\item[{severe}] generalized disorder resulting from widespread GALE
impairment that presents with life-threatening illness
in the newborn period
\end{description}

\item unlike GALT deficiency, even the most severely affected patients
with GALE deficiency exhibit some residual GALE activity, at least
in some tissues.
\item severe form is extremely rare
\end{itemize}

\subsubsection{Metabolic Derangement}
\label{sec:orgffade59}
\begin{itemize}
\item When exposed to milk accumulate galactose, galactitol, Gal-1P, and
UDPgal in blood
\begin{itemize}
\item may also show abnormal glycosylation of proteins in blood
\end{itemize}
\item red cell GALE activity does not correlate well with that seen in
other tissues, such as lymphoblasts,
\begin{itemize}
\item is poor at differentiating between peripheral and generalised forms of the disease
\end{itemize}
\end{itemize}

\subsubsection{Genetics}
\label{sec:org4080c86}
\begin{itemize}
\item AR, GALE
\end{itemize}

\subsubsection{Diagnostic Tests}
\label{sec:org83c343e}
\begin{itemize}
\item may be detected by NBS with \(\uparrow\) total galactose or Gal-1P and normal GALT activity
\item GALE activity in fresh erythrocytes or other cells
\item GALE activity in transformed lymphoblasts
\item red cell Gal-1P or urinary galactitol measured while on and off dietary galactose
\item molecular
\end{itemize}

\subsubsection{Treatment}
\label{sec:orgcbfe931}
\begin{itemize}
\item Low galactose formula until the diagnosis can be confirmed or excluded.
\item Once confirmed, patients with generalized GALE deficiency should be
treated and followed much like patients with classic galactosemia,
\item less stringent dietary galactose restriction may be advisable to
ensure sufficient exogenous galactose for synthesis of galactoproteins and galactolipids.
\end{itemize}
\subsection{Galactokinase (GALK) Deficiency}
\label{sec:org9596032}
\subsubsection{Clinical Presentation}
\label{sec:org452a488}
\begin{itemize}
\item untreated galactokinase deficiency has been considered largely
benign except for diet-dependent cataracts and in rare cases pseudotumour cerebri
\item Symptoms in severe deficiency may include:
\begin{itemize}
\item hypoglycaemia, failure to thrive, microcephaly, intellectual
disability, and hypercholesterolemia
\end{itemize}
\end{itemize}

\subsubsection{Metabolic Derangement}
\label{sec:orgae3bddb}
\begin{itemize}
\item lack the ability to phosphorylate galactose
\item accumulate galactose and galactitol, but not Gal-1P
\item accumulate galactitol in the lens when consuming a high galactose diet
\item causing osmotic swelling, denaturation of proteins, and cataracts
\end{itemize}

\subsubsection{Genetics}
\label{sec:orga5b8e30}
\begin{itemize}
\item AR, 1/40,000 screened births in Germany
\item all mutations associated with clinical GALK deficiency map to GK1.
\end{itemize}

\subsubsection{Diagnostic Tests}
\label{sec:org2e7de43}
\begin{itemize}
\item profound GALK deficiency may be discovered by NBS due to elevated total blood galactose
\item enzyme assay of freshly drawn red cells or another cell type
\item Elevated galactose and galactitol may also be detected in urine if
the patient is on a high galactose diet
\end{itemize}

\subsubsection{Treatment}
\label{sec:org9e84035}
\begin{itemize}
\item Initial treatment of GALK deficiency involves elimination of milk and other high galactose foods from the diet.
\item Cryptic sources of dietary galactose, such as fruits and vegetables, are generally allowed.
\item Once a patient is on a galactose-restricted diet urinary levels of galactitol should normalize
\end{itemize}

\section{Hepatic Glycogenoses}
\label{sec:orga9c23fb}
\subsection{Introduction}
\label{sec:org283bebe}
\begin{itemize}
\item hepatic glycogenosis generally cause hepatomegaly (apart from GSD
Oa) and fasting hypoglycaemia
\item muscle disorders are associated with skeletal and/or
cardiomyopathy
\item The clinical phenotypes are extremely heterogeneous
\end{itemize}
\begin{table}[htbp]
\caption{\label{tab:orgd7dc973}
Hepatic Glycogenoses}
\centering
\begin{tabular}{llll}
Type & Enzyme & Gene & Phenotype\\
\hline
0a & liver glycogen synthase & GYS2 & ketotic hypoglycema\\
Ia & G6Pase \(\alpha\) & GSPC & hypoglycema, hepatomegaly, lactic acidosis\\
Ib & G6P transporter & SLC37A4 & Ia + neutrophil dysfunction, colitis\\
III & Glycogen debrancher & AGL & hypoglycema, hepatomegaly\\
IV & Glycogen brancher & GBE1 & cirrhosis, cardiomyopathy \& myopathy\\
 &  &  & adult polyglucosan body disease\\
VI & Liver glycogen phosphorylase & PYGL & hypoglycema, hepatomegaly, growth delay\\
IXa & phosphorylase kinase \(\alpha\)2 & PHKA2 & hypoglycema, hepatomegaly\\
IXb & phosphorylase kinase \(\beta\) & PHKB & \\
IXc & phosphorylase kinase \(\gamma\)TL & PHKG2 & \\
IXd & phosphorylase kinase \(\alpha\)1 & PHKA1 & myopathy\\
\end{tabular}
\end{table}

\begin{figure}[htbp]
\centering
\includegraphics[width=1\textwidth]{./hepatic_glycogenoses/figures/gggmetab_hepatic.png}
\caption[Hepatic Glycogenoses]{\label{fig:orgc878ad5}
Hepatic Glycogenoses}
\end{figure}

\subsection{GSD Type 0}
\label{sec:org57f465a}
\subsubsection{Metabolic Derangement}
\label{sec:org4170986}
\begin{itemize}
\item Deficiency in hepatic glycogen synthase (GS2)
\item Very low liver glycogen
\item Fasting associated with ketosis
\item Post-prandial hyperglycemia with moderate hyperlactatemia
\begin{itemize}
\item reduced liver uptake of glucose
\end{itemize}
\end{itemize}

\subsubsection{Genetics}
\label{sec:org1e480e0}
\begin{itemize}
\item AR, GYS2 encodes liver isoform of GS2
\end{itemize}

\subsubsection{Clinical Presentation}
\label{sec:org264cedb}
\begin{itemize}
\item ketotic hypoglycema with post-prandial hyperglycemia/uria
\item can have poor growth
\item no hepatomegaly
\end{itemize}

\subsubsection{Diagnostic Tests}
\label{sec:org29ba3df}
\begin{itemize}
\item mutation analysis
\end{itemize}
\subsubsection{Treatment}
\label{sec:org426142b}
\begin{itemize}
\item avoid fasting
\item uncooked cornstarch prior to overnight fast and during infections
\end{itemize}
\subsection{GSD Type I}
\label{sec:org475f3b3}
\subsubsection{Metabolic Derangement}
\label{sec:orgf7023fa}
\begin{itemize}
\item GSD Ia, glucose 6 phosphatase-\(\alpha\)
\item GSD Ib, glucose 6 phosphate transporter
\item Disorder of glycogen metabolism and gluconeogenesis
\item Failure of glucose dephosphorylation inhibits hepatic glycogen breakdown
\item Hyperlactatemia occurs due to lack of gluconeogenesis
\begin{itemize}
\item protective
\end{itemize}
\item hyperlipidemia and hyperuricemia due to \(\uparrow\) G6P
\begin{itemize}
\item \(\uparrow\) G6P \(\to\) \emph{de novo} lipogenesis and flux through pentose phosphate pathway
\end{itemize}
\item G6P transporter required for normal neutrophil function
\end{itemize}

\subsubsection{Genetics}
\label{sec:orga425574}
\begin{itemize}
\item AR, 1:100,000, 80\% Ia
\item GSD Ia: G6PC
\item GSD Ib: SLC37A4
\item no genotype phenotype correlation established
\end{itemize}

\subsubsection{Clinical Presentation}
\label{sec:orga5db06c}
\begin{enumerate}
\item Ia and Ib
\label{sec:org660db38}
\begin{itemize}
\item severe fasting hypoglycema, lactic acidosis
\item hepatomegaly
\item hyperlipidemia, hyperuricemia
\end{itemize}
\item Ib
\label{sec:orgdf450a3}
\begin{itemize}
\item neutrophil dysfunction
\item increased infections
\end{itemize}
\end{enumerate}

\subsubsection{Diagnostic Tests}
\label{sec:org757312d}
\begin{itemize}
\item mutation analysis
\end{itemize}

\subsubsection{Treatment}
\label{sec:org009920f}
\begin{itemize}
\item generally fatal if untreated
\item diet
\item liver transplant
\item treatment of sequelae
\begin{itemize}
\item hepatic tumors
\item GI disease - IBD in GSD Ib
\item renal disease - glycogen deposition
\item hematological disease
\begin{itemize}
\item anemia
\item coagulopathy
\item infections, GSD Ib
\end{itemize}
\item cardiovascular disease
\item bone disease
\end{itemize}
\end{itemize}

\subsection{GSD Type III}
\label{sec:org20d225d}
\subsubsection{Metabolic Derangement}
\label{sec:orgc7bdf9e}
\begin{itemize}
\item Glycogen debrancher enzyme (GDE) deficiency
\item has both glucosidase and transferase activity
\begin{itemize}
\item cleaves \(\alpha\)-1,4 glucose linkages of the terminal glucose
\item then breaks \(\alpha\)-1,6 linkage to remove branch point
\end{itemize}
\item accumulation of abnormal glycogen
\item limited glucose release from glycogen
\item gluconeogenesis functions normally
\end{itemize}
\subsubsection{Genetics}
\label{sec:orgae323cd}
\begin{itemize}
\item AR, AGL gene
\item mutations occur throughout AGL (GSD IIIa)
\begin{itemize}
\item defect in liver and muscle
\end{itemize}
\item two specific mutations in exon 3 (GSD IIIb)
\begin{itemize}
\item liver only
\end{itemize}
\end{itemize}
\subsubsection{Clinical Presentation}
\label{sec:org59651a8}
\begin{itemize}
\item Hepatic glycogenosis and (in most cases) also myopathic
\item First year with poor growth, delayed motor milestones and abdominal
distension
\item Fasting hypoglycaemia 
\begin{itemize}
\item Fasting tolerance is usually longer than in GSD I
\end{itemize}
\item Fasting ketosis is prominent
\item Gluconeogenesis is normal \(\therefore\) no fasting hyperlactataemia
\item Moderate post-prandial \(\uparrow\) lactate
\item Hyperlipdaemia
\item \(\uparrow\) \(\uparrow\) \(\uparrow\) liver transaminases
\item \(\uparrow\) CK in myopathic form
\end{itemize}
\subsubsection{Diagnostic Tests}
\label{sec:org839fa78}
\begin{itemize}
\item DBE activity in leucocytes
\item mutation analysis
\end{itemize}
\subsubsection{Treatment}
\label{sec:org1e255ec}
\begin{itemize}
\item Aim is to maintain normoglycaemia, reduce the hyperlipidaemia and ketosis and
ensure adequate growth
\item Regular meals and uncooked cornstarch
\item Overnight continuous feeding is less commonly needed in GSD III than
in GSD I
\item Long term outcome for individuals with GSD III is generally good
with survival into adulthood
\end{itemize}
\subsection{GSD Type IV}
\label{sec:org3c9ca7f}
\subsubsection{Metabolic Derangement}
\label{sec:org6db0171}
\begin{itemize}
\item GSD IV is caused by deficiency in glycogen brancher enzyme (GBE)
\item GBE transfers short glucosyl chains to form branch points with an
\(\alpha\)-1,6 linkage
\item Deficiency results in an abnormal poorly soluble glycogen with fewer branch points (polyglucosan)
\item This abnormal glycogen accumulates in liver, muscle, heart, nervous system and skin
\begin{itemize}
\item leads to tissue damage
\end{itemize}
\end{itemize}

\subsubsection{Genetics}
\label{sec:orgf49cfbf}
\begin{itemize}
\item AR, GBE1
\item Common mutation in Ashkenazi Jewish pop
\begin{itemize}
\item adult polyglucosan body disease (APBD)
\end{itemize}
\end{itemize}

\subsubsection{Clinical Presentation}
\label{sec:org9215a89}

\begin{itemize}
\item Multiple phenotypes associated with GBE deficiency
\begin{itemize}
\item Ranges from death in utero to adult presentation
\end{itemize}
\end{itemize}

\begin{enumerate}
\item Liver Disease
\label{sec:org23d1ee3}
\begin{itemize}
\item Progressive liver disease in infancy
\begin{itemize}
\item Presents in first months of life with:
\begin{itemize}
\item Failure to thrive and hepatomegaly
\end{itemize}
\item Cirrhosis develops with eventual end stage liver disease and
portal hypertension
\item Death is usual by 5 years of age
\end{itemize}
\item Non-progressive liver disease in childhood.
\begin{itemize}
\item Present with hepatomegaly, liver dysfunction, hypotonia and
myopathy
\item Liver disease does not progress, survival into adulthood
\end{itemize}
\end{itemize}

\item Neuromuscular Disease
\label{sec:orgfb25e2d}
\begin{itemize}
\item Congenital onset
\begin{itemize}
\item fetal loss in pregnancy
\item fetal akinesia deformation sequence (FADS) with athrogryposis, hydrops and perinatal death
\item severe congenital myopathy similar to SMA with \textpm{} cardiomyopathy
\end{itemize}
\item Juvenile onset
\begin{itemize}
\item with a myopathy and/or cardiomyopathy
\end{itemize}
\item Adult onset
\begin{itemize}
\item adult polyglucosan body disease (APBD)
\item rarely myopathy
\end{itemize}
\end{itemize}
\end{enumerate}

\subsubsection{Diagnostic Tests}
\label{sec:orgc160545}

\begin{itemize}
\item \(\uparrow\) transaminases in those with hepatic involvement
\item Fasting hypoglycaemia is uncommon except in end stage liver failure
\item Liver and muscle histology show swollen hepatocytes that contain
periodic acid-Schiff (PAS)-positive and diastase resistance
inclusions and evidence of interstitial fibrosis
\item Enzyme analysis can be undertaken in liver tissue, cultured skin
fibroblast, peripheral lymphocytes and muscle
\item Confirmed by GBE1 mutation analysis
\end{itemize}

\subsubsection{Treatment}
\label{sec:orga8f46d9}
\begin{itemize}
\item Liver transplant is the only treatment for the progressive liver form
\item Heart transplant may be considered in those with heart failure caused by cardiomyopathy
\item There is no specific treatment for the other forms of the disease
\end{itemize}
\subsection{GSD Type VI}
\label{sec:org637a3f4}
\subsubsection{Metabolic Derangement}
\label{sec:org4dbfe03}
\begin{itemize}
\item GSD VI is caused by deficiency in hepatic glycogen phosphorylase
\item Catalyses the release and phosphorylation of terminal glucosyl units
from glycogen forming glucose-1-phosphate
\item Ketosis with or without hypoglycaemia may occur with fasting
\item Although plasma lipids may be raised
\item In severe variants recurrent hypoglycaemia and post-prandial lactic
acidosis can occur
\end{itemize}
\subsubsection{Genetics}
\label{sec:org51aa811}
\begin{itemize}
\item AR, PGYL gene
\end{itemize}

\subsubsection{Clinical Presentation}
\label{sec:org7713e2b}
\begin{itemize}
\item GSD VI is generally a mild disorder often diagnosed due to hepatomegaly
\begin{itemize}
\item can present with symptomatic ketotic hypoglycaemia and growth retardation
\end{itemize}
\end{itemize}
\subsubsection{Diagnostic Tests}
\label{sec:org95275f3}
\begin{itemize}
\item Diagnosis confirmed by mutation analysis or
\item Enzyme deficiency in hepatic tissue, erythrocytes, and leukocytes
\item Enzyme activity may not always be reduced in blood and even in liver
tissue may be difficult to interpret due to residual activity and
the effect of other factors
\item For example, deficiency of glycogen phosphorylase kinase will cause
low activity of glycogen phosphorylase
\end{itemize}
\subsubsection{Treatment}
\label{sec:org1d5eaa7}
\begin{itemize}
\item No treatment required for asymptomatic children
\item Those with growth failure or fasting ketosis benefit from regular
meals,snacks and uncooked cornstarch
\item The outcome for individuals with GSD VI is generally excellent
\begin{itemize}
\item Catch up growth occurring for those with short stature in childhood
\end{itemize}
\end{itemize}
\subsection{GSD Type IX}
\label{sec:org57ab4e3}
\subsubsection{Metabolic Derangement}
\label{sec:orgeb8d90f}
\begin{itemize}
\item GSD IX is caused by deficiency in hepatic glycogen phosphorylase kinase (PHK)
\item PHK phosphorylates glycogen phosphatase \emph{b} \(\to\) \emph{a} form
\begin{itemize}
\item inactive \emph{b} \(\to\)  active \emph{a}
\end{itemize}
\item \(\downarrow\) PHK activity \(\to\) \(\downarrow\) G1P release from glycogen
\item PHK is homotetramer in which each subunit is itself a tetramer
\begin{itemize}
\item \(\alpha\), \(\beta\), \(\gamma\) and \(\delta\) subunits.
\end{itemize}
\item The \(\gamma\) subunit is catalytic and the other subunits regulatory
\item There are tissue specific isoforms of the \(\alpha\) and \(\gamma\) subunits.
\item The \(\delta\) subunit, calmodulin, is ubiquitous
\end{itemize}

\subsubsection{Genetics}
\label{sec:orgb5274c1}
\begin{table}[htbp]
\caption{\label{tab:org1d5806a}
GSD Type IX Genetics}
\centering
\begin{tabular}{lllll}
Type & Gene & Subunit & Inheritance & Tissue\\
\hline
IXa & PHKA2 & \(\alpha\)2 & XLR & liver \& blood\\
IXb & PHKB & \(\beta\) & AR & liver \& muscle\\
IXc & PHKG2 & \(\gamma\)TL & AR & live\\
IXd & PHKA1 & \(\alpha\)1 & AR & muscle\\
\end{tabular}
\end{table}

\subsubsection{Clinical Presentation}
\label{sec:orgbfe581b}
\begin{itemize}
\item Usually a benign disorder, with hepatomegaly often detected
incidentally
\item possible short stature, fasting hypoglycaemia and ketosis, with
raised liver transaminases, cholesterol and triglycerides.
\item Blood lactate and uric acid are normal. There is usually resolution
of signs and symptoms by adulthood.
\item GSD IXc can be more severe with an increased risk of hepatic fibrosis and cirrhosis
\end{itemize}

\subsubsection{Diagnostic Tests}
\label{sec:org37de06c}
\begin{itemize}
\item Considered in children with unexplained hepatomegaly and in those with ketotic hypoglycaemia.
\item PHK can be measured in liver, erythrocytes and leukocytes.
\item However, in view of variable tissue expression enzyme assays may be
difficult to interpret.
\item Diagnosis is best achieved by mutation analysis using a DNA panel.
\end{itemize}
\subsubsection{Treatment}
\label{sec:org2c5aa38}
\begin{itemize}
\item Asymptomatic patients may not need treatment.
\item growth failure or symptomatic hypoglycaemia frequent meals and
uncooked cornstarch may be used.
\item Protein can be increased to 15 to 20\% of calories to provide a gluconeogenesis substrate.
\item The outcome for most patients is good with resolution of
hepatomegaly and catch up growth by adulthood.
\end{itemize}

\section{Muscle Cardiac Glycogenoses}
\label{sec:orgbcc181a}
\subsection{Introduction}
\label{sec:org894ceb9}
\begin{itemize}
\item Muscle predominately utilizes fatty acids for energy when at rest
\begin{itemize}
\item uses blood glucose during sub-maximal exercise.
\end{itemize}
\item Energy is supplied by glycogenolysis during intense exercise.
\begin{itemize}
\item glycogenolysis supplies glucose for anaerobic glycolysis
\end{itemize}
\item hepatic glycogenosis generally cause hepatomegaly (apart from GSD
Oa) and fasting hypoglycaemia
\item muscle disorders are associated with skeletal and/or
cardiomyopathy
\item The clinical phenotypes are extremely heterogeneous
\end{itemize}

\begin{table}[htbp]
\caption{\label{tab:org40e5145}
Muscle and Cardiac Glycogenoses}
\centering
\begin{tabular}{llll}
Type & Enzyme & Gene & Phenotype\\
\hline
0 & Muscle Glycogen Synthase & GYS1 & cardiomyopathy \& myopathy\\
II (Pompe) & Acid \(\alpha\)-Glucosidase & GAA & hypotonia, muscle dysfunction\\
III & Glycogen Debrancher & AGL & hypoglycema, hepatomegaly\\
V (McArdle) & Muscle Glycogen Phosphorylase & PYGM & exercise intolerance\\
Danon & LAMP2 & LAMP2 & cardiomyopathy, \textpm{} skeletal myopathy\\
AMPK & AMP-Activated Protein Kinase & PRKAG2 & cardiomyopathy, \textpm{} skeletal myopathy\\
\end{tabular}
\end{table}

\begin{figure}[htbp]
\centering
\includegraphics[width=1\textwidth]{./muscle_cardiac_glycogenoses/figures/gggmetab_muscle_cardiac.png}
\caption[Muscle and Cardiac Glycogenoses]{\label{fig:org029b411}
Muscle and Cardiac Glycogenoses}
\end{figure}

\subsection{GSD Type 0}
\label{sec:orge48ddc7}
\subsubsection{Metabolic Derangement}
\label{sec:org7a3c55e}
\begin{itemize}
\item Deficiency in muscle glycogen synthase
\item Ubiquitously expressed
\end{itemize}

\subsubsection{Genetics}
\label{sec:org5d1a7a8}
\begin{itemize}
\item AR, GYS1 encodes muscle isoform of GS
\end{itemize}

\subsubsection{Clinical Presentation}
\label{sec:org020b3d9}
\begin{itemize}
\item muscle fatigue
\item hypertrophic cardiomyopathy
\end{itemize}

\subsubsection{Diagnostic Tests}
\label{sec:org708d205}
\begin{itemize}
\item mutation analysis
\end{itemize}
\subsubsection{Treatment}
\label{sec:orgbb5270b}
\begin{itemize}
\item \(\beta_{\text{1}}\)-receptor blockage for cardiac protection
\end{itemize}

\subsection{GSD Type II (Pompe)}
\label{sec:org902570a}
\subsubsection{Metabolic Derangement}
\label{sec:org200cdab}
\begin{itemize}
\item Acid \(\alpha\)-glucosidase deficiency
\item accumulation of glycogen within the lysosomes
\item different critical thresholds depending on the organ
\end{itemize}

\subsubsection{Genetics}
\label{sec:org00c32e1}
\begin{itemize}
\item AR, AGL gene
\item 200 mutations have been reported in GAA
\begin{itemize}
\item \textasciitilde{} 75\% of these being pathogenic mutations (www.pompecenter.nl)
\end{itemize}
\item some genotype-phenotype correlation
\begin{itemize}
\item severe mutations (such as del exon18) associated with the infantile form
\item ›leaky‹ mutations associated with the adult variant
\end{itemize}
\item c.-32-13T>G is the most common mutation in adults and children with
a slowly progressive course
\begin{itemize}
\item approximately 80\% of Caucasian patients
\end{itemize}
\end{itemize}

\subsubsection{Clinical Presentation}
\label{sec:orgdf90b8e}
\begin{enumerate}
\item Infantile
\label{sec:orga72c2a5}
\begin{itemize}
\item first months of life with hypotonia and hypertrophic cardiomyopathy
\item also dysphagia, smooth muscle dysfunction, enlargement of the tongue
and liver
\item Most untreated infantile onset patients die from cardiopulmonary
failure or aspiration pneumonia prior to one year of age
\end{itemize}
\item Juvenile
\label{sec:orgaeb5eb9}
\begin{itemize}
\item predominant skeletal muscle dysfunction
\begin{itemize}
\item with motor and respiratory problems, rarely cardiac involvement
\item Calf hypertrophy can be present, mimicking Duchenne muscular dystrophy in boys
\end{itemize}
\item Myopathy and respiratory insufficiency deteriorate gradually, and patients may become dependent on a ventilator or wheelchair
\end{itemize}
\item Adult
\label{sec:orgccf1906}
\begin{itemize}
\item 3rd or 4th decade and affects the trunk and proximal limb muscles
\begin{itemize}
\item mimics inherited limb-girdle muscle dystrophies
\end{itemize}
\item Involvement of the diaphragm is frequent
\begin{itemize}
\item acute respiratory failure may be the initial symptom in some patients
\end{itemize}
\item the heart is generally not affected
\end{itemize}
\end{enumerate}
\subsubsection{Diagnostic Tests}
\label{sec:orga0d3ff3}
\begin{itemize}
\item Acid \(\alpha\)-glucosidase enzyme assay
\begin{itemize}
\item classic infantile \textasciitilde{} 1\% residual activity
\item Children and Adults \(\le\) 30\% activity
\end{itemize}
\item Skin fibroblasts are best tissue
\begin{itemize}
\item Lower biochemical interference (neutral \(\alpha\)-glucosidases)
\end{itemize}
\item mutation analysis
\end{itemize}
\subsubsection{Treatment}
\label{sec:org3f862dd}
\begin{itemize}
\item Recombinant acid \(\alpha\)-glucosidase (rhGAA)
\begin{itemize}
\item CHO cells (alglucosidase alfa)
\end{itemize}
\item Anti rhGAA IgG antibodies form
\item 1/3 of ERT treated were ventilator free
\item Better outcome if identified by NBS
\end{itemize}

\subsection{GSD Type V}
\label{sec:orgd1ae756}
\subsubsection{Metabolic Derangement}
\label{sec:orgf437faa}
\begin{itemize}
\item GSD V is caused by deficiency in muscle glycogen phosphorylase
\item Catalyses the release and phosphorylation of terminal glucosyl units
from glycogen forming glucose-1-phosphate
\item There are three isoforms of glycogen phosphorylase: brain/heart
liver and muscle, all encoded by different genes
\item GSD V is caused by deficient myophosphorylase activity
\end{itemize}

\subsubsection{Genetics}
\label{sec:org933337b}
\begin{itemize}
\item AR, PYGM
\item \textgreater{} 100 known pathogenic mutations
\item p.R50X mutation, most common in Caucasians
\begin{itemize}
\item 81\% of the alleles in British patients
\item 63\% of alleles in US patients
\end{itemize}
\item No genotype-phenotype correlations have been detected
\item ACE polymorphism may be a phenotype modulator
\end{itemize}

\subsubsection{Clinical Presentation}
\label{sec:orgcf0423d}
\begin{itemize}
\item exercise intolerance with myalgia and stiffness in exercising muscles
\begin{itemize}
\item relieved by rest
\end{itemize}
\item Onset of the disease occurs during childhood
\begin{itemize}
\item diagnosis is frequently missed at an early age
\item affected children are often considered lazy
\end{itemize}
\item Myoglobinuria is the major complication, and occurs in about half of
the patients
\item Creatine kinase (CK) can increase to more than 100,000-1,000,000
UI/l during episodes of rhabdomyolysis
\item Risk of acute renal failure
\end{itemize}

\subsubsection{Diagnostic Tests}
\label{sec:org2b5c453}

\begin{itemize}
\item Ischaemic forearm exercise test (IFET) was first used by McArdle to
describe the absence of elevation of lactate during exercise
\begin{itemize}
\item \textbf{Should not be used}
\end{itemize}
\item Non-ischemic FET has a sensitivity of 100\% in McArdle’s disease
\item Ammonia levels should be also assessed in parallel with lactate
\begin{itemize}
\item an abnormal increase in ammonia always observed in GSD V
\end{itemize}
\item PYGM gene sequencing
\end{itemize}

\subsubsection{Treatment}
\label{sec:orgf427734}
\begin{itemize}
\item no pharmacological treatment
\item exercise intolerance may be alleviated by:
\begin{itemize}
\item aerobic conditioning programs
\item ingestion of oral sucrose
\end{itemize}
\end{itemize}
\subsection{LAMP 2 Deficiency (Danon Disease)}
\label{sec:org8071e06}
\begin{itemize}
\item Danon disease is a rare X-linked disorder
\item caused by a primary deficiency of lysosomal-associated membrane
protein 2 (LAMP2)
\item Presents after 1st decade
\begin{itemize}
\item cardiomyopathy all cases
\item mild skeletal myopathy and developmental delay 70\%
\end{itemize}
\item muscle biopsy shows glycogen filled vacuoles
\item consider cardiac transplantation
\end{itemize}

\subsection{AMPK Deficiency}
\label{sec:orgec66a3d}
\begin{itemize}
\item AMPK controls whole-body glucose homeostasis by regulating metabolism in multiple peripheral tissues, such as
skeletal muscle, liver, adipose tissues, and pancreatic \(\beta\)-cells
\item activated \(\uparrow\) AMP/ATP ratio
\item stimulates glucose uptake and lipid oxidation to produce energy
\item inhibits energy-consuming processes including glucose and lipid production.
\end{itemize}

\subsubsection{Metabolic Derangement}
\label{sec:org8e7c52f}
\begin{itemize}
\item AMPK is a heterotrimeric complex comprising:
\begin{itemize}
\item a catalytic subunit (\(\alpha\))
\item two regulatory subunits (\(\beta\) and \(\gamma\))
\end{itemize}
\item Three isoforms of the gamma subunits are known (\(\gamma\)1, \(\gamma\)2 and \(\gamma\)3) with different tissue
expression
\end{itemize}

\subsubsection{Genetics}
\label{sec:org7b00026}
\begin{itemize}
\item The PRKAG2 gene coding for the \(\gamma\)-subunit of AMPK is located on chromosome 7q36
\item Mutations in the \(\gamma\)2-subunit of AMPK are transmitted as an
autosomal dominant trait with full penetrance
\end{itemize}

\subsubsection{Diagnosis \& Treatment}
\label{sec:orgb18c611}
\begin{itemize}
\item The differential diagnosis includes Pompe, Danon (LAMP2) and Fabry diseases
\item diagnosis, if clinically suspected, is based on ECG,
echocardiography and molecular genetics
\item Treatment includes a pacemaker/defibrillator and heart transplant
\end{itemize}

\section{Glycolysis and PPP}
\label{sec:orga17675e}
\subsection{Introduction}
\label{sec:orge50fa9e}
\begin{figure}[htbp]
\centering
\includegraphics[width=1.0\textwidth]{./glycolysis_ppp/figures/glyc_ppp.png}
\caption{\label{fig:orgb5276a7}
Glycolysis and PPP}
\end{figure}

\subsubsection{Glycolysis}
\label{sec:orge86abca}
\begin{itemize}
\item Glycolysis is an oxygen-independent metabolic pathway
\begin{itemize}
\item converts each molecule of glucose to two of pyruvate.
\item consists of two phases and ten steps.
\end{itemize}
\item The first five steps are the preparatory phase,
\begin{itemize}
\item consumes ATP to convert glucose into two, three-carbon sugar
phosphate molecules.
\end{itemize}
\item In the other five steps ATP and NADH are produced
\end{itemize}

\subsubsection{Pentose Phosphate Pathway}
\label{sec:orga61cb79}
\begin{itemize}
\item Two phases
\end{itemize}
\begin{enumerate}
\item Oxidative Phase
\label{sec:org56103e0}
\begin{itemize}
\item glucose 6-P \(\to\) NADPH + ribose 5-P
\item Glucose 6-P dehydrogenase catalyses first step
\item NADPH is for reducing reactions
\begin{itemize}
\item NADPH/NADP\(^{\text{+}}\) \textgreater{}\textgreater{}\textgreater{} NADH/NAD\(^{\text{+}}\)
\item NADH is rapidly converted to NAD\(^{\text{+}}\) in the ETC
\end{itemize}
\end{itemize}
\item Non-oxidative Phase
\label{sec:org3365fa7}
\begin{itemize}
\item reversible rxns
\item convert glycolytic intermediates to 5 carbon sugars

\item Ribose-5-P required for purine and pyrimidine synthesis
\item NADPH required for detoxification and synthetic reaction
\begin{itemize}
\item Detoxification
\begin{itemize}
\item Reduction of oxidized glutathione
\item Cytochrome p450 monoxygenases
\end{itemize}
\item Synthetic reactions
\begin{itemize}
\item FA synthesis
\item Cholesterol
\item neurotransmitters
\item deoxynucleotide
\item superoxide
\end{itemize}
\end{itemize}
\end{itemize}
\end{enumerate}
\subsection{Disorders}
\label{sec:org5c6b7fe}
\subsubsection{Glycolysis Disorders}
\label{sec:org710c488}
\begin{itemize}
\item Glycolysis is the most important source of energy in erythrocytes
and in some types of skeletal muscle fibres

\begin{itemize}
\item \textbf{\(\therefore\) IMD of glycolysis are mainly characterized by hemolytic}
\textbf{anaemia \textpm{} metabolic myopathy}
\end{itemize}

\item Ten inborn errors of the glycolytic pathway are known,
\begin{itemize}
\item 8 inherited as an autosomal recessive trait
\item X-linked phosphoglycerate kinase and glycerol kinase deficiencies
\end{itemize}

\item Hexokinase (HK), glucose-6-phosphate isomerase (GPI) and pyruvate
kinase (PKD) deficiencies cause severe haemolytic anaemia

\item Muscle phosphofructokinase (PFKM), aldolase A, triosephosphate
isomerase (TPI) and phosphoglycerate kinase (PGK) deficiencies are
characterized by haemolytic anaemia alone or coupled with
neurological disease and/or myopathy

\item Phosphoglycerate mutase (PGAM), enolase and lactate dehydrogenase
(LDH) deficiencies present with a purely myopathic syndrome
\begin{itemize}
\item characterized by exercise induced cramps and myoglobinuria
\end{itemize}

\item Glycerol kinase deficiency (GKD) is either an isolated condition
with hypoglycaemia and acidosis or part of a contiguous
gene deletion where it also associated with congenital adrenal
hypoplasia and/or Duchenne muscular dystrophy

\item Glucose-6-phosphate can also be formed by the conversion of the
glycogen derived glucose-1-phosphate

\begin{itemize}
\item a reaction catalysed by
phosphoglucomutase (PGM). PGM1 deficiency is a CDG
\end{itemize}
\end{itemize}

\subsubsection{PPP Disorders}
\label{sec:org60864ff}

\begin{itemize}
\item Four inborn errors in the pentose phosphate pathway (PPP) are known.
\item Glucose-6-phosphate dehydrogenase deficiency is an X-linked defect
in the first, irreversible step of the pathway.
\begin{itemize}
\item An exclusively haematological disorder.
\end{itemize}
\item Ribose-5-phosphate isomerase (RPI) deficiency has been described in one patient
\begin{itemize}
\item presented with developmental delay and a slowly progressive leukoencephalopathy.
\end{itemize}
\item Transaldolase (TALDO) deficiency often presents in the neonatal or
antenatal period with hepatosplenomegaly, \(\downarrow\) liver function,
hepatic fibrosis and anaemia.
\item Transketolase (TKT) deficiency presents with short stature,
developmental delay and congenital heart defects.
\end{itemize}

\subsubsection{PPP Disorders}
\label{sec:org029142a}
\begin{itemize}
\item Sedoheptulokinase (SHPK) deficiency, a defect related to the PPP has
been described as an isolated disorder and also as part of a 57-kb
deletion in nephropathic cystinosis.
\item Essential pentosuria is the result of a partial deficiency of L-xylulose reductase (xylitol dehydrogenase) an enzyme of the glucuronic acid pathway.
\begin{itemize}
\item Affected individuals excrete large amounts of L-xylulose in urine.
\item benign disorder that occurs almost exclusively in Jewish people
\end{itemize}
\end{itemize}
\section{Fructose}
\label{sec:org773224d}
\subsection{Introduction}
\label{sec:orgfc66b2f}
\begin{itemize}
\item Fructose is mainly metabolised in the liver, renal cortex and small
intestinal mucosa (Figure \ref{fig:org02bf614})
\item Pathway composed of fructokinase, aldolase B and triokinase.
\item Aldolase B is also involved in the glycolytic-gluconeogenic pathway
\end{itemize}

\subsubsection{Disorders of Fructose Metabolism}
\label{sec:orgb2b5503}
\begin{itemize}
\item Three inborn errors are known in the pathway of fructose metabolism.
\item Essential fructosuria is a harmless anomaly
\begin{itemize}
\item characterised by the appearance of fructose in the urine after the intake of fructose-containing food.
\end{itemize}
\item Hereditary fructose intolerance (HFI)
\begin{itemize}
\item fructose may provoke prompt gastrointestinal discomfort and hypoglycaemia upon ingestion
\item fructose may cause liver and kidney failure when taken persistently,
\item IV fructose can be lethal
\end{itemize}
\item Fructose-1,6-bisphosphatase (FBPase) deficiency
\begin{itemize}
\item considered an inborn error of fructose metabolism although, it is a defect of gluconeogenesis.
\item hypoglycaemia and lactic acidosis (neonatally, or later during fasting or induced by fructose)
\item may be life-threatening
\end{itemize}
\end{itemize}


\begin{figure}[htbp]
\centering
\includegraphics[width=1.0\textwidth]{./fructose/figures/fructose.png}
\caption[Fructose]{\label{fig:org02bf614}
Fructose Metabolism: The three enzyme defects in fructose metabolism are boxed and depicted by solid bars across the arrows; the diminished activity of aldolase B toward fructose-1,6-bisphosphate is depicted by a broken bar}
\end{figure}

\subsection{Essential Fructosuria}
\label{sec:org94d630c}
\subsubsection{Clinical Presentation}
\label{sec:orgec14911}
\begin{itemize}
\item Rare non-disease
\item Detected a when test urine is tested for reducing substances
\item Deficiency in Fructokinase
\end{itemize}
\subsubsection{Metabolic Derangement}
\label{sec:orgb5b8a90}
\begin{itemize}
\item 10-20\% of ingested fructose is excreted in urine
\item remainder is slowly metabolised in adipose tissue and muscle
\end{itemize}
\subsubsection{Genetics}
\label{sec:orgb46914f}
\begin{itemize}
\item AR, 1:130,000 (underestimate)
\item KHK undergoes tissue specific alternative splicing
\begin{itemize}
\item Two isoforms
\item ketohexokinase A, widely expressed, no clear role
\item ketohexokinase C, adult liver, kidney and small intestine
\begin{itemize}
\item affected in essential fructosuria
\end{itemize}
\item Only 2 mutations found
\end{itemize}
\end{itemize}
\subsubsection{Diagnosis and Treatment}
\label{sec:org1ed3399}
\begin{itemize}
\item \(\uparrow\) reducing substances in urine
\item glucose oxidase negative
\item fructosuria is dependant on fructose intake
\item no treatent required
\item prognosis is excellent
\end{itemize}

\subsection{Hereditary Fructose Intolerance}
\label{sec:org12ae5d6}
\subsubsection{Clinical Presentation}
\label{sec:org74289fc}
\begin{itemize}
\item perfectly healthy as long as they do not ingest food containing fructose, sucrose and/or sorbitol
\item no metabolic derangement occurs during breast-feeding
\item younger the child and the higher the dietary fructose load, the more severe the reaction
\item first symptoms appear with the intake of fruits and vegetables or fructose containing formula
\begin{itemize}
\item gastrointestinal discomfort, nausea, vomiting, restlessness,
pallor, sweating, trembling, lethargy and, eventually, apathy,
coma, jerks and convulsion
\end{itemize}
\item laboratory tests indicate acute liver failure and generalised dysfunction of the renal proximal tubules
\item hypoglycaemia after fructose ingestion is short-lived and can be easily missed or masked by concomitant glucose intake
\item food aversions form
\item approximately half of all adults with HFI are free of dental caries
\item affected subjects may remain undiagnosed and still have a normal life span
\end{itemize}

\subsubsection{Metabolic Derangement}
\label{sec:orge184711}
\begin{itemize}
\item FI is caused by deficiency of the second enzyme of the fructose pathway, aldolase B
\begin{itemize}
\item splits fructose-1-phosphate (F-1-P) into dihydroxyacetone phosphate and glyceraldehyde
\end{itemize}
\item The high activity of fructokinase after intake of fructose results in accumulation of F-1-P and trapping of phosphate
\item This has two major effects:
\begin{enumerate}
\item inhibition of glucose production by blockage of gluconeogenesis
(inhibition of aldolase A) and glycogenolysis (inhibition of glycogen phosphorylase A)
\begin{itemize}
\item induces a rapid drop in blood glucose
\end{itemize}
\item overutilization and diminished regeneration of ATP
\begin{itemize}
\item depletion of ATP results in an increased production of uric acid
\item a release of magnesium
\item and a series of other disturbances,including impaired protein
synthesis and ultrastructural lesions which are responsible for
hepatic and renal dysfunction
\end{itemize}
\end{enumerate}
\item glycolysis and gluconeogenesis are not impaired in the fasted state in HFI patients due to activity of aldolase A

\item same process happens in IV fructose to normal patients
\item the use of fructose, sorbitol and invert sugar has been strongly discouraged for parenteral nutrition in general
\end{itemize}

\subsubsection{Genetics}
\label{sec:orgf2c10eb}
\begin{itemize}
\item AR
\item Three aldolase genes
\item B is the major fructaldolase of liver, renal cortex, and small intestine
\item A muscle
\item C brain
\end{itemize}

\subsubsection{Diagnosis and Treatment}
\label{sec:org0cd97ef}
\begin{itemize}
\item nutritional history
\item response to fructose withdrawl
\item first tier molecular diagnosis
\item second tier (no mutations) \(\to\) enzymatic
\item liver biopsy Aldo B activity
\begin{itemize}
\item False low Aldo B secondary to liver damage
\end{itemize}

\item acute intoxication may require treatment with fresh frozen plasma

\item remove fructose, sucrose and sorbitol from diet
\item prognosis on diet is excellent with normal growth,
intelligence and life span
\end{itemize}

\subsection{Fructose-1,6-Bisphosphatase Deficiency}
\label{sec:org93640e6}
\subsubsection{Clinical Presentation}
\label{sec:org6e35ba7}
\begin{itemize}
\item 1/2 present in the first 1-4 days of life
\begin{itemize}
\item severe hyperventilation
\begin{itemize}
\item lactic acidosis
\item hypoglycaemia
\end{itemize}
\end{itemize}
\item later irritability, apnoeic spells, tachycardia, muscle hypotonia
\item chronic ingestion of fructose does not lead to gastrointestinal symptoms
\begin{itemize}
\item no aversion to sweet foods or failure to thrive, and only rarely \(\downarrow\) liver function
\end{itemize}
\end{itemize}

\subsubsection{Metabolic Derangement}
\label{sec:orgb946092}
\begin{itemize}
\item deficiency of hepatic FBPase, key enzyme in gluconeogenesis
\begin{itemize}
\item impairs the formation of glucose from all gluconeogenic precursors including dietary fructose
\end{itemize}
\item normoglycaemia in patients is dependent on glucose (and galactose)
intake and degradation of hepatic glycogen
\item hypoglycaemia occurs when glycogen reserves are limited (newborns, fasting)
\item accumulation of the gluconeogenic substrates lactate, pyruvate, alanine, and glycerol.
\end{itemize}
\subsubsection{Genetics}
\label{sec:org7358ee6}
\begin{itemize}
\item AR
\item Liver isoform, FBP1 gene
\item 35 mutations in all regions of the gene have been published
\end{itemize}

\subsubsection{Diagnosis}
\label{sec:org5a30a12}
\begin{itemize}
\item plasma during acute episodes
\begin{itemize}
\item \(\uparrow\) lactate (up to 15–25 mM)
\item \(\downarrow\) pH
\item \(\uparrow\) lactate/pyruvate ratio (up to 40)
\item hyperalaninaemia
\item \(\uparrow\) glycerol which may mimic hypertriglyceridaemia
\item glucagon-resistant hypoglycaemia
\item \(\uparrow\) free fatty acids and uric acid
\end{itemize}
\item urinary analysis reveals
\begin{itemize}
\item \(\uparrow\) lactate, alanine, glycerol
\item in most cases, ketones and glycerol-3-phosphate
\end{itemize}

\item molecular analysis on DNA from peripheral leukocytes
\item if no mutations found
\begin{itemize}
\item enzymatic activity in a liver biopsy
\item the residual activity may vary from zero to 30\% of normal
\end{itemize}
\end{itemize}

\subsubsection{Differential Diagnosis}
\label{sec:org1af0584}
\begin{itemize}
\item other disturbances in gluconeogenesis and pyruvate oxidation should be considered, including:
\begin{enumerate}
\item pyruvate dehydrogenase deficiency characterised by a low
lactate/pyruvate ratio, absence of hypoglycaemia and aggravation
of lactic acidosis by glucose infusion
\item pyruvate carboxylase deficiency
\item respiratory chain disorders
\item glycogenosis type Ia and Ib presenting with the same metabolic profile
\begin{itemize}
\item fasting hypoglycaemia and lactic acidosis and hepato-nephromegaly, hyperlipidaemia, and hyperuricaemia
\end{itemize}
\item fatty acid oxidation defects presenting with fasting hypoketotic hypoglycaemia and hyperlactataemia
\end{enumerate}
\end{itemize}

\subsubsection{Treatment}
\label{sec:orgfd711b2}

\begin{itemize}
\item acute life-threatening episodes should be treated with an IV bolus
of 20\% glucose
\item followed by a continuous infusion of glucose and bicarbonate to
control hypoglycaemia and acidosis
\item Maintenance therapy should be aimed at avoiding fasting,
particularly during febrile episodes
\begin{itemize}
\item slowly absorbed carbohydrates (uncooked starch), and a gastric
drip, if necessary
\end{itemize}
\item absence of any triggering effects leading to metabolic
decompensation, individuals with FBPase deficiency are healthy and
no carbohydrate supplements are needed
\end{itemize}

\section{Insulin}
\label{sec:orge2e9435}
\subsection{Introduction}
\label{sec:orgcc33bfe}
\subsubsection{Insulin Secretion}
\label{sec:org01663c0}
\begin{itemize}
\item glucose is transported into the pancreatic \(\beta\)-cell and phosphorylated to G-6-P by glucokinase
\begin{itemize}
\item GCK K\(_{\text{M}}\) \(\sim\) [glucose] in blood
\item functions as a glucose sensor
\end{itemize}
\item \(\uparrow\) glycolysis \(\to\) \(\uparrow\) ATP
\item \(\uparrow\) ATP/ADP ratio detected by ATP/ADP-sensitive potassium channels (K\(_{\text{ATP}}\))
\begin{itemize}
\item \(\to\) channel closure depolarization of the plasma membrane
\item \(\to\) voltage-sensitive Ca\(^{\text{2+}}\) channel opens
\item influx of extracellular Ca\(^{\text{2+}}\) stimulates insulin secretion by
exocytosis from storage granules
\end{itemize}

\item Other mechanisms regulate the release of insulin
\begin{enumerate}
\item transcription factors, such as HNF1A and HNF4A
\item metabolic factors which modulate the ATP production
\begin{itemize}
\item leucine activation of glutamate dehydrogenase
\item short-chain L-3-hydroxyacyl-CoA dehydrogenase
\item monocarboxylate transporter 1
\item mitochondrial uncoupling protein 2
\end{itemize}
\item receptors for various hormones and neuropeptides including:
\begin{itemize}
\item somatostatin, insulin, GLP1,GIP, etc
\end{itemize}
\end{enumerate}
\end{itemize}


\begin{figure}[htbp]
\centering
\includegraphics[width=0.9\textwidth]{./insulin/figures/insulin.png}
\caption[insulin]{\label{fig:org7f8879f}
Insulin Secretion}
\end{figure}

\subsubsection{Insulin Effects}
\label{sec:org6573475}
\begin{itemize}
\item activation of insulin receptors:
\begin{itemize}
\item \(\uparrow\) glucose utilization
\item \(\downarrow\) lipid utilization
\item \(\uparrow\) cellular growth
\item translocation of GLUT4 to the PM
\end{itemize}
\item cerebral cells are poorly insulin-sensitive
\begin{itemize}
\item highly dependent on circulating glucose
\begin{itemize}
\item in hyperinsulinism, there is a significant risk of brain damage
from neuroglucopenia
\end{itemize}
\end{itemize}
\end{itemize}

\subsection{Congenital Hyperinsulinism}
\label{sec:org8c25833}
\begin{itemize}
\item CHI includes all genetic causes of hyperinsulinaemic
hypoglycaemia due to a primary defect of the pancreatic
\(\beta\)-cell
\item CHI can present throughout childhood, most common in infancy
\item Severe CHI is responsible for recurrent severe hypoglycaemia in neonates
\begin{itemize}
\item delayed diagnosis or inappropriate medical management is responsible for brain damage in about 1/3
\end{itemize}
\item Two main histopathological variants of CHI: diffuse and focal
\item Three forms: transient, syndromic and isolated congenital HI
\end{itemize}

\subsection{Transient and Syndromic HI}
\label{sec:org137629e}
\begin{itemize}
\item transient neonatal HI
\begin{itemize}
\item can occur in newborns from diabetic mothers
\item small for gestational age
\item due to perinatal stress such as fetal distress or following birth asphyxia
\item hypoglycaemia can be severe
\begin{itemize}
\item usually resolves within a few days or months
\end{itemize}
\end{itemize}
\item syndromic HI
\begin{itemize}
\item HI is part of a developmental syndrome
\item hypoglycaemia can be the initial manifestation of a number of
different syndromes during the neonatal period
\begin{itemize}
\item Beckwith Wiedemann Syndrome (BWS)
\item CDGs (PMM2-CDG and PMMI-CDG)
\item Kabuki syndrome
\item Sotos syndrome
\end{itemize}
\end{itemize}
\end{itemize}

\subsection{Isolated Congenital HI}
\label{sec:org5dfc8b0}
\begin{itemize}
\item HI is inherited but occurs primarily as an isolated abnormality
\item hypoglycaemia can reveal the disease in all ages
\item hypoglycaemia occurs both in the fasting and the post-prandial states
\item most neonates (86\%) are resistant to treatment with diazoxide
\end{itemize}

\subsubsection{Metabolic Derangement}
\label{sec:org88118db}
\begin{itemize}
\item functional defect of the pancreatic \(\beta\)-cells
\item inappropriate secretion of insulin \(\to\) hypoglycaemia
\begin{itemize}
\item \(\downarrow\) hepatic glucose release from glycogen and gluconeogenesis
\item \(\uparrow\) glucose uptake in muscular and fatty tissues
\end{itemize}
\item CHI is heterogeneous, caused by various defects in regulation of insulin secretion
\begin{itemize}
\item channelopathies affecting the ATP channel (ABCC8 and KCNJ11 mutations)
\item metabolic defects:
\begin{itemize}
\item enzymes deficiencies: glucokinase, glutamate dehydrogenase, or SCHAD
\item transporter deficiencies: MCT1 and the mitochondrial uncoupling protein 2
\end{itemize}
\item transcription factors impairment, such as HNF1A and HNF4A
\item exceptional cases, defect in the signalling pathway of the insulin
receptor
\end{itemize}
\end{itemize}

\subsubsection{Genetics}
\label{sec:orgb76a144}
\begin{itemize}
\item estimated incidence of severe CHI is 1 in 50,000 live births
\begin{itemize}
\item in countries with substantial consanguinity it may be as high as 1 in 2,500
\end{itemize}
\item pattern of inheritance can be dominant or recessive
\begin{itemize}
\item sometimes \emph{de novo}
\end{itemize}
\item in isolated CHI, the inheritance is:
\begin{itemize}
\item autosomal recessive for ABCC8, KCNJ11 and HADH gene mutations
\item autosomal dominant or \emph{de novo} for GLUD1, GCK, UCP2,SLC16A1, HNF1A, HNF4A mutations
\begin{itemize}
\item some cases for ABCC8 and KCNJ11 mutations.
\end{itemize}
\end{itemize}
\end{itemize}

\subsubsection{Diagnosis}
\label{sec:orgb07b6c2}
\begin{itemize}
\item diagnosis of HI relies on 5 non-essential criteria:
\begin{enumerate}
\item fasting and/or post-prandial hypoglycaemia (<2.5-3 mmol/l).
\item inappropriate plasma insulin levels and c-peptide at the time of
hypoglycaemia (potentially missed by a single sample because of
the pulsatile secretion of insulin).
\item absent/low blood \& urine ketones bodies and non-esterified fatty
acids (NEFA). However, in some cases, ketones bodies and NEFA are
not totally suppressed.
\item \(\uparrow\) blood glucose \textgreater{} 1.7 mmol/l within 30–40 min after
SC/IM or IV administration of 1 mg glucagon.
\item the need for a high glucose infusion rate (GIR) to keep blood
glucose above 3 mmol/l is characteristic of an insulin related
hypoglycaemia
\end{enumerate}

\item once HI is established molecular studies to identify a gene
\end{itemize}

\section{Glucose Transport}
\label{sec:org5fe9820}
\subsection{Introduction}
\label{sec:org89392ab}
\subsubsection{Glucose Transporters}
\label{sec:orgeb22365}
\begin{itemize}
\item glucose is hydrophilic \(\therefore\) cannot easily cross cell membrane
\item transporters exist in almost all cell types
\item glucose transporter proteins can be divided into two groups:
\begin{enumerate}
\item Sodium-dependent glucose transporters (SGLTs)
\begin{itemize}
\item symporter systems, active transporters encoded by members of
the SLC5 gene family
\item couple sugar transport to the electrochemical gradient of sodium
\item transport glucose \(\uparrow\) [gradient]
\end{itemize}
\item Facilitative glucose transporters (GLUTs)
\begin{itemize}
\item uniporter systems, passive transporters encoded by members of the SLC2 gene family
\item transport glucose \(\downarrow\) [gradient]
\end{itemize}
\end{enumerate}
\end{itemize}

\begin{figure}[htbp]
\centering
\includegraphics[width=1.0\textwidth]{./glucose_transport/figures/glut.png}
\caption[glucose transporters]{\label{fig:org4cffb9c}
Glucose Transporters}
\end{figure}

\subsubsection{Congenital Defects of Glucose Transporters}
\label{sec:org62cb01d}
\begin{itemize}
\item Five congenital defects of monosaccharide transporters
\item Their clinical picture depends on tissue-specific expression and
substrate specificity of the affected transporter

\begin{enumerate}
\item SGLT1 : Congenital Glucose/Galactose Malabsorption
\item SGLT2 : Renal Glucosuria
\item GLUT1 : Glucose Transporter-1 Deficiency
\item GLUT2 : Fanconi-Bickel Syndrome
\item GLUT10 : Arterial Tortuosity Syndrome
\end{enumerate}
\end{itemize}

\subsection{Congenital Glucose/Galactose Malabsorption (SGLT1 Deficiency )}
\label{sec:orgf995cb4}

\begin{itemize}
\item SGLT1 is a high-affinity, low-capacity sodium-dependent transporter
of the two monosaccharides, at the brush border of enterocytes

\item SGLT1 deficiency \(\to\) intestinal glucose-galactose malabsorption
\item GGM is a rare autosomal recessive disorder

\item presents with severe osmotic diarrhoea and dehydration soon after a
normal birth and pregnancy
\begin{itemize}
\item patients develop severe hypertonic dehydration, often with fever
\item patients die from hypovolaemic shock
\end{itemize}

\item Treatement is a glucose and galactose free diet
\end{itemize}

\subsection{Renal Glucosuria (SGLT2 Deficiency)}
\label{sec:org5196f16}
\begin{itemize}
\item SGLT2 is the major co-transporter involved in glucose reabsorption in
the kidney
\item SGLT2 deficiency results in isolated renal glucosuria
\begin{itemize}
\item a harmless renal transport defect characterised by:
\begin{itemize}
\item glucosuria
\item normal blood glucose concentrations
\item absence of any other signs of renal tubular dysfunction
\end{itemize}
\end{itemize}
\end{itemize}

\subsection{Glucose Transporter-1 Deficiency (GLUT1 Deficiency)}
\label{sec:orgb549eb0}
\begin{itemize}
\item GLUT1 is a membrane-spanning, glycosylated protein that facilitates
glucose transport across the blood-brain barrier
\begin{itemize}
\item low CSF glucose concentration (hypoglycorrhachia)
\end{itemize}

\item clinical symptoms include: microcephaly, epileptic encephalopathies,
paroxysmal movement disorders, tremor
\item haemolytic anaemia has also been observed

\item AD and AR inheritance described, SLC2A1

\item GLUT1D should be suspected in any child with a CSF glucose
concentration \textless{} 2.5 mmol/L 
\begin{itemize}
\item normal \textgreater{} 3.3 mmol/L
\end{itemize}

\item CSF to blood glucose ratio \textless{} 0.5 
\begin{itemize}
\item normal \textgreater{} 0.6
\item in the absence of hypoglycaemia or a CNS infection is diagnostic.
\end{itemize}
\item ketogenic diet is used in treatment
\end{itemize}

\subsection{Fanconi-Bickel Syndrome (GLUT2 Deficiency )}
\label{sec:org7cae693}

\begin{itemize}
\item GLUT2 is a high-K\(_{\text{M}}\) monosaccharide carrier 
\begin{itemize}
\item located in hepatocytes
\item basolateral membrane of cells in the proximal tubule
\item pancreatic \(\beta\)-cells
\end{itemize}

\item typically presents with a combination of increased hepatic
glycogen storage, generalised renal tubular dysfunction, severe glucosuria

\item in FBS GLUT2 acts as a malfunctioning glucose sensor:
\begin{itemize}
\item in the fasted state, [glucose] and [G-6-P] are inappropriately \(\uparrow\) in hepatocytes
\item stimulates glycogen synthesis, inhibits gluconeogenesis and glycogenolysis
\item predisposes to hypoglycaemia and hepatic glycogen accumulation
\end{itemize}

\item very rare AR, SLC2A2

\item diagnosis suggested by the characteristic combination of an altered
glucose homeostasis, hepatic glycogen accumulation, and the typical
features of a Fanconi-type tubulopathy

\item elevated biotinidase activity in serum has been found to be a useful
screening test for hepatic glycogen storage disorders including FBS

\item only symptomatic treatment is available
\end{itemize}

\subsection{Arterial Tortuosity Syndrome (GLUT10 Deficiency)}
\label{sec:org4c5b0e2}

\begin{itemize}
\item GLUT10 function not entirely clear:
\begin{itemize}
\item localizes to mitochondria of smooth muscle and insulin-stimulated adipocytes
\item facilitates transport of dehydroascorbic acid (DHA), the
oxidized form of vitamin C, into mitochondria
\end{itemize}

\item GLUT10 deficiency is characterised by hyperelastic connective tissue
and generalised tortuosity and elongation of all major arteries
including the aorta

\item presents with acute infarction owing to ischaemic stroke or an
increased risk of thromboses
\item Aortic regurgitation and multiple pulmonary artery stenoses are
typical intrathoracic manifestations
\item closely resembles a connective tissue disorder in presentation

\item rare, AR GLUT10 (SLC2A10)
\item Echocardiography, angiography, and/or CT scan are important to demonstrate vascular changes.
\item Diagnosis is based on molecular genetic methods
\end{itemize}
\end{document}