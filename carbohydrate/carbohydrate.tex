% Created 2019-10-13 Sun 13:53
% Intended LaTeX compiler: pdflatex
\documentclass{scrartcl}
\usepackage[utf8]{inputenc}
\usepackage[T1]{fontenc}
\usepackage{graphicx}
\usepackage{grffile}
\usepackage{longtable}
\usepackage{wrapfig}
\usepackage{rotating}
\usepackage[normalem]{ulem}
\usepackage{amsmath}
\usepackage{textcomp}
\usepackage{amssymb}
\usepackage{capt-of}
\usepackage{hyperref}
\hypersetup{colorlinks,linkcolor=black,urlcolor=blue}
\usepackage{textpos}
\usepackage{textgreek}
\usepackage[version=4]{mhchem}
\usepackage{chemfig}
\usepackage{siunitx}
\usepackage{gensymb}
\usepackage[usenames,dvipsnames]{xcolor}
\usepackage[T1]{fontenc}
\usepackage{lmodern}
\usepackage{verbatim}
\usepackage{tikz}
\usepackage{wasysym}
\usetikzlibrary{shapes.geometric,arrows,decorations.pathmorphing,backgrounds,positioning,fit,petri}
\author{Matthew Henderson, PhD, FCACB}
\date{\today}
\title{Disorders of Carbohydrate Metabolism}
\hypersetup{
 pdfauthor={Matthew Henderson, PhD, FCACB},
 pdftitle={Disorders of Carbohydrate Metabolism},
 pdfkeywords={},
 pdfsubject={},
 pdfcreator={Emacs 26.1 (Org mode 9.1.9)}, 
 pdflang={English}}
\begin{document}

\maketitle
\tableofcontents


\tikzstyle{chemical} = [rectangle, rounded corners, text width=5em, minimum height=1em,text centered, draw=black, fill=none]
\tikzstyle{hardware} = [rectangle, rounded corners, text width=5em, minimum height=1em,text centered, draw=black, fill=gray!30]
\tikzstyle{ms} = [rectangle, rounded corners, text width=5em, minimum height=1em,text centered, draw=orange, fill=none]
\tikzstyle{msw} = [rectangle, rounded corners, text width=7em, minimum height=1em,text centered, draw=orange, fill=none]
\tikzstyle{label} = [rectangle,text width=8em, minimum height=1em, text centered, draw=none, fill=none]
\tikzstyle{hl} = [rectangle, rounded corners, text width=5em, minimum height=1em,text centered, draw=black, fill=red!30]
\tikzstyle{box} = [rectangle, rounded corners, text width=5em, minimum height=5em,text centered, draw=black, fill=none]
\tikzstyle{arrow} = [thick,->,>=stealth]
\tikzstyle{hl-arrow} = [ultra thick,->,>=stealth,draw=red]


\section{Carbohydrate Metabolism}
\label{sec:orgf9a11a7}
\subsection{Introduction}
\label{sec:org9a5eac2}
\begin{enumerate}
\item Carbohydrate Digestion
\label{sec:org2114caf}
\begin{figure}[htbp]
\centering
\includegraphics[width=0.7\textwidth]{./carbohydrate_metabolism/figures/carb_digest.pdf}
\caption{\label{fig:org187a80b}
Carbohydrate Digestions}
\end{figure}

\item Major Pathways of Glucose Metabolism
\label{sec:orgb29701f}

\begin{figure}[htbp]
\centering
\includegraphics[width=0.9\textwidth]{./carbohydrate_metabolism/figures/glucose_pathways.pdf}
\caption{\label{fig:org2156b7c}
Major Pathways of Glucose Metabolism}
\end{figure}

\item Conversion to AAs and FAs
\label{sec:org61665eb}

\begin{figure}[htbp]
\centering
\includegraphics[width=0.65\textwidth]{./carbohydrate_metabolism/figures/glucose_conversion.pdf}
\caption{\label{fig:org8d00f9b}
Conversion of Glucose}
\end{figure}
\end{enumerate}

\subsection{Regulation}
\label{sec:org46d956c}
\begin{enumerate}
\item Major Hormones of Metabolic Homeostasis
\label{sec:org46fce6a}
\begin{itemize}
\item \textbf{Insulin} is the main anabolic hormone
\begin{itemize}
\item promotes use of glucose as fuel
\begin{itemize}
\item \(\uparrow\) transport into cells
\end{itemize}
\item storage of glucose as glycogen
\item conversion of glucose \(\to\) TAGs
\item TAG storage in adipose tissue
\item AA uptake and protein synthesis in muscle
\end{itemize}
\item \textbf{Glucagon} is catabolic
\begin{itemize}
\item maintain fuel availability in the absence of dietary glucose
\item stimulates glycogenolysis
\item stimulates gluconeogenesis from lactate, glycerol and AAs
\item mobilizing FAs from adipose TAGs
\item acts on liver and adipose, muscle has no receptor
\end{itemize}
\end{itemize}

\item Major Hormones of Metabolic Homeostasis
\label{sec:orge5c0e4e}
\begin{figure}[htbp]
\centering
\includegraphics[width=0.9\textwidth]{./carbohydrate_metabolism/figures/regulation.pdf}
\caption{\label{fig:org7dbd655}
Glucose Homeostasis}
\end{figure}

\item Insulin and Counterregulatory Hormones
\label{sec:org0a76c37}

\begin{center}
\begin{tabular}{lll}
Hormone & Function & Pathway\\
\hline
Insulin & \(\uparrow\) storage & glucose \(\to\) glycogen\\
 & \(\uparrow\) growth & FA synthesis and storage\\
 &  & AA uptake, protein synthesis\\
\hline
Glucagon & mobilizes stores & \(\uparrow\) gluconeogenesis\\
 & maintain blood glucose & \(\uparrow\) glycogenolysis\\
 & during a fast & FA release\\
\hline
Epinephrine & mobilize fuel during & \(\uparrow\) glycogenolysis\\
 & acute stress & FA release\\
\hline
Cortisol & long-term fuel requirements & \(\uparrow\) AA mobilization\\
 &  & from muscle\\
 &  & \(\uparrow\) gluconeogenesis for\\
 &  & glycogen synthesis\\
 &  & \(\uparrow\) FA release\\
\end{tabular}
\end{center}

\item Post Prandial Production
\label{sec:org4ddbce4}

\begin{figure}[htbp]
\centering
\includegraphics[width=0.5\textwidth]{./carbohydrate_metabolism/figures/meal.pdf}
\caption{\label{fig:org1035881}
Carbohydrate rich meal}
\end{figure}

\item Counterregulatory Hormones During Fasting
\label{sec:orgc2c9ae5}

\begin{figure}[htbp]
\centering
\includegraphics[width=0.9\textwidth]{./carbohydrate_metabolism/figures/counter_hormones.pdf}
\caption{\label{fig:org09eca28}
Low Blood Glucose}
\end{figure}
\end{enumerate}

\subsection{Transport}
\label{sec:org618df11}
\begin{enumerate}
\item Carbohydrate Digestion
\label{sec:orgbac30b8}

\begin{figure}[htbp]
\centering
\includegraphics[width=0.5\textwidth]{./carbohydrate_metabolism/figures/digestion.pdf}
\caption{\label{fig:orga5a418a}
Digestion of Carbohydrates}
\end{figure}

\item Absorption from the Intestine
\label{sec:orga0f5d53}
\begin{itemize}
\item NA-dependent transporter
\begin{itemize}
\item transport glucose \(\uparrow\) the concentration gradient
\item w co-transport of NA \(\downarrow\) the concentration gradient
\end{itemize}
\item Facilitative Glucose Transport
\begin{itemize}
\item transport glucose \(\downarrow\) the concentration gradient
\item GLUT1 to GLUT5
\end{itemize}
\item Galactose and Fructose Transport
\begin{itemize}
\item Gal uses same mechanism as glucose
\item Fructose relies on facilitated diffusion via GLUT5
\end{itemize}
\end{itemize}

\item Absorption from the Intestine
\label{sec:orgb51259a}

\begin{figure}[htbp]
\centering
\includegraphics[width=0.9\textwidth]{./carbohydrate_metabolism/figures/intestine.pdf}
\caption{\label{fig:orga9a3c56}
Absorption from the intestine}
\end{figure}

\item GLUTs
\label{sec:org7461351}

\begin{center}
\begin{tabular}{lll}
Transporter & Distribution & Comments\\
\hline
GLUT1 & erythrocyte & barrier cells\\
 & brain barrier & \(\uparrow\) affinity transporter\\
 & retina barrier & \\
 & placenta barrier & \\
 & testis barrier & \\
\hline
GLUT2 & Liver & \(\uparrow\) capacity, \(\downarrow\) affinity\\
 & Kidney & may be glucose sensor\\
 & Pancreatic \(\beta\)-cell & in pancreas\\
 & intestine & \\
\hline
GLUT3 & Neurons & \(\uparrow\) affinity  transporter in CNS\\
\hline
GLUT4 & Adipose & insulin sensitive transport\\
 & Skeletal muscle & \(\uparrow\) insulin \(\to\) \(\uparrow\) number\\
 & Heart muscle & \(\uparrow\) affinity\\
\hline
GLUT5 & Intestinal epithelium & fructose transport\\
 & spermatozoa & \\
\end{tabular}
\end{center}
\end{enumerate}

\subsection{Glycogen}
\label{sec:orgda7acbe}
\begin{enumerate}
\item Glycogen
\label{sec:org382fde7}

\begin{itemize}
\item glycogen is the storage form of glucose found in glycogen particles
\item large polymer of branched glucose polysaccharide
\item composed of glucosyl chains linked by \(\alpha\)-1-4-glycosidic bonds
\item \(\alpha\)-1-6-branches every 8 to 10 residues
\begin{itemize}
\item allows parallel processing
\item \(\uparrow\) solubility
\end{itemize}
\end{itemize}

\item Synthesis
\label{sec:org2221d68}
\begin{figure}[htbp]
\centering
\includegraphics[width=0.4\textwidth]{./carbohydrate_metabolism/figures/glycogen_synth.pdf}
\caption{\label{fig:org8f9772e}
Glycogen Synthesis}
\end{figure}

\item Degradation
\label{sec:org79f2ef7}

\begin{figure}[htbp]
\centering
\includegraphics[width=0.5\textwidth]{./carbohydrate_metabolism/figures/glycogen_degradation.pdf}
\caption{\label{fig:orgdf1a653}
Glycogen Degradation}
\end{figure}

\item Synthesis and Degradation
\label{sec:org9776311}

\begin{figure}[htbp]
\centering
\includegraphics[width=0.7\textwidth]{./carbohydrate_metabolism/figures/glycogen_synth_deg.pdf}
\caption{\label{fig:org8fd16a1}
Glycogen Synthesis and Degradation}
\end{figure}

\item Regulation
\label{sec:org2f0b862}

\begin{figure}[htbp]
\centering
\includegraphics[width=0.7\textwidth]{./carbohydrate_metabolism/figures/glycogen_enzyme_reg.pdf}
\caption{\label{fig:org00754c4}
Regulation of Glycogen Synthesis and Degradation}
\end{figure}

\item Regulation
\label{sec:orge49a67d}
\begin{enumerate}
\item Liver
\label{sec:org57bc6a7}

\begin{center}
\begin{tabular}{lll}
state & regulators & response\\
\hline
Fasting & \(\uparrow\) glucagon & \(\uparrow\) degradation\\
 & \(\downarrow\) insulin & \\
 & \(\uparrow\) cAMP & \\
CHO meal & \(\downarrow\) glucagon & \(\uparrow\) synthesis\\
 & \(\uparrow\) insulin & \\
 & \(\downarrow\) cAMP & \\
exercise \& & \(\uparrow\) epinephrine & \(\uparrow\) degradation\\
stress & \(\uparrow\) cAMP & \\
\end{tabular}
\end{center}
\end{enumerate}

\item Regulation
\label{sec:org83fd9b2}
\begin{enumerate}
\item Muscle
\label{sec:orgeb2c841}

\begin{center}
\begin{tabular}{lll}
state & regulators & response\\
\hline
Fasting & \(\downarrow\) insulin & \(\uparrow\) degradation\\
(rest) &  & \(\downarrow\) gluc transport\\
 & \(\uparrow\) cAMP & \\
CHO meal & \(\uparrow\) insulin & \(\uparrow\) synthesis\\
(rest) &  & \(\uparrow\) gluc transport\\
 &  & \\
exercise & \(\uparrow\) epinephrine & \(\uparrow\) glycolysis\\
 & \(\uparrow\) cAMP & \(\downarrow\) synthesis\\
 & \(\downarrow\) AMP & \(\downarrow\) degradation\\
\end{tabular}
\end{center}
\end{enumerate}
\end{enumerate}

\subsection{Sugar Metabolism Pathways}
\label{sec:org4e459c8}
\begin{enumerate}
\item Pathways
\label{sec:org19cfa37}
\begin{itemize}
\item Fructose
\item Galactose
\item Pentose Phosphate Pathway
\end{itemize}
\item Fructose Metabolism
\label{sec:org681b5c6}

\begin{figure}[htbp]
\centering
\includegraphics[width=0.8\textwidth]{./carbohydrate_metabolism/figures/fruc_met.pdf}
\caption{\label{fig:org7bb6b8a}
Fructose Metabolism}
\end{figure}

\item Fructose Synthesis
\label{sec:org75b7646}

\begin{enumerate}
\item :BMCOL:
\label{sec:org19cb3a0}
\begin{figure}[htbp]
\centering
\includegraphics[width=0.5\textwidth]{./carbohydrate_metabolism/figures/fruc_syn.pdf}
\caption{\label{fig:orga459a8a}
Fructose Synthesis}
\end{figure}

\item :BMCOL:
\label{sec:orgb42eb97}
\begin{itemize}
\item polyol pathway
\item present in most tissues
\end{itemize}
\end{enumerate}

\item Galactose Metabolism
\label{sec:orga99d631}

\begin{figure}[htbp]
\centering
\includegraphics[width=0.7\textwidth]{./carbohydrate_metabolism/figures/gal_met.pdf}
\caption{\label{fig:orgd796261}
Galactose Metabolism}
\end{figure}

\item Pentose Phosphate Pathway
\label{sec:org2249bb5}
\begin{enumerate}
\item Oxidative Phase
\label{sec:org5e3abc0}
\begin{itemize}
\item glucose 6-P \(\to\) NADPH + ribose 5-P
\item Glucose 6-P dehydrogenase catalyses first step
\item NADPH is for reducing reactions
\begin{itemize}
\item NADPH/NADP\(^{\text{+}}\) \textgreater{}\textgreater{}\textgreater{} NADH/NAD\(^{\text{+}}\)
\item NADH is rapidly converted to NAD\(^{\text{+}}\) in the ETC
\end{itemize}
\end{itemize}
\item Non-oxidative Phase
\label{sec:orgb408821}
\begin{itemize}
\item reversible rxns
\item convert glycolytic intermediates to 5 carbon sugars
\end{itemize}
\end{enumerate}
\item Pentose Phosphate Pathway
\label{sec:org09d5e67}

\begin{itemize}
\item Ribose-5-P required for purine and pyrimidine synthesis
\item NADPH required for detoxification and synthetic reaction
\begin{itemize}
\item Detoxification
\begin{itemize}
\item Reduction of oxidized glutathione
\item Cytochrome p450 monoxygenases
\end{itemize}
\item Synthetic reactions
\begin{itemize}
\item FA synthesis
\item Cholesterol
\item neurotransmitters
\item deoxynucleotide
\item superoxide
\end{itemize}
\end{itemize}
\end{itemize}
\end{enumerate}

\subsection{Synthesis}
\label{sec:orgf0d11a1}
\begin{enumerate}
\item Interconversion
\label{sec:org6ff1a96}
\begin{itemize}
\item sugars are activated by addition of nucleotides
\item Uridine diphosphate (UDP)-glucose is a precusor of:
\begin{itemize}
\item glycogen, lactate, UDP-glucuronate
\item CHO chains in proteoglycans glycoproteins and glycolipids
\end{itemize}
\end{itemize}
\item UPD-glucose
\label{sec:org01bde84}
\begin{figure}[htbp]
\centering
\includegraphics[width=0.7\textwidth]{./carbohydrate_metabolism/figures/udp_glu.pdf}
\caption{\label{fig:orgbebbeb0}
UDP-glucose metabolism}
\end{figure}

\item UPD-glucuronate
\label{sec:org712efd7}

\begin{figure}[htbp]
\centering
\includegraphics[width=0.7\textwidth]{./carbohydrate_metabolism/figures/udp_gln.pdf}
\caption{\label{fig:org66c70bc}
UDP-glucuronate metabolism}
\end{figure}
\end{enumerate}

\subsection{Gluconeogenesis}
\label{sec:org504965e}

\begin{enumerate}
\item Precusors
\label{sec:orgaf3d79c}

\begin{figure}[htbp]
\centering
\includegraphics[width=0.6\textwidth]{./carbohydrate_metabolism/figures/precusors.pdf}
\caption{\label{fig:org1139db2}
Glucose precusors}
\end{figure}


\item Tissue response to Fasting
\label{sec:orgc3a861f}

\begin{figure}[htbp]
\centering
\includegraphics[width=0.9\textwidth]{./carbohydrate_metabolism/figures/fasting.pdf}
\caption{\label{fig:orgab5abba}
Tissue interrelationships during fasting}
\end{figure}


\item Changes in metabolic fuels during fasting
\label{sec:orgcf68c72}

\begin{figure}[htbp]
\centering
\includegraphics[width=0.9\textwidth]{./carbohydrate_metabolism/figures/fasting_changes.pdf}
\caption{\label{fig:org546a79b}
Changes in metabolic fuels during fasting}
\end{figure}
\end{enumerate}

\section{Galactosemia}
\label{sec:org3a675e5}
\subsection{Classical galactosemia}
\label{sec:orga731338}
\begin{itemize}
\item Caused by deficient activity of galactose-1-phosphate uridylyltransferase
\item prevalence is 1:16,000-60,000 live births
\item autosomal recessive
\begin{itemize}
\item 336 mutations is the GALT gene described
\end{itemize}

\item presents in the neonatal period
\begin{itemize}
\item potentially lethal
\end{itemize}

\item Therapy is life long dietary galactose restriction
\begin{itemize}
\item does not prevent long-term complications
\end{itemize}
\end{itemize}

\subsection{Galactose}
\label{sec:orgc7a25fe}
\begin{itemize}
\item Primary source is dietary lactose
\item Functions
\begin{itemize}
\item energy source in pre-weaning infants
\item glycosylation
\item glycolipid synthesis
\end{itemize}
\end{itemize}


\begin{figure}[htbp]
\centering
\includegraphics[width=0.4\textwidth]{./galt/figures/Beta-D-Lactose.png}
\caption[lactose]{\label{fig:org401a60d}
Lactose is a disaccharide derived from the condensation of galactose and glucose, which form a \(\beta\) 1 \(\to\) 4 glycosidic linkage.}
\end{figure}


\subsection{Galactose metabolism}
\label{sec:orga592eb6}

\begin{figure}[htbp]
\centering
\includegraphics[width=0.8\textwidth]{./galt/figures/Fig1.png}
\caption[met]{\label{fig:orgb2b480a}
Galactose metabolism}
\end{figure}


\subsection{GALT Protein}
\label{sec:orgaa1cebd}
\begin{itemize}
\item ubiquitous enzyme
\end{itemize}

\begin{figure}[htbp]
\centering
\includegraphics[width=0.8\textwidth]{./galt/figures/Fig2.png}
\caption[structure]{\label{fig:org2a0e5ec}
Crystal structure}
\end{figure}


\subsection{GALT catalytic mechanism}
\label{sec:org28ce139}

\begin{figure}[htbp]
\centering
\includegraphics[width=0.8\textwidth]{./galt/figures/Fig3.png}
\caption[mechanism]{\label{fig:orged2dc74}
Catalytic mechanism}
\end{figure}


\subsection{GALT gene}
\label{sec:org28d4803}
\begin{itemize}
\item 9p13, 11 exons, \textasciitilde{}4 kb
\item housekeeping
\end{itemize}
\begin{enumerate}
\item c.563A>G, p.Q188R
\label{sec:org5912363}
\begin{itemize}
\item \textasciitilde{}64\% of galactosemic alleles in caucasian pop
\item Irish Travellers 1:430
\item Non-functional variant: Destabilise UMP-GALT
\begin{itemize}
\item no residual RBC GALT activity
\end{itemize}
\end{itemize}

\item c.855G>T, p.K285N
\label{sec:org58c8f9b}
\begin{itemize}
\item slavic origin?
\item no residual RBC GALT activity
\end{itemize}
\end{enumerate}

\subsection{Duarte and Los Angles variants}
\label{sec:org8cd449f}
\begin{itemize}
\item Duarte (p.N314D) - 50\% RBC GALT enzyme activity
\item LA (p.N314D) - elevated RBC GALT enzyme activity
\item Same electrophoretic pattern
\item D is in linkage disequilibrium w a 4bp promoter deletion
\item p.D314 is the ancestral allele
\item p.N314 arose early in human evolution
\end{itemize}


\subsection{Biochemical Features}
\label{sec:org1b0c888}
\begin{itemize}
\item \(\uparrow\)  Galatactose
\item \(\uparrow\) Gal-1-p - pathogenic
\item \(\uparrow\)  galactitol - cataracts
\item \(\uparrow\) galactonate
\item \(\downarrow\) UPD-Gal - disordered glycosylation, glycolipids
\item \(\downarrow\) UPD-Glc
\end{itemize}

\subsection{Acute Presentation}
\label{sec:org5ac36c8}
\begin{itemize}
\item asymptomatic at birth
\item with feeding:
\begin{itemize}
\item poor weight gain, vomiting, diarrhea
\item hepatocellular damage, lethargy, and hypotonia
\end{itemize}
\item May progress to Gram negative sepsis, cataracts
\end{itemize}

\subsection{NBS}
\label{sec:org7de8986}
\begin{itemize}
\item not universal
\begin{itemize}
\item must be identified early
\item \textless{} 5 days is ideal
\end{itemize}
\end{itemize}

\subsection{Spotcheck Method}
\label{sec:org8f8a1aa}

\ce{Gal-1-P + UDP-Glu ->[GALT] Glu-1-P + UDP-Gal}

\ce{Glu-1-P ->[PGluM] Glu-6-P}

\ce{Glu-6-P + NADP ->[G6PD] 6-PG + NADPH}

\ce{NADPH + MTT ->[methoxy PMS] Coloured Formazan + NADP}

\subsection{Diagnosis}
\label{sec:orgb61c551}
\begin{itemize}
\item reducing substances in urine - not specific or sensitive
\item Gal-1-P, galactose, galactitol in blood or urine
\item RBC Gal-1-P - not specific
\item RBC GALT activity
\begin{itemize}
\item Classic galactosemia - undetectable or 1\% of controls
\end{itemize}
\end{itemize}

\subsection{Therapy and Outcome}
\label{sec:org2d74370}

\begin{enumerate}
\item Therapy
\label{sec:orgc06e822}
\begin{itemize}
\item life long dietary restriction of galactose.
\end{itemize}

\item Outcome
\label{sec:org3d64877}
\begin{itemize}
\item Endogenous galactose synthesis may be responsible for:
\begin{itemize}
\item Cognitive impairment
\item Ovarian insufficiency
\end{itemize}
\item Dairy restrictions
\begin{itemize}
\item Bone health
\end{itemize}
\end{itemize}
\end{enumerate}

\section{Hepatic Glycogenoses}
\label{sec:orga62fa7e}
\subsection{Introduction}
\label{sec:org063b3d5}

\begin{enumerate}
\item Hepatic Glycogenoses
\label{sec:orgf0181d9}

\scriptsize
\begin{center}
\begin{tabular}{llll}
Type & Enzyme & Gene & Phenotype\\
\hline
0a & liver glycogen synthase & GYS2 & ketotic hypoglycema\\
Ia & G6Pase \(\alpha\) & GSPC & hypoglycema, hepatomegaly, lactic acidosis\\
Ib & G6P transporter & SLC37A4 & Ia + neutrophil dysfunction, colitis\\
III & Glycogen debrancher & AGL & hypoglycema, hepatomegaly\\
IV & Glycogen brancher & GBE1 & cirrhosis, cardiomyopathy \& myopathy\\
 &  &  & adult polyglucosan body disease\\
VI & Liver glycogen phosphorylase & PYGL & hypoglycema, hepatomegaly, growth delay\\
IXa & phosphorylase kinase \(\alpha\)2 & PHKA2 & hypoglycema, hepatomegaly\\
IXb & phosphorylase kinase \(\beta\) & PHKB & \\
IXc & phosphorylase kinase \(\gamma\)TL & PHKG2 & \\
IXd & phosphorylase kinase \(\alpha\)1 & PHKA1 & myopathy\\
\end{tabular}
\end{center}

\item Hepatic Glycogenoses
\label{sec:orge5b5a00}

\begin{figure}[htbp]
\centering
\includegraphics[width=0.75\textwidth]{./figures/gggmetab.png}
\caption[Hepatic Glycogenoses]{\label{fig:orgc0048bc}
Hepatic Glycogenoses}
\end{figure}



\item Hepatic Glycogenoses
\label{sec:orgde817f8}

\begin{figure}[htbp]
\centering
\includegraphics[width=0.75\textwidth]{./hepatic_glycogenoses/figures/gggmetab_hepatic.png}
\caption[Hepatic Glycogenoses]{\label{fig:org045e6d6}
Hepatic Glycogenoses}
\end{figure}
\end{enumerate}

\subsection{GSD Type 0}
\label{sec:org91d300d}
\begin{enumerate}
\item Metabolic Derangement
\label{sec:org82429ac}
\begin{itemize}
\item Deficiency in hepatic glycogen synthase (GS2)
\item Very low liver glycogen
\item Fasting associated with ketosis
\item Post-prandial hyperglycemia with moderate hyperlactatemia
\begin{itemize}
\item reduced liver uptake of glucose
\end{itemize}
\end{itemize}

\item Genetics
\label{sec:org8da0405}
\begin{itemize}
\item AR, GYS2 encodes liver isoform of GS2
\end{itemize}

\item Clinical Presentation
\label{sec:orgc521158}
\begin{itemize}
\item ketotic hypoglycema with post-prandial hyperglycemia/uria
\item can have poor growth
\item no hepatomegaly
\end{itemize}

\item Diagnostic Tests
\label{sec:orgf1780fb}
\begin{itemize}
\item mutation analysis
\end{itemize}
\item Treatment
\label{sec:org065dd1e}
\begin{itemize}
\item avoid fasting
\item uncooked cornstarch prior to overnight fast and during infections
\end{itemize}
\end{enumerate}
\subsection{GSD Type I}
\label{sec:orgc5ad080}
\begin{enumerate}
\item Metabolic Derangement
\label{sec:orgd27566d}
\begin{itemize}
\item GSD Ia, glucose 6 phosphatase-\(\alpha\)
\item GSD Ib, glucose 6 phosphate transporter
\item Disorder of glycogen metabolism and gluconeogenesis
\item Failure of glucose dephosphorylation inhibits hepatic glycogen breakdown
\item Hyperlactatemia occurs due to lack of gluconeogenesis
\begin{itemize}
\item protective
\end{itemize}
\item hyperlipidemia and hyperuricemia due to \(\uparrow\) G6P
\begin{itemize}
\item \(\uparrow\) G6P \(\to\) \emph{de novo} lipogenesis and flux through pentose phosphate pathway
\end{itemize}
\item G6P transporter required for normal neutrophil function
\end{itemize}

\item Genetics
\label{sec:org2e9b308}
\begin{itemize}
\item AR, 1:100,000, 80\% Ia
\item GSD Ia: G6PC
\item GSD Ib: SLC37A4
\item no genotype phenotype correlation established
\end{itemize}

\item Clinical Presentation
\label{sec:org6b78b77}
\begin{enumerate}
\item Ia and Ib
\label{sec:org43a8514}
\begin{itemize}
\item severe fasting hypoglycema, lactic acidosis
\item hepatomegaly
\item hyperlipidemia, hyperuricemia
\end{itemize}
\item Ib
\label{sec:org837f133}
\begin{itemize}
\item neutrophil dysfunction
\item increased infections
\end{itemize}
\end{enumerate}

\item Diagnostic Tests
\label{sec:org49c279f}
\begin{itemize}
\item mutation analysis
\end{itemize}

\item Treatment
\label{sec:orge822f29}
\begin{itemize}
\item generally fatal if untreated
\item diet
\item liver transplant
\item treatment of sequelae
\begin{itemize}
\item hepatic tumors
\item GI disease - IBD in GSD Ib
\item renal disease - glycogen deposition
\item hematological disease
\begin{itemize}
\item anemia
\item coagulopathy
\item infections, GSD Ib
\end{itemize}
\item cardiovascular disease
\item bone disease
\end{itemize}
\end{itemize}
\end{enumerate}

\subsection{GSD Type III}
\label{sec:org4cf9b68}
\begin{enumerate}
\item Metabolic Derangement
\label{sec:orga9d220e}
\begin{itemize}
\item Glycogen debrancher enzyme (GDE) deficiency
\item has both glucosidase and transferase activity
\begin{itemize}
\item cleaves \(\alpha\)-1,4 glucose linkages of the terminal glucose
\item then breaks \(\alpha\)-1,6 linkage to remove branch point
\end{itemize}
\item accumulation of abnormal glycogen
\item limited glucose release from glycogen
\item gluconeogenesis functions normally
\end{itemize}
\item Genetics
\label{sec:org79f1b52}
\begin{itemize}
\item AR, AGL gene
\item mutations occur throughout AGL (GSD IIIa)
\begin{itemize}
\item defect in liver and muscle
\end{itemize}
\item two specific mutations in exon 3 (GSD IIIb)
\begin{itemize}
\item liver only
\end{itemize}
\end{itemize}
\item Clinical Presentation
\label{sec:org46168b7}
\begin{itemize}
\item Hepatic glycogenosis and (in most cases) also myopathic
\item First year with poor growth, delayed motor milestones and abdominal
distension
\item Fasting hypoglycaemia 
\begin{itemize}
\item Fasting tolerance is usually longer than in GSD I
\end{itemize}
\item Fasting ketosis is prominent.
\item Gluconeogenesis is normal \(\therefore\) no fasting hyperlactataemia
\item Moderate post-prandial \(\uparrow\) lactate
\item Hyperlipdaemia
\item \(\uparrow\) \(\uparrow\) \(\uparrow\) liver transaminases
\item \(\uparrow\) CK in myopathic form
\end{itemize}
\item Diagnostic Tests
\label{sec:orgec98d9b}
\begin{itemize}
\item DBE activity in leucocytes
\item mutation analysis
\end{itemize}
\item Treatment
\label{sec:org6b188de}
\begin{itemize}
\item Aim is to maintain normoglycaemia, reduce the hyperlipidaemia and ketosis and
ensure adequate growth.
\item Regular meals and uncooked cornstarch
\item Overnight continuous feeding is less commonly needed in GSD III than
in GSD I
\item Long term outcome for individuals with GSD III is generally good
with survival into adulthood.
\end{itemize}
\end{enumerate}
\subsection{GSD Type IV}
\label{sec:orgd54857d}
\begin{enumerate}
\item Metabolic Derangement
\label{sec:orgd5c427e}
\begin{itemize}
\item GSD IV is caused by deficiency in glycogen brancher enzyme (GBE).
\item GBE transfers short glucosyl chains to form branch points with an
\(\alpha\)-1,6 linkage.
\item Deficiency results in an abnormal poorly soluble glycogen with fewer branch points (polyglucosan)
\item This abnormal glycogen accumulates in liver, muscle, heart, nervous system and skin.
\begin{itemize}
\item leads to tissue damage.
\end{itemize}
\end{itemize}

\item Genetics
\label{sec:org11d4e0f}
\begin{itemize}
\item AR, GBE1
\item Common mutation in Ashkenazi Jewish pop
\begin{itemize}
\item adult polyglucosan body disease (APBD)
\end{itemize}
\end{itemize}

\item Clinical Presentation
\label{sec:org5c465df}

\begin{itemize}
\item Multiple phenotypes associated with GBE deficiency
\begin{itemize}
\item Ranges from death in utero to adult presentation
\end{itemize}
\end{itemize}

\begin{enumerate}
\item Liver Disease
\label{sec:orgf5bb3f4}
\begin{itemize}
\item Progressive liver disease in infancy.
\begin{itemize}
\item Presents in first months of life with:
\begin{itemize}
\item Failure to thrive and hepatomegaly.
\end{itemize}
\item Cirrhosis develops with eventual end stage liver disease and
portal hypertension.
\item Death is usual by 5 years of age.
\end{itemize}
\item Non-progressive liver disease in childhood.
\begin{itemize}
\item Present with hepatomegaly, liver dysfunction, hypotonia and
myopathy.
\item Liver disease does not progress, survival into adulthood.
\end{itemize}
\end{itemize}
\end{enumerate}

\item Clinical Presentation
\label{sec:orgc7aebe4}

\begin{enumerate}
\item Neuromuscular Disease
\label{sec:org2105d31}
\begin{itemize}
\item Congenital onset
\begin{itemize}
\item fetal loss in pregnancy,
\item fetal akinesia deformation sequence (FADS) with athrogryposis, hydrops and perinatal death
\item severe congenital myopathy similar to SMA with \textpm{} cardiomyopathy
\end{itemize}
\item Juvenile onset
\begin{itemize}
\item with a myopathy and/or cardiomyopathy
\end{itemize}
\item Adult onset
\begin{itemize}
\item adult polyglucosan body disease (APBD)
\item rarely myopathy
\end{itemize}
\end{itemize}
\end{enumerate}

\item Diagnostic Tests
\label{sec:org0270868}

\begin{itemize}
\item \(\uparrow\) transaminases in those with hepatic involvement
\item Fasting hypoglycaemia is uncommon except in end stage liver failure
\item Liver and muscle histology show swollen hepatocytes that contain
periodic acid-Schiff (PAS)-positive and diastase resistance
inclusions and evidence of interstitial fibrosis.
\item Enzyme analysis can be undertaken in liver tissue, cultured skin
fibroblast, peripheral lymphocytes and muscle
\item Confirmed by GBE1 mutation analysis.
\end{itemize}

\item Treatment
\label{sec:orgb31d97f}
\begin{itemize}
\item Liver transplant is the only treatment for the progressive liver form
\item Heart transplant may be considered in those with heart failure caused by cardiomyopathy.
\item There is no specific treatment for the other forms of the disease.
\end{itemize}
\end{enumerate}
\subsection{GSD Type VI}
\label{sec:orga6270a3}
\begin{enumerate}
\item Metabolic Derangement
\label{sec:orgbc6d5f3}
\begin{itemize}
\item GSD VI is caused by deficiency in hepatic glycogen phosphorylase
\item Catalyses the release and phosphorylation of terminal glucosyl units
from glycogen forming glucose-1-phosphate
\item Ketosis with or without hypoglycaemia may occur with fasting
\item Although plasma lipids may be raised
\item In severe variants recurrent hypoglycaemia and post-prandial lactic
acidosis can occur
\end{itemize}
\item Genetics
\label{sec:org24b99a6}
\begin{itemize}
\item AR, PGYL gene
\end{itemize}

\item Clinical Presentation
\label{sec:orga86ebcf}
\begin{itemize}
\item GSD VI is generally a mild disorder often diagnosed due to hepatomegaly.
\begin{itemize}
\item can present with symptomatic ketotic hypoglycaemia and growth retardation
\end{itemize}
\end{itemize}
\item Diagnostic Tests
\label{sec:org7e1c352}
\begin{itemize}
\item Diagnosis confirmed by mutation analysis or
\item Enzyme deficiency in hepatic tissue, erythrocytes, and leukocytes.
\item Enzyme activity may not always be reduced in blood and even in liver
tissue may be difficult to interpret due to residual activity and
the effect of other factors.
\item For example, deficiency of glycogen phosphorylase kinase will cause
low activity of glycogen phosphorylase.
\end{itemize}
\item Treatment
\label{sec:orgf5259a0}
\begin{itemize}
\item No treatment required for asymptomatic children
\item Those with growth failure or fasting ketosis benefit from regular
meals,snacks and uncooked cornstarch.
\item The outcome for individuals with GSD VI is generally excellent
\begin{itemize}
\item Catch up growth occurring for those with short stature in childhood.
\end{itemize}
\end{itemize}
\end{enumerate}
\subsection{GSD Type IX}
\label{sec:org30bfe0d}
\begin{enumerate}
\item Metabolic Derangement
\label{sec:org0e09c7d}
\begin{itemize}
\item GSD IX is caused by deficiency in hepatic glycogen phosphorylase kinase (PHK)
\item PHK phosphorylates glycogen phosphatase \emph{b} \(\to\) \emph{a} form
\begin{itemize}
\item inactive \emph{b} \(\to\)  active \emph{a}
\end{itemize}
\item Decreased PHK activity \(\to\) \(\uparrow\)
\item PHK is homotetramer in which each subunit is itself a tetramer
\begin{itemize}
\item \(\alpha\), \(\beta\), \(\gamma\) and \(\delta\) subunits.
\end{itemize}
\item The \(\gamma\) subunit is catalytic and the other subunits regulatory
\item There are tissue specific isoforms of the \(\alpha\) and \(\gamma\) subunits.
\item The \(\delta\) subunit, calmodulin, is ubiquitous
\end{itemize}

\item Genetics
\label{sec:org19ef54d}
\begin{center}
\begin{tabular}{lllll}
Type & Gene & Subunit & Inheritance & Tissue\\
\hline
IXa & PHKA2 & \(\alpha\)2 & XLR & liver \& blood\\
IXb & PHKB & \(\beta\) & AR & liver \& muscle\\
IXc & PHKG2 & \(\gamma\)TL & AR & live\\
IXd & PHKA1 & \(\alpha\)1 & AR & muscle\\
\end{tabular}
\end{center}

\item Clinical Presentation
\label{sec:org290fea1}
\begin{itemize}
\item Usually a benign disorder, with hepatomegaly often detected
incidentally
\item possible short stature, fasting hypoglycaemia and ketosis, with
raised liver transaminases, cholesterol and triglycerides.
\item Blood lactate and uric acid are normal. There is usually resolution
of signs and symptoms by adulthood.
\item GSD IXc can be more severe with an increased risk of hepatic fibrosis and cirrhosis
\end{itemize}

\item Diagnostic Tests
\label{sec:org475b966}
\begin{itemize}
\item Considered in children with unexplained hepatomegaly and in those with ketotic hypoglycaemia.
\item PHK can be measured in liver, erythrocytes and leukocytes.
\item However, in view of variable tissue expression enzyme assays may be
difficult to interpret.
\item Diagnosis is best achieved by mutation analysis using a DNA panel.
\end{itemize}
\item Treatment
\label{sec:org82bf7a0}
\begin{itemize}
\item Asymptomatic patients may not need treatment.
\item growth failure or symptomatic hypoglycaemia frequent meals and
uncooked cornstarch may be used.
\item Protein can be increased to 15 to 20\% of calories to provide a gluconeogenesis substrate.
\item The outcome for most patients is good with resolution of
hepatomegaly and catch up growth by adulthood.
\end{itemize}
\end{enumerate}

\section{Muscle Cardiac Glycogenoses}
\label{sec:orgd3a1d5d}
\subsection{Introduction}
\label{sec:orgd256891}
\begin{enumerate}
\item Muscle and Cardiac Glycogenoses
\label{sec:org4133777}

\begin{itemize}
\item Muscle predominately utilizes fatty acids for energy when at rest
\begin{itemize}
\item uses blood glucose during sub-maximal exercise.
\end{itemize}
\item Energy is supplied by glycogenolysis during intense exercise.
\begin{itemize}
\item glycogenolysis supplies glucose for anaerobic glycolysis
\end{itemize}
\end{itemize}

\item Muscle and Cardiac Glycogenoses
\label{sec:org2ca4bcf}

\scriptsize
\begin{center}
\begin{tabular}{llll}
Type & Enzyme & Gene & Phenotype\\
\hline
0 & Muscle Glycogen Synthase & GYS1 & cardiomyopathy \& myopathy\\
II (Pompe) & Acid \(\alpha\)-Glucosidase & GAA & hypotonia, muscle dysfunction\\
III & Glycogen Debrancher & AGL & hypoglycema, hepatomegaly\\
V (McArdle) & Muscle Glycogen Phosphorylase & PYGM & exercise intolerance\\
Danon & LAMP2 & LAMP2 & cardiomyopathy, \textpm{} skeletal myopathy\\
AMPK & AMP-Activated Protein Kinase & PRKAG2 & cardiomyopathy, \textpm{} skeletal myopathy\\
\end{tabular}
\end{center}


\item Muscle and Cardiac Glycogenoses
\label{sec:orgc3d0a7a}

\begin{figure}[htbp]
\centering
\includegraphics[width=0.75\textwidth]{./muscle_cardiac_glycogenoses/figures/gggmetab_muscle_cardiac.png}
\caption[Muscle and Cardiac Glycogenoses]{\label{fig:org05b9b00}
Muscle and Cardiac Glycogenoses}
\end{figure}
\end{enumerate}

\subsection{GSD Type 0}
\label{sec:orgaec116e}
\begin{enumerate}
\item GSD Type 0
\label{sec:orgec3210d}
\begin{enumerate}
\item Metabolic Derangement
\label{sec:org8d5fd62}
\begin{itemize}
\item Deficiency in muscle glycogen synthase
\item Ubiquitously expressed
\end{itemize}

\item Genetics
\label{sec:orgab8eb47}
\begin{itemize}
\item AR, GYS1 encodes muscle isoform of GS
\end{itemize}

\item Clinical Presentation
\label{sec:org49bd2be}
\begin{itemize}
\item muscle fatigue
\item hypertrophic cardiomyopathy
\end{itemize}

\item Diagnostic Tests
\label{sec:orge2a5d4b}
\begin{itemize}
\item mutation analysis
\end{itemize}
\item Treatment
\label{sec:org0e384b1}
\begin{itemize}
\item \(\beta_{\text{1}}\)-receptor blockage for cardiac protection
\end{itemize}
\end{enumerate}
\end{enumerate}

\subsection{GSD Type II (Pompe)}
\label{sec:orgbe9575e}
\begin{enumerate}
\item Metabolic Derangement
\label{sec:org946efdb}
\begin{itemize}
\item Acid \(\alpha\)-glucosidase deficiency
\item accumulation of glycogen within the lysosomes
\item with different critical thresholds depending on the organ
\end{itemize}

\item Genetics
\label{sec:orgadedbaa}
\begin{itemize}
\item AR, AGL gene
\item 200 mutations have been reported in GAA
\begin{itemize}
\item \textasciitilde{} 75\% of these being pathogenic mutations (www.pompecenter.nl)
\end{itemize}
\item some genotype-phenotype correlation
\begin{itemize}
\item severe mutations (such as del exon18) associated with the infantile form
\item ›leaky‹ mutations associated with the adult variant
\end{itemize}
\item c.-32-13T>G is the most common mutation in adults and children with
a slowly progressive course
\begin{itemize}
\item approximately 80\% of Caucasian patients.
\end{itemize}
\end{itemize}

\item Clinical Presentation
\label{sec:org460f7d7}
\begin{enumerate}
\item Infantile
\label{sec:orgec115c9}
\begin{itemize}
\item first months of life with hypotonia and hypertrophic cardiomyopathy
\item also dysphagia, smooth muscle dysfunction, enlargement of the tongue
and liver
\item Most untreated infantile onset patients die from cardiopulmonary
failure or aspiration pneumonia prior to one year of age
\end{itemize}
\item Juvenile
\label{sec:org937fee7}
\begin{itemize}
\item predominant skeletal muscle dysfunction
\begin{itemize}
\item with motor and respiratory problems, rarely cardiac involvement.
\item Calf hypertrophy can be present, mimicking Duchenne muscular dystrophy in boys.
\end{itemize}
\item Myopathy and respiratory insufficiency deteriorate gradually, and patients may become dependent on a ventilator or wheelchair.
\end{itemize}
\end{enumerate}
\item Clinical Presentation
\label{sec:org7c1428a}
\begin{enumerate}
\item Adult
\label{sec:orgeb0cb77}
\begin{itemize}
\item 3rd or 4th decade and affects the trunk and proximal limb muscles
\begin{itemize}
\item mimicks inherited limb-girdle muscle dystrophies.
\end{itemize}
\item Involvement of the diaphragm is frequent,
\begin{itemize}
\item acute respiratory failure may be the initial symptom in some patients.
\end{itemize}
\item the heart is generally not affected.
\end{itemize}
\end{enumerate}
\item Diagnostic Tests
\label{sec:orga39e07a}
\begin{itemize}
\item Acid \(\alpha\)-glucosidase enzyme assay
\begin{itemize}
\item classic infantile \textasciitilde{} 1\% residual activity
\item Children and Adults \(\le\) 30\% activity
\end{itemize}
\item Skin fibroblasts are best tissue
\begin{itemize}
\item Lower biochemical interference (neutral \(\alpha\)-glucosidases)
\end{itemize}
\item mutation analysis
\end{itemize}
\item Treatment
\label{sec:org4680b6f}
\begin{itemize}
\item Recombinant acid \(\alpha\)-glucosidase (rhGAA)
\begin{itemize}
\item CHO cells (alglucosidase alfa)
\end{itemize}
\item Anti rhGAA IgG antibodies form
\item 1/3 of ERT treated were ventilator free
\item Better outcome if identified by NBS
\end{itemize}
\end{enumerate}

\subsection{GSD Type V}
\label{sec:org78fcd80}
\begin{enumerate}
\item Metabolic Derangement
\label{sec:org2f6425f}
\begin{itemize}
\item There are three isoforms of glycogen phosphorylase: brain/heart,
liver and muscle, all encoded by different genes.
\item GSD V is caused by deficient myophosphorylase activity.
\end{itemize}

\item Genetics
\label{sec:orgf1f640a}
\begin{itemize}
\item AR, PYGM
\item \textgreater{} 100 known pathogenic mutations
\item p.R50X mutation, most common in Caucasians
\begin{itemize}
\item 81\% of the alleles in British patients
\item 63\% of alleles in US patients
\end{itemize}
\item No genotype-phenotype correlations have been detected
\item ACE polymorphism may be a phenotype modulator
\end{itemize}

\item Clinical Presentation
\label{sec:org4cfde5d}
\begin{itemize}
\item exercise intolerance with myalgia and stiffness in exercising muscles
\begin{itemize}
\item relieved by rest.
\end{itemize}
\item Onset of the disease occurs during childhood
\begin{itemize}
\item diagnosis is frequently missed at an early age
\item affected children are often considered lazy.
\end{itemize}
\item Myoglobinuria is the major complication, and occurs in about half of
the patients.
\item Creatine kinase (CK) can increase to more than 100,000-1,000,000
UI/l during episodes of rhabdomyolysis
\item Risk of acute renal failure
\end{itemize}

\item Diagnostic Tests
\label{sec:orgcf3fc66}

\begin{itemize}
\item ischaemic forearm exercise test (IFET) was first used by McArdle to
describe the absence of elevation of lactate during exercise.
\begin{itemize}
\item \textbf{Should not be used}
\end{itemize}

\item Non-ischemic FET has a sensitivity of 100\% in McArdle’s disease
\item Ammonia levels should be also assessed in parallel with lactate
\begin{itemize}
\item an abnormal increase in ammonia always observed in GSD V.
\end{itemize}
\item PYGM gene sequencing
\end{itemize}

\item Treatment
\label{sec:org048491f}
\begin{itemize}
\item no pharmacological treatment
\item exercise intolerance may be alleviated by:
\begin{itemize}
\item aerobic conditioning programs
\item ingestion of oral sucrose
\end{itemize}
\end{itemize}
\end{enumerate}
\subsection{LAMP 2 Deficiency (Danon Disease)}
\label{sec:org1526a74}
\begin{enumerate}
\item LAMP2 Deficiency (Danon Disease)
\label{sec:org83fb61f}
\begin{itemize}
\item Danon disease is a rare X-linked disorder
\item caused by a primary deficiency of lysosomal-associated membrane
protein 2 (LAMP2).
\item Presents after 1st decade
\begin{itemize}
\item cardiomyopathy all cases
\item mild skeletal myopathy and developmental delay 70\%
\end{itemize}
\item muscle biopsy shows glycogen filled vacuoles
\item consider cardiac transplantation
\end{itemize}
\end{enumerate}

\subsection{AMPK Deficiency}
\label{sec:org05fcc9f}
\begin{enumerate}
\item AMPK Deficiency
\label{sec:org9193901}
\begin{itemize}
\item AMPK controls whole-body glucose homeostasis by regulating metabolism in multiple peripheral tissues, such as
skeletal muscle, liver, adipose tissues, and pancreatic \(\beta\)-cells
\item activated \(\uparrow\) AMP/ATP ratio
\item stimulates glucose uptake and lipid oxidation to produce energy
\item inhibits energy-consuming processes including glucose and lipid production.
\end{itemize}

\item Metabolic Derangement
\label{sec:org9a4dd8c}
\begin{itemize}
\item AMPK is a heterotrimeric complex comprising:
\begin{itemize}
\item a catalytic subunit (\(\alpha\))
\item two regulatory subunits (\(\beta\) and \(\gamma\)).
\end{itemize}
\item Three isoforms of the gamma subunits are known (\(\gamma\)1, \(\gamma\)2 and \(\gamma\)3) with different tissue
expression
\end{itemize}

\item Genetics
\label{sec:org0bc4acb}
\begin{itemize}
\item The PRKAG2 gene coding for the \(\gamma\)-subunit of AMPK is located on chromosome 7q36.
\item Mutations in the \(\gamma\)2-subunit of AMPK are transmitted as an
autosomal dominant trait with full penetrance.
\end{itemize}

\item Diagnosis \& Treatment
\label{sec:org1102ba0}
\begin{itemize}
\item The differential diagnosis includes Pompe, Danon (LAMP2) and Fabry diseases.

\item diagnosis, if clinically suspected, is based on ECG,
echocardiography and molecular genetics.

\item Treatment includes a pacemaker/defibrillator and heart transplant.
\end{itemize}
\end{enumerate}

\section{Glycolysis and PPP}
\label{sec:org600e212}
\subsection{Introduction}
\label{sec:org5367e45}
\begin{enumerate}
\item Glycolysis
\label{sec:org4d0ef9d}
\begin{itemize}
\item Glycolysis is an oxygen-independent metabolic pathway
\begin{itemize}
\item converts each molecule of glucose to two of pyruvate.
\item consists of two phases and ten steps.
\end{itemize}
\item The first five steps are the preparatory phase,
\begin{itemize}
\item consumes ATP to convert glucose into two, three-carbon sugar
phosphate molecules.
\end{itemize}
\item In the other five steps ATP and NADH are produced
\end{itemize}

\item Pentose Phosphate Pathway
\label{sec:org269246a}
\begin{enumerate}
\item Oxidative Phase
\label{sec:org7f22f07}
\begin{itemize}
\item glucose 6-P \(\to\) NADPH + ribose 5-P
\item Glucose 6-P dehydrogenase catalyses first step
\item NADPH is for reducing reactions
\begin{itemize}
\item NADPH/NADP\(^{\text{+}}\) \textgreater{}\textgreater{}\textgreater{} NADH/NAD\(^{\text{+}}\)
\item NADH is rapidly converted to NAD\(^{\text{+}}\) in the ETC
\end{itemize}
\end{itemize}
\item Non-oxidative Phase
\label{sec:org5fc5e9c}
\begin{itemize}
\item reversible rxns
\item convert glycolytic intermediates to 5 carbon sugars
\end{itemize}
\end{enumerate}
\item Pentose Phosphate Pathway
\label{sec:org4407ba1}


\begin{itemize}
\item Ribose-5-P required for purine and pyrimidine synthesis
\item NADPH required for detoxification and synthetic reaction
\begin{itemize}
\item Detoxification
\begin{itemize}
\item Reduction of oxidized glutathione
\item Cytochrome p450 monoxygenases
\end{itemize}
\item Synthetic reactions
\begin{itemize}
\item FA synthesis
\item Cholesterol
\item neurotransmitters
\item deoxynucleotide
\item superoxide
\end{itemize}
\end{itemize}
\end{itemize}

\item Glycolysis and PPP
\label{sec:org011cfc0}

\begin{figure}[htbp]
\centering
\includegraphics[width=1\textwidth]{./glycolysis_ppp/figures/glyc_ppp.png}
\caption{\label{fig:org05e8d03}
Glycolysis and PPP}
\end{figure}
\end{enumerate}

\subsection{Disorders}
\label{sec:org0d33b2c}
\begin{enumerate}
\item Glycolysis Disorders
\label{sec:orga21b5aa}
\begin{itemize}
\item Glycolysis is the most important source of energy in erythrocytes
and in some types of skeletal muscle fibres

\begin{itemize}
\item \(\therefore\) IMD of glycolysis are mainly characterized by hemolytic
anaemia \textpm{} metabolic myopathy.
\end{itemize}

\item Ten inborn errors of the glycolytic pathway are known,
\begin{itemize}
\item 8 inherited as an autosomal recessive trait
\item X-linked phosphoglycerate kinase and glycerol kinase deficiencies.
\end{itemize}

\item Hexokinase (HK), glucose-6-phosphate isomerase (GPI) and pyruvate
kinase (PKD) deficiencies cause severe haemolytic anaemia.

\item Muscle phosphofructokinase (PFKM), aldolase A, triosephosphate
isomerase (TPI) and phosphoglycerate kinase (PGK) deficiencies are
characterized by haemolytic anaemia alone or coupled with
neurological disease and/or myopathy.
\end{itemize}

\item Glycolysis Disorders
\label{sec:orged485df}
\begin{itemize}
\item Phosphoglycerate mutase (PGAM), enolase and lactate dehydrogenase
(LDH) deficiencies present with a purely myopathic syndrome
\begin{itemize}
\item characterized by exercise induced cramps and myoglobinuria.
\end{itemize}

\item Glycerol kinase deficiency (GKD) is either an isolated condition
with hypoglycaemia and acidosis or part of a contiguous
gene deletion where it also associated with congenital adrenal
hypoplasia and/or Duchenne muscular dystrophy.

\item Glucose-6-phosphate can also be formed by the conversion of the
glycogen derived glucose-1-phosphate, a reaction catalysed by
phosphoglucomutase (PGM). PGM1 deficiency is a CDG
\end{itemize}

\item PPP Disorders
\label{sec:orgea4abc1}

\begin{itemize}
\item Four inborn errors in the pentose phosphate pathway (PPP) are known.
\item Glucose-6-phosphate dehydrogenase deficiency is an X-linked defect
in the first, irreversible step of the pathway.
\begin{itemize}
\item An exclusively haematological disorder.
\end{itemize}
\item Ribose-5-phosphate isomerase (RPI) deficiency has been described in one patient
\begin{itemize}
\item presented with developmental delay and a slowly progressive leukoencephalopathy.
\end{itemize}
\item Transaldolase (TALDO) deficiency often presents in the neonatal or
antenatal period with hepatosplenomegaly, \(\downarrow\) liver function,
hepatic fibrosis and anaemia.
\item Transketolase (TKT) deficiency presents with short stature,
developmental delay and congenital heart defects.
\end{itemize}
\end{enumerate}

\subsection{The non-ischemic forearm exercise test}
\label{sec:org5a75a85}
\begin{enumerate}
\item NIET in Myopathy
\label{sec:orgf1ec22f}

\begin{center}
\begin{tabular}{lll}
 & Lactate & Ammonia\\
\hline
GSD I & N & N\\
GSD III (L\&M) & \(\downarrow\) \(\downarrow\) & N/\(\uparrow\)\\
GSD V & \(\downarrow\) \(\downarrow\) & N/\(\uparrow\)\\
\textbf{GSD VII (PFK)} & \(\downarrow\) \(\downarrow\) & N/\(\uparrow\)\\
\textbf{GSD IX (PGK)} & \(\downarrow\) \(\downarrow\) & N/\(\uparrow\)\\
\textbf{GSD X (PGAM)} & \(\downarrow\) & N/\(\uparrow\)\\
Alcoholic myopathy & N & N\\
CFS & N & N\\
Poor effort & N/\(\downarrow\) & N/\(\downarrow\)\\
\end{tabular}
\end{center}

\item NIET Method
\label{sec:org8a4ca23}

\begin{figure}[htbp]
\centering
\includegraphics[width=0.9\textwidth]{./glycolysis_ppp/figures/niet_method.png}
\caption{\label{fig:org7d7d537}
NIET Method}
\end{figure}


\item Exercising Muscle: Lactate
\label{sec:orgda2e8a6}
\begin{itemize}
\item Lactate, ammonia and purine compounds are generated by exercising muscle.
\item Exercising muscle generates lactic acid from the anaerobic breakdown
of glycogen to pyruvate
\begin{itemize}
\item \ce{pyruvate \to lactate}
\end{itemize}
\item Lactate enters the circulation and is converted back to pyruvate in the liver.
\end{itemize}

\begin{figure}[htbp]
\centering
\includegraphics[width=0.5\textwidth]{./glycolysis_ppp/figures/Lactate_dehydrogenase_mechanism.png}
\caption{\label{fig:org7c8d65e}
LDH}
\end{figure}

\item Exercising Muscle: ATP
\label{sec:org63cc7bf}

\begin{itemize}
\item Some ATP regeneration is provided by glycolytic metabolism of fuels,
but this is relatively slow
\item Most ATP regeneration relys on creatine kinase catalysed transfer of
phosphate from phosphocreatine.

\begin{itemize}
\item \ce{phosphocreatine + ADP ->[CK] creatine + ATP}
\end{itemize}

\item adenylatekinase transphosphorylates ATP to be regenerated with the formation
of AMP

\begin{itemize}
\item \ce{2ADP ->[ADK] ATP + AMP}
\end{itemize}

\item AMP deaminase
\begin{itemize}
\item \ce{AMP ->[AMPD] IMP + NH4+}
\end{itemize}

\item IMP degraded to hypoxanthine
\item recycled back to AMP in the purine nucleotide cycle.
\end{itemize}

\item Exercising Muscle: Ammonia
\label{sec:org2958d03}
\begin{itemize}
\item Most ammonia produced by exercising muscle removed by formation of glutamine
\begin{itemize}
\item ultimately excreted as urea
\end{itemize}
\end{itemize}

\begin{figure}[htbp]
\centering
\includegraphics[width=0.6\textwidth]{./glycolysis_ppp/figures/nitrogen_glutamine.png}
\caption[gln]{\label{fig:org7afe822}
Glutamine and Ammonia}
\end{figure}

\begin{itemize}
\item Some ammonia is released by exercising skeletal muscle directly into the circulation
\begin{itemize}
\item removed with a half-life of 20\textpm{}30 min.
\end{itemize}
\item In resting skeletal muscle ammonia is consumed rather than produced
\item \textasciitilde{}50\% of arterial ammonia can be taken up and metabolized by skeletal muscle.
\end{itemize}

\item Interpretation
\label{sec:org4a76ab6}

\begin{figure}[htbp]
\centering
\includegraphics[width=.8\textheight]{./glycolysis_ppp/figures/niet_results.png}
\caption[interp]{\label{fig:orge70c662}
NIET Results}
\end{figure}
\end{enumerate}
\end{document}