% Created 2020-01-11 Sat 11:09
% Intended LaTeX compiler: pdflatex
\documentclass{scrartcl}
\usepackage[utf8]{inputenc}
\usepackage[T1]{fontenc}
\usepackage{graphicx}
\usepackage{grffile}
\usepackage{longtable}
\usepackage{wrapfig}
\usepackage{rotating}
\usepackage[normalem]{ulem}
\usepackage{amsmath}
\usepackage{textcomp}
\usepackage{amssymb}
\usepackage{capt-of}
\usepackage{hyperref}
\hypersetup{colorlinks,linkcolor=black,urlcolor=blue}
\usepackage{textpos}
\usepackage{textgreek}
\usepackage[version=4]{mhchem}
\usepackage{chemfig}
\usepackage{siunitx}
\usepackage{gensymb}
\usepackage[usenames,dvipsnames]{xcolor}
\usepackage[T1]{fontenc}
\usepackage{lmodern}
\usepackage{verbatim}
\usepackage{tikz}
\usepackage{wasysym}
\usetikzlibrary{shapes.geometric,arrows,decorations.pathmorphing,backgrounds,positioning,fit,petri}
\usepackage{fancyhdr}
\pagestyle{fancy}
\author{Matthew Henderson, PhD, FCACB}
\date{\today}
\title{Organic Acidemias}
\hypersetup{
 pdfauthor={Matthew Henderson, PhD, FCACB},
 pdftitle={Organic Acidemias},
 pdfkeywords={},
 pdfsubject={},
 pdfcreator={Emacs 26.1 (Org mode 9.1.9)}, 
 pdflang={English}}
\begin{document}

\maketitle
\tableofcontents


\section{Organic Acids}
\label{sec:orgfb9146d}
\subsection{What are Urine Organic Acids?}
\label{sec:orgcd3a925}
\begin{enumerate}
\item What are urine organic acids?
\label{sec:org4182190}
\begin{itemize}
\item Water soluble compounds containing \(\ge\) one carboxyl group(s) and
non-amino functional groups
\end{itemize}


\centering
\chemfig{X-C(-[2]X)(-[6]X)-C(-[2]X)(-[6]X)-C(-[7]OH)=[1]O}


\begin{enumerate}
\item Acylglycines
\label{sec:orgd409fbb}
\begin{itemize}
\item Acylglycines are also detected in UOA analysis
\begin{itemize}
\item conjugation of acyl-CoA species to glycine
\item catalysed by glycine N-acylase
\end{itemize}
\end{itemize}
\end{enumerate}

\item Nomenclature
\label{sec:org4e9a557}

\begin{center}
\begin{tabular}{lll}
Length & Monocarboxylic acid & Dicarboxylic acid\\
\hline
C2 & Acetic & Oxalic\\
C3 & Propionic & Malonic\\
C4 & Butyric & Succinic\\
 & Isobutyric & \\
C5 & Valeric & Glutaric\\
 & Isovaleric & \\
 & 2-Methylbutyric & \\
C6 & Hexanoic (caprioc) & Adipic\\
C7 & Heptanoic (enanthic) & Pimelic\\
C8 & Octanoic (caprylic) & Suberic\\
C9 & Nonanoic (pelargonic) & Azelaic\\
C10 & Decanoic (capric) & Sebacic\\
\end{tabular}
\end{center}

\item Functional Groups
\label{sec:org0cf5145}

\centering
\chemfig{X-C(-[2]X)(-[6]X)-C(-[2]X)(-[6]X)-C(-[7]OH)=[1]O}

\begin{center}
\begin{tabular}{ll}
Functional group & Formula\\
\hline
hydrogen & -H\\
keto & .=O\\
hydroxyl & -OH\\
carboxyl & -C(=O)OH\\
side chain & -(CH\(_2\))\(_n\)\\
\end{tabular}
\end{center}

\item Side Chains
\label{sec:org728a27d}
\centering
\chemfig{X-C(-[2]X)(-[6]X)-C(-[2]X)(-[6]X)-C(-[7]OH)=[1]O}

\begin{center}
\begin{tabular}{ll}
Side chain & Structure\\
\hline
Methyl & \chemfig{CH_3-}\\
Ethyl & \chemfig{CH_3-CH_2-}\\
Propyl & \chemfig{CH_3-CH_2-CH_2-}\\
Butyl & \chemfig{CH_3-CH_2-CH_2-CH_2-}\\
\end{tabular}
\end{center}

\item Classes of organic acids detected in the urine of  healthly subjects
\label{sec:org9afc343}

\begin{itemize}
\item Tricarboxylic acid cycle acids
\begin{itemize}
\item citric
\end{itemize}
\item hydroxyaliphatic acids
\begin{itemize}
\item 3-hydroxybutyric
\end{itemize}
\item aliphatic keto acids
\begin{itemize}
\item pyruvic
\end{itemize}
\item aliphatic acids
\begin{itemize}
\item oxalic
\end{itemize}
\item aldonic and deoxyaldonic acids (sugar acids)
\begin{itemize}
\item 3,4-dihyroxybutanoic
\end{itemize}
\item aromatic acids
\begin{itemize}
\item hippuric
\end{itemize}
\item Short chain acids
\begin{itemize}
\item formic
\end{itemize}
\end{itemize}

\item Where do they come from?
\label{sec:orgc88c531}
\begin{itemize}
\item Intermediate metabolism of all major groups of organic cellular
components
\end{itemize}

\begin{enumerate}
\item :BMCOL:
\label{sec:org591720d}
\begin{enumerate}
\item Endogenous Sources
\label{sec:org4f1d939}
\begin{itemize}
\item amino acids
\begin{itemize}
\item 2-hydroxyisocaproate - MSUD
\end{itemize}
\item fatty acids
\begin{itemize}
\item suberic - MCADD
\end{itemize}
\item carbohydrates
\begin{itemize}
\item lactate - multiple causes
\end{itemize}
\item nucleic acids
\begin{itemize}
\item uracil - dihyropyrimidine dehydrogenase def
\end{itemize}
\item steroids
\begin{itemize}
\item 3-hydroxypropionic - PA, MMA
\end{itemize}
\item neurotransmitters
\begin{itemize}
\item vanillactic - neuroblastoma
\end{itemize}
\end{itemize}
\end{enumerate}
\item :BMCOL:
\label{sec:org4cb6b4a}
\begin{enumerate}
\item Exogenous
\label{sec:orgb7c5a38}
\begin{itemize}
\item food
\begin{itemize}
\item adipic - gelatin
\item furan dicarboxylate - heated sugar
\item vanillactic - bananas
\end{itemize}
\item environment
\begin{itemize}
\item palmitic - soap
\end{itemize}
\item medications
\begin{itemize}
\item ibuprofen
\end{itemize}
\item gut bacteria
\begin{itemize}
\item methylmalonic
\end{itemize}
\end{itemize}
\end{enumerate}
\end{enumerate}

\item Why are they measured in urine?
\label{sec:org8ce693f}

\begin{itemize}
\item not extensively reabsorbed in the kidney tubules after glomerular
filtration
\begin{itemize}
\item can be present at 100x concentration in blood
\end{itemize}
\item readily available sample type
\item less invasive than serum
\item Over 500 organic acids have been identified in urine.
\begin{itemize}
\item a clinical metabolomics test
\end{itemize}
\end{itemize}
\end{enumerate}

\subsection{Why is Urine Organic Acid Analysis Requested}
\label{sec:org3fcdf93}

\begin{enumerate}
\item Clinical Indications for UOA analysis
\label{sec:org6065974}

\begin{itemize}
\item Neonatal or late-onset acute illness associated with:
\begin{itemize}
\item hyperammonemia
\item hypoglycemia, and/or ketolactic acidosis
\item neurologic abnormalities
\begin{itemize}
\item seizures
\item ataxia
\item hypotonia
\item lethargy
\item coma
\item developmental delay
\item unexplained intellectual disability
\end{itemize}
\item failure to thrive
\item pancreatitis
\item unexplained metabolic acidosis
\item unusual odor
\item macrocephaly
\item liver failure
\end{itemize}
\item Some symptoms, including lethargy and acidosis, can be due to exogenous intoxication
\begin{itemize}
\item ethylene glycol poisoning
\item ibuprofen overdose
\item \(\gamma\)-hydroxybutyric acid
\end{itemize}
\end{itemize}

\item Reasons for Abnormal Urine Organic Acids profiles
\label{sec:org4ef8e31}
\begin{itemize}
\item Elevated concentration of normal metabolites
\begin{itemize}
\item fumaric acid in fumarase deficiency
\item adipic, suberic, and sebacic acids in MCADD
\item ketones in fasting
\begin{itemize}
\item 3-hydroxybutyric
\item acetoacetic
\end{itemize}
\end{itemize}

\item Pathological metabolites
\begin{itemize}
\item succinylacetone, methylcitric acid
\end{itemize}

\item Food, medications, environment
\begin{itemize}
\item ethosuximide
\item adipic acid
\item cresol
\item 2-furaldehyde
\end{itemize}
\end{itemize}

\item Disorders that can be identified by UOA analysis
\label{sec:org279a24e}

\begin{enumerate}
\item classic organic acidemias
\label{sec:orge46ef15}
\begin{itemize}
\item isovaleric acidemia [MIM 243500]
\item methylmalonic acidemia[s] propionic acidemia [MIM 606054]
\item glutaric acidemia type I [MIM 231670]
\end{itemize}
\item amino acidopathies
\label{sec:orga3fa0f1}
\begin{itemize}
\item phenylketonuria [MIM 261600]
\item tyrosinemia type I [MIM 276700]
\item alkaptonuria [MIM 203500]
\item 3-methylglutaconic aciduria type I [MIM 250950]
\item maple syrup urine disease [MIM 248600]
\end{itemize}
\end{enumerate}

\item Disorders that can be identified by UOA analysis
\label{sec:org98efd57}
\begin{enumerate}
\item Mitochondrial disorders
\label{sec:org299e02c}
\begin{itemize}
\item pyruvate dehydrogenase deficiency
\item fumarase deficiency [MIM 606812]
\item SUCLA2 deficiency [MIM 603921]
\end{itemize}
\item fatty acid oxidation
\label{sec:orgfb59c1c}
\begin{itemize}
\item short-chain acyl-CoA dehydrogenase deficiency [MIM 201470]
\item medium-chain acyl-CoA dehydrogenase deficiency [MIM 201450]
\item multiple acyl-CoA dehydrogenase deficiency [MIM 231680]
\end{itemize}

\item purine and pyrimidine metabolism
\label{sec:orgc43c9b4}
\begin{itemize}
\item uridine monophosphate synthetase deficiency [MIM 613891]
\item dihydropyrimidine dehydrogenase deficiency [MIM 274270]
\end{itemize}
\end{enumerate}

\item Disorders that can be identified by UOA analysis
\label{sec:org573180a}

\begin{enumerate}
\item neurotransmission
\label{sec:orgd34e509}
\begin{itemize}
\item aromatic L-amino acid decarboxylase deficiency [MIM 608643]
\item ethylmalonic encephalopathy [MIM 602473]
\item Canavan disease [MIM 271900]
\end{itemize}
\item others
\label{sec:org98bb61d}
\begin{itemize}
\item Ornithine transcarbamylase deficiency [MIM 311250]
\item glutathione synthetase deficiency [MIM 266130]
\item glycerol kinase deficiency [MIM 307030]
\item primary hyperoxaluria type I [MIM 259900]
\item primary hyperoxaluria type II [MIM 260000]
\end{itemize}
\end{enumerate}

\item Naturopathic Medicine
\label{sec:org67ac492}

\begin{itemize}
\item labs provide the urine organic acid testing 
\begin{itemize}
\item Great Plains Laboratory - OAT
\item Genova Diagnostics - Organix
\item Analytical Reference Laboratories - Urinary Organic Acids

\item Targeted to Naturopaths and  "Functional Medicine"
\item Provide impressive graphical reportes with recomendations for diet and supplements
\end{itemize}
\end{itemize}
\end{enumerate}

\subsection{Examples}
\label{sec:org8e3a066}

\begin{enumerate}
\item Case 1
\label{sec:org618e744}

\begin{itemize}
\item A 2-week-old female is referred to metabolic clinic following positive newborn screen.
\item 3 OH Glutaric acid = 121 mmol/mol UCR, RI <1
\end{itemize}

\item Case 1 UOA
\label{sec:org6763056}

\begin{center}
\includegraphics[width=.9\linewidth]{./organic_acids/figures/case1uoa.png}
\end{center}

\pause

\begin{itemize}
\item Elevated glutaric acid and 3-hydroxyglutaric acid is consistent with
Glutaryl-CoA dehydrogenase deficiency (GA-1).
\end{itemize}

\item Case 2
\label{sec:orgc43c410}

\begin{itemize}
\item 5 day old girl, with C3 acylcarnitine positive newborn screen.
\item MMA = 357 mmol/mol UCR, RI <10
\end{itemize}


\item Case 2 UOA
\label{sec:org875441c}

\begin{center}
\includegraphics[width=.9\linewidth]{./organic_acids/figures/case2uoa.png}
\end{center}

\pause

\begin{itemize}
\item Significant elevation in methylmalonic acid, with the presence of
methylcitric acid is consistent with methlymalonic acidemia
and cobalamin metabolism defects.
\end{itemize}

\item Case 3
\label{sec:orge2b888f}

\begin{itemize}
\item - a 5 yo female who had come to the ED twice in the last year with
vomiting, dehydration and hypoglycemia.

\item urine organic acids elevation in alpha-ketoacids 
\begin{itemize}
\item 2-ketoisocaproic
\item 2-ketoisovaleric
\item 2-keto-3-methylvaleric
\item urine creatinine = 4,113 umol/L
\end{itemize}
\end{itemize}


\item Case 3 UOA
\label{sec:orgcb631aa}

\begin{center}
\includegraphics[width=.9\linewidth]{./organic_acids/figures/fasting.png}
\end{center}

\pause

\begin{itemize}
\item Elevation in alpha-keto acids: 2-ketoisocaproic, 2-ketoisovaleric
and 2-keto-3-methylvaleric acids,
\item These intermediate metabolites of branched chain amino acid
metabolism can be elevate in fasting and maple syrup urine
disease.
\end{itemize}
\end{enumerate}

\section{Disorders of Branched Chain Amino Acid Catabolism}
\label{sec:orge78a730}
\subsection{Introduction}
\label{sec:org7c8e843}
\begin{enumerate}
\item BCAAs
\label{sec:org2d5706d}

\centering
\chemname{\chemfig[][scale=.75]{^{+}H_3N-C(-[2]COO^{-})(-[6]CH(-[7]CH_3)(-[5]CH_3))-H}}{\small valine}
\chemname{\chemfig[][scale=.75]{^{+}H_3N-C(-[2]COO^{-})(-[6]CH_2-[6]CH(-[7]CH_3)(-[5]CH_3))-H}}{\small leucine}
\chemname{\chemfig[][scale=.75]{^{+}H_3N-C(-[2]COO^{-})(-[6]CH(-CH_3)-[6]CH_2-[6]CH_3)-H}}{\small isoleucine}


\item BCAA Catabolism
\label{sec:orgb37dfdc}

\begin{figure}[htbp]
\centering
\includegraphics[width=0.9\textwidth]{./bcaa_oa/figures/bcaa.png}
\caption[BCAA Catabolism]{\label{fig:orgca0cfe3}
BCAA Catabolism}
\end{figure}
\end{enumerate}

\subsection{BCAA Organic Acidurias}
\label{sec:org1f28cbb}

\begin{enumerate}
\item Blocks Distal to BCKD
\label{sec:orgd58ba74}
\begin{itemize}
\item Do no accumulate AAs or 2-oxo-acids
\item Lead to accumulation of intermediates proximal to the block
\item Metabolites detected by plasma acylcarnitines or urine organic acids
\end{itemize}
\begin{enumerate}
\item Leucine
\label{sec:org366699b}
\begin{itemize}
\item isovaleric acidemia
\item 3-MCC carboxylase
\item 3-Methyl-glutaconic aciduria
\item 3-OH-2-methylglutaryl-CoA lyase
\end{itemize}

\item Isoleucine
\label{sec:org3eab091}
\begin{itemize}
\item mitochondrial acetoacetyl-CoA thiolase
\item short/branched chain acyl-CoA dehydrogenase
\item 2-methyl-3-OH-butyryl-CoA dehydrogenase
\end{itemize}

\item Valine
\label{sec:orgd7e4d6c}
\begin{itemize}
\item isobutyryl-CoA dehydrogenase
\item Short-Chain enoyl-CoA Hydratase Deficiency
\begin{itemize}
\item AKA: Crotonase
\end{itemize}
\item 3-OH-isobutyryl-CoA hydrolase
\begin{itemize}
\item 2-OH-isobutyric aciduria
\end{itemize}
\item methylmalonic semialdehyde dehydrogenase
\end{itemize}
\end{enumerate}

\item Nomenclature
\label{sec:orgadb80f4}

\begin{center}
\begin{tabular}{lll}
Length & Monocarboxylic acid & Dicarboxylic acid\\
\hline
C2 & Acetic & Oxalic\\
C3 & Propionic & Malonic\\
C4 & Butyric & Succinic\\
 & Isobutyric & \\
C5 & Valeric & Glutaric\\
 & Isovaleric & \\
 & 2-Methylbutyric & \\
C6 & Hexanoic (caprioc) & Adipic\\
C7 & Heptanoic (enanthic) & Pimelic\\
C8 & Octanoic (caprylic) & Suberic\\
C9 & Nonanoic (pelargonic) & Azelaic\\
C10 & Decanoic (capric) & Sebacic\\
\end{tabular}
\end{center}

\item Functional Groups
\label{sec:orgad7dd11}

\centering
\chemfig{X-C(-[2]X)(-[6]X)-C(-[2]X)(-[6]X)-C(-[7]OH)=[1]O}

\begin{center}
\begin{tabular}{ll}
Functional group & Formula\\
\hline
hydrogen & -H\\
keto & .= O\\
hydroxyl & -OH\\
carboxyl & -COOH\\
side chain & -(CH\(_2\))\(_n\)\\
\end{tabular}
\end{center}

\item Side Chains
\label{sec:orgc1dd867}

\centering
\chemfig{X-C(-[2]X)(-[6]X)-C(-[2]X)(-[6]X)-C(-[7]OH)=[1]O}

\begin{center}
\begin{tabular}{ll}
Side chain & Structure\\
\hline
Methyl & \chemfig{CH_3-}\\
Ethyl & \chemfig{CH_3-CH_2-}\\
Propyl & \chemfig{CH_3-CH_2-CH_2-}\\
Butyl & \chemfig{CH_3-CH_2-CH_2-CH_2-}\\
\end{tabular}
\end{center}


\item Propionic Acidemia / Methylmalonic Acidemia
\label{sec:org1f52101}

\begin{itemize}
\item Final step in valine and isoleucine metabolism
\begin{itemize}
\item Propiogenic substrates
\begin{itemize}
\item Met
\item Thr
\item Odd-chain fatty acids
\item Cholesterol
\end{itemize}
\end{itemize}
\end{itemize}
\end{enumerate}

\section{Propionic Acidemia}
\label{sec:org11528dd}
\subsection{Introduction}
\label{sec:org8961c3e}
\begin{enumerate}
\item History
\label{sec:org8602665}
\begin{itemize}
\item Was called Ketotic Hyperglycinemia
\item A patient with PA was reported in 1961 with hyperglycinemia
\begin{itemize}
\item recurrent ketoacidosis
\item \(\uparrow\) \(\uparrow\) glycine in blood and urine
\item attacks related to the intake of protein
\begin{itemize}
\item administration of BCAA, Thr, Met \(\to\) ketonuria
\end{itemize}
\end{itemize}
\item Index patient and sister had defective propionyl carboxylase.
\item \(\uparrow\) clinical penetrance
\begin{itemize}
\item incidence of PA has not \(\uparrow\) with NBS
\end{itemize}
\end{itemize}

\item Propionic Acid and Derivatives
\label{sec:orgb81e8a2}

\vspace{2em}
\chemname{\chemfig[][scale=.5]{-[7]-[1]([2]=O)-[7]OH}}{\tiny propionic acid}
\hspace{4em}
\chemname{\chemfig[][scale=.5]{-[7]-[1]([2]=O)-[7]CoA}}{\tiny propionyl CoA}

\vspace{2em}
\chemname{\chemfig[][scale=.5]{-N^{+}([2]-)([6]-)-[1]-[7]([6]-O-([5]=O)-[7,.6]-[1,.6])-[1]-[7]([7]=O)([1]-O^{-})}}{\tiny propionyl-carnitine}
\hspace{4em}
\chemname{\chemfig[][scale=.5]{OH-[1]-[7]-[1]([2]=O)-[7]OH}}{\tiny 3-hydroxypropionic acid}


\item Propionic Acidemia Pathway
\label{sec:org3d3d552}

\begin{figure}[htbp]
\centering
\includegraphics[width=0.9\textwidth]{./pa/figures/pa_path.png}
\caption[PA]{\label{fig:orgca80434}
Propionic Acidemia}
\end{figure}

\item BCAA catabolism
\label{sec:org64b7d97}

\begin{figure}[htbp]
\centering
\includegraphics[width=0.9\textwidth]{./pa/figures/bcaa.png}
\caption[BCAA catabolism]{\label{fig:orgcf151ac}
BCAA catabolism}
\end{figure}

\item AA catabolism
\label{sec:orgc624603}

\begin{figure}[htbp]
\centering
\includegraphics[width=0.9\textwidth]{./pa/figures/aa_met.png}
\caption[AA catabolism]{\label{fig:orgc5d1b7a}
Amino Acid Catabolism}
\end{figure}

\item Odd-Chain Length Fatty Acids
\label{sec:orgcba3ff2}

\begin{figure}[htbp]
\centering
\includegraphics[width=0.9\textwidth]{./pa/figures/23_10.png}
\caption[Odd-chain length FAs]{\label{fig:org3d32a46}
Odd-chain length fatty acids}
\end{figure}

\item Long-Chain Branched-Chain Fatty Acids
\label{sec:orgf5fe214}

\begin{figure}[htbp]
\centering
\includegraphics[width=0.9\textwidth]{./pa/figures/ff22.png}
\caption[Long-Chain Branched-Chain Fatty Acids]{\label{fig:orgfa0a79a}
Long-Chain Branched-Chain Fatty Acids}
\end{figure}

\begin{itemize}
\item \(\alpha\)-oxidation of phytanic acid takes place in peroxisomes.
\item Pristanic acid can then undergo \(\beta\)-oxidation.
\begin{itemize}
\item Propionyl-CoA is released when the \(\alpha\) carbon is substituted
\end{itemize}
\end{itemize}

\item Sterol Catabolism
\label{sec:org3b0ba5b}

\begin{figure}[htbp]
\centering
\includegraphics[width=0.9\textwidth]{./pa/figures/gr3.jpg}
\caption[Sterol Catabolism]{\label{fig:org1c18b6c}
Sterol Catabolism}
\end{figure}

\item Propionyl Carboxylase
\label{sec:orgc833501}

\begin{figure}[htbp]
\centering
\includegraphics[width=0.9\textwidth]{./pa/figures/pc.jpg}
\caption[PC]{\label{fig:org40e358d}
Propionyl Carboxylase}
\end{figure}

\begin{itemize}
\item Composed of \(\alpha\) and \(\beta\) subunits
\begin{itemize}
\item \(\alpha_{\text{4}} \beta\)\(_{\text{4}}\) heteropolymer
\end{itemize}
\item Apoenzyme activated by covalent binding to biotin
\end{itemize}
\end{enumerate}

\subsection{Genetics and Pathogenisis}
\label{sec:org0ae85d3}

\begin{enumerate}
\item Genetics
\label{sec:org785212f}
\begin{itemize}
\item autosomal recessive trait
\item Two complementation groups
\begin{itemize}
\item PccA \(\to\) \(\alpha\)
\item PccBC \(\to\) \(\beta\)
\end{itemize}
\item \(\alpha\) 13q32
\item \(\beta\) 3q13.3-22
\end{itemize}

\item Propionyl CoA
\label{sec:org34f372a}
\begin{enumerate}
\item Hyperglycinemia
\label{sec:orge662baf}
\begin{itemize}
\item PA inhibits synthesis of glycine cleaving enzyme.
\end{itemize}
\item CBC
\label{sec:org691eeca}
\begin{itemize}
\item Propionyl-CoA toxic to bone marrow
\begin{itemize}
\item neutropenia
\item transient thrombocytopenia in infancy
\end{itemize}
\end{itemize}
\item Hyperammonemia
\label{sec:org7d95f14}
\begin{itemize}
\item PA inhibits carbamylphosphate synthetase
\end{itemize}
\item Ketosis
\label{sec:org4e2e206}
\begin{itemize}
\item PA inhibits mitochondrial oxidation of succinic acid and 2-ketoglutaric acid
\end{itemize}
\end{enumerate}

\item Biochemical Markers
\label{sec:orgd2bf8dc}
\begin{itemize}
\item 3-hydroxypropionic acid
\item tiglic acid / tiglyglycine
\item propionylglycine
\item methylcitrate
\begin{itemize}
\item condensation of propionyl-CoA \& oxaloacetic acid
\item metabolic end product and very stable
\end{itemize}
\end{itemize}
\end{enumerate}

\subsection{Laboratory Investigations}
\label{sec:orgc770de8}
\begin{enumerate}
\item NSO PA/MMA Screening Logic
\label{sec:org5e3bb2f}
\begin{enumerate}
\item Inital positive \textless{} 7 days
\label{sec:org0cc5d4c}
(C3/C2 \(\ge\) 0.21 AND C3 \(\ge\) 4.0)
OR
(C3/C2 \(\ge\) 0.23 AND C3 \(\ge\) 3.5)
\item Inital positive \textgreater{} 7 days
\label{sec:orgbb55bca}
(C3/C2 \(\ge\) 0.21 AND C3 \(\ge\) 2.6)
OR
(C3/C2 \(\ge\) 0.23 AND C3 \(\ge\) 2.4)
\begin{itemize}
\item Repeat overnight
\item No weekend reporting
\end{itemize}
\item Alert
\label{sec:orgfe61764}
C3/C2 \(\ge\) 0.3 AND C3 \(\ge\) 9.0
\begin{itemize}
\item Repeat same day
\item Weekend reporting
\end{itemize}
\item Confirmation
\label{sec:org5dda07e}
C3/C2 \(\ge\) 0.23 AND MCA \(\ge\) 0.5
\end{enumerate}
\item Elevated C3 ACT algorithm
\label{sec:org552efd4}
\begin{figure}[htbp]
\centering
\includegraphics[width=0.9\textwidth]{./pa/figures/pa_act.png}
\caption[C3 ACT algorithm]{\label{fig:org1d1034b}
C3 ACT algorithm}
\end{figure}

\item Clinical Chemistry
\label{sec:orgb203989}
\begin{itemize}
\item Acidosis in acute episodes
\begin{itemize}
\item accumulation of \(\beta\)-hydroxybutyrate and acetoacetate
\item Arterial pH as low as 6.9
\item Bicarb as low as 5 mEq/L
\end{itemize}
\item \(\uparrow\) lactic acid
\item Hypoglycemia
\item Hyperammonemia
\end{itemize}

\item Biochemical Genetics
\label{sec:orgc60c1d7}
\begin{enumerate}
\item Plasma Amino Acids
\label{sec:org70d31d5}
\begin{itemize}
\item \(\Uparrow\) glycine
\item \(\uparrow\) glutamine when hyperammonemia
\end{itemize}
\item Plasma Acylcarnitines
\label{sec:orge8a9dd0}
\begin{itemize}
\item \(\uparrow\) propionyl carnitine (C3)
\end{itemize}
\item Urine Organic Acids
\label{sec:orgb1fbf3f}
\begin{itemize}
\item 3-hydroxypropionic acid
\item methylcitric acid
\item lactic acid
\item BHB
\item acetoacetate
\item tiglic acid / tiglyglycine
\end{itemize}
\end{enumerate}

\item Urine Organic Acids
\label{sec:org013e323}
\begin{figure}[htbp]
\centering
\includegraphics[width=0.9\textwidth]{./pa/figures/pa_uoa.png}
\caption[pa OAs]{\label{fig:orgb053f2e}
PA Organic Acids}
\end{figure}
\end{enumerate}

\subsection{Clinical Findings}
\label{sec:org57f373c}
\begin{enumerate}
\item Acute presentation
\label{sec:org46dcb0f}
\begin{itemize}
\item Life-threatening illness early in life
\begin{itemize}
\item ketonuria
\begin{itemize}
\item acidosis
\item dehydration
\end{itemize}
\item vomiting
\item lethargy \(\to\) coma
\end{itemize}
\end{itemize}

\item Recurrent Symptoms
\label{sec:orgc39ce4e}
\begin{itemize}
\item ketotic episodes
\item infection
\item protein intolerance
\end{itemize}

\item Long term
\label{sec:orgfbf64cc}
\begin{itemize}
\item Variable developmental/cognitive outcomes
\begin{itemize}
\item appears linked to incidence of illness
\end{itemize}
\item hypotonic
\begin{itemize}
\item developmental delay
\end{itemize}
\item A subset with exclusively neurological presentation
\begin{itemize}
\item \textpm{} ketoacidosis
\item hypotonia \(\to\) hypertonia
\end{itemize}
\item Propionyl-CoA toxic to bone marrow
\begin{itemize}
\item neutropenia
\item transient thrombocytopenia in infancy
\end{itemize}
\item Osteoporosis
\item Pancreatitis
\item Cardiomyopathy
\item Fatty infiltration of liver on PM
\end{itemize}

\item Neurological Findings
\label{sec:org82c313d}
\begin{itemize}
\item Neonatal death
\begin{itemize}
\item spongy degeneration of white matter
\end{itemize}
\item Later death
\begin{itemize}
\item shrinkage and marbling in basal ganglia
\item neuronal loss
\item gliosis
\end{itemize}
\end{itemize}

\begin{figure}[htbp]
\centering
\includegraphics[width=0.9\textwidth]{./pa/figures/pa_mri.png}
\caption[PA MRI]{\label{fig:org1317659}
PA MRI}
\end{figure}

\item Long-term Treatment
\label{sec:org3141b95}
\begin{enumerate}
\item Diet
\label{sec:org814b1be}
\begin{itemize}
\item Limit Val, Ile, Thr, Met
\begin{itemize}
\item Monitor urine metabolites, plasma amino acids
\item Urine ketones (daily in infancy)
\item Monitor weight, nitrogen balance
\end{itemize}
\item Avoid fasting
\begin{itemize}
\item Catabolism
\item Propionate release from lipids
\end{itemize}
\end{itemize}

\item Supplementation
\label{sec:org531350c}
\begin{itemize}
\item Carnitine
\begin{itemize}
\item excretion of carnitine esters \(\to\) detoxification
\item Daily dose 60 to 100 mg/kg
\end{itemize}
\item Biotin
\begin{itemize}
\item conflicting information
\end{itemize}
\end{itemize}
\end{enumerate}
\end{enumerate}

\section{Methylmalonic Acidemia}
\label{sec:org9a95f4c}
\subsection{Introduction}
\label{sec:orgb74bf66}
\begin{enumerate}
\item History
\label{sec:orgbf1cdc7}
\begin{itemize}
\item First reported in 1967 by Oberholzer and Stokke
\item Defect in methylmalonic mutase
\begin{itemize}
\item catalyzes the isomerization of methylmalonyl-CoA to succinyl-CoA
\end{itemize}
\item Requires adenosylcobalamin as a cofactor
\begin{itemize}
\item derived from vitamin B\(_{\text{12}}\)
\end{itemize}
\item Genetic heterogeneity was observed early
\begin{itemize}
\item response to large doses of B\(_{\text{12}}\)
\end{itemize}
\item B\(_{\text{12}}\) responsive have defects in adenosylcobalamin synthesis
\item B\(_{\text{12}}\) unresponsive have defects in the apoenzyme methylmalonyl mutase
\begin{description}
\item[{mut\^{}-}] little activity
\item[{mut\(^{\text{0}}\)}] no activity
\end{description}
\end{itemize}

\item Methylmalonyl CoA Mutase
\label{sec:org0f35ba0}
\begin{itemize}
\item Upon entry to the mitochondria, the 32 amino acid mitochondrial
leader sequence at the N-terminus of the protein is cleaved, forming
the fully processed monomer.
\item The monomers associate into homodimers, and bind AdoCbl (one
for each monomer active site) to form the active holoenzyme
\end{itemize}

\begin{center}
\includegraphics[width=.9\linewidth]{./mma/figures/mut.pdf}
\end{center}

\item Genetics
\label{sec:org288b514}
\begin{itemize}
\item autosomal recessive
\item 1/50000
\begin{itemize}
\item 1/2-2/3 have mutase apoenzyme defect
\end{itemize}
\item Methymalonyl pathway
\item Cobalamin transport and metabolism
\end{itemize}

\item Methylmalonyl CoA Pathway
\label{sec:orgbb45449}
\begin{itemize}
\item Methylmalonyl CoA Epimerase deficiency
\item Methylmalonyl CoA mutase deficiency
\begin{itemize}
\item mut\(^{\text{0}}\), mut\(^{\text{-}}\)
\item >200 mutations in MUT locus identified
\end{itemize}
\item Succinyl CoA ligase deficiency (SUCLG1, SUCLA1)
\end{itemize}

\item Methylmalonyl-CoA  Pathway
\label{sec:org759e880}

\begin{figure}[htbp]
\centering
\includegraphics[width=0.9\textwidth]{./mma/figures/expanded_mma_path.png}
\caption{\label{fig:org71e1c5e}
Methylmalonyl-CoA  Pathway}
\end{figure}

\item Cobalamin Transport and Metabolism
\label{sec:orge0f98b3}
\begin{itemize}
\item Adenosyltransferase deficiency (Cbl B)
\item Cbl A
\item Homocystinuria w MMA (Cbl C, D)
\item B\(_{\text{12}}\) deficiency
\begin{itemize}
\item Dietary (vegan)
\item Pernicious anemia
\end{itemize}
\item Transcobalamin II deficiency
\item B\(_{\text{12}}\) transport from lysosome defect (Cbl F)
\end{itemize}

\item Cobalamin Transport and Metabolism
\label{sec:orgecf1dd8}
\begin{figure}[htbp]
\centering
\includegraphics[width=0.9\textwidth]{./mma/figures/cbl_path.png}
\caption[cbl transport]{\label{fig:org3fadf67}
Cobalamin Transport and Metabolism}
\end{figure}

\item Methylmalonic Acid and Derivatives
\label{sec:org24cd350}

\vspace{6em}
\chemname{\chemfig[][scale=.5]{OH-[1]([2]=O)-[7]([6]<)-[1]([2]=O)-[7]S-CoA}}{\tiny S-methylmalonyl-CoA}
\hspace{2em}
\chemname{\chemfig[][scale=.5]{OH-[1]([2]=O)-[7]([6]<:)-[1]([2]=O)-[7]S-CoA}}{\tiny R-methylmalonyl-CoA}
\hspace{2em}
\chemname{\chemfig[][scale=.5]{OH-[1]([2]=O)-[7]([6]-)-[1]([2]=O)-[7]OH}}{\tiny methylmalonic acid}


\item Sources of Methylmalonic Acid
\label{sec:org56bd3eb}
\begin{itemize}
\item AA catabolism
\begin{itemize}
\item isoleucine
\item valine
\item threonine
\item methionine
\end{itemize}
\item FA
\begin{itemize}
\item \(\downarrow\) contribution to urine MMA
\end{itemize}
\end{itemize}

\item Methylmalonyl-CoA
\label{sec:orgeea7298}

\begin{itemize}
\item inhibits PCC \(\to\) \(\uparrow\) PA
\end{itemize}
\begin{enumerate}
\item Hyperglycinemia
\label{sec:org6742ad2}
\begin{itemize}
\item inhibits synthesis of glycine cleaving enzyme.
\item inhibits transport of malate, 2-ketoglutarate and isocitrate
\end{itemize}
\item CBC
\label{sec:org71f5fa3}
\begin{itemize}
\item megaloblastosis in cblC defects
\begin{itemize}
\item \(\downarrow\) methylcobalamin
\end{itemize}
\end{itemize}

\item Hyperammonemia
\label{sec:orge2b51e5}
\begin{itemize}
\item PA inhibits carbamylphosphate synthetase
\end{itemize}
\item Ketosis
\label{sec:org7abea2f}
\begin{itemize}
\item PA inhibits mitochondrial oxidation of succinic acid and 2-ketoglutaric acid
\end{itemize}
\end{enumerate}
\end{enumerate}

\subsection{Laboratory Investigations}
\label{sec:org6daf558}
\begin{enumerate}
\item NSO PA/MMA Screening Logic
\label{sec:org955a4c6}
\begin{enumerate}
\item Inital positive \textless{} 7 days
\label{sec:org9faeec4}
(C3/C2 \(\ge\) 0.21 AND C3 \(\ge\) 4.0)
OR
(C3/C2 \(\ge\) 0.23 AND C3 \(\ge\) 3.5)
\item Inital positive \textgreater{} 7 days
\label{sec:orgbdf4015}
(C3/C2 \(\ge\) 0.21 AND C3 \(\ge\) 2.6)
OR
(C3/C2 \(\ge\) 0.23 AND C3 \(\ge\) 2.4)
\begin{itemize}
\item Repeat overnight
\item No weekend reporting
\end{itemize}
\item Alert
\label{sec:orga189ea0}
C3/C2 \(\ge\) 0.3 AND C3 \(\ge\) 9.0
\begin{itemize}
\item Repeat same day
\item Weekend reporting
\end{itemize}
\item Confirmation
\label{sec:org7ea11bc}
C3/C2 \(\ge\) 0.23 AND MCA \(\ge\) 0.5
\end{enumerate}

\item Clinical Chemistry
\label{sec:org9d2f899}
\begin{itemize}
\item Acidosis in acute episodes
\begin{itemize}
\item accumulation of \(\beta\)-hydroxybutyrate and acetoacetate
\item Arterial pH as low as 6.9
\item Bicarb as low as 5 mEq/L
\end{itemize}
\item \(\uparrow\) lactic acid
\item Hypoglycemia
\item Hyperammonemia
\item Measure B12
\end{itemize}

\item Biochemical Genetics
\label{sec:orgcb95f86}
\begin{enumerate}
\item Plasma Amino Acids
\label{sec:org3ec2b7c}
\begin{itemize}
\item \textpm{} glycine
\item \(\uparrow\) glutamine when hyperammonemia
\end{itemize}
\item Plasma Acylcarnitines
\label{sec:orgb0cfde2}
\begin{itemize}
\item \(\uparrow\) propionyl carnitine (C3)
\item \(\uparrow\) methylmalonyl carnitine (C4DC)
\end{itemize}

\item Urine Organic Acids
\label{sec:orgc14f1d3}
\begin{itemize}
\item methylmalonic acid
\item 3-hydroxypropionic acid
\item methylcitric acid
\item tiglic acid / tiglyglycine
\item Ketones
\begin{itemize}
\item BHB
\item acetoacetate
\end{itemize}
\item lactic acid
\end{itemize}
\end{enumerate}

\item Urine Organic Acids
\label{sec:orga41ff9b}

\begin{figure}[htbp]
\centering
\includegraphics[width=0.9\textwidth]{./mma/figures/mma_uoa.png}
\caption[MMA UOA]{\label{fig:org8926ff4}
MMA UOA}
\end{figure}

\item Typical Urine Methylmalonic Acid Values
\label{sec:orgc85d85e}

\begin{center}
\begin{tabular}{lr}
Clinical Status & mmol/mol creatinine\\
\hline
normal & 0-2\\
Mut\(^{\text{0}}\);presentation & 3000-13000\\
Mut\(^{\text{0}}\);steady state & 200-2000\\
B\(_{\text{12}}\) responsive;presentation & 2000\\
B\(_{\text{12}}\) responsive;steady-state & 90-300\\
B\(_{\text{12}}\) deficient infant & 4500-5700\\
Transcobalamin II deficiency & 600\\
Cobalamin C,D & 270\\
Atypical-normal mutase & 200\\
Succinyl CoA ligase & 80-120\\
Methylmalonyl CoA epimerase deficiency & 30-300\\
\end{tabular}
\end{center}
\end{enumerate}

\subsection{Clinical Findings}
\label{sec:org7764f83}
\begin{enumerate}
\item Initial presentation
\label{sec:org62012b4}
\begin{itemize}
\item Failure to thrive
\begin{itemize}
\item \(\downarrow\) linear growth
\end{itemize}
\item Skin lesions - candidasis
\item Life-threatening illness early in life
\begin{itemize}
\item ketonuria
\begin{itemize}
\item acidosis
\item dehydration
\end{itemize}
\item vomiting
\item lethargy \(\to\) coma
\end{itemize}
\end{itemize}

\item Acute Treatment
\label{sec:org2991a09}
\begin{itemize}
\item aggressive intravenous hydration
\begin{itemize}
\item efficient renal excretion of MMA
\end{itemize}
\item Insulin and glucose \(\to\) anabolism
\item Acute HGH
\end{itemize}

\item Recurrent Symptoms
\label{sec:org269d521}
\begin{itemize}
\item ketotic episodes
\item infection
\item protein intolerance
\end{itemize}

\item Long term
\label{sec:orge8a8df5}
\begin{itemize}
\item Variable developmental/cognitive outcomes
\begin{itemize}
\item appears linked to incidence of illness
\end{itemize}
\item severe hypotonia
\item stroke
\item hepatomegaly - normal LFTs
\item \downarrown renal function
\item pancreatitis
\item candidasis
\begin{itemize}
\item MMA inhibits maturation of hematopoietic cells and T cells
\end{itemize}
\end{itemize}

\item Neurological Findings
\label{sec:org3368e23}
\begin{itemize}
\item due to acute episodes
\begin{itemize}
\item \(\downarrow\) cerebral perfusion
\item hypoglycemia
\item hyperammonemia
\end{itemize}
\item more common in apo enzyme defect than Cbl
\item Successful treatment \(\to\) normal IQ
\item 25 of 33 patients :
\begin{itemize}
\item ataxia = lack of coordination
\item dystonia = muscle contraction
\item dyskinesia = involuntary movement
\item dsyarthria = speech
\item chorea = rythmic contractions
\item clonus = jerky movements
\item tremors
\end{itemize}
\end{itemize}

\item Long-term Treatment
\label{sec:org24ce59b}
\begin{enumerate}
\item Diet
\label{sec:orgb12816e}
\begin{itemize}
\item Limit Val, Ile, Thr, Met
\begin{itemize}
\item Monitor urine metabolites, plasma amino acids
\item Urine ketones (daily in infancy)
\item Monitor weight, nitrogen balance
\end{itemize}
\item Avoid fasting
\begin{itemize}
\item Catabolism
\item Propionate release from lipids
\end{itemize}
\end{itemize}

\item Supplementation
\label{sec:orgf6c604f}
\begin{itemize}
\item test for B\(_{\text{12}}\) response
\begin{itemize}
\item pharmacologic does of B\(_{\text{12}}\)
\end{itemize}
\item carnitine
\begin{itemize}
\item excretion of carnitine esters \(\to\) detoxification
\item Daily dose 60 to 100 mg/kg
\end{itemize}
\item ?alanine supplementation?
\end{itemize}
\end{enumerate}

\item Long-term Treatment
\label{sec:orgdd4e92e}
\begin{enumerate}
\item Medication
\label{sec:org5bbd5ff}
\begin{itemize}
\item Production of PA by intestinal bacteria
\begin{itemize}
\item metronidazole
\item neomycin
\end{itemize}
\item HGH \(\to\) anabolism
\end{itemize}

\item Transplantation
\label{sec:org4f8bfe0}
\begin{itemize}
\item liver transplantation does not stop progressive neurological symptoms
\item liver \& kidney may be considered
\end{itemize}
\end{enumerate}
\end{enumerate}

\section{Multiple Carboxylase Deficiency}
\label{sec:org9f0344e}
\subsection{Introduction}
\label{sec:orgd99323b}
\begin{enumerate}
\item Multiple Carboxylase Deficiency
\label{sec:org12e67ff}
\begin{itemize}
\item Four carboxylases:
\begin{itemize}
\item pyruvate carboxylase
\item acetyl-CoA carboxylase
\item propionyl carboxylase
\item 3-methylcrotonyl carboxylase
\end{itemize}
\end{itemize}

\item Carboxylases
\label{sec:org71efd38}

\begin{figure}[htbp]
\centering
\includegraphics[width=0.9\textwidth]{./mcd/figures/carboxylases.png}
\caption[Carboxylases]{\label{fig:orga0e7191}
Carboxylases}
\end{figure}

\item Causes of Multiple Carboxylase Deficiency
\label{sec:orgf9da1ad}
\begin{itemize}
\item Biotin Deficiency
\item Holocarboxylase Synthetase Deficiency
\item Biotinidase Deficiency
\end{itemize}

\item Holocarboxylase Synthetase
\label{sec:org18ba16c}
\begin{itemize}
\item HCS activates biotin to D-biotin-5'-adenylate
\item catalyzes attachment to an apocarboxylase
\begin{itemize}
\item lysine \(\epsilon\)-amino group
\end{itemize}
\end{itemize}

\begin{figure}[htbp]
\centering
\includegraphics[width=0.9\textwidth]{./mcd/figures/hcs.png}
\caption[Holocarboxylase Synthase]{\label{fig:orgae92fdc}
Holocarboxylase Synthase}
\end{figure}

\item Holocarboxylase Synthetase Deficiency
\label{sec:org56a5da2}
\begin{itemize}
\item Increased Km for Biotin
\begin{itemize}
\item Normally 1-6 nmol/L, patients 9-12 nmol/L
\end{itemize}
\end{itemize}

\begin{figure}[htbp]
\centering
\includegraphics[width=0.9\textwidth]{./mcd/figures/kinetics.png}
\caption[Kinetics]{\label{fig:orga1e6a89}
Holocarboxylase Synthetase Kinetics}
\end{figure}


\item Biotinidase
\label{sec:org0e270e5}

\begin{figure}[htbp]
\centering
\includegraphics[width=0.9\textwidth]{./mcd/figures/biot.png}
\caption[Biotinidase]{\label{fig:orgab579ed}
Biotinidase}
\end{figure}

\item Biotin and HCS
\label{sec:org8000bcd}

\begin{figure}[htbp]
\centering
\includegraphics[width=0.9\textwidth]{./mcd/figures/biotHCS.png}
\caption[Biotin and HCS]{\label{fig:org8459d2c}
Biotin and HCS}
\end{figure}


\item Treatment
\label{sec:orgb442825}
\begin{itemize}
\item All symptoms and biochemical abnormalities treated with biotin.
\begin{itemize}
\item Except optic and auditory nerve atrophy
\item Biocytin?
\end{itemize}
\end{itemize}
\end{enumerate}

\subsection{Biotinidase Deficiency}
\label{sec:org620f304}

\begin{enumerate}
\item Clinical
\label{sec:org80cf7d7}
\begin{itemize}
\item neurologic abnormalities including:
\item seizures, hypotonia, ataxia, developmental delay, vision problems, hearing
loss, and cutaneous abnormalities (e.g., alopecia, skin rash,
candidiasis).
\item Older children and adolescents with profound biotinidase deficiency
often exhibit motor limb weakness, spastic paresis, and decreased
visual acuity.
\item Once vision problems, hearing loss, and developmental
delay occur, they are usually irreversible,

\item Autosomal recessively inherited disorder of biotin recycling
\begin{itemize}
\item associated with secondary alterations in amino acid, carbohydrate,
and fatty acid metabolism.
\end{itemize}
\item Caused by absent or markedly deficient activity of biotinidase
\begin{itemize}
\item cytosolic enzyme that liberates biotin from biocytin during the
normal proteolytic turnover of holocarboxylases and other
biotiny-lated proteins.
\end{itemize}

\item Based on newborn screening outcome data from 2006, the incidence of
profound biotinidase deficiency in the United States is estimated at
1/80,000 births
\item partial biotinidase deficiency between 1/31,000 and 1/40,000
\end{itemize}


\item Diagnostic Testing
\label{sec:orga22640b}
\begin{itemize}
\item Diagnosis of biotinidase deficiency is based on demonstrating
deficient enzyme activity in serum or plasma

\item Patients with profound biotinidase deficiency have less than 10\% of
mean normal serum activity

\item Patients with the partial biotinidase deficiency variant have 10-30\%
of mean normal serum activity
\begin{itemize}
\item are largely asymptomatic
\end{itemize}

\item Confirmation of biotinidase deficiency by DNA analysis, by either
allele-targeted methods or full-gene sequencing, may be useful.
\end{itemize}

\begin{figure}[htbp]
\centering
\includegraphics[width=0.9\textwidth]{./mcd/figures/biot_cycle.png}
\caption[Biotin Cycle]{\label{fig:org3384ba3}
Biotin Cycle}
\end{figure}


\item Biotinidase
\label{sec:org4871d21}

\begin{itemize}
\item Biotinidase is a monomeric enzyme encoded by a single gene (BTD) located on chromosome 3p25
\begin{itemize}
\item comprises 543 amino acid residues, including 41 amino acids of a potential signal peptide.
\end{itemize}

\item Three publicly available databases of biotinidase variants:
\begin{itemize}
\item \href{https://grenada.lumc.nl/LOVD2/shared1/home.php?select\_db=BTD}{Leiden Open Variation Database}
\item \href{https://www.ncbi.nlm.nih.gov/clinvar/}{ClinVar}
\item \href{http://www.arup.utah.edu/database/BTD/BTD\_welcome.php}{ARUP:Biotinidase Deficiency and BTD}
\end{itemize}

\item 204 biotinidase variants are consolidated in the ARUP database,
\begin{itemize}
\item >150 categorized as pathogenic
\item 145 missense changes
\item Four common pathogenic variants cause profound biotinidase deficiency.
\end{itemize}
\end{itemize}

\item Pathogenic Variants
\label{sec:org4c51d1d}
\begin{itemize}
\item Among children ascertained because of clinical symptoms, the two
most commonly reported variants are:

\begin{itemize}
\item c.98\(_{\text{104delinsTCC}}\) in exon 2
\begin{itemize}
\item seven-base deletion/three-base insertion
\item occurring in at least one allele in approximately 50\% of affected individuals
\end{itemize}

\item p.Arg538Cys in exon 4
\begin{itemize}
\item occurring at least once in 30\% of affected individuals
\end{itemize}

\item These variants result in complete absence of biotinidase protein.
\end{itemize}

\item Other relatively common variants discovered by newborn screening are:
\begin{itemize}
\item p.Gln456His, associated with profound deficiency

\item p.Asp444His, a substitution that reduces enzymatic activity by about 50\%.

\item The p.Asp444His variant in trans with a severe BTD pathogenic variant is associated with partial biotinidase deficiency,
\item p.Asp444His in cis with p.Ala171Thr (i.e., as the double mutant p.[(Ala171Thr); (Asp444His)]), results in a profound biotinidase deficiency allele.
\end{itemize}
\end{itemize}


\item Partial \& Profound Deficiency
\label{sec:org218ee90}

\begin{enumerate}
\item Profound Deficiency
\label{sec:org68dc4d2}
\begin{itemize}
\item Initially, most symptomatic children with biotinidase deficiency
were found to have 3\% of mean serum biotinidase activity of normal
individuals.
\item Three standard deviations above this mean, corresponding to 10\% of
mean normal activity, was taken as the threshold below which
individuals were considered to have profound biotinidase deficiency.
\end{itemize}

\item Partial Deficiency
\label{sec:org975a3fb}
\begin{itemize}
\item With NBS for biotinidase deficiency babies were identified with about 25\% of mean normal activity.
\item Essentially all of these children had the p.Asp444His variant as one of their alleles
\item This variant, together with a variant for profound deficiency on the other allele, results in 10–30\% of mean normal biotinidase activity.
\item These children are considered to have partial biotinidase deficiency.
\end{itemize}
\end{enumerate}


\item NBS for partial deficiency
\label{sec:orgc8c088b}

\begin{itemize}
\item A retrospective study reviewing clinical histories of
individuals with profound (22) or partial (120) biotinidase
deficiency identified by newborn screening supports the long-term
benefit of treatment and management of both populations. \footnote{Outcomes of individuals with profound and partial
biotinidase deficiency ascertained by newborn screening in Michigan
over 25 years, Genetics In Medicine, 2014/08/21/}
\end{itemize}
\end{enumerate}

\subsection{Laboratory Methods}
\label{sec:org62339eb}

\begin{enumerate}
\item Biotinidase NBS
\label{sec:org76fec4a}

\begin{itemize}
\item DBS is eluted and incubated
\end{itemize}

\centering
\ce{Biotin-PAB <=>>[Biotinidase][pH=6] Biotin + PABA}

\begin{itemize}
\item proteins removed by TCA precipitation and filtration.
\end{itemize}

\ce{PABA <=>>[\ce{NO2, NH2SO3}][NED] Purple chromophore}

\begin{itemize}
\item measured at 550 nm, reference 690 nm
\item Sulfonamide antibiotics can cause false negative results:
\begin{itemize}
\item sulfamethoxazole, trimethoprim, sulfioxazole
\end{itemize}
\end{itemize}

\begin{enumerate}
\item Interpretation
\label{sec:org8bf589d}
\begin{description}
\item[{screen positive}] \textless{} 27.0 MRU
\item[{units}] 1 MRU = 1 umol of PABA produced from Biotin-PAB
\end{description}
\end{enumerate}

\item Biotinidase Diagnostic
\label{sec:org48e19c7}

\begin{itemize}
\item serum or plasma
\end{itemize}

#+BEGIN_LaTeX
\ce{Biotin-PAB <=>>[Biotinidase][pH=6] Biotin + PABA}
\begin{itemize}
\item proteins removed by TCA precipitation and centrifugation.
\end{itemize}
\ce{PABA <=>>[\ce{NaNO2, NH2SO3}][NED] Purple chromophore}
\begin{itemize}
\item measured at 546 nm
\end{itemize}

\begin{enumerate}
\item Interpretation
\label{sec:org0dde6ed}
\begin{description}
\item[{Deficiency}] \(\le\) 10\% of normal
\item[{Partial}] \textgreater{} 10\% and \(\le\) 30\%
\item[{units}] nmoles/min/L plasma or serum.
\end{description}
\end{enumerate}

\item HCS NBS
\label{sec:orgbef1bde}
\begin{itemize}
\item C5OH acylcarnitine
\item No longer include C5OH acylcarnitine in the NSO AACC screen
\item 3-methylcrotonyl-CoA carboxylase (3MCC) deficiency (infant or mother)
\item 3-hydroxy-3-methylglutaryl (HMG)-CoA lyase deficiency
\item \(\beta\)-ketothiolase deficiency
\item multiple carboxylase deficiency (MCD) including biotinidase deficiency and holocarboxylase synthetase deficiency
\item 2-methyl-3-hydroxybutyric acidemia (2M3HBA)
\item 3-methylglutaconic aciduria (3MGA)
\end{itemize}

\item HCS Diagnostic
\label{sec:org8f5d2be}

\begin{itemize}
\item Urine organic acids
\begin{itemize}
\item \(\beta\)-hydroxyisovalerate
\item \(\beta\)-methylcrotonylglycine
\item \(\beta\)-hydroxypropionate
\item methylcitrate
\item lactate
\item tiglylglycine
\end{itemize}
\end{itemize}
\end{enumerate}


\section{{\bfseries\sffamily TODO} Lysine Catabolism}
\label{sec:orgc201038}
\end{document}