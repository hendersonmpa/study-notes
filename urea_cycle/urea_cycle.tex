% Created 2020-01-27 Mon 09:25
% Intended LaTeX compiler: pdflatex
\documentclass{scrartcl}
\usepackage[utf8]{inputenc}
\usepackage[T1]{fontenc}
\usepackage{graphicx}
\usepackage{grffile}
\usepackage{longtable}
\usepackage{wrapfig}
\usepackage{rotating}
\usepackage[normalem]{ulem}
\usepackage{amsmath}
\usepackage{textcomp}
\usepackage{amssymb}
\usepackage{capt-of}
\usepackage{hyperref}
\hypersetup{colorlinks,linkcolor=black,urlcolor=blue}
\usepackage{textpos}
\usepackage{textgreek}
\usepackage[version=4]{mhchem}
\usepackage{chemfig}
\usepackage{siunitx}
\usepackage{gensymb}
\usepackage[usenames,dvipsnames]{xcolor}
\usepackage[T1]{fontenc}
\usepackage{lmodern}
\usepackage{verbatim}
\usepackage{tikz}
\usepackage{wasysym}
\usetikzlibrary{shapes.geometric,arrows,decorations.pathmorphing,backgrounds,positioning,fit,petri}
\usepackage{fancyhdr}
\pagestyle{fancy}
\author{Matthew Henderson, PhD, FCACB}
\date{\today}
\title{Urea Cycle}
\hypersetup{
 pdfauthor={Matthew Henderson, PhD, FCACB},
 pdftitle={Urea Cycle},
 pdfkeywords={},
 pdfsubject={},
 pdfcreator={Emacs 26.1 (Org mode 9.1.9)}, 
 pdflang={English}}
\begin{document}

\maketitle
\tableofcontents


\section{Ammonia}
\label{sec:orgcacc9e6}
\subsection{Background}
\label{sec:org3e7cc43}
\begin{enumerate}
\item Production
\label{sec:org397aa12}
\begin{itemize}
\item Ammonia is produced via the metabolism of nitrogen containing compounds:
\begin{itemize}
\item amino acids
\item purines and pyrimidines
\end{itemize}
\item Intestinal bacteria produce ammonia by splitting urea in the gut.
\item Ammonia in the body of a healthy individual is typically present as ammonium ions.
\ce{NH4+ <=>[pKa = 9.3] NH3 + H+}
\item Ammonium ions cannot cross membranes
\end{itemize}

\item Transport from Muscle: Glucose/Alanine Cycle
\label{sec:orga78a2e0}
\begin{figure}[htbp]
\centering
\includegraphics[width=0.9\textwidth]{./urea_cycle/ammonia/figures/glucose_alanine_cycle.png}
\caption{\label{fig:orgb059c7d}
Glucose/Alanine Cycle}
\end{figure}

\item Transport from Muscle: Glutamine
\label{sec:orgb908692}

\begin{figure}[htbp]
\centering
\includegraphics[width=0.9\textwidth]{./urea_cycle/ammonia/figures/nitrogen_glutamine.png}
\caption{\label{fig:org168ef37}
Glutamine}
\end{figure}

\item Exchange: Glutamate
\label{sec:org86ceb70}
\begin{figure}[htbp]
\centering
\includegraphics[width=0.9\textwidth]{./urea_cycle/ammonia/figures/nitrogen_glutamate.png}
\caption{\label{fig:org480ae4b}
Glutamate Exchange}
\end{figure}

\chemname{\chemfig{H_2{\color{red}N}-[1](=[2]O)-[7]{\color{red}N}H_2}}{Urea}

\item Liver Lobule
\label{sec:org11ac251}

\begin{figure}[htbp]
\centering
\includegraphics[width=0.9\textwidth]{./urea_cycle/ammonia/figures/liver_lobule.png}
\caption{\label{fig:org87f04df}
Liver Lobule}
\end{figure}

\begin{description}
\item[{periportal hepatocytes}] \(\uparrow\) capacity, \(\downarrow\) affinity
\item[{perivenous hepatocytes}] \(\downarrow\) capacity, \(\uparrow\) affinity

\item Figure \ref{fig:org87f04df} focuses on periportal and perivenous
hepatocytes as the two ammonia detoxifying compartments in
liver.

\item Ammonia is metabolized with high capacity but low affinity in
the urea cycle which is solely expressed in periportal
hepatocytes.

\item As back-up, ammonia is detoxified by the action of glutamine
synthetase that is solely expressed in perivenous hepatocytes and
has a low capacity but high affinity towards ammonia. Urea and
glutamine re-enter the circulation to be excreted in urine or
further metabolized in the kidney, respectively.
\end{description}
\end{enumerate}


\subsection{Laboratory}
\label{sec:orgad96a8c}
\begin{enumerate}
\item Measurement at CHEO
\label{sec:org7f6e9a6}

\begin{itemize}
\item The VITROS AMON Slide has a multilayered analytical element coated
on a polyester support.
\item A drop of patient sample is deposited on the slide.
\item An  alkaline buffer converts ammonium ions to gaseous ammonia.
\item A semipermeable membrane allows only ammonia to pass through
\item After incubation the reflection density of the dye is measured.
\end{itemize}

\centering
\ce{NH3 + bromophenol blue -> blue dye (600 nm)}

\begin{itemize}
\item bromophenol blue changes from yellow at pH 3.0 to blue at pH 4.6
\end{itemize}

\item Measurement at TOH
\label{sec:org03acb6c}
\begin{itemize}
\item The Dimension Vista Ammonia (AMM) method uses glutamate dehydrogenase (GLDH) and a NADPH analog.
\end{itemize}

\centering
\ce{\alpha-ketoglutarate + NH4+ + NADPH ->[GLDH] L-glutamate + NADP+ + H2O}

\begin{itemize}
\item The decrease in absorbance due to the oxidation of the reduced
cofactor is monitored at 340/700 nm and is proportional to the
ammonia concentration.
\end{itemize}

\item Ammonia Interpretation
\label{sec:org261ef71}
\begin{itemize}
\item In a healthy person, ammonia is relatively tightly controlled;
\end{itemize}
\begin{enumerate}
\item CHEO
\label{sec:org93532b1}
\begin{center}
\begin{tabular}{lrr}
Age & RI (umol/L) & Critical\\
\hline
0-<1 month & 10-55 & >55\\
1-<3 months & <30 & >55\\
3 months-<18 yrs & <30 & >100\\
>18 years & <30 & >200\\
\end{tabular}
\end{center}

\item TOH
\label{sec:orgf42518d}
\begin{center}
\begin{tabular}{lrl}
Age & RI (umol/L) & Critical\\
\hline
All & <35 & None\\
\end{tabular}
\end{center}
\end{enumerate}
\end{enumerate}


\subsection{Clinical}
\label{sec:org405e836}
\begin{enumerate}
\item Neonatal Symptoms
\label{sec:org78db77b}
\begin{itemize}
\item poor feeding
\item vomiting
\item seizures
\item respiratory distress
\item poor peripheral blood circulation
\item hypotonia
\item vomiting
\item "abnormal neurologic changes"
\begin{itemize}
\item stupor
\end{itemize}
\item inhibition of insulin secretion
\end{itemize}

\begin{enumerate}
\item Outcome
\label{sec:org4d74f30}
\begin{itemize}
\item outcome \(\propto\) \(\frac{1}{duration + [\ce{NH4+}]}\)
\begin{itemize}
\item irreparable brain damage
\end{itemize}
\end{itemize}
\end{enumerate}
\item Causes of Hyperammonemia
\label{sec:org4301b62}
\begin{enumerate}
\item Increased ammonia production
\label{sec:org1a97ae8}
\begin{itemize}
\item High protein diets
\item Massive hemolysis
\item Parenteral nutrition with high nitrogen content
\item Protein catabolism (kwashiorkor)
\item Infection
\item \textbf{Pre-analytical}
\end{itemize}

\item Decreased ammonia elimination
\label{sec:orgca87935}
\begin{itemize}
\item liver disease
\item IEM
\begin{itemize}
\item urea cycle defects
\item fatty acid oxidation defects
\item organic acidemias
\end{itemize}
\end{itemize}
\end{enumerate}

\item Pre-analytical Considerations in Ammonia Testing
\label{sec:orgf55c1a8}
\begin{itemize}
\item Capillary ammonia is significantly higher than arterial and venous
\begin{itemize}
\item Capillary samples - sweat contamination
\end{itemize}
\item Delayed analysis
\begin{itemize}
\item erythrocytes and platlets \(\to\) ammonia
\item GGT activity
\item Serum is unsuitable
\end{itemize}
\item Hemolysis -  \(\uparrow\) [ammonia] RBC
\item Detergent contamination
\end{itemize}

\item Specimen Collection and Handling
\label{sec:orgcd621d7}
\begin{itemize}
\item Free flowing venous or arterial sample
\item Pre-chilled lithium heparin (green top)
\item Transport on ice
\item Separated w/in 15 min of collection
\item Analysed immediately
\item Once separated stable: 4 hr @ 4\textdegree C , 24hr @-20\textdegree C
\end{itemize}

\item Biochemical Testing in Neonate with Hyperammonemia
\label{sec:org58e737e}
\begin{itemize}
\item First line
\begin{itemize}
\item Blood gas analysis
\begin{itemize}
\item UCD \(\to\) Respiratory alkalosis
\item UCD rarely acidotic
\item Acidosis suggests OAD or mitochondrial disorder.
\end{itemize}
\item Urea
\item Glucose
\begin{itemize}
\item hypoglycemia - FAOD, HI, liver failure
\end{itemize}
\item Liver Function tests
\item Lactate
\begin{itemize}
\item mitochondrial disorders,organic acidemias and FAODs
\end{itemize}
\end{itemize}

\item Specialist Investigations
\begin{itemize}
\item Urine and Plasma amino acids
\begin{itemize}
\item citruline
\item argininosuccinic acid
\end{itemize}
\item Urine Organic Acids
\begin{itemize}
\item orotic acid
\end{itemize}
\end{itemize}
\end{itemize}

\item Differential Diagnosis in the Neonate
\label{sec:org5b7113e}

\begin{itemize}
\item Inborn errors of metabolism are a very important part of the
differential diagnosis in a neonate who has hyperammonemia
\begin{itemize}
\item Ammonia 100 umol/L or higher
\end{itemize}

\item Sepsis
\item Liver dysfunction
\item Portocaval shunt
\item Perinatal asphyxia
\item Sampling artifact
\item IEMs
\begin{itemize}
\item Urea cycle disorders
\item Organic acidemias
\item Fatty acid oxidation disorders
\item Mitochondrial disorders
\item Amino acid transporter deficiency
\end{itemize}
\end{itemize}

\begin{figure}[htbp]
\centering
\includegraphics[width=0.9\textwidth]{./urea_cycle/ammonia/figures/THANvIEM.png}
\caption{\label{fig:org3efc980}
THAN v IEM}
\end{figure}

\item Ammonia Interpretation
\label{sec:orgf019b35}

\begin{center}
\begin{tabular}{ll}
Ammonia (umol/L) & Conditions\\
\hline
> 1500 & THAN\\
> 600 & UCD, PA, Valproate\\
200 - 600 & OA, FAOD,\\
< 200 & Acquired\\
\end{tabular}
\end{center}
\end{enumerate}

\section{Diagnostic Testing for OTC Deficiency}
\label{sec:orgf1b1106}
\subsection{Background}
\label{sec:org758460f}
\begin{enumerate}
\item The Urea Cycle
\label{sec:org13a72dc}
\begin{enumerate}
\item Mitochondrial enzymes:
\label{sec:org6a31c03}
\begin{itemize}
\item Carbamoylphosphate synthetase I (CPS1, AR)
\begin{itemize}
\item rate-limiting reaction of the urea cycle
\item N-acetylglutamate is an obligate activator
\end{itemize}
\item N-acetylglutamate synthetase (NAGS, AR)
\item Ornithine transcarbamylase (OTC, X-linked)
\end{itemize}
\item Cytoplasmic enzymes:
\label{sec:orgbe49efd}
\begin{itemize}
\item Argininosuccinic acid synthetase (ASS1, AR)
\item Argininosuccinic acid lyase (ASL, AR)
\item Arginase (ARG1, AR)
\end{itemize}

\item Transporters:
\label{sec:orgbd52b1b}
\begin{itemize}
\item Ornithine translocase (SLC25A15, AR)
\end{itemize}

\item Ancillary Enzymes
\label{sec:org5316d4c}
\begin{itemize}
\item Carbonic Anhydrase Va (CA5A, AR)
\item Citrin (SLC25A13, AR)
\item \(\Delta\)-pyrroline-5 carboxylate synthetase (ALDH18A1)
\end{itemize}

\begin{figure}[htbp]
\centering
\includegraphics[width=0.9\textwidth]{./urea_cycle/otc_diag/figures/urea_cycle_crop.png}
\caption{\label{fig:orga345304}
Urea Cycle}
\end{figure}
\end{enumerate}

\item Secondary Effects on the Urea Cycle
\label{sec:org3e19d99}

\begin{figure}[htbp]
\centering
\includegraphics[width=0.9\textwidth]{./urea_cycle/otc_diag/figures/2nd_ammonemia.png}
\caption{\label{fig:org1d28047}
Secondary Effects on the Urea Cycle}
\end{figure}

\item OTC Deficiency
\label{sec:org12dc7d0}

\ce{ornithine + carbamoyl phosphate ->[OTC] citrulline}

\begin{itemize}
\item Incidence 1:56,500
\item X-linked inheritance
\end{itemize}

\item OTCD Neonatal Onset
\label{sec:org5ae067e}
\begin{itemize}
\item Severe neonatal-onset disease in males (but rarely in females)
\item Males with severe neonatal-onset OTCD are typically normal
at birth
\begin{itemize}
\item become symptomatic on day two to three of life.
\end{itemize}
\item After successful treatment of neonatal hyperammonemic coma these
infants can easily become hyperammonemic again despite appropriate
treatment
\begin{itemize}
\item typically require liver transplant by age six months to improve quality of life.
\end{itemize}
\end{itemize}

\item OTCD Post-Neonatal Onset
\label{sec:orgbed845a}
\begin{itemize}
\item Post-neonatal-onset (partial deficiency) disease in males and females.
\item Males and heterozygous females with post-neonatal-onset (partial)
OTC deficiency can present from infancy to later childhood,
adolescence, or adulthood.
\item In all OTCD a hyperammonemic crisis can be precipitated by stress
\item Typical neuropsychological complications include developmental delay, learning disabilities,
intellectual disability, attention deficit hyperactivity disorder
(ADHD), and executive function deficits.
\end{itemize}

\item OTCD Metabolic Derangements
\label{sec:org2076056}


\begin{itemize}
\item Hyperammonemia due to product inhibition of CPS by CP.
\item Excess CP \(\to\) \(\uparrow\) pyrimidine biosysthesis
\begin{itemize}
\item \(\uparrow\) orotic acid and uracil in urine
\end{itemize}
\item \(\uparrow\) \ce{NH4+} \(\to\) \(\uparrow\) glutamine
\begin{itemize}
\item \(\uparrow\)  glycine, serine, glutamate, alanine
\end{itemize}
\end{itemize}

\begin{enumerate}
\item Routine Biochemical Testing
\label{sec:org9c8ed47}
\begin{itemize}
\item Routine Biochemical Testing
\begin{itemize}
\item hyperammonemia
\item absence of hypoglycemia, lactic acidosis, ketonuria
\end{itemize}
\end{itemize}

\item Plasma Amino Acids
\label{sec:org46143d3}
\begin{itemize}
\item Plasma Amino Acids
\begin{description}
\item[{glutamine}] \textgreater{} 1000 \si{\micro\mol/\liter}
\item[{alanine}] \textgreater{} 600 \si{\micro\mol/\liter}
\item[{citruline}] \textless{} 10 \si{\micro\mol/\liter}
\item[{arginine}] \textless{} 30 \si{\micro\mol/\liter}
\end{description}
\end{itemize}
\end{enumerate}

\item Female Carriers of OTC
\label{sec:org1785ef1}
\begin{itemize}
\item Variable inactivation of the X-chromosome
\item Variable phenotype
\begin{itemize}
\item low residual OTC function
\item asymptomatic
\item long term symptoms consistent with undiagnosed hyperammonemia
\end{itemize}
\end{itemize}
\end{enumerate}

\subsection{Case Study}
\label{sec:orgdb75688}
\begin{enumerate}
\item A Case of Severe Neonatal Hyperammonemia
\label{sec:org7bfbb04}

\begin{itemize}
\item Roy W.A. Peake and Edward G. Neilan, Clin Chem 2017
\begin{itemize}
\item Department of Laboratory Medicine and Division of Genetics and Metabolism
\item Boston Children’s Hospital, Boston, MA.
\end{itemize}
\end{itemize}
\item Clinical History
\label{sec:orgf4dec2c}
\begin{itemize}
\item Male child delivered by C-section at 39 wks
\item Emerged limp and cyanotic w/o respiratory effort
\begin{itemize}
\item intubated, suctioned, positive pressure ventilation
\item developed spontaneous respiration
\end{itemize}
\item Treated with boluses of saline and glucose for hypotension and hypoglycemia
\item Thrombocytopenia was treated with empiric antibiotics

\item Full enteral feeding by day 4

\item Day 5 developed apnea and seizures

\begin{itemize}
\item intubated, ventilation, phenobarbital, antibiotics
\end{itemize}
\end{itemize}

\item Screening and Diagnostic Testing
\label{sec:org7862e28}
\begin{itemize}
\item Newborn Screening results became available
\begin{itemize}
\item "concerning for low-normal citrulline levels"
\end{itemize}
\item A metabolic disorder was considered
\item Plasma amino acids are useful for diagnosis for OTCD
\end{itemize}

\begin{enumerate}
\item Lab results
\label{sec:org64a049a}
\begin{description}
\item[{Ammonia}] 2090 \si{\micro\mol/\liter} (RI < 90)
\item Plasma Amino Acids
\begin{description}
\item[{Glutamine}] 1536 \si{\micro\mol/\liter} (RI 330-1080)
\item[{Alanine}] 1160 \si{\micro\mol/\liter} (RI 120-500)
\item[{Citrulline}] 3 \si{\micro\mol/\liter} (RI 2-50)
\end{description}
\end{description}
\end{enumerate}

\item Urine Orotic Acid
\label{sec:orgb4424f4}
\begin{itemize}
\item OA is a surrogate marker for increased carbamoyl phosphate.
\begin{itemize}
\item \(\uparrow\) CP overwhelms the pyrimidine synthesis pathway
\end{itemize}
\end{itemize}

\centering
\ce{CP ->[aspartate transcarbamylase] carbamoyl aspartate ->[dihydroorotase] dihydroorotate ->[dihydroorotate dehydrogenase] orotic acid}

\begin{itemize}
\item Qualitative urine organic acids have limited value in the diagnosis of UCDs.
\item Methodological issues
\begin{itemize}
\item OA is a charged molecule and is not efficiently extracted
\item OA often coelutes with cis-aconitic acid
\begin{itemize}
\item Presence of OA ion pair 254 and 357 mz should be confirmed.
\end{itemize}
\end{itemize}
\end{itemize}

\item Urine Organic Acids
\label{sec:orgfc208f2}

\begin{figure}[htbp]
\centering
\includegraphics[width=0.9\textwidth]{./urea_cycle/otc_diag/figures/F1_large.jpg}
\caption{\label{fig:orgfcf9b35}
Urine Organic Acids}
\end{figure}

\item Treatment
\label{sec:org8fffc8b}
\begin{itemize}
\item IV sodium phenylacetate and sodium benzoate (ammonul)
\begin{itemize}
\item phenylacetate + CoA + glutamine \(\to\) phenylactylglutamine
\item benzoate + CoA + glycine \(\to\) hippuric acid
\end{itemize}

\item Emergency hemodialysis
\item At day 7 ammonia was normal
\begin{itemize}
\item veno-venous hemofiltration
\item ammonul and arginine
\end{itemize}
\end{itemize}

\item Follow-up
\label{sec:org74641d5}

\begin{itemize}
\item Molecular testing of OTC gene
\begin{itemize}
\item hemizygous pathogenic variant (c.596A>G, p.Asn199Ser)
\begin{itemize}
\item ornithine binding site
\end{itemize}
\end{itemize}
\end{itemize}
\url{http://www.uniprot.org/uniprot/P00480\#showFeaturesViewer}
\begin{itemize}
\item previously reported in neonatal onset hyperammonemia
\end{itemize}
\begin{itemize}
\item At 9 months of age:
\begin{itemize}
\item delayed by steady developmental progress
\item 1 episode of hyperammonemia ( 250 \si\{\textmu{}\mol/L)
\end{itemize}
\item Liver transplant is being pursued
\end{itemize}
\end{enumerate}

\subsection{Local Testing}
\label{sec:orgc9f0744}
\begin{enumerate}
\item Urine Organic Acids
\label{sec:orgf03d9e1}
\begin{itemize}
\item Oximated with 10\% hydroxylamine-HCL
\begin{itemize}
\item avoids multiple TMS species due to keto-enol tautomerism
\end{itemize}
\end{itemize}

\centering
\schemedebug{false}
\schemestart
\chemname{\chemfig[][scale=.5]{R=[1](-[2]OH)-[7]R}}{\tiny enol}
\arrow{<=>}
\chemname{\chemfig[][scale=.5]{R-[1](=[2]O)-[7]R}}{\tiny ketone}
\+
\chemname{\chemfig[][scale=.5]{N(<:[::-160]H)(<[::-120]H)-O-[1]H}}{\tiny hydroxylamine}
\arrow{->}
\chemname{\chemfig[][scale=.5]{R-[1](=[2]N-[1]OH)-[7]R}}{\tiny ketoxime}
\schemestop

\begin{itemize}
\item Acidified and extracted twice with ethyl ether
\item Derivatised with BSTFA (N,O-bis(trimethylsilyl)trifluoroacetamide)
\begin{itemize}
\item forms organic acid TMS ethers
\end{itemize}
\item DB-1 0.25 mm x 30 m x 0.25 \si{\micro\meter} column
\end{itemize}

\item Quantitative Orotic Acid
\label{sec:org49a495f}

\begin{itemize}
\item \ce{1,3 -^15 N2} Orotic acid isotope IDMS
\item Silicic acid SPE
\item Eluted with chloroform: tertiary-amyl alcohol mixture
\item Derivatised with BSTFA (N,O-bis(trimethylsilyl)trifluoroacetamide)
\item DB-1 0.25 mm x 30 m x 0.25 \si{\micro\meter}
\end{itemize}

\item Genetic Testing for UCDs at NSO
\label{sec:orgd7ec727}
\begin{enumerate}
\item Targets of Newborn Screening
\label{sec:orgc2094ce}
\begin{itemize}
\item ASS1
\item ASL
\item ARG1
\end{itemize}
\item Mitochrondrial Gene Panel
\label{sec:org9cb1115}
\begin{itemize}
\item NAGS
\item CPS1
\item OTC
\item SLC25a13 (Citrin)
\item SLC25a15 (ORNT1)
\end{itemize}

\item Not available
\label{sec:orgd774028}
\begin{itemize}
\item ALDH18A1 (PC5S)
\item CA5A - added to list to consider for Mito panel
\end{itemize}
\end{enumerate}
\end{enumerate}

\section{NBS for UCD}
\label{sec:org3a6478a}
\subsection{Background}
\label{sec:org07a2bfe}
\begin{enumerate}
\item The Urea Cycle
\label{sec:org73a7935}
\begin{itemize}
\item Free ammonium ion, generated from the glutaminase and glutamate
dehydrogenase reactions,
\begin{itemize}
\item Condensed with bicabonate and converted to urea for excretion.
\end{itemize}
\item All required enzymes are expressed in the periportal cells of the liver lobule.
\end{itemize}

\begin{center}
\includegraphics[width=.9\linewidth]{./urea_cycle/ucd_nbs/figures/liver_lobule.png}
\end{center}

\begin{enumerate}
\item Mitochondrial enzymes:
\label{sec:org751207f}
\begin{itemize}
\item Carbamoylphosphate synthetase I (CPS1, AR)
\begin{itemize}
\item rate-limiting reaction of the urea cycle
\item N-acetylglutamate is an obligate activator
\end{itemize}
\item N-acetylglutamate synthetase (NAGS, AR)
\item Ornithine transcarbamylase (OTC, X-linked)
\end{itemize}
\item Cytoplasmic enzymes:
\label{sec:org70e68ae}
\begin{itemize}
\item Argininosuccinic acid synthetase (ASS1, AR)
\item Argininosuccinic acid lyase (ASL, AR)
\item Arginase (ARG1, AR)
\end{itemize}

\item Transporters:
\label{sec:org92b0d12}
\begin{itemize}
\item Ornithine translocase (SLC25A15, AR)
\end{itemize}
\end{enumerate}

\item The Urea Cycle: Ancillary Enzymes
\label{sec:org3273824}
\begin{itemize}
\item Carbonic Anhydrase Va (CAVA)
\item Citrin (SLC25A13, AR)
\item \(\Delta\)-pyrroline-5 carboxylate synthetase (P5CS)
\end{itemize}


\item The Urea Cycle
\label{sec:org2d4eb0a}
\begin{center}
\includegraphics[width=.9\linewidth]{./urea_cycle/ucd_nbs/figures/ucd-overview-Image001.jpg}
\end{center}

\item Biochemical Book Keeping
\label{sec:org36d7a5b}
\begin{itemize}
\item Beginning and ending with ornithine, the reactions of the cycle
consume three equivalents of ATP and a total of four high-energy
nucleotide phosphates.
\item Urea is the only new compound generated by the cycle; all other
intermediates and reactants are recycled.
\item The energy consumed in the production of urea is more than recovered
by the release of energy formed during the synthesis of the urea
cycle intermediates.
\item Ammonia released during the glutamate dehydrogenase reaction is
coupled to the formation of NADH. In addition, when fumarate is
converted back to aspartate, the malate dehydrogenase reaction used
to convert malate to oxaloacetate generates a mole of NADH.
\item These two moles of NADH are subsequently oxidized in the mitochondria yielding six moles of ATP.
\end{itemize}

\item Incidence of UCDs
\label{sec:orgc5315b4}
\begin{center}
\begin{tabular}{llll}
Urea Cycle Disorder & Estimated Incidence & RUSP & NSO\\
\hline
NAGS deficiency & <1:2,000,000 & No & No\\
CPS1 deficiency & 1:1,300,000 & No & No\\
\textbf{OTC deficiency} & \textbf{1:56,500} & \textbf{No} & \textbf{No}\\
ASS1 deficiency & 1:250,000 & Yes & Yes\\
ASL deficiency & 1:218,750 & Yes & Yes\\
ARG1 deficiency & 1:950,000 & No & No\\
Ornithine translocase deficiency & Unknown & No & No\\
Citrin deficiency & 1:100,000-1:230,000 in Japan 1 & No & No\\
\end{tabular}
\end{center}
\end{enumerate}

\subsection{NBS for UCDs}
\label{sec:orgb78eb6d}
\begin{enumerate}
\item NBS Challenges
\label{sec:orgd1f2a7c}
\begin{itemize}
\item Biochemical 
\begin{itemize}
\item Sensitivity of NBS methods and markers for the mitochondrial UCDs.
\end{itemize}
\item Clinical 
\begin{itemize}
\item Severely affected patients - very early onset of disease before NBS results are available
\item Patients with mild disease - pre-symptomatic detection of disease remains controversial
\end{itemize}
\end{itemize}

\item Markers for NBS for Mitochondrial UCDs
\label{sec:org32f2293}
\begin{itemize}
\item Measurement of ammonia is not feasible
\begin{itemize}
\item post-collection elevation, no methods in lit
\end{itemize}
\item Glutamine is another metabolite that is generally elevated in UCDs.
\begin{itemize}
\item spontaneous glutamate and pyroglutamate formation.
\end{itemize}
\item Orotic acid is often elevated in the urine of patients with
OTC deficiency, cannot be used as a marker in blood.
\end{itemize}

\item NBS for Mitochondrial UCDs
\label{sec:org6df4204}
\begin{itemize}
\item Mitochondrial enzymes: NAGS, CPS1 and OTC
\end{itemize}

\begin{enumerate}
\item Biomarkers
\label{sec:orgb3f52cf}
\begin{itemize}
\item Low or decreased plasma levels of citrulline and arginine
\item Urine orotic acid is often elevated in OTCD
\end{itemize}
\end{enumerate}

\item NBS for Cytoplasmic UCDs
\label{sec:orge68f150}
\begin{itemize}
\item Cytoplasmic enzymes: ASS1, ASL, ARG1
\end{itemize}

\begin{enumerate}
\item Biomarkers
\label{sec:org7615aad}
\begin{itemize}
\item ASSD: \(\uparrow\) citrulline
\item ASLD: \(\uparrow\) argininosuccinate, \(\uparrow\) citrulline
\item ARG1D: \(\uparrow\) arginine
\end{itemize}
\end{enumerate}

\item Transporter Defects
\label{sec:org0c927b9}
\begin{itemize}
\item Membrane bound transporters : ORNT1, Citrin
\end{itemize}

\begin{enumerate}
\item Biomarkers
\label{sec:org32b780d}
\begin{itemize}
\item Hyperammonemia-hyperornithinemia-homocitrullinuria syndrome (ORNT1): \(\uparrow\) ornithine
\begin{itemize}
\item Ornithine not elevated in newborns
\end{itemize}
\item Citrullinemia type II (Citrin): \(\uparrow\) citrulline
\end{itemize}
\end{enumerate}

\item NBS for UCDs in the US
\label{sec:orgf559f51}
\begin{itemize}
\item Current newborn screening panels in the United States using tandem
mass spectrometry detect abnormal concentrations of analytes
associated with ASS1 deficiency, and ASL deficiency in all states.

\item Other disorders are screened for in some states only:
\begin{itemize}
\item CPS1 deficiency is screened for in Florida, Maine, Massachusetts,
Mississippi, New Hampshire, Pennsylvania, Rhode Island, and
Vermont.
\item OTC deficiency is screened for in Connecticut, Maine,
Massachusetts, New Hampshire, Rhode Island, and Vermont, and is
likely to be detected in Kentucky and Utah.
\item Arginase deficiency is screened for in 35 states and likely to be
detected in four more.
\item Citrin deficiency is screened for in 36 states and likely to be
detected in 13 more.
\end{itemize}
\end{itemize}

\item Newborn Screening for UCD: OTC
\label{sec:orga640e2b}
\begin{itemize}
\item The sensitivity and specificity of a low citrulline level as a
marker for OTC deficiency in NBS has been questioned.
\begin{itemize}
\item common causes of low citrulline in premature infants or in sick
babies such as those with pathological conditions involving the
small intestine, i.e. short-bowel syndrome
\end{itemize}
\item The detection of OTC deficiency on NBS may be improved by using
Collaborative Laboratory Integrated Reports (CLIR) which includes
glutamine, glutamate, and amino acid ratios in the analysis.
\end{itemize}

\item Newborn Screening for UCD in Ontario
\label{sec:orgec141a0}
\begin{itemize}
\item Screen for ASA and ASL
\item Primary marker is citrulline
\item Secondary markers are:
\begin{itemize}
\item ASA
\item CIT/ORN
\item ASA/ARG
\end{itemize}
\end{itemize}

\item Quantitative FIA-MS/MS
\label{sec:org78d5ebc}
\begin{itemize}
\item Amino acids in the DBS eluate are esterified as butyl esters with butanol-hydrogen chloride
\end{itemize}

\centering
\schemedebug{false}
\schemestart
\chemname{\chemfig[][scale=.33]{H_2N-[::30,,2,](=[::60]O)-[::-60]NH-[::60]-[::-60]-[::60]-[::-60](<[::-60]NH_3^+)-[::60](=[::60]O)-[::-60]OH}}{\tiny citrulline 175 Da}
\+
\chemname{\chemfig[][scale=.33]{HO-[::30]-[::-60]-[::60]-[::-60]}}{\tiny n-butanol 74 Da}
\arrow{-U>[][{\tiny \ce{H2O}}]}
\chemname{\chemfig[][scale=.33]{H_2N-[::30,,2,](=[::60]O)-[::-60]NH-[::60]-[::-60]-[::60]-[::-60](<[::-60]NH_3^+)-[::60](=[::60]O)-[::-60]O-[::60]-[::-60]-[::60]-[::-60]}}{\tiny 232 Da}
\schemestop
\begin{itemize}
\item Citrulline contains a labile amino group that fragments together with butyl formate.
\item CID results in net neutral fragmentation of butyl formate (102 Da) plus \ce{NH3} (17 Da)
\item \href{https://en.wikipedia.org/wiki/Selected\_reaction\_monitoring}{SRM} Citrulline-Bu 232.15 Da \(\to\) 113 Da , loss of 119 Da
\end{itemize}

\centering
\schemedebug{false}
\schemestart
\chemname{\chemfig[][scale=.33]{H_2N-[::60]-[::-60]-[::60]-[::-60]-[::60]N=O=C}}{\tiny 113 Da}
\+
\chemname{\chemfig[][scale=.33]{H-[::60](=[::60]O)-[::-60]O-[::60]-[::-60]-[::60]-[::-60]}}{\tiny 102 Da}
\+
\chemname{\chemfig[][scale=.43]{NH_3}}{\tiny 17 Da}
\schemestop

\begin{itemize}
\item Its name is derived from citrullus, the Latin word for watermelon, from which it was first isolated in 1914 by Koga and Odake.
\end{itemize}

\item Screening Thresholds
\label{sec:org168579f}
CIT \(\ge\) 70 OR \\
[CIT \(\ge\) 40 AND (ASA≥2.5 OR CIT/ARG\(\ge\) 6.61 OR CIT/ORN\(\ge\) 2.40 OR ASA/ORN\(\ge\) 0.10 OR ASA/ARG\(\ge\) 0.12)]

\begin{itemize}
\item Citrulline > 70 \textmu{}mol/L is confirmed

\item Citrulline > 100 \textmu{}mol/L also prompt measurement of ASA

\begin{itemize}
\item Daughter ions of 459 Da
\end{itemize}
\end{itemize}
\end{enumerate}

\subsection{Clinical Challenges}
\label{sec:org694557f}
\begin{enumerate}
\item Severe Mitochondrial UCDs
\label{sec:orgcf37a70}
\begin{itemize}
\item First symptoms occur soon after birth between 12 and 72 h of age
\item Although results were late, some patients still benefited from the
availability of results shortly after presentation.
\end{itemize}

\item Mild UCDs
\label{sec:org48ca195}
\begin{itemize}
\item there are patients described with a late-onset of disease with only
a single, few or even absence of symptom(s) and only a biochemical
phenotype.
\item case of ASSD, described as suffering from mild citrullinemia type 1,
\begin{itemize}
\item a condition allelic to classical citrullinemia type 1 but much
milder and with less, if any need for medical intervention.
\end{itemize}
\item Such patients were often identified in neonatal screening programs
and it has been discussed whether their mild phenotype would result
from early detection and initiation of treatment or from a relevant
residual enzyme or transporter function.
\item Metabolites and/or mutation analysis may help to identify attenuated
patients to avoid medicalization of non-diseases
\end{itemize}
\end{enumerate}
\end{document}